\Topic{三角函数的单调性与奇偶性}
  \Teach{三角函数的单调性与奇偶性}
  \Grade{高一}
  \Name{郑皓天}\FirstTime{20181207}\CurrentTime{20181214}
  % \Name{林叶}\FirstTime{20180908}\CurrentTime{20181125}
  %\Name{1v2}\FirstTime{20181028}\CurrentTime{20181117}
  % \Name{林叶}\FirstTime{20180908}\CurrentTime{20181125}
  % \Name{郭文镔}\FirstTime{20181111}\CurrentTime{20181117}
  % \Name{马灿威}\FirstTime{20181111}\CurrentTime{20181111}
  \newtheorem*{Theorem}{定理}
  \makefront
\vspace{-1.5em}
\startexercise
\begin{exercise}{\heiti 课前检测}\\
  \item
    填写下表,写出下列三角函数的最小正周期、单调性、奇偶性以及对称轴.
    \begin{center}
      \begin{tabular}{|c|c|c|c|c|}
        \hline
      $f(x)$&\mbox{最小正周期}&\mbox{\hspace{8em}单调性\hspace{8em}}&奇偶性&\mbox{\hspace{3em}对称轴\hspace{3em}}\\
        \hline
        $\sin x$&&&&\\
        \hline
        $\cos x$&&&&\\
        \hline
        $\tan x$&&&&\\
        \hline
      \end{tabular}\\
    \end{center}
  \item
    画出$f(x)=2\sin \left(3x-\dfrac{\piup}4 \right)+2$的图像.\\
  \vspace{4cm}
\end{exercise}
\section{习题}
  % \begin{description}
  %   \item [label]
  % \end{description}
  \begin{exercise}
    \item%《习题化知识清单》P72例1
      函数$\dfrac{\sin x+2}{\sin x+1},x\in\left[0,\dfrac{\piup}2\right]$的值域为\tk.
      \begin{answer}
        $\left[\dfrac32,2\right]$
      \end{answer}
    \item%《习题化知识清单》P73知识1-1
      若函数$f(x)=\sin \omega x(\omega>0)$在区间$\left[0,\dfrac{\piup}3 \right]$上单调递增,在区间$\left[ \dfrac{\piup}3,\dfrac{\piup}2 \right]$上单调递减,则$\omega$可以为\xz
      \xx{$\dfrac23$}
        {$\dfrac32$}
        {2}
        {3}
      \begin{answer}
        B
      \end{answer}
    \item%《习题化知识清单》P73知识1-3
      设函数$f(x)=\sqrt{2}\sin \left(\omega x+\varphi+\dfrac{\piup}4 \right)$$\left(\omega>0,\abs{\varphi}<\dfrac{\piup}2 \right)$的最小正周期为$\piup$,且$f(-x)=f(x)$,则\xz
      \xx{$f(x)$在$\left(0,\dfrac{\piup}2 \right)$上单调递减}
        {$f(x)$在$\left(\dfrac{\piup}4,\dfrac{3\piup}4 \right)$上单调递减}
        {$f(x)$在$\left(0,\dfrac{\piup}2 \right)$上单调递增}
        {$f(x)$在$\left(\dfrac{\piup}4,\dfrac{3\piup}4 \right)$上单调递增}
      \begin{answer}
        A
      \end{answer}
    \item%《习题化知识清单》P76单元检测6
      当$x=\dfrac{\piup}4$时,函数$f(x)=A\sin(x+\varphi)(A>0)$取得最小值,则函数$y=f\left(\dfrac{3\piup}4-x \right)$是\xz
      \xx{奇函数且图像关于点$\left(\dfrac{\piup}2,0 \right)$对称}
        {偶函数且图像关于点$\left(\piup,0 \right)$对称}
        {奇函数且图像关于直线$x=\dfrac{\piup}2$对称}
        {偶函数且图像关于点$\left(\dfrac{\piup}2,0 \right)$对称}
      \begin{answer}
        C
      \end{answer}
    \item%《习题化知识清单》P72知识2-1
      函数$y=\tan2\left(x+\dfrac{\piup}4 \right)$\xz
      \xx{是奇函数}
        {是偶函数}
        {既是奇函数又是偶函数}
        {是非奇非偶函数}
      \begin{answer}
        A
      \end{answer}
    \item%《习题化知识清单》P72知识2-2
      不等式$\tan x>a$在$x\in\left(-\dfrac{\piup}4,\dfrac{\piup}2 \right)$上恒成立,则$a$的取值范围是\xz
      \xx{$(-\infty,-1]$}
        {$(-\infty,-1)$}
        {$(-\infty,1]$}
        {$(-\infty,1]$}
    \item%LaTeX-master/sanjiaohanshu/sanjiaohanshu-gaokao.tex 4
      函数$f(x)=\cos\left(\omega x+\varphi\right)$的部分图象如图所示,则$f(x)$的单调递减区间为\xz
      \begin{center}
      \begin{tikzpicture}
        \node[below left](O) at(0,0) {\small$\bm{O}$};
        \draw(0,1)node[right]{\tiny$1$}--(0.1,1);
        \clip(-1.2,-1.2) rectangle (2,1.5);
        \draw[->,>=stealth](-1.2,0)--(2,0) node[below left] (x){$x$};
        \draw[->,>=stealth](0,-1.2)--(0,1.5) node[below right] (y){$y$};
        \draw[domain=-1.2:2,samples=1000] plot(\x,{cos((pi*(\x)+1/4*pi) r)});
        \node[below] (A)at (0.25,0){$\frac{1}{4}$};
        \node[below] (B)at (1.25,0){$\frac{5}{4}$};
      \end{tikzpicture}
      \end{center}
      \xx{$ \left(k\pi-\dfrac{1}{4},k\pi+\dfrac{3}{4}\right),k\in\mathbb{Z}$}
        {$ \left(2k\pi-\dfrac{1}{4},2k\pi+\dfrac{3}{4}\right),k\in\mathbb{Z}$}
        {$ \left(k-\dfrac{1}{4},k+\dfrac{3}{4}\right),k\in\mathbb{Z}$}
        {$\left(2k-\dfrac{1}{4},2k+\dfrac{3}{4}\right),k\in\mathbb{Z} $}
      \begin{answer}
        D
      \end{answer}
    \item%LaTeX-master/sanjiaohanshu/gaokaosection.tex 11
      已知函数$f(x)=2\sin \left(\omega x+\varphi\right),\ x \inR$,\ 其中$ \omega>0,\ -\pi <\varphi\le \pi ,\  $若$f(x)$的最小正周期为$ 6\pi  $,且当$ x=\dfrac{\pi}{2} $时,$f(x)$取得最大值,则\xz
      \xx{$f(x)$在区间$ \left[-2\pi,0\right] $上是增函数}
        {$f(x)$在区间$ \left[-3\pi,-\pi \right] $上是增函数}
        {$f(x)$在区间$ \left[3\pi,5\pi \right] $上是减函数}
        {$f(x)$在区间$ \left[4\pi,6\pi \right] $上是减函数}
      \begin{answer}
        A
      \end{answer}
    \item%《习题化知识清单》P73易混清单例
      函数$y=2\sin\left(\dfrac{\piup}3-2x \right)$的单调增区间为\tk.
      \begin{answer}
        $\left[k\piup+\dfrac{5\piup}{12},k\piup+\dfrac{11\piup}{12} \right],k\in\mathbb{Z}$
      \end{answer}
    \item%LaTeX-master/sanjiaohanshu/gaokaosection.tex 31
      把函数$ y=\sin 2x $的图象沿$x$轴向左平移$ \dfrac{\pi}{6} $个单位,纵坐标伸长到原来的2倍(横坐标不变)后得到函数$ y=f(x) $的图象,对于函数$ y=f(x) $有以下四个判断:\\
      \ding{192} 该函数的解析式为$ y=2\sin \left(2x+\dfrac{\pi}{6}\right) $;\\
      \ding{193} 该函数图象关于点$ \left(\dfrac{\pi}{3},0\right) $对称;\\
      \ding{194} 该函数在$ \left[0,\dfrac{\pi}{6}\right] $上是增函数;\\
      \ding{195} 若函数$ y=f(x)+a $在$ \left[0,\dfrac{\pi}{2}\right] $上的最小值为$ \sqrt{3},\  $则$ a=2\sqrt{3} .$\\
      其中,正确判断的序号是\tk.
      \begin{answer}
        \circled{2}\circled{3}\circled{4}
      \end{answer}
    \item%《习题化知识清单》P77单元检测15
      已知函数$f(x)=\sin(\omega x+\varphi)$$(\omega>0,0<\varphi<\piup)$的最的最小正周期为$\piup$,且函数$f(x)$的图像过点$\left(\dfrac{\piup}2,-1\right)$.\\
      (1)求$\omega$和$\varphi$的值;
      (2)设$g(x)=f(x)+f\left(\dfrac{\piup}4-x \right)$,求函数$g(x)$的单调递增区间.
      \begin{answer}
        (1)$\omega=2$,$\varphi=\dfrac{\piup}2$.
        (2)$\left[k\piup-\dfrac{3\piup}8,k\piup+\dfrac{\piup}8\right](k\in\mathbb{Z})$
      \end{answer}
  \end{exercise}
\newpage
\section{课后作业}
  \begin{exercise}
    \item%《习题化知识清单》P72知识3-3
      若函数$y=2\cos(2x+\varphi)$是偶函数,且在$\left(0,\dfrac{\piup}4\right)$上是增函数,则实数$\varphi$可能是\xz
      \xx{$-\dfrac{\piup}2$}
        {0}
        {{$\dfrac{\piup}2$}}
        {$\piup$}
      \begin{answer}
        D
      \end{answer}
    \item%《习题化知识清单》P72方法2-1
      函数$\abs{\sin x}$的一个单调区间是\xz
      \xx{$\left(\dfrac{\piup}2,\piup\right)$}
        {$\left(\piup,2\piup\right)$}
        {$\left(\piup,\dfrac{3\piup}2\right)$}
        {$\left(0,\piup\right)$}
      \begin{answer}
        C
      \end{answer}
    \item%LaTeX-master/sanjiaohanshu/gaokaosection.tex 13
       已知函数$f(x)=\Bigg\{\begin{aligned}
      \sin(x+a),x\le 0\\\cos (x+b),x>0
      \end{aligned}$是偶函数,则下列结论可能成立的是\xz
       \xx{$ a=\dfrac{\pi}{4},b=-\dfrac{\pi}{4}$}
        {$ a=\dfrac{2\pi}{3},b=\dfrac{\pi}{6}$}
        {$a=\dfrac{\pi}{3},b=\dfrac{\pi}{6} $}
        {$ a=\dfrac{5\pi}{6},b=\dfrac{2\pi}{3}$}
      \begin{answer}
        C
      \end{answer}
    \item%《习题化知识清单》P77单元检测10
      定义在$\mathbb{R}$上的偶函数$f(x)$满足$f(x+1)=-\dfrac2{f(x)}(f(x)\neq0)$,且在区间$(2013,2014)$上单调递增.已知$\alpha,\beta$是锐角三角形的两个内角,则$f(\sin\alpha),f(\cos\beta)$的大小关系是\xz
      \xx{$f(\sin\alpha)<f(\cos\beta)$}
        {$f(\sin\alpha)>f(\cos\beta)$}
        {$f(\sin\alpha)=f(\cos\beta)$}
        {以上情况均有可能}
    \item%《习题化知识清单》P72例1-1
      函数$\dfrac{\sin x-2}{2+\sin x}$的最大值为\tk.
      \begin{answer}
        $-\dfrac13$
      \end{answer}
    \item%LaTeX-master/sanjiaohanshu/gaokaosection.tex 26
      已知函数$f(x)=\sin (2x+\varphi)$,若$    f\left(\dfrac{\piup}{12}\right)-f\left(-\dfrac{5\piup}{12}\right)=2 $,则函数$f(x)$的单调增区间为\tk.
      \begin{answer}
        $\left[k\piup-\dfrac{5\piup}{12},k\piup+\dfrac{\piup}{12}\right],k\in\mathbb{Z}$
      \end{answer}
    \item%函数y=Asin(ωx+φ)的图象及简单应用P11.9
      若$f(x)=\cos\left(2x+\dfrac{\piup}3+\varphi\right)$$(\abs{\varphi}<\dfrac{\piup}2)$是奇函数,则$\varphi=$\tk.
      \begin{answer}
        $\dfrac{\piup}6$
      \end{answer}
    \item%《习题化知识清单》P74易混清单练
      函数$y=2\sin\left(\dfrac{\piup}3-2x \right)$的单调减区间为\tk.
      \begin{answer}
        $\left[k\piup+\dfrac{\piup}6,k\piup+\dfrac{2\piup}3 \right],k\in\mathbb{Z}$
      \end{answer}
    \item%《习题化知识清单》P77单元检测12
      设$\omega>0$,若函数$f(x)=2\sin \omega x(\omega>0)$在区间$\left[-\dfrac{\piup}3,\dfrac{\piup}4 \right]$上单调递增,则$\omega$取值范围是\tk.
      \begin{answer}
        $\left(0,\dfrac32\right]$
      \end{answer}
    \item%函数y=Asin(ωx+φ)的图象及简单应用P11.14
      已知曲线$y=A\sin(\omega x+\varphi)$$(A>0,\omega>0,\abs{\varphi}\leqslant\dfrac{\piup}2)$上最高点为$(2,\sqrt{2})$,该最高点与相邻的最低点间的曲线与$x$轴交于点$(6,0)$.\\
      (1)该函数的解析式;\\
      (2)该函数在$x\in[-6,0]$上的值域.
      \begin{answer}
        (1)$y=\sqrt{2}\sin(\dfrac{\piup}8x+\dfrac{\piup}4)$;
        (2)$[-\sqrt{2},0]$
      \end{answer}
  \end{exercise}
\stopexercise
\newpage
\section{参考答案}
\printanswer

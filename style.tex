%\usepackage[BoldFont,SlantFont]{xeCJK}
\usepackage[utf8x]{inputenc}
\usepackage[left=1.5cm,right=1.5cm,top=2.5cm,bottom=2cm,footskip=1.5cm,headheight=2.5cm,headsep=0.1cm,\footskip=0.2cm]{geometry}
\usepackage{graphicx}
\usepackage{fancyhdr}
\usepackage[export]{adjustbox}
\usepackage{tabularx}
\usepackage[hidelinks]{hyperref}
\usepackage{latexexercise}
%数学符号
\usepackage{amsmath,amsfonts,amssymb,times }%AMS宏包
%绘图
\usepackage{tikz,
            pgfplots,
            tkz-euclide}
\usetikzlibrary{automata,
                positioning}

% \setCJKmainfont[BoldFont = simhei.ttf]{simfang.ttf}

% \defaultfontfeatures{Scale=1.2}
% \linespread{1.2}

\usepackage{zhnumber}

\fancypagestyle{plain}
{\fancyhf{}
\renewcommand{\headrulewidth}{1.2pt}
\fancyhead[L]{\begin{tabular}[c]{c@{}c @{}}
{\vspace{-0.8em}}
\hspace{-4em}\includegraphics[height=2.32cm]{Logo.jpg}
\end{tabular}
}
% \fancyhead[R]{\Large 清华大学\phantom{a} \\ 数学科学系}
\fancyhead[R]{\begin{tabular}[t]{@{} c @{}}
      {\vspace{0.3em}}
      \small{福州清大教育\hspace{6em}} \\
      \small{FuZhou Qingda Education\hspace{6em}} \\
\end{tabular}
}

\renewcommand{\footrulewidth}{.4pt}
\fancyfoot[C]{
\resizebox{!}{0.9\baselineskip}
{\begin{tabular}[c]{@{} c @{}}
      \large{\textbf{鼓楼校区:87500166\hspace{2em}}\textbf{金山校区:87521588\hspace{2em}}\textbf{台江校区:83310089\hspace{2em}}} \\
      \small{\vspace{1em}清大教学部内部版权所有\hspace{2em}未经允许\hspace{2em}严禁复制} \\
      \small{第\thepage 页}
\end{tabular}}
}
}

\pagestyle{plain}

\pagenumbering{arabic} % switch off page numbering

%选择题括号的配置
\newcommand{\xz}[1][1]{\nolinebreak\dotfill\mbox{\raisebox{-1.8pt}
   {$\cdots$}(\hspace{#1 cm})}\\ }
%z选择题四个选项的样式
%       当你排版选择题的时候,你只需输入
%\xx{选项 A 的内容}{选项 B 的内容}{选项 C 的内容}{选项 D 的内容}
%       上面代码的意思是取四个选项的最长宽度,加上 “A.” %以及左右留空,作为选项的最长宽度。将
%它和行宽的 1/2 以及 1/4 作比较, 来决定把 4 个选项排成一行、两行或者四行及以上。
\usepackage{ifthen} 
\newlength{\la}
\newlength{\lb}
\newlength{\lc}
\newlength{\ld}
\newlength{\lhalf}
\newlength{\lquarter}
\newlength{\lmax}
\newcommand{\xx}[4]{\ \\[.5pt]%
  \settowidth{\la}{A.~#1~~~}
  \settowidth{\lb}{B.~#2~~~}
  \settowidth{\lc}{C.~#3~~~}
  \settowidth{\ld}{D.~#4~~~}
  \ifthenelse{\lengthtest{\la > \lb}}
    {\setlength{\lmax}{\la}}
    {\setlength{\lmax}{\lb}}
  \ifthenelse{\lengthtest{\lmax < \lc}}
    {\setlength{\lmax}{\lc}}  {}
  \ifthenelse{\lengthtest{\lmax < \ld}}
    {\setlength{\lmax}{\ld}}  {}
  \setlength{\lhalf}{0.5\linewidth}
  \setlength{\lquarter}{0.25\linewidth}
  \ifthenelse{\lengthtest{\lmax < \lquarter}}
    {\noindent\makebox[\lquarter][l]{A.~#1~~~}%
     \makebox[\lquarter][l]{B.~#2~~~}%
     \makebox[\lquarter][l]{C.~#3~~~}%
     \makebox[\lquarter][l]{D.~#4~~~}}%
    {\ifthenelse{\lengthtest{\lmax < \lhalf}}
       {\noindent\makebox[\lhalf][l]{A.~#1~~~}%
        \makebox[\lhalf][l]{B.~#2~~~}\\%
        \makebox[\lhalf][l]{C.~#3~~~}%
        \makebox[\lhalf][l]{D.~#4~~~}}
       {\noindent A.~#1  \\ B.~#2 \\ C.~#3 \\ D.~#4 }
    }
}
%(6) 填空题横线的排版
%       当编写到填空题的时候,加上 \tk 即可,可以加在一句话中间,也可以加到一句话末尾,当在
%末尾的时候,可以在后面加上句号。
%       横线的默认长度是 2.5cm, 可以使用\tk[3]将横线改为 3cm。
\newcommand{\tk}[1][2.5]{\,\underline{\mbox{\hspace{#1 cm}}}\,}












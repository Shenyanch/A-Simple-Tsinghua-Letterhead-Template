\Topic{函数及其性质}
  \Teach{函数的单调性}
  \Grade{高一}
  \Name{郑皓天}\FirstTime{20181207}\CurrentTime{20181207}
  % \Name{林叶}\FirstTime{20180908}\CurrentTime{20181125}
  %\Name{1v2}\FirstTime{20181028}\CurrentTime{20181117}
  % \Name{林叶}\FirstTime{20180908}\CurrentTime{20181125}
  % \Name{郭文镔}\FirstTime{20181111}\CurrentTime{20181117}
  % \Name{马灿威}\FirstTime{20181111}\CurrentTime{20181111}
  \newtheorem*{Theorem}{定理}
  \makefront
\vspace{-1.5em}
\startexercise
\begin{exercise}{\heiti 课前检测}\\
  填写下表,写出各函数的定义域、值域 、单调性以及奇偶性.
  \begin{center}
    \begin{tabular}{|c|c|c|c|c|}
      \hline
    $f(x)$&\mbox{\hspace{1.5em}定义域\hspace{1.5em}}&\mbox{\hspace{2em}值域\hspace{2em}}&\mbox{\hspace{8em}单调性\hspace{8em}}&\mbox{\hspace{1.2em}奇偶性\hspace{1.2em}}\\
      \hline
      $x$&&&&\\
      \hline
      $x^2$&&&&\\
      \hline
      $\log_2x$&&&&\\
      \hline
      $3^x$&&&&\\
      \hline
      $\frac{1}{x}$&&&&\\
      \hline
      $\sqrt{x}$&&&&\\
      \hline
      $\log_x2$&&&&\\
      \hline
    \end{tabular}\\
  \end{center}
\end{exercise}
\section{函数的概念与表示}
  \begin{description}
    \item [定义] 一般地,有:\\
      设 $A$,$B$ 是非空的数集,如果按照某种确定的对应关系 $f$,使对于集合$A$中的任意一个数 $x$,在集合 $B$ 中都有唯一确定的数 $f(x)$ 和它对应,那么就称 $f: A\mapsto B$ 为从集合 $A$ 到集合 $B$ 的一个函数,记作
      $$y=f(x),\qquad x\in A.$$
      其中,$x$ 叫做自变量,$x$ 的取值范围 $A$ 叫做函数的定义域;与 $x$ 的值相对应的 $y $值叫做函数值,函数值的集合$\{f(x)|x\in A\}$叫做函数的值域,值域是集合$B$ 的子集.
      \begin{itemize}[leftmargin=*]
        \kaishu
        \item 函数是两个数集间的一种对应关系;
        \item 未指明定义域的情况下,默认定义域取使得对应关系有意义的所有实数. 具体如下:
        \begin{enumerate}[label=\circled{\arabic*}]
          \item 分式的分母不为0;
          \item 偶次根式的被开方数不小于0;
          \item 零次或负次指数次幂的底数不为零;
          \item 对数的真数大于0;
          \item 指数、对数函数的底数大于0且不等于1;
          \item 实际问题对自变量的限制.
        \end{enumerate}
        \item 若函数$f(x)$定义域为$D$,且$f(A)$存在,则$$A\in D.$$
      \end{itemize}
  \end{description}
  \begin{exercise}
    \item
      函数$f(x)=\sqrt{2^x-1}$的定义域是\xz
      \xx{$ \left[0,+\infty\right)$}{$ \left[1,+\infty\right)$}{$ \left(-\infty,0\right]$}{$ \left(-\infty,1\right]$}
      \begin{answer}
        A
      \end{answer}
    \item
      函数$f(x)=\dfrac{1}{\sqrt{\left(\log_2x\right)^2-1}}$的定义域为\xz
      \xx{$ \left(0,\dfrac{1}{2}\right)$}{$ \left(2,+\infty\right)$}{$ \left(0,\dfrac{1}{2}\right)\bigcup\left(2,+\infty\right)$}{$ \left(0,\dfrac{1}{2}\right]\bigcup\left[2,+\infty\right)$}
      \begin{answer}
        C
      \end{answer}
    \item
      已知函数$f(x)$的定义域为$(-1,0)$,则函数$f(2x+1)$的定义域为\xz
      \xx{$(-1,1)$}{$\left(-1,-\dfrac{1}{2}\right)$}{$(-1,0)$}{$\left(\dfrac{1}{2},1\right)$}
      \begin{answer}
        B
      \end{answer}
    \item
      已知函数$f(2x+1)$的定义域为$\left(-2,\dfrac{1}{2}\right)$,则函数$f(x)$的定义域为\xz
      \xx{$ \left(-\dfrac{3}{2},-\dfrac{1}{4}\right)$}{$ \left(-1,\dfrac{3}{2}\right)$}{$ \left(-3,2\right)$}{$ \left(-3,3\right)$}
      \begin{answer}
        A
      \end{answer}
    \item
      下列函数中,其定义域和值域分别与函数$y=10^{\lg x}$的定义域和值域相同的是\xz
      \xx{$y=x$}{$y=\lg x$}{$y=2^x$}{$y=\dfrac{1}{\sqrt{x}}$}
      \begin{answer}
        D
      \end{answer}
  \end{exercise}
\section{函数的奇偶性}
  \begin{description}
    \item[几何定义] 一般地,图像关于$y$轴对称的函数称为偶函数,图像关于原点对称的函数称为奇函数.
    \item[代数定义]
    \begin{enumerate}[label=\arabic*)]
      \item[] \hspace{-2em}若对于函数$f(x)$定义域内任意一个$x$,都有$f(-x)=f(x)$,则函数$f(x)$称为偶函数;
      \item[] 若对于函数$f(x)$定义域内任意一个$x$,都有$f(-x)=-f(x)$,那么函数$f(x)$称为奇函数;
    \end{enumerate}
    {\kaishu
      奇函数与偶函数的定义域关于原点对称
    }
    \item[性质]
      \begin{itemize}
        \item
          奇函数左右对应中会有负号,偶函数没有负号,此处的规律可以参考“负负得正”.{\kaishu (以下假设奇偶函数都不恒为$ 0 $)}
          \begin{enumerate}[label=\circled{\arabic*}]
            \kaishu
            \item 奇$\pm$奇=奇;\  偶$\pm $偶=偶;
            \  奇$\pm$偶=非奇非偶
            \item 奇$\times(\div)$奇=偶;\ 偶$\times(\div)$偶=偶;\  奇$\times(\div)$偶=奇.
            \item 当复合函数的内外两层函数都具有奇偶性时,有偶即偶,两奇为奇.
          \end{enumerate}
        \item
          奇(偶)函数在关于原点对称的两个区间上具有相同(相反)的单调性;
        \item 若奇函数$f(x)$在原点有定义,则$f(x)=0$.
      \end{itemize}
  \end{description}
  \begin{exercise}
    \item
      设奇函数$f(x)$在$ \left(0,+\infty\right) $上增函数且$ f(1)=0 $,则不等式$ \dfrac{f(x)-f(-x)}{x}<0 $的解集为\xz
      \xx{$ \left(-1,0\right)\bigcup \left(1,+\infty\right)$}{$ \left(-\infty,-1\right)\bigcup \left(0,1\right)$}{$ \left(-\infty,-1\right)\bigcup \left(1,+\infty\right)$}{$ \left(-1,0\right)\bigcup \left(0,1\right)$}
      \begin{answer}
        D
      \end{answer}
    \item
      奇函数$f(x)$的定义域为$ \mathbf{R} ,~$若$ f(x+2) $为偶函数,且$ f(1)=1,~ $则 $f(8)+f(9)=$\xz
      \xx{$ -2 $}{$ -1 $}{$ 0 $}{$ 1 $}
      \begin{answer}
        D
      \end{answer}
    \item
      设函数$f(x),g(x)$的定义域都为$\mathbf{R}$,且$f(x)$是奇函数,$g(x)$是偶函数,则下列结论正确的是\xz
      \xx{$f(x)g(x)$是偶函数}{$\abs{f(x)}g(x)$是奇函数}{$f(x)\abs{g(x)}$是奇函数}{$\abs{f(x)g(x)}$是奇函数}
      \begin{answer}
        C
      \end{answer}
    \item
      已知函数$f(x)=\ln \left(\sqrt{1+9x^2}-3x\right)+1$,则$ f(\lg2)+f\left(\lg\dfrac{1}{2}\right) $等于\xz
      \xx{$-1$}{$0$}{$1$}{$2$}
      \begin{answer}
        D
      \end{answer}
    \item
      已知函数$f(x)$是定义在$ \mathbf{R} $上的偶函数,且在区间$ \left[0,+\infty\right) $上单调递增,若实数$ a $满足$ f(\log_2a) +f(\log_\frac{1}{2}a)\le 2f(1)$,则$ a $的取值范围是\xz
      \xx{$ \left[1,2\right]$}{$ \left(0,\dfrac{1}{2}\right]$}{$ \left[\dfrac{1}{2},2\right]$}{$ \left(0,2\right]$}
      \begin{answer}
        C
      \end{answer}
    \item
      %30次课学完高中数学P3.7(2)
      已知函数$f(x)$是定义在$\mathbb{R}$上的奇函数,$g(x)$是定义在$\mathbb{R} $的偶函数,且$f(x)-g(x)=1-x^2-x^3 $,则$g(x) $的解析式为\xz
      \xx{$1-x^2$}{$2-2x^2$}{$x^2-1 $}{$2x^2-2 $}
      \begin{answer}
      C
      \end{answer}
    \item
      若$f(x)=x\ln (x+\sqrt{a+x^2})$为偶函数,则$ a= $\tk.
      \begin{answer}
        1
      \end{answer}
  \end{exercise}
\vspace{2em}
\section{函数的单调性}
  \begin{description}
    \item [定义] 一般地,设函数 $f(x)$ 的定义域为$I$:
      \begin{enumerate}[label=\arabic*)]
        \item 如果对于定义域 $I$ 内某个区间 $D$ 上的任意两个自变量的值 $x_1$,$x_2$,当 $x_1<x_2$ 时,都有 $f(x_1)<f(x_2)$,那么就说函数$f(x)$ 在区间 $D$ 上是增函数;
        \item 如果对于定义域 $I$ 内某个区间 $D$ 上的任意两个自变量的值 $x_1$,$x_2$,当 $x_1<x_2$ 时,都有 $f(x_1)>f(x_2)$,那么就说函数$f(x)$ 在区间 $D$ 上是减函数.
      \end{enumerate}
      \hspace{2em}如果函数 $f(x)$ 在区间 $D$ 上是增函数或减函数,那么就说函数 $f(x)$ 在区间 $D$ 具有(严格的)单调性,区间 $D$ 叫做函数 $f(x)$ 的单调区间.
      \begin{itemize}[leftmargin=*]
        \kaishu
        \item 函数的单调性是定义在区间上的,即单调性是函数在某个区间上的性质;
        \item 单调区间是定义域的子集;
        \item 单调区间的写法: 尽可能地使用闭区间(不能写成闭区间的三种情形: $\infty$符号旁; 端点不在函数定义域内; 端点处函数增减性发生变化);
        \item 自变量量和函数值:变化趋势相同时,函数单调增;变化趋势相反时,函数单调减;简记为:同增异减.\\\vspace{-8pt}
        $$\text{单调递增}\Leftrightarrow(
        x_1-x_2)[f(x_1)-f(x_2)]>0\Leftrightarrow \dfrac{f(x_1)-f(x_2)}{x_1-x_2}>0$$\\\vspace{-24pt}
        $$\text{单调递减}\Leftrightarrow(x_1-x_2)[f(x_1)-f(x_2)]<0\Leftrightarrow \dfrac{f(x_1)-f(x_2)}{x_1-x_2}<0$$
      \end{itemize}
    \item [判定] 函数单调性的判断目前有以下几种常见方法:
    \begin{itemize}[leftmargin=*]
      \item 根据图像判断;
      \item 根据定义;
        由定义证明函数 $f(x)$ 在给定区间 $D$ 上单调性的步骤 :\\
        \begin{enumerate}[label=\circled{\arabic*}]
          \kaishu
          \item 取值: 任取 $x_1,x_2\in D$ ,且$x_1<x_2$ ;
          \item 作差或作商: $f(x_1)-f(x_2)$或$f(x_1)/f(x_2)$;(当$f(x)$在区间$D$内恒大于0或恒小于0时才可使用作商法)
          \item 变形: 因式分解、配方、通分、根式有理化等等,化简至能够简单判断正负号的式子; \item 定号: 判断 $f(x_1)-f(x_2)$的正负(或$f(x_1)/f(x_2)$与1比大小),进一步判断 $f(x_1)$与$f(x_2)$的大小值关系;
          \item 得出结论:$f(x_1)<f(x_2)$时函数$f(x)$单调递增;$f(x_1)>f(x_2)$时函数$f(x)$单调递减.
        \end{enumerate}
      \item 根据单调性已知的函数,并利用函数单调性的几个结论判断:
        \begin{enumerate}[label=\circled{\arabic*}]
          \kaishu
          \item $f(x)$ 与 $f(x)+C$($C$是常数)具有相同的单调性;
          \item $k>0$时,$k f(x)$ 与 $f(x) $单调性相同; $k<0$时,$k f(x)$ 与 $f(x)$ 单调性相反;
          \item 在公共定义域内,两增函数相加仍为增函数; 减函数相减仍为减函数;
          \item 对于复合函数,“同增异减”,即:\\
          若$\mu=g(x)$在 $[a , b]$ 上是增(减)函数,函数$y=f(\mu)$ 在区间 $[g(a) , g(b)]$ (或区间 $[g(b) , g(a)]$)上是增(减)函数,那么复合函数$y=f[g(x)]$在区间$[a , b]$ 上一定是单调的,且若 $f(\mu ) $与 $g(x)$ 单调性相同, 则复合函数$y=f[g(x)]$单调递增; 若 $f(\mu ) $与 $g(x)$ 单调性相反, 则复合函数$y=f[g(x)]$单调递减.
        \end{enumerate}
    \end{itemize}
  \end{description}
  \begin{exercise}
    \item
      设$f(x),\ g(x)$都是单调函数,有如下四个命题:\\
      \ding{192}若$f(x)$单调递增,$g(x)$单调递增,则$f(x)-g(x)$单调递增;\\
      \ding{193}若$f(x)$单调递增,$g(x)$单调递减,则$f(x)-g(x)$单调递增;\\
      \ding{194}若$f(x)$单调递减,$g(x)$单调递增,则$f(x)-g(x)$单调递减;\\
      \ding{195}若$f(x)$单调递减,$g(x)$单调递减,则$f(x)-g(x)$单调递减;\\
      其中,正确的命题是\xz
      \xx{\ding{192}\ding{194}}{\ding{192}\ding{195}}{\ding{193}\ding{194}}{\ding{193}\ding{195}}
      \begin{answer}
        C
      \end{answer}
    \item
      函数$y=-\sqrt{1-4x^2}$的单调递减区间是\xz
      \xx{$\left(-\infty,\dfrac{1}2\right]$}
        {$\left[\dfrac{1}2,+\infty\right)$}
        {$\left[-\dfrac{1}2,0\right]$}
        {$\left[0,\dfrac{1}2\right]$}
      \begin{answer}
        C
      \end{answer}
    \item
      (福州八中 15-16 高一期中考,2)设偶函数 $f(x)$的定义域为$\mathbb{R}$,当 $x\in[0,+\infty)$时,$f(x)$是增函数,则$f(-2)$,$f(\pi)$,$f(-3)$的大小关系是\xz
      \xx{$f(\pi)>f(-3)>f(-2)$}
          {$f(\pi)>f(-2)>f(-3)$}
          {$f(\pi)<f(-3)<f(-2)$}
          {$f(\pi)<f(-2)<f(-3)$}
      \begin{answer}
        A
      \end{answer}
    \item
      (福州高级中学 16-17 高一期中考,11)定义在 $\mathbb{R}$上的偶函数$f(x)$,当$x\in[1,2]$时,$f(x)<0$且$f(x)$增函数,给出下列四个结论:\\
      (1)$f(x)$在$[-2,-1]$上单调递增;\hspace{4em}(2)当$x\in[-2,-1]$时,有$f(x)<0$;\\
      (3)$f(-x)$在$[-2,-1]$上单调递减;\hspace{3.5em}(4)$|f(x)|$在$[-2,-1]$上单调递减.\\
      其中正确的结论是\xz
      \xx{(1)(3)}{(2)(4)}{(2)(3)}{(3)(4))}
      \begin{answer}
        C
      \end{answer}
    \item
      【2016师大附中18】 (本小题满分12分)
      已知函数$f(x)$为$\mathbb{R}$上的偶函数. $x\leq0$时$f(x)=4^{-x}-a\cdot~2^{-x},(a>0)$\\
      (\Rmnum{1})求函数$f(x)$在$(0,+\infty)$上的解析式;
      (\Rmnum{2})求函数$f(x)$在$[0,+\infty)$上的最小值.
      \begin{answer}
        (I)$x\in(0,+\infty)$时,$f(x)=f(-x)=4^x-a\cdot 2^x$;\hspace{2em}
        (II)$a\geq2$时,$f(x)_{\min}=f(\dfrac{a}2)=-\dfrac{a^2}4$;$0<a<2$时,$f(x)_{\min}=f(0)=1-a$
      \end{answer}
    \vspace{12em}
    \item
      %福州重点中学期中考真题分类汇编 2函数的相关性质.pdf P21
      (福州市格致中学 2016-2017 高一上期中考试数学学科试卷22)已知二次函数 $f ( x )= ax^2+ bx+3$ 是偶函数,且 过点$(-1,4)$,$ g ( x )= x + 4$ .\\
      (\Rmnum{1} )求 $f (x) $的解析式;\\
      (\Rmnum{2} )求函数 $F ( x )= f (2^x )+ g (2^{x+1} )$ 的值域; \\
      (\Rmnum{3} )若 $f ( x ) \geq g ( mx +m )$ 对 $x\in [2, 6] $恒成立,求实数 $m$ 的取值范围.\\
      \begin{answer}
      (I) $f(x)=ax^2+3$; (II) $(7,+\infty)$; (III) $m\leq1$
      \end{answer}
      \vspace{13em}
    \vspace{12em}
  \end{exercise}

\newpage
\section{课后作业}
  \begin{exercise}
    \item
      如果$f(x)$是定义在$\mathbf{R}$上的奇函数,那么下列函数中一定是偶函数的是\xz      \xx{$ x+f(x)$}{$ xf(x)$}{$ x^2+f(x)$}{$ x^2f(x)$}
      \begin{answer}
        B
      \end{answer}
    \item
      已知函数$g(x)=f(x)-x$是偶函数,且$ f(3)=4 $,则$ f(-3)= $\xz
      \xx{$-4$}{$-2$}{$0$}{$4$}
      \begin{answer}
        B
      \end{answer}
    \item
      设函数$f(x),g(x)$的定义域都为$\mathbf{R}$,且$f(x)$是奇函数,$g(x)$是偶函数,则下列结论正确的是\xz
      \xx{$f(x)+\abs{g(x)}$是偶函数}{$f(x)-\abs{g(x)}$是奇函数}{$\abs{f(x)}+g(x)$是偶函数}{$\abs{f(x)}-g(x)$是奇函数}
      \begin{answer}
        C
      \end{answer}
    \item
      (福州格致中学 16-17 高一期中考,10)若$f(x)=-x^2+2ax$与$g(x)=\dfrac a{x+1}$ 在区间$[1,2]$上都是减函数,则实数$a$ 的取值范围\xz
      \xx{$(-1,0)\cup(0,1)$}
          {$(-1,0)\cup(0,1]$}
          {$(0,1)$}{$(0,1]$}
      \begin{answer}
        D
      \end{answer}
    \item
      设函数$f(x)=\lg \dfrac{2+x}{2-x}$,则$ f\left(\dfrac{x}{2}\right)+f\left(\dfrac{2}{x}\right) $的定义域为\xz
      \xx{$\left(-4,0\right)\bigcup \left(0,4\right) $}{$\left(-4,-1\right)\bigcup \left(1,4\right) $}{$ \left(-2,-1\right)\bigcup \left(1,2\right)$}{$ \left(-4,-2\right)\bigcup \left(2,4\right)$}
      \begin{answer}
        B
      \end{answer}
    \item
      %30次课学完高中数学P3.7例6(4)
      (2009四川卷文理12) 已知函数$f(x)$是定义在实数集$\mathbb{R}$上的不恒为零的偶函数,且对任意实数$x$都有$xf(x+1)=(1+x)f(x) $,则$f(\frac{5}{2}) $的值是\xz
      \xx{0}
      {$\frac12$}
      {1}
      {$\frac52$}
      \begin{answer}
      A
      \end{answer}
    \item
      若函数$f(x)=\ln (e^{3x}+1)+ax$为偶函数,则$ a= $\tk.
      \begin{answer}
        $-\dfrac{3}2$
      \end{answer}
    \item
      若$f(x)$是定义在 $\mathbf{R} $上的奇函数,当$ x\le0 $时,$f(x)=2x^2-x$,则$f(1)=$\tk.
      \begin{answer}
        -3
      \end{answer}
    \item
      设函数$f(x)$在$ \left(-\infty,+\infty\right) $内有定义,下列函数:\\
      \ding{192} $ y=-\left|f(x)\right| $\qquad\ding{193} $ y=xf(x^2) $;\\
      \ding{194} $ y=-f(-x) $\qquad \ding{195} $ y=f(x)-f(-x) $.\\
      中必为奇函数的有\tk.(要求填写正确答案的序号)
      \begin{answer}
        \circled{2}\circled{4}
      \end{answer}
    \item
      【2016 福州三中 17】(本小题满分 12 分)
      已知函数$f(x)=\log_39x\cdot\log_3x+2 $,$x\in[\frac19,3]$.\\
      (1) 求$f(x)$最小值和最大值;\\
      (2) 若不等式$f(x)-2m+1>0 $恒成立,求实数$m$ 的取值范围.
      \begin{answer}
        (1) $f_{\min}(x)=f(\frac13)=1$
              $f_{\max}(x)=f(3)=5$
        (2) $m\in(-\infty,1)$
      \end{answer}
    \vspace{22em}
    \item
      %福州重点中学期中考真题分类汇编 2函数的相关性质.pdf P11
      (福州八中 2015—2016 高一上学期期中考试23)设 $f (x )$ 是定义在 $\mathbb{R}$ 上的奇函数,且对任意 $a,b\in \mathbb{R}$ ,当$a+b\neq0$时,都有 $\frac{f(a)+f(b)}{a+b}>0$\\
      (1)若 $a> b$ ,试比较 $f (a ) $与 $f (b)$ 的大小关系;\\
      (2)若 $f (9^x- 2\cdot 3^x )+ f ( 2\cdot 9^x-k )> 0 $对任意 $x\in[0,\infty )$ 恒成立,求实数 $k$ 的取值范围.
      \begin{answer}
      (1)$f(a)>f(b)$;
      (2)$k<1$.
      \end{answer}
    \vspace{22em}
    \item
      %福州重点中学期中考真题分类汇编 2函数的相关性质.pdf P23
      (福州市屏东中学 2016-2017 高一上期中22)已知函数$f(x)=2^x-2^{-2} $,定义域为$\mathbb{R} $;函数 $g(x)=2^{x+1}-2^{2x} $,定义域为$[-1,1] $.\\
      (1)判断函数$f(x) $的奇偶性,不用证明;\\
      (2)求函数$g(x) $ 的最值;\\
       (3) 若不等式$f(g(x))\leq f(-3am+m^2+1) $对$x\in[-1,1] $,$a\in[-2,2] $ 上恒成立,求 $m$ 的取值范围.
       \begin{answer}
       (1) 增函数; (2) $g_{\max}(t)=g(1)=1 $; $g_{\min}(t)=g(2)=0 $; (3) $m\in (-\infty,-6)\cup[6,+\infty)\cup\{0\} $
      \end{answer}
      \vspace{20em}
  \end{exercise}
\stopexercise
\newpage
\section{部分参考答案}
\printanswer

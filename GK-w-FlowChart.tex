\Topic{数列、框图}
  \Teach{}
  \Grade{高三}
  % \Name{郑皓天}\FirstTime{20181207}\CurrentTime{20181207}
  % \Name{林叶}\FirstTime{20180908}\CurrentTime{20181125}
  %\Name{1v2}\FirstTime{20181028}\CurrentTime{20181117}
  % \Name{林叶}\FirstTime{20180908}\CurrentTime{20181125}
  % \Name{郭文镔}\FirstTime{20181111}\CurrentTime{20181117}
  % \Name{马灿威}\FirstTime{20181111}\CurrentTime{20181111}
  % \Name{黄亭燏}\FirstTime{20181231}\CurrentTime{20181231}
  % \Name{王睿妍}\FirstTime{20190129}\CurrentTime{}
  \Name{郑旭晶}\FirstTime{20190423}\CurrentTime{20190509}
  \newtheorem*{Theorem}{定理}
  \makefront
\vspace{-1.5em}
  \tikzstyle{startstop} = [rectangle,rounded corners,minimum height=0.7cm,minimum width=1.2cm,text centered, draw=black]
  \tikzstyle{io} = [trapezium, trapezium left angle = 70,trapezium right angle=110,minimum height=0.7cm,minimum width=1.8cm,text centered,draw=black]
  \tikzstyle{process} = [rectangle,minimum height=0.7cm,minimum width=1.8cm,text centered,draw=black]
  \tikzstyle{decision} = [diamond,shape aspect=2.5,minimum height=0.5cm,text centered,draw=black]
  \tikzstyle{arrow} = [thick,->,>=stealth]
\startexercise
\section{数列基本概念}
  \begin{exercise}
    \begin{multicols}{2}
      \item %《2019金考卷双测20套(文)ISBN978-7-5371-9890-5》题型15程序框图 P15p1【2018•全国II卷】【框图】\\
        \source{2018文}{全国II卷}
        为计算$S=1-\dfrac12+\dfrac13-\dfrac14+\cdots+\dfrac1{99}-\dfrac1{100}$,设计了如图所示的程序框图,则在空白框中应填入\xz
        \begin{center}\vspace{-1.8em}\begin{tikzpicture}[node distance=1.3cm,scale=0.6,transform shape]
          \node (start) [startstop] {开始};
          \node (init1) [process,below of=start] {$N=0,T=0$};
          \node (init2) [process,below of=init1] {$i=1$};
          \node (dec1) [decision,below of=init2,yshift=-0.5cm] {$i<100$};
          \node (pro1a) [process,below left of=dec1,xshift=-1cm,yshift=-0.5cm] {$N=N+\dfrac1{i}$};
          \node (pro2a) [process,below of=pro1a] {$T=T+\dfrac1{i+1}$};
          \node (pro3a) [process,below of=pro2a] {};

          % \node (p2) [right of=dec1,xshift=1cm,coordinate]  {};
          \node (pro1b) [process,below right of=dec1,xshift=1cm,yshift=-0.5cm] {$S=N-T$};
          \node (out1) [io,below of=pro1b] {输出$S$};
          \node (stop) [startstop,below of=out1] {结束};

          \draw [arrow] (start) -- (init1);
          \draw [arrow] (init1) -- (init2);
          \draw [arrow] (init2) -- (dec1);
          \draw [arrow] (dec1) -| node[anchor=south,xshift=0.1cm] {是} (pro1a);
          \draw [arrow] (pro1a) -- (pro2a);
          \draw [arrow] (pro2a) -- (pro3a);
          \path (init2) -- (dec1) coordinate[pos=0.5](p);
          \draw [arrow] (pro3a.west) -- ++(-0.8cm,0) |- (p);
          \draw [arrow] (dec1) -| node[anchor=south,xshift=-0.1cm] {否} (pro1b);
          \draw [arrow] (pro1b) -- (out1);
          \draw [arrow] (out1) -- (stop);
          \end{tikzpicture}
        \end{center}\vspace{-0.8em}
        \xx{$i=i+1$}{$i=i+2$}{$i=i+3$}{$i=i+4$}
        \begin{answer}
          B
        \end{answer}
      \item %《2019金考卷双测20套(文)ISBN978-7-5371-9890-5》题型15程序框图 P15p2【2018•天津卷】【框图】\\
        \source{2018文}{天津卷}
        阅读如图所示的程序框图,运行相应的程序,若输入$N$的值为20,则输出$T$的值为\xz
        \begin{center}\vspace{-1.8em}\begin{tikzpicture}[node distance=1.3cm,scale=0.6,transform shape]
          \node (start) [startstop] {开始};
          \node (init1) [io,below of=start] {输入$N$};
          \node (init2) [process,below of=init1] {$i=1$,$T=0$};
          \node (dec1) [decision,below of=init2,yshift=-0.3cm] {$\frac{N}{i}$\small 是整数?};
          \node (pro1a) [process,below of=dec1,yshift=-0.3cm] {$T=T+1$};
          \node (pro1) [process,below of=pro1a] {$i=i+1$};
          \node (dec2) [decision,below of=pro1,yshift=-0.2cm] {$i\geqslant5}$?};

          \node (out1) [io,below of=dec2,yshift=-0.3cm] {输出$T$};
          \node (stop) [startstop,below of=out1] {结束};

          \draw [arrow] (start) -- (init1);
          \draw [arrow] (init1) -- (init2);
          \draw [arrow] (init2) -- (dec1);
          \draw [arrow] (dec1) -- node[anchor=west] {是} (pro1a);
          \draw [arrow] (pro1a) -- (pro1);
          \draw [arrow] (pro1) -- (dec2);
          \draw [arrow] (dec2) -- node[anchor=west] {是} (out1);
          \path (init2) -- (dec1) coordinate[pos=0.5](p1);
          \draw [arrow] (dec2.west) -- node[anchor=south]{否}++(-0.7cm,0) |- (p1);
          \path (pro1a) -- (pro1) coordinate[pos=0.5](p2);
          \draw [arrow] (dec1.east) -- node[anchor=south] {否}++(0.5cm,0) |-(p2);
          \draw [arrow] (out1) -- (stop);
          \end{tikzpicture}
        \end{center}\vspace{-0.8em}
        \xx{1}{2}{3}{4}
        \begin{answer}
          B
        \end{answer}
      \item %《2019金考卷双测20套(文)ISBN978-7-5371-9890-5》题型15程序框图 P15p3【2018•太原一模】【框图】\\
        \source{2018文}{太原一模}
        执行如图所示的程序框图,输出的$S$的值为\xz
        \begin{center}\vspace{-1.8em}\begin{tikzpicture}[node distance=1.3cm,scale=0.6,transform shape]
          \node (start) [startstop] {开始};
          \node (init1) [process,below of=start] {$S=3$,$i=1$};
          \node (dec1) [decision,below of=init1,yshift=-0.3cm] {$i\leqslant3$};
          \node (pro1a) [process,below of=dec1,yshift=-0.3cm] {$S=S+\log_2\sqrt{\frac{i+1}i}$};
          \node (pro2a) [process,below of=pro1a] {$i=i+1$};
          \node (pro1b) [process,below right of=dec1,xshift=2.5cm,yshift=-0.1cm] {$S=\log_2S$};
          \node (out1) [io,below of=pro1b] {输出$S$};
          \node (stop) [startstop,below of=out1] {结束};

          \draw [arrow] (start) -- (init1);
          \draw [arrow] (init1) -- (dec1);
          \draw [arrow] (dec1) -- node[anchor=east] {是} (pro1a);
          \draw [arrow] (pro1a) -- (pro2a);
          \path (init1) -- (dec1) coordinate[pos=0.5](p1);
          \draw [arrow] (pro2a.south) |- node[anchor=south]++(-2cm,-0.2cm) |- (p1);
          \draw [arrow] (dec1) -| node[anchor=south,xshift=-1cm] {否} (pro1b);
          \draw [arrow] (pro1b) -- (out1);
          \draw [arrow] (out1) -- (stop);
          \end{tikzpicture}
        \end{center}\vspace{-0.8em}
        \xx{$3+\dfrac12\log_23$}{$\log_23$}{3}{2}
        \begin{answer}
          D
        \end{answer}
      \item %《2019金考卷双测20套(文)ISBN978-7-5371-9890-5》题型15程序框图 P15p4【2018•郑州测试】【框图】\\
        \source{2018文}{郑州测试}
        如图所示的程序框图的算法思路源于数学名著《几何原本》中的“辗转相除法”,执行该程序框图(图中“$m$~MOD~$n$”表示
        $m$除以$n$的余数),若输入的$m$,$n$分别为495,135,则输出的$m=$\xz
        \begin{center}\vspace{-1.8em}\begin{tikzpicture}[node distance=1.25cm,scale=0.6,transform shape]
          \node (start) [startstop] {开始};
          \node (init1) [io,below of=start] {输入$m$,$n$};
          \node (pro1) [process,below of=init1] {$r=m$~MOD~$n$};
          \node (pro2) [process,below of=pro1] {$m=n$};
          \node (pro3) [process,below of=pro2] {$n=r$};
          \node (dec1) [decision,below of=pro3,yshift=-0.2cm] {$r=0$?};
          \node (out1) [io,below of=dec1,yshift=-0.2cm] {输出$m$};
          \node (stop) [startstop,below of=out1] {结束};

          \draw [arrow] (start) -- (init1);
          \draw [arrow] (init1) -- (pro1);
          \draw [arrow] (pro1) -- (pro2);
          \draw [arrow] (pro2) -- (pro3);
          \draw [arrow] (pro3) -- (dec1);
          \draw [arrow] (dec1) -- node[anchor=west] {是} (out1);
          \path (init1) -- (pro1) coordinate[pos=0.5](p1);
          \draw [arrow] (dec1.east) -- node[anchor=south,xshift=0.1cm]{否}++(0.8cm,0) |- (p1);
          \draw [arrow] (out1) -- (stop);
          \end{tikzpicture}
        \end{center}\vspace{-0.8em}
        \xx{0}{5}{45}{90}
        \begin{answer}
          C
        \end{answer}
      \item %《2019金考卷双测20套(文)ISBN978-7-5371-9890-5》题型15程序框图 P15p5【2018•合肥二检】【框图】\\
        \source{2018文}{合肥二检}
        执行如图所示的程序框图,若输出的结果为1,则输出的$x$的值是\xz
        \begin{center}\vspace{-1.8em}\begin{tikzpicture}[node distance=1.2cm,scale=0.6,transform shape]
          \node (start) [startstop] {开始};
          \node (init1) [io,below of=start] {输入$x$};
          \node (dec1) [decision,below of=init1,yshift=-0.2cm] {$x>2$?};
          \node (pro1a) [process,below of=dec1,yshift=-0.2cm] {$y=-2x-3$};
          \node (pro1b) [process,right of=pro1a,xshift=2cm] {$y=\log_3(x^2-2x)$};
          \node (out1) [io,below of = pro1a,yshift=-0.2cm] {输出$y$};
          \node (stop) [startstop,below of=out1] {结束};

          \draw [arrow] (start) -- (init1);
          \draw [arrow] (init1) -- (dec1);
          \draw [arrow] (dec1) -- node[anchor=west] {否} (pro1a);
          \draw [arrow] (dec1) -| node[anchor=south,xshift=-1cm]{是} (pro1b);
          \path (out1) -- (pro1a) coordinate[pos=0.5](p);
          \draw [arrow] (pro1b) |- (p);
          \draw [arrow] (pro1a) -- (out1);
          \draw [arrow] (out1) -- (stop);
          \end{tikzpicture}
        \end{center}\vspace{-0.8em}
        \xx{$3$或$-2$}{$2$或$-2$}{$3$或$-1$}{$3$或$-1$或$-2$}
        \begin{answer}
          A
        \end{answer}
      \item %《2018天利38套:高考真题单元专题训练(文)ISBN978-7-223-03161-5》专题33算法、复数P117p15【2017•北京】【框图】\\
          \source{2017文}{北京}
          执行如图所示的程序框图,输出的$s$值是\xz
          \begin{center}\vspace{-1.8em}\begin{tikzpicture}[node distance=1.2cm,scale=0.6,transform shape]
            \node (start) [startstop] {开始};
            \node (init1) [process,below of=start] {$k=0$,$s=1$};
            \node (dec1) [decision,below of=init1,yshift=-3cm] {$k<3$};
            \node (pro1a) [process,above right of=dec1,xshift=1.6cm,yshift=0.5cm] {$k=k+1$};
            \node (pro2a) [process,above of=pro1a,yshift=0.2cm] {$a=\dfrac{s+1}s$};
            \node (out1) [io,below of = dec1,yshift=-0.2cm] {输出$s$};
            \node (stop) [startstop,below of=out1] {结束};

            \draw [arrow] (start) -- (init1);
            \draw [arrow] (init1) -- (dec1);
            \draw [arrow] (dec1) -| node[anchor=south,xshift=-0.2cm]{是}  (pro1a);
            \draw [arrow] (pro1a) -- (pro2a);
            \path (init1) -- (dec1) coordinate[pos=0.1](p);
            \draw [arrow] (pro2a) |- (p);
            \draw [arrow] (dec1) -| node[anchor=west,yshift=-0.2cm]{否} (out1);
            \draw [arrow] (out1) -- (stop);
            \end{tikzpicture}
          \end{center}\vspace{-0.8em}
          \xx{$2$}{$\dfrac32$}{$\dfrac53$}{$\dfrac85$}
          \begin{answer}
            C
          \end{answer}
      \item %《2018天利38套:高考真题单元专题训练(文)ISBN978-7-223-03161-5》专题33算法、复数P118p17【2017•全国新课标】【框图】\\
        \source{2017}{全国新课标}
        执行如图所示的程序框图,为使输出$S$的值小于91,则输入的正整数$N$的最小值为\xz
        \begin{center}\vspace{-1.8em}\begin{tikzpicture}[node distance=1.3cm,scale=0.6,transform shape]
          \node (start) [startstop] {开始};
          \node (init0) [io,below of=start] {输入$N$};
          \node (init1) [process,below of=init0] {$t=1$,$M=100$,$S=0$};
          \node (dec1) [decision,below of=init1,yshift=-0.3cm] {$t \leqslant N$};
          \node (pro1a) [process,below of=dec1,yshift=-0.3cm] {$S=S+M$};
          \node (pro2a) [process,below of=pro1a] {$M=-\dfrac{M}{10}$};
          \node (pro3a) [process,below of=pro2a] {$t=t+1$};
          \node (out1) [io,right of=pro1a,xshift=1.6cm] {输出$S$};
          \node (stop) [startstop,below of=out1] {结束};

          \draw [arrow] (start) -- (init0);
          \draw [arrow] (init0) -- (init1);
          \draw [arrow] (init1) -- (dec1);
          \draw [arrow] (dec1) -- node[anchor=west] {是} (pro1a);
          \draw [arrow] (pro1a) -- (pro2a);
          \draw [arrow] (pro2a) -- (pro3a);
          \path (init1) -- (dec1) coordinate[pos=0.5](p1);
          \draw [arrow] (pro3a) -- ++(-2cm,0cm) |- (p1);
          \draw [arrow] (dec1) -| node[anchor=south,xshift=-1cm] {否} (out1);
          \draw [arrow] (out1) -- (stop);
          \end{tikzpicture}
        \end{center}\vspace{-0.8em}
        \xx{5}{4}{3}{2}
        \begin{answer}
          D
        \end{answer}
      \item %《2018天利38套:高考真题单元专题训练(文)ISBN978-7-223-03161-5》专题33算法、复数P118p19【2016•全国新课标】【框图】\\
        \source{2016}{全国新课标}
        执行如图的程序框图,如果输入的$x=0$,$y=1$,$n=1$,则输出$x$,$y$的值满足\xz
        \begin{center}\vspace{-1.8em}\begin{tikzpicture}[node distance=1.2cm,scale=0.6,transform shape]
          \node (start) [startstop] {开始};
          \node (init1) [io,below of=start] {输入$x$,$y$,$n$};
          \node (pro1) [process,below of=init1,yshift=-0.3cm] {$x=x+\dfrac{n-1}2$,$y=ny$};
          \node (dec1) [decision,below of=pro1,yshift=-0.7cm] {$x^2+y^2\geqslant36$};
          \node (pro1b) [process,left of=pro1,xshift=-2.5cm] {$n=n+1$};
          \node (out1) [io,below of = dec1,yshift=-0.6cm] {输出$x$,$y$};
          \node (stop) [startstop,below of=out1] {结束};

          \draw [arrow] (start) -- (init1);
          \draw [arrow] (init1) -- (pro1);
          \draw [arrow] (pro1) -- (dec1);
          \draw [arrow] (dec1) -- node[anchor=east] {是} (out1);
          \draw [arrow] (dec1) -| node[anchor=south,xshift=0.4cm]{否} (pro1b);
          \path (init1) -- (pro1) coordinate[pos=0.3](p);
          \draw [arrow] (pro1b) |- (p);
          \draw [arrow] (out1) -- (stop);
          \end{tikzpicture}
        \end{center}\vspace{-0.8em}
        \xx{$y=2x$}{$y=3x$}{$y=4x$}{$y=5x$}
        \begin{answer}
          C
        \end{answer}
        \item %《2019金考卷双测20套(文)ISBN978-7-5371-9890-5》题型15程序框图 P15p9【2018•南昌一模】【框图】\\
          \source{2018文}{南昌一模}
          执行如图所示的程序框图,则输出的$n$等于\tk.
          \begin{center}\vspace{-1.8em}\hspace{-0.7em}\begin{tikzpicture}[node distance=1.8cm,scale=0.6,transform shape]
            \node (start) [startstop] {开始};
            \node (init1) [process,right of=start,xshift=0.6cm] {$n=0,\,x=\frac{13\piup}{12}$};
            \node (pro1) [process,right of=init1,xshift=1.8cm] {$a=\sin x$};
            \node (dec1) [decision,right of=pro1,xshift=1.8cm] {$a=\frac{\sqrt3}2$?};
            \node (pro1a) [process,below left of=dec1,xshift=-0.1cm] {$n=n+1$};
            \node (pro2a) [process,left of=pro1a,xshift=-0.9cm] {$x=x-\frac{2n-1}{12}\piup$};
            \node (out1) [io,right of = dec1,xshift=1.6cm] {输出$n$};
            \node (stop) [startstop,right of=out1,xshift=0.1cm] {结束};

            \draw [arrow] (start) -- (init1);
            \draw [arrow] (init1) -- (pro1);
            \draw [arrow] (pro1) -- (dec1);
            \draw [arrow] (dec1) -- node[anchor=south,xshift=-0.1cm] {是} (out1);
            \draw [arrow] (dec1) |- node[anchor=south,xshift=0.2cm]{否} (pro1a);
            \draw [arrow] (pro1a) -- (pro2a);
            \path (init1) -- (pro1) coordinate[pos=0.15](p);
            \draw [arrow] (pro2a) -| (p);
            \draw [arrow] (out1) -- (stop);
            \end{tikzpicture}
          \end{center}\vspace{-0.5em}
          \begin{answer}
            3
          \end{answer}
    \end{multicols}
  \end{exercise}

\newpage
\section{课后作业}
  % \begin{exercise}{\heiti 练习}
  % \end{exercise}
  \begin{exercise}
  \end{exercise}
\stopexercise

\newpage
\section{参考答案}
\begin{multicols}{2}
  \printanswer
\end{multicols}

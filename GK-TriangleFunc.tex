\Topic{三角函数复习}
  \Teach{}
  \Grade{高三}
  % \Name{郑皓天}\FirstTime{20181207}\CurrentTime{20181207}
  % \Name{林叶}\FirstTime{20180908}\CurrentTime{20181125}
  %\Name{1v2}\FirstTime{20181028}\CurrentTime{20181117}
  % \Name{林叶}\FirstTime{20180908}\CurrentTime{20181125}
  % \Name{郭文镔}\FirstTime{20181111}\CurrentTime{20181117}
  % \Name{马灿威}\FirstTime{20181111}\CurrentTime{20181111}
  \Name{黄亭燏}\FirstTime{20181231}\CurrentTime{20181231}
  \newtheorem*{Theorem}{定理}
  \makefront
\vspace{-1.5em}
\startexercise
% \section{基本性质}
%   \subsection{任意角的三角函数}
%   \subsubsection{任意角的概念}
%   \begin{enumerate}[1)]
%   \item 以$x$轴正方向为角度的起始边,把终边按逆时针方向旋转所成的角叫做\CJKunderdot{正角};按顺时针旋转的角叫做\CJKunderdot{负角}, 没有旋转所成的角叫\CJKunderdot{零角};
%   \item 终边相同的角:所有与$ \alpha $终边相同的角连同$ \alpha $在内可以构建一个集合$ S=\bigl\{\beta \bigm|\beta =\alpha+k\cdot360\degree,{k\inZ}\bigr\} $.
%   \end{enumerate}
%   \subsubsection{弧度制}
%   把长度等于半径长的弧所对的圆心角叫做$ 1 $弧度的角,用符号$ \rad $表示,读作弧度.\par
%   一般的,正角的弧度是正数,负角的弧度是负数,零角的弧度是$ 0. $如果半径为$ r $的圆的圆心角$ \alpha $所对的弧的长为$ l $,那么角$ \alpha $的弧度数的绝对值是:\[\abs{\alpha}=\dfrac{l}{r}.\]
%   角度与弧度对应关系:
%   $$\begin{array}{ll}
%   360\degree =2\pi\rad,&180\degree =\pi \rad;\\
%   1\degree =\dfrac{\pi}{180}\rad&1\rad=\dfrac{180\degree }{\pi}\approx57.30\degree
%   \end{array}$$
%   \[
%   \begin{array}{|c*{11}{|c}|}
%   \hline
%   \text{度}&0\degree & 30\degree & 45\degree & 60\degree & 90\degree & 120\degree & 135\degree &150\degree &180\degree &270\degree &360\degree \\\hline
%   \text{弧度}&0&\Gape[6pt]{\dfrac{\pi}{6}}&\dfrac{\pi}{4}&\dfrac{\pi}{3}&\dfrac{\pi}{2}&\dfrac{2\pi}{3}&\dfrac{3\pi}{2}&\dfrac{5\pi}{6}&\pi&\dfrac{3\pi}{2}&2\pi\\\hline
%   \end{array}
%   \]
%   \subsubsection{任意角的三角函数}
%   $ P(x,y) $是角$ \alpha $终边上异于原点的一点,$ \abs{OP} =r=\sqrt{x^2+y^2}$,则\[\sin\alpha=\dfrac{y}{r},\cos\alpha=\dfrac{x}{r},\tan\alpha=\dfrac{y}{x}.\]
%   其中$ x,y $都是带符号数,所以可以根据各象限内$ x,y $的正负性得到三角函数的符号规律:一 全正,二正弦,三两切(余切高考不涉及),四余弦.\par
%
%   \subsubsection{同角三角函数关系}
%   两个重要的三角函数关系式:
%   \ding{192} $\sin^2\alpha+\cos^2\alpha=1;$\qquad
%   \ding{193} $ \tan\alpha=\dfrac{\sin\alpha}{\cos\alpha}.$
%   \subsubsection{诱导公式}
%
%   \begin{center}
%   \begin{tikzpicture}
%   \tikzmath{
%   \a =sqrt(3)/2;
%   \b =1/2;
%   \c =-\b;
%   }
%   \coordinate[label=below right:\footnotesize $O$](O) at(0,0);
%   \draw (0,0) circle (1cm);
%   \draw[->,>=latex] (-1.4,0)--(1.4,0)node[below](x){$x$};
%   \draw[->,>=latex] (0,-1.4)--(0,1.4)node[right](y){$y$};
%   \coordinate[label= right:\tiny $M$] (M) at(30:1);
%   \draw (0,0)--(30:1.4);
%   \draw[densely dashed] (30:1)|-(0.5,0);
%   \draw[densely dashed] (30:1)-|(0,0);
%   \coordinate[label=below:\tiny $M'$](M1)at(\a ,0);
%   \draw (0.3,0) arc(0:30:0.3);
%   \node[right](a) at (16:0.4) {\footnotesize $ \alpha$};
%   \coordinate[label=left:\tiny $N$] (N) at(120:1);
%   \coordinate[label=below:\tiny $N'$](N1)at(\c ,0);
%   %\coordinate[label=left:$N''$](M2)at(0 ,\a);
%   \draw (0,0)--(120:1.4);
%   \draw[densely dashed] (120:1)|-(0,0);
%   \draw[densely dashed] (120:1)-|(0,0);
%   \draw (0.4,0) arc(0:120:0.4);
%   \node[above](b) at (110:0.4) {\footnotesize$ \beta$};
%   \draw[rotate=30] (0,0) rectangle +(0.2,0.2);
%   \end{tikzpicture}
%   \end{center}
%
%   如上图所示,当$\beta=\dfrac{\pi}{2}+\alpha\text{时}, \triangle OMM' $和$ \triangle ONN' $全等,根据三角函数定义,可以得到:\[\cos\beta=\dfrac{ON'}{ON}=-\dfrac{MM'}{OM}=-\sin\alpha\]
%   即:\[\cos\left(\dfrac{\pi}{2}+\alpha\right)=-\sin\alpha \]
%   以此类推,可得:
%   $$ \sin\left(\dfrac{k\pi}{2}\pm\alpha\right)=\Bigg\{\begin{aligned}
%   &+\textfractionsolidus - \sin\alpha&k\text{为偶数},\\
%   &+\textfractionsolidus - \cos\alpha&k\text{为奇数}.
%   \end{aligned}~{(\kaishu \text{奇变偶不变,符号看象限})} $$
%   {\kaishu 此公式为自创精简写法,分析如下:当$ k $为奇数时,正(余)弦仍对应正(余)弦,当$ k $为偶数时,正(余)弦对应余(正)弦,右侧的正负号根据$ \dfrac{k\pi}{2}\pm\alpha $所在象限的正(余)弦值决定.}
%
%   \subsection{函数图象}
%   \subsubsection{正弦函数图象}
%   \begin{center}
%   \begin{tikzpicture}[scale=0.7]
%   \coordinate[label=below right:$O$] (O) at(0,0);
%   \coordinate[label=below :\small$\pi$] (t1) at(pi,0);
%   \coordinate[label=below :\small$2\pi$] (t2) at(2*pi,0);
%   \draw[->,>=latex](-pi,0)--(2.5*pi,0)node[below](x) {$x$};
%   \draw[->,>=latex](0,-1.5)--(0,1.5)node[right](y) {\small $y=\sin(x)$};
%   \draw [domain=-pi/2:2*pi,samples=1000] plot(\x,{sin(\x r)});
%   \draw[densely dashed](pi/2,0)node[below](pi){$\frac{\pi}{2}$}--++(0,1.2);
%   \end{tikzpicture}
%   \end{center}
%   \begin{enumerate}[(1)]
%   \item 定义域:$x\inR$;\quad 值域:$ \left[-1,1\right] $ ;\quad 奇偶性:奇函数;
%   \item 对称轴:$ x=k\pi+\dfrac{\pi}{2}\left(k\inZ\right) $;\quad 对称中心:$\left(k\pi,0\right)\left(k\inZ\right)$;\quad 最小正周期:$ T=2\pi  $;
%   \item 单调区间:\begin{enumerate}[(i)]
%   \item 单调递增区间:$ \left[2k\pi-\dfrac{\pi}{2},2k\pi+\dfrac{\pi}{2}\right]\left(k\inZ\right) $;
%   \item 单调递减区间:$ \left[2k\pi+\dfrac{\pi}{2},2k\pi+\dfrac{3\pi}{2}\right] \left(k\inZ\right)$.
%   \end{enumerate}
%   \end{enumerate}
%   \subsubsection{余弦函数图象}
%   \begin{center}
%   \begin{tikzpicture}[scale=0.7]
%   \coordinate[label=below right:\small$O$] (O) at(0,0);
%   \coordinate[label=below :\small $\frac{\pi}{2}$] (t1) at(pi/2,0);
%   \coordinate[label=below :\small $2\pi$] (t2) at(2*pi,0);
%   \draw[->,>=latex](-pi,0)--(2.5*pi,0)node[below](x) {$x$};
%   \draw[->,>=latex](0,-1.5)--(0,1.5)node[right](y) {\small $y=\cos(x)$};
%   \draw [domain=-pi/2:2*pi,samples=1000] plot(\x,{cos(\x r)});
%   \draw[densely dashed](pi,1.2)--++(0,-1.2)node[below left](pi){\small $\pi$}--++(0,-1.2);
%   \end{tikzpicture}
%   \end{center}
%   \begin{enumerate}[(1)]
%   \item 定义域:$x\inR$;\quad 值域:$ \left[-1,1\right] $;\quad 奇偶性:偶函数;
%   \item 对称轴:$ x=k\pi \left(k\inZ\right) $;\quad 对称中心:$\left(k\pi+\dfrac{\pi}{2},0\right)\left(k\inZ\right)$;\quad 最小正周期:$ T=2\pi  $;
%   \item 单调区间:\begin{enumerate}[(i)]
%   \item 单调递增区间:$ \left[2k\pi-\pi,2k\pi\right] \left(k\inZ\right)$;
%   \item 单调递减区间:$ \left[2k\pi,2k\pi+\pi\right]\left(k\inZ\right) $.
%   \end{enumerate}
%   \end{enumerate}
%   \subsubsection{正切函数图象}
%   \begin{center}
%   \begin{tikzpicture}[scale=0.7]
%   \coordinate[label=below right:$O$] (O) at(0,0);
%   %\coordinate[label=below :$\dfrac{\pi}{2}$] (t1) at(pi/2,0);
%   %\coordinate[label=below :$2\pi$] (t2) at(2*pi,0);
%   \draw[->,>=latex](-pi,0)--(pi,0)node[below](x) {$x$};
%   \draw[->,>=latex](0,-1.5)--(0,2)node[right](y) {\small $y=\tan(x)$};
%   \draw [domain=-pi/3:1/3*pi,samples=1000] plot(\x,{tan(\x r)});
%   \draw[densely dashed](2*pi/5,1.5)--++(0,-1.5)node[below right](pi){$\frac{\pi}{2}$}--++(0,-1.5);
%   \draw[densely dashed](-2*pi/5,1.5)--++(0,-1.5)node[below left](pi){$-\frac{\pi}{2}$}--++(0,-1.5);
%   \end{tikzpicture}
%   \end{center}
%   \begin{enumerate}[(1)]
%   \item 定义域:$\left\{x\left|x\ne k\pi+\dfrac{\pi}{2}\right.\right\}\left(k\inZ\right)$;\quad 值域:$ \mathbf{R} $;\quad 奇偶性:奇函数;
%   \item 对称中心:$\left(k\pi,0\right)\left(k\inZ\right)$;\quad 最小正周期:$ T=\pi  $;
%   \item 单调区间:单调递增区间:$ \left(k\pi-\dfrac{\pi}{2},k\pi+\dfrac{\pi}{2}\right) \left(k\inZ\right)$;
%   \end{enumerate}
%   \subsection{$y=A\sin\left(\omega x+\varphi\right)$}
%   \subsubsection*{$y=A\sin\left(\omega x+\varphi\right)$图象}
%   \begin{enumerate}[1)]
%   \item 用“五点法”作图:设$ z=\omega x+\varphi $,由$ z $取$ 0,\dfrac{\pi}{2},\pi,\dfrac{3\pi}{2},2\pi $来求出相应的$ x $,通过描点连线的方法画出图象.\par
%   {\kaishu {\heiti (注:}此处使用的$ z=\omega x+\varphi $的方法同样可以应用于求单调区间、最值等问题)}
%   \item 由函数$y=\sin(x)$的图象经过变换得到$y=A\sin\left(\omega x+\varphi\right)$的图象,有两种主要的途径:“先平移后伸缩”和“先伸缩后平移”
%   \begin{enumerate}[i)]
%   \item 先平移后伸缩\begin{equation*}
%   \begin{aligned}
%   y=\sin x&\xrightarrow[\text{平移}\abs{\varphi}\text{个单位}]{\text{向左}(\varphi>0)\text{或向右}(\varphi<0)}y=\sin\left(x+\varphi\right)\\
%   &\xrightarrow[\text{纵坐标不变}]{\text{横坐标变为原来的}\tfrac{1}{\omega}}y=\sin\left(\omega x+\varphi\right)\\
%   &\xrightarrow[\text{横坐标不变}]{\text{纵坐标变为原来的}A\text{倍}}y=A\sin\left(\omega x+\varphi\right)
%   \end{aligned}
%   \end{equation*}
%   \item 先伸缩后平移
%   \begin{equation*}
%   \begin{aligned}
%   y=\sin x&\xrightarrow[\text{纵坐标不变}]{\text{横坐标变为原来的}\tfrac{1}{\omega}}y=\sin\omega x\\
%   &\xrightarrow[\text{平移}\abs{\tfrac{\varphi}{\omega}}\text{个单位}]{\text{向左}(\varphi>0)\text{或向右}(\varphi<0)}y=\sin\left(\omega x+\varphi\right)\\&\xrightarrow[\text{横坐标不变}]{\text{纵坐标变为原来的}A\text{倍}}y=A\sin\left(\omega x+\varphi\right)
%   \end{aligned}
%   \end{equation*}
%   \end{enumerate}
%   \item 由图象求函数$y=A\sin\left(\omega x+\varphi\right)$的解析式一般步骤:
%   \begin{enumerate}[i)]
%   \item 由函数的最值确定$ A $的取值;
%   \item 由函数的周期确定$ \omega $的值, 周期:$ T=\dfrac{2\pi}{\abs{\omega}} $;
%   \item 由函数图象最高点(最低点)的坐标得到关于$ \varphi $的方程,再由$ \varphi $的范围求$ \varphi $的值.
%   \end{enumerate}
%
%   \item 最值:当$ x $没有范围要求时,$  A $和$ -A $分别为最大值和最小值;当$ x $有范围时,切忌将范围两端分别代入得到所谓取值范围.
%   \end{enumerate}
%   \subsubsection*{$y=A\sin\left(\omega x+\varphi\right)$的单调区间问题}
%   \begin{enumerate}[1)]
%   \item 对于选择填空题,可以直接作图得到单调区间(不推荐);
%   \item 通用流程:\begin{enumerate}[1)]
%   \item 确定$ \omega $为正,若为负,则用诱导公式转化为正;
%   \item 确定$ A $为正,若为负,去掉负号反向取值(求$\nearrow$改成求$ \searrow $,求$ \searrow $改成求$ \nearrow $.)
%   \item 令$ t=\omega x+\varphi $,得到$ y=\sin t $,根据$ y=\sin t $增区间和减区间得到$ \omega x+\varphi $的范围,进而得到$ x $的取值范围.
%   \end{enumerate}
%   \end{enumerate}
%   \subsubsection*{$y=A\sin\left(\omega x+\varphi\right)$在给定区间最值问题}\label{123}
%   对于给定区间$ x\in\left[x_1,x_2\right] $,有:
%   \begin{enumerate}
%   \item 设$ t=\omega x+\varphi $;
%   \item 将$ x $的取值代入$ \omega x+\varphi $中计算$ t$的取值范围;
%   \item 根据$ y=\sin t $的图象(标准图象)得到$ y $的最值及此时$ x $的取值$ x_0 $.
%   \end{enumerate}
%   {\kaishu \textbf{注:}对于类似$ y=f\left[g\left(x\right)\right] $类型的复合函数的相关计算问题(定义域、单调区间、比较大小等),一般可以分解为 $\begin{dcases}
%   		y=f\left(u\right)\\
%   		u=g\left(x\right)
%   	\end{dcases} $,通过两个基本函数的性质解题.\par
%   例如:$ y=sin\left(2x+\dfrac{\pi}{3}\right) $可以分解为$\begin{dcases}
%   	y=sin\left(t\right)\\
%   	t=2x+\dfrac{\pi}{3}
%   	\end{dcases}$,根据$ y=sin\left(t\right) $单调增区间有$ t\in\left[2k\pi-\dfrac{\pi}{2},2k\pi+\dfrac{\pi}{2}\right] \left(k\inZ\right)$,代入$ t=2x+\dfrac{\pi}{3} $即可以得到$ y $关于$ x $的单调区间.
%   }%
\section{习题}
  \begin{exercise}
    \item%【三角函数的概念】
      若角$\alpha$的终边经过点$P(1,-2)$,则$\tan{2\alpha}$的值为\tk.
      \begin{answer}
        $\dfrac43$
      \end{answer}
    \item%【同角三角函数的关系】
      若$\cos\alpha+2\sin\alpha=-\sqrt{5}$,则$\tan\alpha=$\xz
      \xx{$\dfrac12$}{$2$}{$-\dfrac12$}{$-2$}
      \begin{answer}
        B
      \end{answer}
    \item%【同角三角函数的关系】
      $\alpha$是第四象限角,$\tan\alpha=-\dfrac5{12}$,则$\sin\alpha=$\xz
      \xx{$\dfrac15$}{$-\dfrac15$}{$\dfrac5{13}$}{$-\dfrac5{13}$}
      \begin{answer}
        D
      \end{answer}
    \item%【诱导公式】
      若$\sin\Bp{\dfracp{}2+\theta}=\dfrac35$,则$\cos{2\theta}=$\tk.
      \begin{answer}
        $-\dfrac7{25}$
      \end{answer}
    \item%【三角函数的图像和性质】
      设$a=\sin{\dfracp57}$,$b=\cos{\dfracp27}$,$a=\tan{\dfracp27}$,则\xz
      \xx{$a<b<c$}{$a<c<b$}{$b<c<a$}{$b<a<c$}
      \begin{answer}
        D
      \end{answer}
    \item%【三角函数的图像和性质】
      函数$y=\ln\cos x$,$\Bp{-\dfracp{}2,\dfracp{}2}$的图像是\xz
  \end{exercise}
\newpage
\section{课后作业}
  \begin{exercise}

  \end{exercise}
\stopexercise

\newpage
\section{参考答案}
\begin{multicols}{2}
  \printanswer
\end{multicols}

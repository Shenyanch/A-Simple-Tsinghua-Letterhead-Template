% sty文件使用 \RequirePackage{latexexercise0}
% 主文件使用 \documentclass[a3paper,twocolumn,2twoside,landscape,12pt,UTF8]{ctexart}
\hspace{3cm}\\
\vspace{0.5cm}
\centering{\heiti \xiaoer 福州三中2018-2019学年第二学期半期考}\\
% \vspace{0.5cm}
\centering{\heiti \xiaoer 高一\quad 数学试卷}\\
\vspace{0.4cm}
\rightline{\wuhao 命题人:高一数学集备组\hspace{3em}}\\
\rightline{\wuhao 审卷人:黄炳锋\hspace{7em}}\\
\begin{flushleft}
  \wuhao 注意事项:\\
  1. 答卷前,考生务必将自己的姓名、准考证号填写在试卷和答题卡上。\\
  2. 回答选择题时,用铅笔把答题卡上对应题目的答案标号涂黑,如需改动,用橡皮檫干净后,再选涂其他答案标号。回答非选择题时,将答案写在答题卡上,写在本试卷上无效。
\end{flushleft}
\vspace{-1.6em}
\part{第I卷}
\vspace{-3em}
\startexercise
\begin{exercise}
\section{选择题(本大题共12小题,每小题5分,共60分.在每小题给出的四个选项中,只有一个选项是符合题目要求的.}
  \item%福州三中2018-2019学年高一数学期中考试卷.pdf p1
    % \source{2019}{福州三中 1}
    若$a,b,c\inR$,且$a>b$,则下列不等式一定成立的是\xz
      \xx{$a+c\geqslant b-c$}
       {$(a-b)c^2\geqslant0$}
       {$\mfrac{c^2}{a-b}>0$}
       {$ac>bc$}
    \begin{answer}
      B
    \end{answer}
  \item%福州三中2018-2019学年高一数学期中考试卷.pdf p2
    % \source{2019}{福州三中 2}
    在正方体$ABCD-A_1B_1C_1D_1$中,异面直线$AD_1$与$BD$所成角的大小为\xz
    \xx{30\degree}{45\degree}{60\degree}{90\degree}
    \begin{answer}
      C
    \end{answer}
  \item%福州三中2018-2019学年高一数学期中考试卷.pdf p3
    % \source{2019}{福州三中 3}
    已知等差数列$\{a_n\}$中,$a_3=3$,$a_7=1$,则$a_{11}$等于\xz
    \xx{$-2$}{$-1$}{$1$}{$2$}
    \begin{answer}
      B
    \end{answer}
  \item%福州三中2018-2019学年高一数学期中考试卷.pdf p4
    % \source{2019}{福州三中 4}
    设$M=2a(a-2)$,$N=(a+1)(a-3)$,则有\xz
    \xx{$M>N$}{$M=N$}{$M<N$}{$M$与$N$大小不定}
    \begin{answer}
      A
    \end{answer}
  \item%福州三中2018-2019学年高一数学期中考试卷.pdf p5
    % \source{2019}{福州三中 5}
    Rt$\triangle{ABC}$中,$\angle{CAB}=90\degree$,$AB=3$,$AC=4$,以$AC$所在直线为轴将$\triangle{ABC}$旋转一周,所得几何体的体积等于\xz
      \xx{$8\piup$}{$12\piup$}{$24\piup$}{$36\piup$}
    \begin{answer}
      B
    \end{answer}
  \item%福州三中2018-2019学年高一数学期中考试卷.pdf p6
    % \source{2019}{福州三中 6}
    在$\triangle{ABC}$中,角$A$,$B$,$C$的对边分别为$a$,$b$,$c$,若$A=60\degree$,$a=\sqrt3$,$b=1$,则$c$等于\xz
      \xx{$1$}{$2$}{$\sqrt3-1$}{$3$}
    \begin{answer}
      B
    \end{answer}
  \item%福州三中2018-2019学年高一数学期中考试卷.pdf p7
    % \source{2019}{福州三中 7}
    首项为$-24$的等差数列从第10项起开始为正数,则公差$d$的取值范围是\xz
      \xx{$\Bigl(\mfrac83,+\infty\Bigr)$}
       {$(-\infty,3)$}
       {$\Bigl[\mfrac83,3\Bigr)$}
       {$\Bigl(\mfrac83,3\Bigr]$}
    \begin{answer}
      D
    \end{answer}
  \item%福州三中2018-2019学年高一数学期中考试卷.pdf p8
    % \source{2019}{福州三中 8}
    在$\triangle{ABC}$中,已知$\sin^2A+\sin^2B-\sin A \sin B=\sin^2C$,且满足$ab=4$,则$\triangle{ABC}$的面积为\xz
      \xx{$1$}{$2$}{$\sqrt3$}{$2\sqrt3$}
    \begin{answer}
      C
    \end{answer}
  \item%福州三中2018-2019学年高一数学期中考试卷.pdf p9
    % \source{2019}{福州三中 9}
    由实数构成的等比数列$\{a_n\}$的前$n$项和为$S_n$,$a_1=2$,且$a_2-4$,$a_3$,$a_4$成等差数列,则$S_6$等于\xz
    \xx{62}{124}{126}{154}
    \begin{answer}
      C
    \end{answer}
  \clearpage
  \item%福州三中2018-2019学年高一数学期中考试卷.pdf p10
    % \source{2019}{福州三中 10}
    已知不等式$(x+y)\Bp{\mfrac1x+\mfrac{a}y}\geqslant9$对任意正实数$x$,$y$恒成立,则正实数$a$的最小值为\xz
    \xx{6}{5}{4}{3}
    \begin{answer}
      C
    \end{answer}
  \item%福州三中2018-2019学年高一数学期中考试卷.pdf p11
    % \source{2019}{福州三中 11}
    如图所示,位于$A$处的信息中心获悉:在其正东方向相距40 n mile 的$B$处有一艘渔船遇险,在原地等待救援,信息中心立即把消息告知在其正南偏西30\degree,相距20 n mile $C$处的乙船,乙船立即沿直线$CB$前往救援,则$\sin\angle{ACB}$等于\xz
    \begin{minipage}[t]{0.5\linewidth}\vspace{-0.5\baselineskip}
      \xx{$\mfrac{\sqrt{21}}7$\hspace{5em}}
       {$\mfrac{\sqrt{21}}{14}$\hspace{5em}}
       {$\mfrac{\sqrt5}7$\hspace{5em}}
       {$\mfrac{\sqrt5}{14}$\hspace{5em}}
    \end{minipage}
    \begin{minipage}[t]{0.5\linewidth}\vspace{-0.5\baselineskip}
      \begin{flushright}\begin{tikzpicture}[scale=0.8]
        \coordinate[label=above left:$A$] (A) at(0,0);
        \coordinate[label=right:$B$] (B) at(4,0);
        \coordinate[label=below:$C$] (C) at(-1,-1.73);
        % \draw (A)--(B)--(C)--cycle;
        \draw (A)--node[anchor=south]{40 n mile}(B);
        \draw (A)--node[anchor=east]{20 n mile}(C) (B)--(C);
        \draw [dashed] ($(A)+(0,-2)$)--++(0,2.7);
        \draw [->,>=latex] ($(B)+(0,0.3)$)--node[anchor=west]{北} ++(0,0.8);
        \end{tikzpicture}
      \end{flushright}
    \end{minipage}
    \begin{answer}
      A
    \end{answer}
  \item%福州三中2018-2019学年高一数学期中考试卷.pdf p12
    % \source{2019}{福州三中 12}
    已知四棱锥$S-ABCD$的所有顶点都在同一球面上,底面$ABCD$是矩形且和球心$O$在同一平面内,若此四棱锥的最大体积为$\mfrac{16}3$,则球$O$的表面积等于\xz
    \xx{$8\piup$}{$16\piup$}{$32\piup$}{$64\piup$}
    \begin{answer}
      B
    \end{answer}
\vskip
\part{第II卷}
\vspace{-1em}
\section{填空题:本题共4小题,每小题5分,共20分.}
  \item%福州三中2018-2019学年高一数学期中考试卷.pdf p13
    % \source{2019}{福州三中 13}
    已知数列$\{a_n\}$的前$n$项和为$S_n=n^2+3n-2$,则$a_6=$\tk.
    \begin{answer}
      14
    \end{answer}
  \item%福州三中2018-2019学年高一数学期中考试卷.pdf p1
    % \source{2019}{福州三中 14}
    已知正$\triangle{ABC}$的边长为2,则由斜二测画法画出的直观图的面积为\tk.
     \begin{answer}
       $\mfrac{\sqrt6}4$
     \end{answer}
  \item%福州三中2018-2019学年高一数学期中考试卷.pdf p15
    % \source{2019}{福州三中 15}
    若不等式$2x-x^2\geqslant a$对一切的$x\in(-1,3]$成立,则实数$a$的取值范围为\tk.
    \begin{answer}
      $(\infty,3]$
    \end{answer}
  \item%福州三中2018-2019学年高一数学期中考试卷.pdf p16
   % \source{2019}{福州三中 16}
   在$\triangle{ABC}$中,$B=60\degree$,点$M$为$BC$中点,且$AM=AC$,则$\sin\angle{BAC}=$\tk.
   \begin{answer}
     $\mfrac{\sqrt{21}}4$
   \end{answer}
\section{解答题:本题共6小题,共70分. 解答应写出文字说明、证明过程或演算步骤.}
  \item%福州三中2018-2019学年高一数学期中考试卷.pdf p17
    % \source{2019}{福州三中 17}
    (本小题满分10分)\\
    已知锐角三角形$ABC$的角$A$,$B$,$C$的对边分别为$a$,$b$,$c$,$a=2b\sin A$.\\
    (I)求角$B$的大小;\\
    (II)若$a=3\sqrt3$,$c=5$,求$b$的值.
    \begin{answer}
      【解】:
      (I)由$a=2b\sin A$,根据正弦定理得$\sin A=2\sin B\sin A$,\\
         所以$\sin B=\mfrac12$,由$\triangle{ABC}$为锐角三角形得\\
         $B=\mfrac{\piup}6$.\\
      (II)根据余弦定理,得\\
          $b^2=a^2+c^2-2ac\cos B=27+25-45=7$.\\
          $\therefore$ $b=\sqrt7$
    \end{answer}
  \clearpage
  \item%福州三中2018-2019学年高一数学期中考试卷.pdf p18
    % \source{2019}{福州三中 18}
    (本小题满分12分)\\
    已知不等式$ax^2-3x+2<0$的解集为$\{x\mid 1<x<b\}$.\\
    (I)求实数$a$,$b$的值;\\
    (II)解关于$x$的不等式$(b-a)x^2-(ac+b)x+2ac\geqslant0$($c\inR$).
    \begin{answer}
      【解】:
      (I)依题意,$1$与$b$是方程$ax^2-3x+2=0$的两个实数根.
         $\therefore$ $\left\{\begin{aligned}
             &1+b=\mfrac{a}3\\
             &1\times b=\mfrac2a
           \end{aligned}\right.$,解得$\left\{\begin{aligned}
               &a=1\\
               &b=2
             \end{aligned}\right.$\\
      (II)将$a$、$b$值代入可得不等式$x^2-(2+c)x+2c\geqslant0$.
         即$(x-2)(x-c)\geqslant0$.\\
         当$c=2$时, 原不等式解集为$\RR$:\\
         当$c>2$时, 原不等式解集为$(-\infty,2]\cup[c,+\infty)$;\\
         当$c<2$时, 原不等式解集为$(-\infty,c]\cup[2,+\infty)$.
    \end{answer}
  \vspace{5cm}
  \item%福州三中2018-2019学年高一数学期中考试卷.pdf p19
    % \source{2019}{福州三中 19}
    (本小题满分12分)\\
    如图所示,正方体$ABCD-A_1B_1C_1D_1$的棱长为2,$E$、$F$分别是$BC$、$C_1D_1$的中点.\\
    \begin{minipage}[t]{0.7\linewidth}
      \vspace{-0.5\baselineskip}
      (I)求异面直线$AC$与$A_1F$所成角的余弦值;\\
      (II)求证:$EF\varparallel\text{平面}BB_1D_1D$.
    \end{minipage}
    \begin{minipage}[t]{0.3\linewidth}
      \vspace{-0.5\baselineskip}
      \begin{flushright}\begin{tikzpicture}[scale=0.25,line width=0.5 pt]
        %\draw[help lines,very thin](0,0) grid (6,6);
        \tikzmath{
        \w=12;\d=12;\h=12;%宽度width,深度depth,高度height
        \arcwd=30;\arcwh=90;%直观图中 x y轴夹角;x、z轴夹角;
        \hh=\h*cos(\arcwh);
        \hv=\h*sin(\arcwh);
        \dh =\d/2*cos(\arcwd);
        \dv =\d/2*sin(\arcwd);
        }
        \coordinate[label=below:\footnotesize$A$] (A) at(0,0);
        \coordinate[label=below:\footnotesize$B$] (B) at(\w,0);
        \coordinate[label=below:\footnotesize$D$] (D) at($(A)+(\dh,\dv)$);
        \coordinate[label=below:\footnotesize$C$] (C) at($(B)+(\dh,\dv)$);
        \foreach \p in{C,D}
        \coordinate[label=above:\footnotesize$\p_1$] (\p_1) at($(\p)+(\hh,\hv)$);
        \foreach \p in{A,B}
        \coordinate[label=below left:\footnotesize$\p_1$] (\p_1) at($(\p)+(\hh,\hv)$);
        \draw (A)--(B)--(C) (A)--(A_1) (B)--(B_1) (C)--(C_1);
        \draw (A_1)--(B_1)--(C_1)--(D_1)--cycle;
        \draw[dashed] (A)--(D)--(C) (D_1)--(D);
        \coordinate[label=below right:\footnotesize$E$](E) at($(B)!0.5!(C)$);
        \coordinate[label=above:\footnotesize$F$](F) at($(C_1)!0.5!(D_1)$);
        \draw (A)--(C) (A_1)--(F) (B_1)--(D_1);
        \draw[dashed] (B)--(D) (E)--(F);
        % \foreach \p in{A,D}
        % \coordinate[label=left:\footnotesize$\p_1$] (\p_1) at($(\p)+(0,3)$);
        \end{tikzpicture}
      \end{flushright}
    \end{minipage}
    \begin{answer}
      【解】:
      \begin{center}\vspace{-1em}\begin{tikzpicture}[t,scale=0.16,line width=0.5 pt]
        %\draw[help lines,very thin](0,0) grid (6,6);
        \tikzmath{
        \w=12;\d=12;\h=12;%宽度width,深度depth,高度height
        \arcwd=30;\arcwh=90;%直观图中 x y轴夹角;x、z轴夹角;
        \hh=\h*cos(\arcwh);
        \hv=\h*sin(\arcwh);
        \dh =\d/2*cos(\arcwd);
        \dv =\d/2*sin(\arcwd);
        }
        \coordinate[label=below:\footnotesize$A$] (A) at(0,0);
        \coordinate[label=below:\footnotesize$B$] (B) at(\w,0);
        \coordinate[label=below:\footnotesize$D$] (D) at($(A)+(\dh,\dv)$);
        \coordinate[label=below:\footnotesize$C$] (C) at($(B)+(\dh,\dv)$);
        \foreach \p in{C,D}
        \coordinate[label=above:\footnotesize$\p_1$] (\p_1) at($(\p)+(\hh,\hv)$);
        \foreach \p in{A,B}
        \coordinate[label=below left:\footnotesize$\p_1$] (\p_1) at($(\p)+(\hh,\hv)$);
        \draw (A)--(B)--(C) (A)--(A_1) (B)--(B_1) (C)--(C_1);
        \draw (A_1)--(B_1)--(C_1)--(D_1)--cycle;
        \draw[dashed] (A)--(D)--(C) (D_1)--(D);
        \coordinate[label=below right:\footnotesize$E$](E) at($(B)!0.5!(C)$);
        \coordinate[label=above:\footnotesize$F$](F) at($(C_1)!0.5!(D_1)$);
        \draw (A)--(C) (A_1)--(F) (B_1)--(D_1);
        \draw[dashed] (B)--(D) (E)--(F);
        \coordinate[label=below:\footnotesize$O$] (O) at($(B)!0.5!(D)$);
        \draw[dashed] (A_1)--(C_1) (O)--(D_1) (O)--(E);
        % \foreach \p in{A,D}
        % \coordinate[label=left:\footnotesize$\p_1$] (\p_1) at($(\p)+(0,3)$);
        \end{tikzpicture}
      \end{center}\vspace{-1.2em}
      (1)连结$A_1C$,则$A_1C_1\varparallel AC$,异面直线$AC$与$A_1F$所成角为$\angle{FA_1C_1}$.\\
         $F$是$C_1D_1$的中点,$\therefore$ $D_1F=FC_1=\mfrac12C_1D_1=1$,\\
         在$\triangle{FA_1C_1}$中,$A_1C_1=\sqrt2C_1D_1=2\sqrt2$,$FC_1=1$,$A_1F=\sqrt{A_1D_1^2+D_1F^2}=\sqrt5$\\
         故由余弦定理得:\\
         $\cos\angle{FA_1C_1}=\mfrac{A_1F^2+A_1C_1^2-C_1F^2}{2A_1F\cdot A_1C_1}
          =\mfrac{5+8-1}{2\times\sqrt5\times2\sqrt2}=\mfrac{3\sqrt{10}}{10}$,\\
         即异面直线$AC$与$A_1F$所成角的余弦值为$\mfrac{3\sqrt{10}}{10}$;\\
      (II)记$AC\cap BD=O$,则$OE\varpareq\mfrac12CD\varpareq \mfrac12C_1D_1$,即$OE\varpareq D_1F$,\\
          $\therefore$ 四边形$D_1FEO$为平行四边形,$\therefore$ $EF\varparallel OD_1$,\\
          又$OD_1\subset \text{平面}BB_1D_1D$,$EF\nsubset \text{平面}BB_1D_1D$,\\
          $\therefore$ $EF\varparallel\text{平面}BB_1D_1D$.
    \end{answer}
  \vspace{2cm}
  \item%福州三中2018-2019学年高一数学期中考试卷.pdf p20
    % \source{2019}{福州三中 20}
    (本小题满分12分)\\
    已知等比数列$\{a_n\}$的各项均为正数,且$a_2-2a_1=3$,$9a_3^2=a_2a_6$.\\
    (I)求数列$\{a_n\}$的通项公式;\\
    (II)设$b_n=\log_3a_1+\log_3a_2+\cdots+\log_3a_n$,求数列$\Bigl\{\mfrac1{b_n}\Bigr\}$的前$n$项和.
    \begin{answer}
      【解】:
      (I)设等比数列$\{a_n\}$的公比为$q$,则由$\{a_n\}$的各项均为正数,$q>0$.\\
         故由$9a_3^2=a_2a_6=a_4^2$得$q=\mfrac{a_4}{a_3}=3$(负值舍去).\\
         $\therefore$ $a_2-2a_1=a_1q-2a_1=a_1=3$\\
         $\therefore$ $a_n=a_1q^{n-1}=3^n$\\
      (II)$b_n=\log_3a_1+\log_3a_2+\cdots+\log_3a_n=1+2+\cdots+n=\mfrac{n(n+1)}2$.\\
          $\therefore$ $\mfrac1{b_n}=\mfrac{n(n+1)}2=2\Bp{\mfrac1n-\mfrac1{n+1}}$.\\
          $\therefore$ 记数列$\Bigl\{\mfrac1{b_n}\Bigr\}$的前$n$项和为$T_n$,则\\
          $T_n=2\Bp{1-\mfrac12}+2\Bp{\mfrac12-\mfrac13}+\cdots+2\Bp{\mfrac1n-\mfrac1{n+1}}
           =2\Bp{1-\mfrac1{n+1}}=\mfrac{2n}{n+1}$,$n\inN^*$.\\
    \end{answer}
  \clearpage
  \item%福州三中2018-2019学年高一数学期中考试卷.pdf p21
    % \source{2019}{福州三中 21}
    (本小题满分12分)\\
    $\triangle{ABC}$的三个内角$A$,$B$,$C$的对边分别为$a$,$b$,$c$,满足$(2b-c)\cos A=a\cos C$.\\
    (I)求$A$的值;\\
    (II)若$a=2$,求$\triangle{ABC}$周长的最大值.
    \begin{answer}
      【解】:
      (I)由正弦定理,$(2\sin B-\sin C)\cos A=\sin A \cos C$,\\
         $\therefore$ $2\sin B\cos A=\sin A \cos C+\sin C \cos A=\sin(A+C)=\sin B$\\
        $\therefore$ $\cos A=\mfrac12$, $\therefore$ $A=\mfrac{\piup}3$\\
      (II)由余弦定理 $4=b^2+c^2-2bc\cos\mfrac{\piup}3
          =b^2+c^2-bc=(b+c)^2-3bc\geqslant \mfrac14(b+c)^2$(当且仅当$b=c$时取等号).\\
          故$b+c\leqslant4$,$\therefore$ $a+b+c\leqslant6$\\
          即$\triangle{ABC}$周长的最大值为6.
    \end{answer}
  \vspace{7cm}
  \item%福州三中2018-2019学年高一数学期中考试卷.pdf p22
    % \source{2019}{福州三中 22}
    (本小题满分12分)\\
    某工艺品生产企业,拟在2019年度进行系列促销活动.经市场调查和测算,该企业一款新产品的年销售量$x$(单位:万件)与年促销费用$t$(单位:万元)之间满足$3-x$于$t+1$成反比例.若不搞促销活动,该新产品的年销售量只有1万件.已知该企业2019年生产这款新产品的固定投资为3万元,每生产1万件这款新产品需另外投资32万元.当企业把这款新产品每件的售价定为“年平均每件生产成本的1.5倍”与“年平均每件所占促销费用的一半”之和时,则当年的产量和销量相等.(利润$=$收入$-$生产成本$-$促销费用)\\
    (I)把该企业2019年的年利润$y$(单位:万元)表示成年促销费用$t$(单位:万元)的函数;\\
    (II)当2019年的促销费用投入多少万元时,该企业的年利润最大?并求出利润的最大值.
    \begin{answer}
      【解】:
      (I)设反比例系数为$k$($k\neq0$),有$3-x=\mfrac{k}{t+1}$.\\
        $\because$ $t=0$时,$x=1$,代入上式得$k=2$,\\
        $\therefore$ $x=3-\mfrac2{t+1}$($t\geqslant0$);\\
        故年利润$y=x\cdot\Bp{\mfrac{3+32x}x\cdot1.5+\mfrac{t}{2x}}-(3+32x)-t
                 =\mfrac12\Bp{99-\mfrac{64}{t+1}-t}$($t\geqslant0$).\\
      (II)$y=\mfrac12\bigl(99-\mfrac{64}{t+1}-t\bigr)
            =\mfrac12\Bigl[100-\bigl(t+1+\mfrac{64}{t+1}\bigr)\Bigr]
            \leqslant \mfrac12\Bp{100-2\sqrt{(t+1)\cdot\mfrac{64}{t+1}}}=42$,\\
          当且仅当$t+1=\mfrac{64}{t+1}$即$t=7$时(负值舍去)取等号;\\
          所以当2019年的促销费用投入$7$万元时,该企业的年利润最大,最大利润为$43$万元.
    \end{answer}

\end{exercise}
\stopexercise
\hspace{2em}
\newpage
\part{参考答案}
\begin{multicols}{2}
  \printanswer
\end{multicols}

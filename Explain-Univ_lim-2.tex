\section{复合函数的极限}
首先,回顾一下极限的定义(关于定义域的一些细节就省略不写了):
\begin{framed}
    设$A$为给定常数,如果对任给的$\varepsilon>0$,存在正数$\delta$,使得当$0<|x-x_0|<\delta$时,有$|f(x)-A|<\varepsilon$,
    则称{\fangsong 函数$f$当$x$趋于$x_0$时以$A$为极限}.记为:
    \[
        \lim_{x \to x_0} f(x)=A.
    \]
\end{framed}

$\varepsilon$这东西就相当于给定一个误差范围一样,无论把这个误差范围限制得多么小,都可以找到在$x_0$附近的$x$值,
使得在这范围内的$f(x)$与$A$之间的误差总是在$\varepsilon$之内.
简单来说,极限的核心要点就是:~$f(x)$与$A$可以\CJKunderdot{要多接近有多接近}.

注意到所谓“$x_0$附近的$x$值”$0<|x-x_0|<\delta$可以用邻域符号表示为$x\in\mathring{U}(x_0;\delta)$,
不等式$|f(x)-A|<\varepsilon$也能用邻域符号表示为$f(x)\in{U}(A;\varepsilon)$,于是“函数$f$当$x$趋于$x_0$时以$A$为极限”表明:
\begin{framed}
    任给$\varepsilon>0$,存在$\delta>0$,使得对一切$x\in\mathring{U}(x_0;\delta)$,
    有$f(x)\in{U}(A;\varepsilon)$. 
\end{framed}

也就是说,如果当$x$趋于$x_0$时,函数$f$以$A$为极限,那么:以$A$为中心,无论设定多么“小”的一个邻域${U}(A;\varepsilon)$,
都能找出一个$x_0$的$\delta$空心邻域$\mathring{U}(x_0;\delta)$,使得在此范围内的$f(x)$,都属于邻域${U}(A;\varepsilon)$.
例如:\\
要使$f(x)$在邻域${U}(A;1)$内,可找出一个$\delta_1$,这时只需$x\in\mathring{U}(x_0;\delta_1)$即可;\\
要使$f(x)$在邻域${U}(A;0.1)$内,可找出一个$\delta_2$,这时只需$x\in\mathring{U}(x_0;\delta_2)$即可;\\
要使$f(x)$在邻域${U}(A;0.01)$内,可找出一个$\delta_3$,这时只需$x\in\mathring{U}(x_0;\delta_3)$即可.

另外,还需要特别注意的一点是:$f(x_0)$无论是否存在,取值多少,都和极限$\lim\limits_{x \to x_0} f(x)$没有任何关系;
并且函数$f$在$x$趋于$x_0$时极限为$A$,并不能说明$f(x_0)$在邻域${U}(A;\varepsilon)$内.\\

\newpage
现在,简单说明下要判断的问题:\par
\begin{framed}
    设函数$y=f\bigl(\varphi(x)\bigr)$由$u=\varphi(x)$与$y=f(u)$复合而成,
    若$\lim\limits_{x \to x_0}\varphi(x)=a$,且$\lim\limits_{u \to a}f(u)=A$,那么是否有下式成立?
    \begin{equation}
        \lim_{x \to x_0} f(\varphi(x))=\lim_{u \to a} f(u)=A\label{eq:conclusion}
    \end{equation}
\end{framed}
\begin{framed}
    \eqref{eq:conclusion}式~$\iff$~任给$\varepsilon>0$,存在$\delta>0$,
    使得对一切$x\in\mathring{U}(x_0;\delta)$,有$f\bigl(\varphi(x)\bigr)=f(u)\in{U}(A;\varepsilon)$.
\end{framed}
而由$\lim\limits_{u \to a}f(u)=A$知:
\begin{framed}
    要使$f(u)$在邻域${U}(A;\varepsilon)$内,可以找出一个$h>0$,只需$u\in\mathring{U}(a;h)$即可.
\end{framed}
那么,何时$u=\varphi(x)\in\mathring{U}(a;h)$~?由$\lim\limits_{x \to x_0}\varphi(x)=a$知:
\begin{framed}
    要使$\varphi(x)$在邻域${U}(a;h)$内,可以找出一个$\delta_1>0$,只需$x\in\mathring{U}(x_0;\delta_1)$即可.
\end{framed}

可以注意到,${U}(a;h)$与$\mathring{U}(a;h)$只差了一点.可以分以下三种情况讨论:\par
\circled{1}~如果$x\in\mathring{U}(x_0;\delta_1)$时,恰好$u=\varphi(x)\neq a$.
            那么这时$u=\varphi(x)$就在空心邻域$\mathring{U}(a;h)$内,
            于是$f(u)=f\bigl(\varphi(x)\bigr)$在邻域${U}(A;\varepsilon)$内.
            从而\eqref{eq:conclusion}式成立.\par
\circled{2}~如果$x\in\mathring{U}(x_0;\delta_1)$时,存在某个$u=\varphi(x)= a$.
            这时可以试着将$\delta_1$再取小一些,例如取成$\delta$,且当$x\in\mathring{U}(x_0;\delta)$时,
            不存在$u=\varphi(x)= a$,即可化为\circled{1}中的情况.于是\eqref{eq:conclusion}式仍成立.
            \\【这时,由于$x\in\mathring{U}(x_0;\delta)\subseteq \mathring{U}(x_0;\delta_1)$,因此$\varphi(x)\in{U}(a;h)$;
            又因为$u=\varphi(x)\neq a$,从而$\varphi(x)\in\mathring{U}(a;h)$.】\par
            那么,能不能做到这点呢?可以先考虑一下反面情形,如果取不到上述的$\delta$,也就是说:对于任意的$\delta>0$,
            只要$x\in\mathring{U}(x_0;\delta)$,就一定存在某个$u=\varphi(x)= a$.
            这种情况在下面的\circled{3}中讨论.由这反面情形,可以取到上述的$\delta$,就一定表明:存在某个$\delta_2>0$,
            使得$x\in\mathring{U}(x_0;\delta_2)$时,$u=\varphi(x)\neq a$.这时,我们取$\delta=\min\{\delta_1,\delta_2\}$即可.\par
\circled{3}~对于任意的$\delta_0>0$,只要$x\in\mathring{U}(x_0;\delta_0)$,就一定存在某个$u=\varphi(x)= a$.
            这时,对于前述的$\delta_1$,假设$u^*=\varphi(x^*)=a$,其中$x^*\in\mathring{U}(x_0;\delta_1)$,
            这时$f(u^*)\bigl(=f(a)\bigr)$是否在邻域${U}(A;\varepsilon)$内?\par
            如果不在,那么就无法从$x\in\mathring{U}(x_0;\delta_1)$顺推到$f(u)\in{U}(A;\varepsilon)$【有$f\bigl(\varphi(x^*)\bigr)\notin{U}(A;\varepsilon)$这个反例】.\par
            如果在,也就是说,$f(a)\in{U}(A;\varepsilon)$,那么这时,对于$x\in\mathring{U}(x_0;\delta_1)$,
            $f\bigl(\varphi(x^*)\bigr)\in{U}(A;\varepsilon)$就一定都能够成立.于是\eqref{eq:conclusion}式得以成立.\par
            注意到上述讨论中,$\varepsilon$是任取的,也就是说,\eqref{eq:conclusion}式要在当前情况\circled{3}下成立,
            就要求对任意$\varepsilon>0$,都必须有$f(a)\in{U}(A;\varepsilon)$,而$A$与$f(a)$都是定值,这就表明:$f(a)=A$.
            也就是说:函数$f$在点$a$处连续.【后面学到函数的连续性之时,会再讲这种情况.】

\newpage
最后总结归纳一下复合函数求极限的情形,
前面的\circled{1}、\circled{2}两种情形其实都可以归结为一种情形,也就是:
\begin{framed}
    设函数$y=f\bigl(\varphi(x)\bigr)$由$u=\varphi(x)$与$y=f(u)$复合而成,
    若$\lim\limits_{x \to x_0}\varphi(x)=a$,$\lim\limits_{u \to a}f(u)=A$,
    并且存在某个$\delta_2>0$,使得$x\in\mathring{U}(x_0;\delta_2)$时,$\varphi(x)\neq a$.那么:
    \begin{equation*}
        \lim_{x \to x_0} f(\varphi(x))=\lim_{u \to a} f(u)=A
    \end{equation*}
\end{framed}
这就是你当前学习的复合函数极限运算法则.其中的一种特殊情形就是:
\begin{framed}
    设函数$y=f\bigl(\varphi(x)\bigr)$由$u=\varphi(x)$与$y=f(u)$复合而成,
    若$\lim\limits_{x \to x_0}\varphi(x)=\infty$,$\lim\limits_{u \to a}f(u)=A$,
    则有:
    \begin{equation*}
        \lim_{x \to x_0} f(\varphi(x))=\lim_{u \to \infty} f(u)=A
    \end{equation*}
\end{framed}

而前面讨论的情形\circled{3}将在后续学习函数的连续性之时,再次讲到:
\begin{framed}
    设函数$y=f\bigl(\varphi(x)\bigr)$由$u=\varphi(x)$与$y=f(u)$复合而成,
    若$\lim\limits_{x \to x_0}\varphi(x)=a$,$\lim\limits_{u \to a}f(u)=A$,
    并且函数$y=f(u)$在$u=a$处连续,则:
    \begin{equation*}
        \lim_{x \to x_0} f(\varphi(x))=\lim_{u \to a} f(u)=A
    \end{equation*}
\end{framed}




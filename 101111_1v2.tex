\Topic{高一必修一期中复习}
  \Teach{分类讨论;含参数问题}
  \Grade{高一}
  \Name{马灿威}
  \FirstTime{20181111}
  \CurrentTime{20181111}
  \makefront
\vspace{-1.7em}
\startexercise

\section{集合}
  \begin{exercise}{}
    \item
      (福州三中高一半期考)已知全集$U=\{-2,-1,2,3,4\}$集合$A=\{-1,2,3\}$,$B=\{-2,2\}$,则$(\complement_UA)\cup B=$\xz
      \xx{$\{-2\}$}
            {$\{-2,2,4\}$}
            {$\{-2,-1,2\}$}
            {$\{-2,2,3,4\}$}
      \begin{answer}
        B
      \end{answer}
    \item
      (福州三中高一半期考)设集合$M=\{x\in \mathbb{R}|x-1<0 \} $, $N=\{y|y=x^2,x\in\mathbb{R} \} $,则$M\cap N$\tk
      \begin{answer}
        $[0,1)$
      \end{answer}
    \item
      (格致中学高一半期考)已知集合$M=\{-2\leq x\leq 5\} $,$N=\{x|a+1\leq x\leq 2a+1\} $.
      (1)若$a=3 $,求$M\displaystyle \cap(\complement_{\mathbb{R} }{\large N}) $;
      (2)若$M\cup N=M $,求实数$a $的取值范围.
      \begin{answer}
        (1) $[-2,4)$
        (2) $(-\infty,2] $
      \end{answer}
  \end{exercise}
\vspace{12em}
\section{指数对数幂函数}
  \begin{exercise}{\large \bf 运\hspace{0.6em}算}
    \item
      【2015 福州八中 4】 设$a=0.7^{\frac12} $,$b=0.8^{0.5} $,$c=\log_30.7$,则\xz
      \xx{$c<b<a$}{$c<a<b$}{$a<b<c$}{$b<a<c$}
      \begin{answer}
        B
      \end{answer}
    \item
      【2016 师大附中 13】已知$2^a=3 $,$3^b=7$,则$\log_756= $\tk.(结果用 a, b 表示)
      \begin{answer}
        $3+ab\over ab$
      \end{answer}
    \item
      【2015 福州八中 14】(本小题满分 10 分)
      计算:\\
      (1)$\displaystyle (2\frac{3}{5})^0+2^{-2}\cdot(2\frac14)^{-\frac12}+(\frac{25}{36})^{0.5}+\sqrt{(-2)^2} $
      \hspace{5em}
      (2)$\displaystyle \frac12 \lg{\frac{32}{49}}-\frac43\lg{\sqrt 8}+\lg{\sqrt {245}} $
      \begin{answer}
        (1) 4 (2)$\frac12$
      \end{answer}
    \vspace{12em}
    \item
      【2016 福州三中 15】(本小题满分 10 分) 根据已知条件,求下列各式的值.\\
      (1) 已知$a=2^{-1}$,$b=3^{\sqrt2}$,求$4a^{\frac23}b^{-\frac13}\div(-\frac23a^{-\frac13}b^{-\frac13})$的值;
      (2)已知$f(x)=3^x$,求$f(\log_32)+f(2)$的值
      \begin{answer}
        (1)原式 = $-6a=-3$;
        (2) 11
      \end{answer}
    \vspace{9em}
  \end{exercise}
  \begin{exercise}{\bf 与二次函数的复合}
    \item
      【2015 福州三中14】已知$a>0$且$a\neq1$,函数$f(x)=a^{-x^2-2x-3}$存在最小值,且最小值为16,则$a=$\tk.\\
    \item
      【2016师大附中18】 (本小题满分12分)
      已知函数$f(x)$为$\mathbb{R}$上的偶函数. $x\leq0$时$f(x)=4^{-x}-a\cdot 2^{-x}(a>0)$\\
      (\Rmnum{1})求函数$f(x)$在$(0,+\infty)$上的解析式;
      (\Rmnum{2})求函数$f(x)$在$(0,+\infty)$上的最小值.
    \vspace{12em}
    \item
      【2016 福州三中 17】(本小题满分 12 分)
      已知函数$f(x)=\log_39x\cdot\log_3x+2 $,$x\in[\frac19,3]$.\\
      (1) 求$f(x)$最小值和最大值;\\
      (2) 若不等式$f(x)-2m+1>0 $恒成立,求实数$m$ 的取值范围.
      \begin{answer}
        (1) $f_{\min}(x)=f(\frac13)=1$
              $f_{\max}(x)=f(3)=5$
        (2) $m\in(-\infty,1)$
      \end{answer}
    \vspace{14em}
  \end{exercise}
\section{函数零点问题,函数模型}
  \begin{exercise}{}
    \item
      (15-16 附中)已知函数$f(x)=\begin{cases}\mathrm{e}^x+a,x\leq0\\2x-1,x>0 \end{cases} $,若函数$f(x)$在$\mathbb{R}$上有两个不同零点,则$a$的取值范围是\xz
      \xx{$[-1,+\infty)$}{$(-1,+\infty)$}{$(-1,0)$}{$[-1,0)$}
      \begin{answer}
        B
      \end{answer}
    \item
      (15-16 八中)若方程$x^2-2mx+4=0$ 的两根满足一根大于 1,一根小于 1,则$m$ 的取值范围是\xz
      \xx{$(-\infty,-\frac52)$}{($\frac52,+\infty)$}{$(-\infty,-2)\cup(2,+\infty)$}{$(-\frac52,+\infty)$}
      \begin{answer}
        D
      \end{answer}
    \item
      (16-17 三中)设函数$f(x)=ax^2+bx+c$,($a>0,b,c\in\mathbb{R}$).\\
      (1) 若$f(1)=c$,$f(x)$在$(k,+\infty)$单调递增,求实数$k$的取值范围;\\
      (2) 若$f(1)=-\dfrac a2$,求证:函数$f(x)$在$(0,2)$ 内至少有一个零点.
    \vspace{15em}
    \item
      (16-17 三中)某城市现有人口 300 万,而汽车保有量为 100 万辆,已知汽车保有量每年以 21\% 递增,而人 口每年以 10\% 递增.\\
      (1)写出该城市人口$y$ (单位:万)关于从现在起经过的年数 $x$ 的函数关系式;\\ (2)问该城市经过多少年人均将拥有一辆汽车?(精确到个位).\\
      参考数据:$\lg3=0.4771$,$\lg11=1.041$,$\lg21=1.322$
      \begin{answer}
        (2)12
      \end{answer}
    \vspace{14em}
  \end{exercise}
\section{函数的相关性质}
  \begin{exercise}{\bf 定义域、分段函数}
    \item
      (福州高级中学16-17高一期中考)已知函数$f(x+1)=2x+5$,则$f(3)=$\xz
      \xx{5}{7}{9}{11}
    \item
      【2016.11 福高高一期中考】函数$f(x)=\sqrt{\log_{\frac13}(x-2)}+\dfrac1{2x-5}$的定义域为\tk.
      \begin{answer}
        $(2,\frac52)\cup(\frac52,3] $
      \end{answer}
    \item
      (福高 2016―2017学年第一学期期中考试)设函数$\displaystyle f(x)=\begin{cases}x^{\frac12},x>0\\(\frac12)^x-1,x\leq0\end{cases} $,已知$f(a)>1$,则$a$的取值范围是\xz
      \xx{$(-1,1)$}
      {$(-\infty,-1)\cup(1,+\infty)$}
      {$(-\infty,-2)\cup(0,+\infty)$}
      {$(1,+\infty)$}
      \begin{answer}
        B
      \end{answer}
  \end{exercise}
  \begin{exercise}{\bf 奇偶性与单调性}
    \item
      (福建师大附中15-16高一期中考,6)下列函数中,既是偶函数又在$(0,+\infty)$单调递增的函数是\xz
      \xx{$y=x^3$}{$y=|x|+1$}{$y=-x^2+1$}{$y=2^{-|x|}$}
    \item
      (福州八中 15-16 高一期中考,2)设偶函数 $f(x)$的定义域为$\mathbb{R}$,当 $x\in[0,+\infty)$时,$f(x)$是增函数,则$f(-2)$,$f(\pi)$,$f(-3)$的大小关系是\xz
      \xx{$f(\pi)>f(-3)>f(-2)$}
          {$f(\pi)>f(-2)>f(-3)$}
          {$f(\pi)<f(-3)<f(-2)$}
          {$f(\pi)<f(-2)<f(-3)$}
    \item
      (福建师大附中16-17高一期中考,7)已知定义在 $\mathbb{R}$上的函数$f(x)$在$(-\infty,2)$内为减函数,且$f(x+2)$为偶函数,则$f(-1)$,$f(4)$,$f(\frac{11}2)$的大小为\xz
      \xx{$f(4)<f(-1)<f(\frac{11}2)$}
          {$f(-1)<f(4)<f(\frac{11}2)$}
          {$f(-1)>f(4)>f(\frac{11}2)$}
          {$f(4)>f(\frac{11}2)>f(-1)$}
    \item
      (福州高级中学 16-17 高一期中考,11)定义在 $\mathbb{R}$上的偶函数$f(x)$,当$x\in[1,2]$时,$f(x)<0$且$f(x)$增函数,给出下列四个结论:\\
      (1)$f(x)$在$[-2,-1]$上单调递增;\hspace{4em}(2)当$x\in[-2,-1]$时,有$f(x)<0$;\\
      (3)$f(-x)$在$[-2,-1]$上单调递减;\hspace{4em}(4)$|f(x)|$在$[-2,-1]$上单调递减.其中正确的结论是\xz
      \xx{(1)(3)}{(2)(4)}{(2)(3)}{(3)(4))}
    \item
      (福州格致中学 16-17 高一期中考,10)若$f(x)=-x^2+2ax$与$g(x)=\dfrac a{x+1}$ 在区间$[1,2]$上都是减函数,则实数$a$ 的取值范围\xz
      \xx{$(-1,0)\cup(0,1)$}
          {$(-1,0)\cup(0,1]$}
          {$(0,1)$}{$(0,1]$}
    \item
      (福建师大附中 16-17 高一期中考,15)定义在 $\mathbb{R}$上的奇函数$f(x)$满足 $f(x-2)=f(x+2)$,且当$x\in(-1,0)$,时,$f(x)=2^x+\dfrac15$,则$f(\log_220)=$\tk.\\
    \item
      (福州格致中学 16-17 高一期中考,14) 已知定义在$\mathbb{R}$上的奇函数$f(x)$ ,当$x>0$时$f(x)=x^2+x-1$,那么$x<0$时,$f(x)=$\tk.\\
    \item%福州重点中学期中考真题分类汇编 2函数的相关性质.pdf P11
      (福州八中 2015—2016 高一上学期期中考试23)设 $f (x )$ 是定义在 $\mathbb{R}$ 上的奇函数,且对任意 $a,b\in \mathbb{R}$ ,当$a+b\neq0$时,都有 $\frac{f(a)+f(b)}{a+b}>0$\\
      (1)若 $a> b$ , $f (a ) $与 $f (b)$ 的大小关系;\\
      (2)若 $f (9^x- 2\cdot 3^x )+ f ( 2\cdot 9^x-k )> 0 $对任意 $x\in[0,\infty )$ 恒成立,求实数 $k$ 的取值范围.\\
      \begin{answer}
      (1)$f(a)>f(b)$;\\
      (2)$k<1$.\\
      \end{answer}
    \vspace{15em}
    \item
      (福建师大附属中学 2016-2017 高一年级期中考试19)定义在$(0,+\infty)$的函数$f(x)$满足下面三个条件:\\ \textcircled{1}对任意正数$a,b$ ,都有$f(a)+f(b)=f(ab)$;\\
      \textcircled{2}当 $x>1$时,$f(x)<0$;\\
      \textcircled{3}$f(2)=-1$.\\
      (\Rmnum{1})求$f(1)$的值;\\
      (\Rmnum{2})试用单调性定义证明:函数$f(x)$ 在$(0,+\infty)$是减函数; \\
      (\Rmnum{3})求满足$f(3x+1)>2$的$x$的取值集合.
      \begin{answer}
      (I)$f(1)=0$;(II)证略;(III)$x\in(\frac13,\frac5{12})$
      \end{answer}
    \vspace{18em}
    \item%福州重点中学期中考真题分类汇编 2函数的相关性质.pdf P19
      (福州市高级中学 2016-2017 高一上期中22)已知函数$f(x)=x^2-2ax+5(a>1)$\\
      (I)若 $f (x )$ 的定义域和值域均是 $[1, a]$ ,求实数 $a$ 的值; \\
      (II)若 $f (x ) $在区间 $[4,+\infty)$上是增函数,且对任意的$ x \in[1, a+ 2]$,都有 $f( x )\leq 0$ ,求实数$a$的取值范围;\\
      (III)若 $g( x )=2^x+\log_2{x+ 1 }$ ,且对任意的 $x \in[0,1]$ ,都存在$f(x_0)=g(x)$ 成立,求实数$a$的取值范围.\\
      \begin{answer}
      (I) $a=2$; (II) $a\geq3$; (III) $a\geq\frac52$
      \end{answer}
    \vspace{18em}
  \end{exercise}

{\hspace{2em}}
{\hspace{2em}}
{\hspace{2em}}
{\hspace{2em}}
\stopexercise

\Topic{集合运算;函数概念;对数与指数运算}
  \Teach{}
  \Grade{高三}
  % \Name{郑皓天}\FirstTime{20181207}\CurrentTime{20181207}
  % \Name{林叶}\FirstTime{20180908}\CurrentTime{20181125}
  %\Name{1v2}\FirstTime{20181028}\CurrentTime{20181117}
  % \Name{林叶}\FirstTime{20180908}\CurrentTime{20181125}
  % \Name{郭文镔}\FirstTime{20181111}\CurrentTime{20181117}
  % \Name{马灿威}\FirstTime{20181111}\CurrentTime{20181111}
  % \Name{黄亭燏}\FirstTime{20181231}\CurrentTime{20181231}
  % \Name{王睿妍}\FirstTime{20190129}\CurrentTime{}
  \Name{郑旭晶}\FirstTime{20190423}\CurrentTime{20190428}
  \newtheorem*{Theorem}{定理}
  \makefront
\vspace{-1.5em}
\startexercise
% \begin{exercise}{\heiti 课前检测}\\
% \end{exercise}
\section{集合}
  \subsection{集合的表示与常用符号}
    一般地,我们把研究对象统称为{\fangsong 元素}(element),把一些具有相同性质的元素组成的总体叫做{\fangsong 集合}(set)(简称为{\fangsong 集}).
    \begin{itemize}
      \item 元素$a${\fangsong 属于}(belong to)集合$A$,记为$a\in A$;
        反之,若元素$a${\fangsong 不属于}集合$A$,记作$a\notin A$;
      \item 不包含任何元素的集合称为{\fangsong 空集},记为$\varnothing$;
      \item 两个集合具有完全一样的元素,则称两集合{\fangsong 相等},记作$A=B$
      \item 如果集合$A$中的元素都属于集合$B$,那么称
        \begin{itemize}
          \item 集合$A${\fangsong 包含于}(contained)集合$B$,记作$A\subseteq B$,并称集合$A$是集合$B$的{\fangsong 子集}(subset);
          \item 集合$B${\fangsong 包含}(contain)集合$A$,记作$B\supseteq A$,并称集合$B$是集合$A$的{\fangsong 超集}(superset);
        \end{itemize}
        进一步,如果$A\subseteq B$,且$A\neq B$(即:$B$中含有集合$A$中没有的元素),则称
          \begin{itemize}
            \item 集合$A$是集合$B$的{\fangsong 真子集}(proper subset),记作$A\subsetneqq B$;
            \item 集合$B$是集合$A$的{\fangsong 真超集}(proper superset),记作$B\supsetneqq A$;
          \end{itemize}
        {\kaishu
          规定空集 $\varnothing$是任何集合的子集
          }
      \item 几个特殊的集合:
        \begin{itemize}
          \item 全体{\fangsong 自然数}$0,1,2,3,\ldots ,$组成的集合,记作$\mathbb{N}$(natural);
          \item 全体正自然数组成的集合,记作$\mathbb{N}_+$或$\mathbb{N}^*$
          \item 全体{\fangsong 整数}$0,\pm1,\pm2,\pm3,\ldots ,$组成的集合,记作$\mathbb{Z}$;
          \item 全体{\fangsong 有理数}组成的集合,记作$\mathbb{Q}$(quotient);
          \item 全体{\fangsong 实数}组成的集合,记作$\mathbb{R}$(real)
          \item 全体{\fangsong 复数}组成的集合,记作$\mathbb{C}$(complex)
        \end{itemize}
      \item 集合的表示方法$\{x\in A\mid p(x)\}$.其中:
        \begin{itemize}
          \item $x\inA$指明集合元素用$x$表示,且元素$x$的取值范围是$A$;
          \item $\mid$为分隔符号(有时用$\colon$表示);
          \item $p(x)$指明集合中的元素所具有的特征
        \end{itemize}
        几个示例:
        \begin{itemize}
          \item 所有偶数组成的集合:$\{x\mid x=2n,\,n\inZ \}$
          \item 所有直线$y=x$上的点组成的集合:$\{(x,y)\mid y=2x\}$
          \item {\fangsong 区间}$[a,b)$:$\{x\mid a\leqslant x<b,\,a<b\}$
          \item {\fangsong 区间}$[a,+\infty)$:$\{x\mid x\geqslant a\}$
        \end{itemize}
    \end{itemize}
  \subsection{集合的运算}
    \begin{itemize}
      \item {\fangsong 交集}:既属于集合$A$,又属于$B$的所有元素组成的集合,记为$A\cap B$;
      \item {\fangsong 并集}:集合$A$中所有元素与集合$B$中的所有元素共同组成的集合,记为$A\cup B$;
      \item {\fangsong 全集}:包含所研究问题中涉及的所有元素的集合;
      \item {\fangsong 补集}:对于全集$U$的一个子集$A$(即$A\subseteq U$),全集$U$中所有不属于集合$A$的元素组成的集合
        称为子集$A$在全集$U$中的补集,记为$\complement_UA$;
    \end{itemize}
  \begin{exercise}
    \item %《2019金考卷双测20套(文)ISBN978-7-5371-9890-5》题型1集合的运算P1p1【2018•全国I卷】【集合,交集】\\
      \source{2018文}{全国I卷}
      已知集合$A=\{0,2\}$,$B=\{-2,-1,0,1,2\}$,则$A\cap B=$\xz
      \xx{$\{0,2\}$}
       {$\{1,2\}$}
       {$\{0\}$}
       {$\{-1,-2.0,1,2\}$}
      \begin{answer}
        A
      \end{answer}
    \item %《2019金考卷双测20套(文)ISBN978-7-5371-9890-5》题型1集合的运算P1p2【2018•天津卷】【集合,补集交集,区间】\\
      {\kaishu [2018 \textbullet 天津卷(文)]}
      设全集为$\mathbb{R}$,集合$A=\{x\mid 0<x<2\}$,$B=\{x\mid x\geqslant1\}$,
      则$A\bigcap\bigl(\complement_{\mathbb R}B\bigr)=$\xz
      \xx{$\{x\mid 0<x\leqslant1\}$}
       {$\{x\mid 0<x<1\}$}
       {$\{x\mid 1\leqslant x<2\}$}
       {$\{x\mid 0<x<2\}$}
      \begin{answer}
        B
      \end{answer}
    \item %《2019金考卷双测20套(文)ISBN978-7-5371-9890-5》题型1集合的运算P1p3【2018•全国II卷】【集合,元素】\\
      {\kaishu [2018 \textbullet 全国II卷(文)]}
      已知集合$A=\{(x,y)\mid x^2+y^2\leqslant3,x\inZ,y\inZ\}$,则$A$中元素的个数为\xz
      \xx{9}{8}{5}{4}
      \begin{answer}
        A
      \end{answer}
    \item %《2019金考卷双测20套(文)ISBN978-7-5371-9890-5》题型1集合的运算P1p4【2018•济南模拟】【集合,并集,二次方程】\\
      {\kaishu [2018 \textbullet 济南模拟(文)]}
      已知集合$A=\{x\mid x^2+2x-3=0\}$,$B=\{-1,1\}$则$A\cup B=$\xz
      \xx{$\{1\}$}
       {$\{-1,1,3\}$}
       {$\{-3,-1,1\}$}
       {$\{-3,-1,1,3\}$}
      \begin{answer}
        C
      \end{answer}
    \item %《2019金考卷双测20套(文)ISBN978-7-5371-9890-5》题型1集合的运算P1p8【2018•福州质检】【集合,元素,交集】\\
      \source{2018}{福州质检(文)}
      已知集合$A=\{x\mid x=2k+1,k\inZ\}$,$B=\{x\mid -1<x\leqslant4\}$,则集合$A\cap B$中元素的个数为\xz
      \xx{1}{2}{3}{4}
      \begin{answer}
        B
      \end{answer}
    \end{exercise}
  \end{exercise}
\vspace{4em}
\section{指数与对数运算}
  \begin{exercise}{\heiti 课前检测}\\
    求下列各式的值:
    \begin{enumerate}[label=\arabic*)]
      \begin{multicols}{2}
        \item $\sqrt[3]{-8}$;
        \item $\sqrt{(-10)^2}$;
        \item $\sqrt[4]{(3-\piup)^4}$;
        \item $\sqrt{(a-b)^2}$\quad($a>b$);
      \end{multicols}
    \end{enumerate}
  \end{exercise}
  \subsection{乘方与开方}
    \begin{description}[leftmargin=0pt]
      \item [乘方] $n$个相同数字的乘积的运算叫做{\fangsong 乘方},乘方的结果叫做{\fangsong 幂}. \par
        实数$a$的$n$次幂$a^n=\underbrace{{a\cdot a\cdot a\cdot \cdots \cdot a}}_{n\text{个}a}$.
        其中,$a$叫做{\fangsong 底数},$n$叫做{\fangsong 指数}.由于$n$是正整数,因此又称乘方$a^n$为{\fangsong 正整数指数幂}.
      \item [开方] 乘方的逆运算称为{\fangsong 开方}.一般地,对于方程$x^n=a$,若$n>1$且$n\inN$,则解出所有$x$值的运算就叫做
        {\fangsong $~a$开$n$次方},相应地,$a$叫做{\fangsong 被开方数}(radicand);方程的解$x$叫做$a$的 {\fangsong $~n$次方根}(nth root).
        开二次方又称为{\fangsong 开平方},开三次方又称为{\fangsong 开立方}.不致误解的情况下,开方常作为开平方的简称.

        一个数的$n$次方根可能有不止一个值,其中与被开方数$a$符号相同的一个记为:$\sqrt[n]a$,此式称为{\fangsong 根式}(radical),
        这里$n$被称为{\fangsong 根指数(radical exponent)}.$n=2$时常省略不写,即:写成$\sqrt a$.
      \item [运算性质]对于正整数指数幂,有以下运算律:(其中$a\inR$,$r,s\inN_+$)
        \begin{itemize}%[leftmargin=*]
          \item $a^ra^s=a^{r+s}$
          \item $(a^r)^s=a^{rs}$
          \item $(ab)^r=a^rb^r$
        \end{itemize}
    \end{description}
  % \subsection{指数幂运算}
  %   在乘方(正整数指数幂)的运算性质中,引入除法与开方,则为使运算性质保持一致性,可分别引入非负数的整数指数幂与分数指数幂.有:
  %   \begin{itemize}
  %     \item $a^0=1$,$a\neq 0$;
  %     \item $a^{-n}=\dfrac1{a^n}$($n\inN_+$),$a\neq 0$;
  %     \item $a^{\frac{m}{n}}=\sqrt[n]m$($m,n\inN_+$,n>1),$a\geqslant 0$;
  %     \item $a^{-\frac{m}{n}}=\dfrac{1}{\sqrt[n]m}$($m,n\inN_+,~n>1$),$a>0$;
  %   \end{itemize}
  \subsection{指数运算与对数运算}
    对于$a^k=N$,满足一定条件时:
    \begin{itemize}
      \item {\fangsong 指数幂运算}:已知$a$、$k$可以求出$N$.如$2^3=8$,$2^{-3}=\dfrac18$,$2^{1/2}=\sqrt2$;
      \item {\fangsong 开方运算}:已知$N$、$k$可以求出$a$.如$\sqrt[3]{-8}=-2$,$\sqrt4=2$,$\sqrt[4]{64}=2\sqrt2$;
        需满足的条件:$n>1$且$n\inN$,且$n$为偶数时$N>0$;
      \item {\fangsong 对数运算}:已知$a$、$N$可以求出$k$.如$\log_28=3$,$\log_2{\frac1{16}}=-4$.
        需满足的条件:$a>0$且$a\neq 1$,$N>0$,
    \end{itemize}
    \begin{exercise}
      \item %《2018天利38套:高考真题单元专题训练(文)》专题9幂函数、指数函数、对数函数P27p1【2016文•全国新课标】【对数指数,比大小】
        \source{2016}{全国新课标(文)}
        若$a>b>0$,$0<c<1$,则\xz
        \xx{$\log_ac<\log_bc$}
         {$\log_ca<\log_cb$}
         {$a^c<b^c$}
         {$c^a>c^b$}
        \begin{answer}
          B
        \end{answer}
      \item %《2018天利38套:高考真题单元专题训练(文)》专题9幂函数、指数函数、对数函数P27p3【2016文•浙江】【对数,定义】
        \source{2010文}{浙江}
        已知函数$f(x)=\log_2(x+1)$,若$f(a)=1$,则$a=$\xz
        \xx{0}{1}{2}{3}
        \begin{answer}
          B
        \end{answer}
      \item %《2018天利38套:高考真题单元专题训练(文)》专题9幂函数、指数函数、对数函数P27p18【2016文•】【对数指数,定义】
        \source{2014}{陕西(文)}
        已知$4^a=2$,若$\lg x=a$,则$x=$\tk.
        \begin{answer}
          $\sqrt{10}$
        \end{answer}
    \end{exercise}
\section{函数概念}
  \begin{exercise}{\heiti 课前检测}\\
    填写下表,写出各函数的定义域、值域 、单调性以及奇偶性.\
    \vspace{-2em}
    \begin{center}
      \renewcommand{\arraystretch}{1.4}
      \begin{tabular}{|c|c|c|c|c|}
        \hline
      $f(x)$&\mbox{\hspace{1.5em}定义域\hspace{1.5em}}&\mbox{\hspace{2em}值域\hspace{2em}}&\mbox{\hspace{8em}单调性\hspace{8em}}&\mbox{\hspace{1.2em}奇偶性\hspace{1.2em}}\\
        \hline
        $x$&&&&\\
        \hline
        $x^2$&&&&\\
        \hline
        $\log_2x$&&&&\\
        \hline
        $3^x$&&&&\\
        \hline
        $\dfrac{1}{x}$&&&&\\
        \hline
        $\sqrt{x}$&&&&\\
        \hline
        $\log_x2$&&&&\\
        \hline
      \end{tabular}\\
    \end{center}
  \end{exercise}
  \begin{description}
    \item [定义] 一般地,有:\\
      设 $A$,$B$ 是非空的数集,如果按照某种确定的对应关系 $f$,使对于集合$A$中的任意一个数 $x$,在集合 $B$ 中都有唯一确定的数 $f(x)$ 和它对应,那么就称 $f\colon A\mapsto B$ 为从集合 $A$ 到集合 $B$ 的一个函数,记作
      $$y=f(x),\qquad x\in A.$$
      其中,$x$ 叫做自变量,$x$ 的取值范围 $A$ 叫做函数的定义域;与 $x$ 的值相对应的 $y $值叫做函数值,函数值的集合$\{f(x)|x\in A\}$叫做函数的值域,值域是集合$B$ 的子集.
      \begin{itemize}[leftmargin=*]
        \kaishu
        \item 函数是两个数集间的一种对应关系;
        \item 未指明定义域的情况下,默认定义域取使得对应关系有意义的所有实数. 具体如下:
        \begin{enumerate}[label=\circled{\arabic*}]
          \item 分式的分母不为0;
          \item 偶次根式的被开方数不小于0;
          \item 零次或负次指数次幂的底数不为零;
          \item 对数的真数大于0;
          \item 指数、对数函数的底数大于0且不等于1;
          \item 实际问题对自变量的限制.
        \end{enumerate}
        \item 若函数$f(x)$定义域为$D$,且$f(a)$存在,则$a\in D$.
      \end{itemize}
  \end{description}
  \clearpage
  % \begin{exercise}
  % \end{exercise}

\newpage
\section{课后作业}
  % \begin{exercise}{\heiti 练习}
  % \end{exercise}
  \begin{exercise}
    \item %《2019金考卷双测20套(文)ISBN978-7-5371-9890-5》题型1集合的运算P1p7【2018•郑州二测】【集合,并集,对数】\\
      \source{2018}{郑州二测(文)}
      已知集合$A=\{x\inR\mid \log_2(3-x)\leqslant 1\}$,$B=\{x\inR\mid 0\leqslant x\leqslant 2\}$,则$A\cup B=$\xz
      \xx{$[0,3]$}
       {$[1,2]$}
       {$[0,3)$}
       {$[1,3]$}
      \begin{answer}
        C
      \end{answer}
    \item %《2019金考卷双测20套(文)ISBN978-7-5371-9890-5》题型1集合的运算P1p8【2018•南昌调研】【集合,交集,对数】\\
      \source{2018}{南昌调研(文)}
      设集合$A=\{x\mid -2\leqslant x\leqslant 1\}$,$B=\{x\mid y=\log_2{(x^2-2x-3)}\}$,则$A\cap B=$\xz
      \xx{$[-2,1)$}
       {$(-1,1]$}
       {$[-2,-1)$}
       {$[-1,1)$}
      \begin{answer}
        C
      \end{answer}
    \item %《2019金考卷双测20套(文)ISBN978-7-5371-9890-5》题型1集合的运算P1p9【2018•合肥一检】【集合,交集,定义域,值域】\\
      \source{2018}{合肥一检(文)}
      已知集合$M$是函数$y=\dfrac1{\sqrt{1-2x}}$的定义域,集合$N$是函数$y=x^2-4$的值域,则$M\cap N=$\xz
      \xx{$\{x\mid x\leqslant \dfrac12\}$}
       {$\{x\mid -4\leqslant x< \dfrac12\}$}
       {$\{(x,y)\mid x< \dfrac12\text{且}\geqslant -4\}$}
       {$\varnothing$}
      \begin{answer}
        B
      \end{answer}
    \item %《2019金考卷双测20套(文)ISBN978-7-5371-9890-5》题型1集合的运算P1p16【2018•江苏卷】【集合,交集】\\
      {\kaishu (2018 \textbullet 江苏卷(文))}
      已知集合$A=\{0,1,2,8\}$,$B=\{-1,1,6,8\}$,那么$A\cap B=$\tk.
      \begin{answer}
        $\{1,8\}$
      \end{answer}
  \end{exercise}
\stopexercise

\newpage
\section{参考答案}
\begin{multicols}{2}
  \printanswer
\end{multicols}

\Topic{平面向量坐标表示,数量积}
  \Teach{向量共线与数量积的坐标表示}
  \Grade{高一}
  % \Name{郑皓天}\FirstTime{20181207}\CurrentTime{20181207}
  % \Name{林叶}\FirstTime{20180908}\CurrentTime{20181125}
  % \Name{1v2}\FirstTime{20181028}\CurrentTime{20181117}
  % \Name{林叶}\FirstTime{20180908}\CurrentTime{20181125}
  % \Name{郭文镔}\FirstTime{20181111}\CurrentTime{20181117}
  % \Name{马灿威}\FirstTime{20181111}\CurrentTime{20181111}
  \newtheorem*{Theorem}{定理}
  \makefront
\vspace{-1.5em}

\startexercise
% \begin{exercise}{\heiti 课前检测}\\
%   表格实例:
%   \begin{center}
%     \renewcommand{\arraystretch}{1.4}
%     \begin{tabular}{|*{8}{c|}}
%       \hline
%         $x$
%         &$-\dfrac{\piup}6$
%         &$-\dfrac{\piup}3$
%         &$-\dfrac{5\piup}6$
%         &$-\dfrac{4\piup}3$
%         &$-\dfrac{11\piup}6$
%         &$-\dfrac{7\piup}3$
%         &$-\dfrac{17\piup}6$\\
%       \hline
%         $y$
%         &$-1$
%         &$1$
%         &$3$
%         &$1$
%         &$-1$
%         &$1$
%         &$3$\\
%       \hline
%     \end{tabular}\\
%   \end{center}
% \end{exercise}

\section{平面向量基本定理及坐标表示}
  \subsection{基底}
    \begin{Theorem}[平面向量基本定理]
      如果$ \bm{e}_1,\bm{e}_2 $是同一平面内的两个\CJKunderdot{不共线}的向量,那么对于这一平面内的任意向量$ \bm{a} $,有且只有一对实数$ \lambda_1,~\lambda_2 $,使$\bm{a}=\lambda_1\bm{e}_1+\lambda_2\bm{e}_2$.\par 其中,不共线的向量$ \bm{e}_1,~\bm{e}_2 $叫做表示这一平面内所有向量的一组\CJKunderdot{基底}.
    \end{Theroem}
    {\kaishu 解决向量问题,需要注意两点:一是向量共线定理,一个是平面向量基本定理.\par
    向量的基底的重要性在于一旦有了基底,你就可以将题目中涉及的所有向量都用基底向量唯一的表示出来(坐标表示就是一组特殊的基底),计算和变形都有了方向,便于寻找和发现关系.如果题目没有明确给出基底,那么就需要自己指定了}
  \subsection{坐标表示}
    在不共线的向量中,垂直是一种重要的情形,把一个向量分解为两个互相垂直的向量,叫做把向量\CJKunderdot{正交分解}.\par
    在平面直角坐标系$xOy$中,分别取与$x$轴、$y$轴方向相同的两个\CJKunderdot{单位向量}$~ \bm{i},~\bm{j} $作为基底.对于平面内的一个向量$ \bm{a} $,由平面向量基本定理可知,有且只有一对实数$ x,~y $使得\[\bm{a}=x\bm{i}+y\bm{j}\]
    这样,平面内的任一向量$ \bm{a} $都可以由$ x,~y $唯一确定,我们把有序数对$ (x,y) $叫做向量$\bm{a}  $的坐标,记作
    \begin{equation}\label{eq:axy}
      \bm{a}=(x,y)
    \end{equation}
    其中$ x $叫做$ \bm{a} $在$x$轴上的坐标,$ y $叫做$ \bm{a} $在$y$轴上的坐标,(\ref{eq:axy})式叫做\textbf{向量的坐标表示}
    \begin{center}
    \begin{tikzpicture}[scale=0.7]
      \draw[->,>=stealth] (-1,0)--(3,0) node[below](x){$x$};
      \draw[->,>=stealth] (0,-1)--(0,3) node[right](y){$y$};
      \draw[very thick,->,>=stealth](0,0)--(1,0)node[midway,below](i) {$\bm{i}$};
      \draw[very thick,->,>=stealth](0,0)--(0,1)node[midway,left](j) {$\bm{j}$};
      \coordinate(A) at (1.5,1.5);
      \node[right](a1)at(1.5,1.5){$A(x,y)$};
      \draw[dashed](A)--++(-1.5,0)node[left](y){$y$};
      \draw[dashed](A)--++(0,-1.5)node[below](x){$x$};
      \draw[->,>=stealth](0,0)--(A) node[midway,left] (a) {$\bm{a}$};
    \end{tikzpicture}
    \end{center}
    \begin{description}
      \item[三点共线的判定] 若$ A,~B,~C $三点共线,有$ \vv{OA}=\lambda \vv{OB}+\mu\vv{OC}~(\lambda+\mu=1) $.或$\vv{AB}=\lambda \vv{AC}$.
      %\begin{enumerate}[1)]
      %
      %\end{enumerate}
    \end{description}
  \subsection{平面向量的坐标计算}
    \begin{enumerate}
      \item
        设点$ A(x_1,y_1),~B(x_2,y_2) $,则$ \vv{AB}=(x_2-x_1,y_2-y_1) $.\par
        一个向量的坐标等于表示此向量的有向线段的终点的坐标减去起始点的坐标.
      \item
        若$\bm{a}=\left(x_1,y_1\right),\bm{b}=\left(x_2,y_2\right)$.
        \begin{description}
          \item[加法:]
            $\bm{a}+\bm{b}=(x_1+x_2,y_1+y_2)$
            \begin{equation*}
            \begin{aligned}
            \bm{a}+\bm{b}=&\left(x_1\bm{i}+y_1\bm{j}\right)+\left(x_2\bm{i}+y_2\bm{j}\right)\\
            =&\left(x_1+x_2\right)\bm{i}+\left(y_1+y_2\right)\bm{j}\\
            \text{即:}\bm{a}+\bm{b}=&(x_1+x_2,y_1+y_2)
            \end{aligned}
            \end{equation*}
          \item[减法:] $\bm{a}-\bm{b}=\left(x_1-x_2,y_1-y_2\right)$.同加法可得
          \item[数乘:]
            $ \lambda \bm{a}=\left(\lambda x_1,\lambda y_1\right) $\begin{equation*}
            \begin{aligned}
             \lambda \bm{a} =&\lambda\left(x_1\bm{i}+y_1\bm{j}\right)=\lambda x_1\bm{i}+\lambda y_1\bm{j}\\
            =&\left(\lambda x_1,\lambda y_1\right)
            \end{aligned}
            \end{equation*}
          \item[模长]
           $\abs{\bm{a}}=\sqrt{x_1^2+y_1^2}$\qquad
           $\abs{\vv{AB}}=\sqrt{(x_2-x_1)^2+(y_2-y_1)^2}$\\
           \qquad $\abs{\bm{a}+\bm{b}}=\sqrt{(\bm{a}+\bm{b})^2}=\sqrt{\bm{a}^2+2\bm{a}\bm{\cdot}\bm{b}+\bm{b}^2}$\\
          \item[共线]$\bm{a} \varparallel \bm{b}\Leftrightarrow x_1y_2=y_2x_1 $\\
            由向量共线的性质知$ \bm{a} $与$ \bm{b}(\bm{b}\ne\bm{0}) $共线,当且仅当存在实数$ \lambda $使得$ \bm{a}=\lambda \bm{b} .$\\用坐标表示为:
            $$(x_1,y_1)=\lambda(x_2,y_2)$$
            即$$\Bigg\{\begin{aligned}
            x_1=&\lambda x_2\\
            y_1=&\lambda y_2
            \end{aligned}$$
            消去$ \lambda $得到\[x_1y_2-x_2y_1=0\]
          \item[垂直]
            $\bm{a}\perp\bm{b}\Leftrightarrow\bm{a}\bm{\cdot}\bm{b}=0\Leftrightarrow x_1x_2+y_1y_2=0 $
            \begin{proof}
              \begin{description}
                \item[方法一]
                  设$ \bm{a},~\bm{b} $所在直线分别为$ l_1,l_2 $,当$ \bm{a},~\bm{b} $所在直线的斜率都存在时,由直线垂直的性质,有$$ k_{l_1}\bm{\cdot}k_{l_2}=-1 $$
                  其中$$ k_{l_1} =\dfrac{y_1-0}{x_1-0}=\dfrac{y_1}{x_1},\quad k_{l_2} =\dfrac{y_2-0}{x_2-0}=\dfrac{y_2}{x_2}$$
                  即$$\dfrac{y_1}{x_1}\bm{\cdot}\dfrac{y_2}{x_2}=-1$$
                  $$x_1x_2+y_1y_2=0$$
                \item[方法二]
                  由向量的数量积性质,当$ \bm{a}\perp\bm{b} $时,
                  $\text{由}\cos\theta=\dfrac{\bm{a\cdot b}}{\abs{\bm{a}}\abs{\bm{b}}}\text{得到}$\\
                  \centering $\bm{a\cdot b}=0$
              \end{description}
            \end{proof}
        \end{description}
    \end{enumerate}
  \begin{exercise}{{\textbf{基础练习}}}
    \item%《习题化知识清单》P82知识1-3【平面向量基本定理】
      已知向量$\bm a=(1,2)$,$\bm b=(-2,3)$,$\bm c=(4,1)$,若用$\bm a$和$\bm b$表示$\bm c$,则$\bm c=$\tk.
      \begin{answer}
        $2\bm a-\bm b$
      \end{answer}
    \item%《习题化知识清单》P82知识2-1【向量坐标运算】
      设平面向量$\bm a=(3,5)$,$\bm b=(-2,1)$,则$\bm a-2\bm b=$\xz
      \xx{(7,3)}{(7,7)}{(1,7)}{(1,3)}
      \begin{answer}
        A
      \end{answer}
    \item%《习题化知识清单》P82知识2-2【向量坐标运算,单位向量】
      已知$\bm a=(3,4)$,则与$\bm a$同向的单位向量的坐标是\xz
      \xx{$(3,4)$}
       {$(-\dfrac{3}5,\dfrac{4}5)$}
       {$(-\dfrac{3}5,-\dfrac{4}5)$}
       {$(\dfrac{3}5,\dfrac{4}5)$}
      \begin{answer}
        D
      \end{answer}
    \item%《习题化知识清单》P82知识2-3【向量坐标运算,中点坐标公式】
      已知平面直角坐标系$xOy$内的三点分别是$A(2,-5)$,$B(3,4)$,$C(-1,-3)$,$D$为线段$BC$的中点,则向量$\vv{DA}$的坐标为\tk.
    \item%《习题化知识清单》P82知识3-1【向量共线】
      设向量$\bm a=(m,1)$,$\bm b=(1,m)$,如果$\bm a$与$\bm b$共线且方向相反,那么$m$的值为\xz
      \xx{$1$}{$-1$}{$\pm 1$}{$0$}
      \begin{answer}
        B
      \end{answer}
    \item%《习题化知识清单》P82知识3-3【向量共线,三角函数的定义】
      若$\bm a=\Bigl(\dfrac{3}2,\sin\alpha\Bigr)$,$\bm b=\Bigl(\sin\alpha,\dfrac{1}3\Bigr)$,且$\bm a\varparallel \bm b$,则锐角$\alpha$为\xz
      \xx{30\degree}{45\degree}{60\degree}{75\degree}
      \begin{answer}
        B
      \end{answer}
    \item%《习题化知识清单》P82知识3-4【向量共线坐标表示】
      若向量$\vv{OA}=(k,6)$,$\vv{OB}=(4,5)$,$\vv{OC}=(1-k,10)$,且$A$,$B$,$C$三点共线,则$k=$\tk.
      \begin{answer}
        $\dfrac{17}6$
      \end{answer}
  \end{exercise}
\section{平面向量的数量积}
  \subsection{定义}
    \begin{description}
      \item[定义] 已知两个非零向量$ \bm{a} $与$\bm{b}$,我们把数量$ \abs{\bm{a}}\abs{\bm{b}}\cos\theta $叫做$ \bm{a} $与$ \bm{b} $的\textbf{数量积}(又称点积、内积),记作$ \bm{a}\bm{\cdot}\bm{b} $,即\[\bm{a}\bm{\cdot}\bm{b}=\abs{\bm{a}}\abs{\bm{b}}\cos\theta\]
      其中$ \theta $为$ \bm{a} $与$ \bm{b} $的夹角.\\
      $\abs{\bm{a}}\cos\theta$($\abs{\bm{b}}\cos\theta$)叫做向量$\bm a$在$\bm b$方向上($\bm b$在$\bm a$方向上)的\textbf{投影},记作$\mathrm{Prj}_{\bm b}{\bm a}$(或$\mathrm{Prj}_{\bm b}{\bm a}$)
      \item[几何意义] 两个向量的数量积等于其中一个向量的模长与另一个向量在此向量方向上的投影的乘积\\%$ \bm{a}\bm{\cdot}\bm{b} $等于$\bm{a} $的模长$ \abs{\bm{a}} $与$ \bm{b} $在$ \bm{a} $的方向上的\textbf{投影}$ \abs{\bm{b}}\cos \theta $的乘积.\par
      {\kaishu \textbf{注:}当$ \theta=0 $时,$ \cos\theta=1 $,所以有$ \bm{a\cdot b}=\bm{\abs{a}\abs{b}} $;\\\phantom{注:\ }当$ \theta=90^{\circ} $时,有$ \cos\theta =0$,所以有$ \bm{a\cdot b}=0 $ \\\phantom{注:\ }当$ \theta=180^{\circ} $时,有$ \cos\theta =-1$,所以有$ \bm{a\cdot b}=\abs{\bm{a}}\abs{\bm{b}} $   }
      \item[数量积计算]
      \begin{equation*}
      \begin{aligned}
      &\because \bm{a}=x_1\bm{i}+y_1\bm{j},~\bm{b}=x_2\bm{i}+y_2\bm{j},\\
      &\therefore \bm{a}\bm{\cdot}\bm{b}=(x_1\bm{i}+y_1\bm{j})\bm{\cdot}(x_2\bm{i}+y_2\bm{j})\\
      &\phantom{\therefore\bm{a}\bm{\cdot}\bm{b}~}=x_1x_2\bm{i}^2+x_1y_2\bm{i}\bm{\cdot}\bm{j}+x_2y_1\bm{i}\bm{\cdot}\bm{j}+y_1y_2\bm{j}^2.\\
      & \bm{i}^2=\bm{j}^2=1,\bm{i}\bm{\cdot}\bm{j}=\bm{j}\bm{\cdot}\bm{i}=0\\
      &\therefore \bm{a}\bm{\cdot}\bm{b}=x_1x_2+y_1y_2.
      \end{aligned}
      \end{equation*}
      \item[运算律]已知向量$\bm a$,$\bm b$,$\bm c$和实数$\lambda$,则:
        \begin{itemize}%[leftmargin=*]
          \item 交换律:$\bm{a}\cdot\bm{b}=\bm{b}\cdot\bm{a}$;
          \item 分配律:$(\bm a+\bm b)\cdot \bm c=\bm a\cdot\bm c+\bm b\cdot \bm c$;
          \item $(\lambda \bm a)\cdot\bm{b}=\lambda(\bm a\cdot\bm b)=\bm{a}\cdot(\lambda\bm{b})$.
        \end{itemize}
      \item[夹角公式] \[ \cos\theta=\dfrac{\bm{a}\bm{\cdot}\bm{b}}{\abs{\bm{a}}\abs{\bm{b}}}=\dfrac{x_1x_2+y_1y_2}{\sqrt{x_1^2+y_1^2}\sqrt{x_2^2+y_2^2}} \quad \left(\theta\in\left[0,\pi\right],~\theta\text{也写作}\left<\bm{a},\bm{b}\right>\right).\]
    \end{description}\par
    {\kaishu 直接求向量的数量积的方是近年高考的重点,其关键是根据向量的加减法则对向量进行基底分解.分解以后可以直接使用题目的已知条件,要么出现所要求的表达式(此时通过解一元一次方程).分解过程中,往往利用\CJKunderdot{垂直}将数量积消掉.\CJKunderdot{整体}的思想在数学中占据着极其重要的位置,求解整体的值时,往往不需要分别求出各个元素的值,而是将元素进行有效的分解、整合,提取有效的信息,从而求出整体的值.}
    %\subsection{求向量夹角的方法}
    %\begin{description}
    %\item[坐标法] $\cos\theta=\dfrac{x_1x_2+y_1y_2}{\sqrt{x_1^2+y_1^2}\sqrt{x_2^2+y_2^2}}$
    %\item[向量法] $\cos\theta=\dfrac{\bm{a}\bm{\cdot}\bm{b}}{\abs{\bm{a}}\abs{\bm{b}}}$
    %\end{description}
  \subsection{数量积相关补充}
    \begin{enumerate}[label=\circled{\arabic*}]
      \item 若$\bm{a}=\left(x,y\right)$,则$ \bm{a}\bm{\cdot}\bm{a}=\bm{a}^2=\abs{\bm{a}}^2=x^2+y^2 $;
      \item $ \left(\bm{a}\pm \bm{b}\right)^2=\abs{\bm{a}\pm \bm{b}}^2=\abs{\bm a}^2\pm2\bm{a}\bm{\cdot}\bm{b}+\abs{\bm b}^2=\bm{a}^2\pm2\bm{a}\bm{\cdot}\bm{b}+\bm{b}^2$;
      \item $\bm a^2-\bm b^2=(\bm a+\bm b)\cdot(\bm a-\bm b)=\abs{\bm a}^2-\abs{\bm b}^2$;
      \item $\bm a^2+\bm b^2=0\Leftrightarrow \bm a=\bm b=0$
      \item $\bigm|\abs{\bm{a}}-\abs{\bm{b}}\bigm|\le\abs{\bm{a\pm b}}\le\abs{\bm{a}}+\abs{\bm{b}}$;
      \item 若点$ A(x_1,y_1),~B(x_2,y_2) $,则$ \abs{\vv{AB}}=\sqrt{(x_2-x_1)^2+(y_2-y_1)^2} $;
      \item \textbf{柯西-施瓦兹不等式:}若$\bm{a}=\left(x_1,y_1\right),\bm{b}=\left(x_2,y_2\right)$,则:$$ -\abs{\bm{a}}\abs{\bm{b}}\le\bm{a}\bm{\cdot}\bm{b}\le\abs{\bm{a}}\abs{\bm{b}}\Leftrightarrow -\sqrt{x_1^2+y_1^2}\sqrt{x_2^2+y_2^2}\le x_1x_2+y_1y_2\le \sqrt{x_1^2+y_1^2}\sqrt{x_2^2+y_2^2}$$
      \item 若$ \abs{\bm{a}+\bm{b}}=\abs{\bm{a}-\bm{b}} $,则$ \bm{a}\perp\bm{b} $.对角线相等的平行四边形必然是矩形.
      \item 若$ \left(\bm{a}+\bm{b}\right)\perp\left(\bm{a}-\bm{b}\right) $,则$\abs{\bm{a}}=\abs{\bm{b}} $.对角线垂直的平行四边形必然是菱形.
      \item 平面上$ O,~A,~B $三点不共线,设$\vv{OA}=\bm{a}=(x_1,y_1) ,~\vv{OB}=\bm{b}=(x_2,y_2)$,则$$ S_{\triangle OAB}=\dfrac{1}{2}\sqrt{\abs{\bm{a}}^2\abs{\bm{b}}^2-\left(\bm{a}\bm{\cdot}\bm{b}\right)^2}=\dfrac{1}{2}\abs{x_1y_2-x_2y_1} .$$
      \item 给定两个长度为$ a $的平面向量$ \vv{OA},~\vv{OB} $,其夹角为$ \theta\in\left[0,\pi \right), ~$点$ C $在以$ O $为圆心的圆弧$ AB $上变动,若$ \vv{OC}=x\vv{OA}+y\vv{OB},~x,y\inR $,则$ x+y $的最大值为$ \sqrt{\dfrac{2}{\cos\theta+1}}. $
    \end{enumerate}
  \begin{exercise}{{\textbf{基础练习}}}
    \item%《习题化知识清单》P83知识1-1【数量积的定义、性质】
      在$\triangle{ABC}$中,$AB=BC=2$,$\angle{B}=\dfrac{\piup}4$,$AD$是边$BC$上的高,则$\vv{AD}\cdot\vv{AC}$的值为\xz
      \xx{0}{2}{4}{8}
      \begin{answer}
        B
      \end{answer}
    \item%《习题化知识清单》P83知识1-2【数量积的定义、性质】
      已知$\triangle{ABC}$中,$AB=AC=BC=6$,平面内一点$M$满足$\vv{BM}=\dfrac{2}3\vv{BC}-\dfrac{1}3\vv{BA}$,则$\vv{AC}\cdot\vv{MB}$等于\xz
      \xx{$-9$}{$-18$}{$12$}{$18$}
      \begin{answer}
        B
      \end{answer}
    \item%《习题化知识清单》P83知识2-2【向量的夹角】
      已知$\abs{\bm a}=2$,$\abs{\bm b}=4$,且$(\bm a+\bm b)\perp \bm a$,则$\bm a$与$\bm b$的夹角为\xz
      \xx{$\dfrac{2\piup}3$}
       {$\dfrac{\piup}3$}
       {$\dfrac{4\piup}3$}
       {$-\dfrac{2\piup}3$}
      \begin{answer}
        A
      \end{answer}
    \item%《习题化知识清单》P84知识4-2【数量积的坐标表示】
      已知$\bm a=(2,-3)$,$\bm b=(1,-2)$,且$\bm c\perp \bm a$,$\bm b\cdot\bm c=1$,则$\bm c$的坐标为\xz
      \xx{$(3,-2)$}{$(3,2)$}{$(-3,-2)$}{$(-3,2)$}
      \begin{answer}
        C
      \end{answer}
    \item%《习题化知识清单》P84知识4-3【数量积的坐标表示】
      在以$OA$为一边,$OB$为一条对角线的矩形中,$\vv{OA}=(-3,1)$,$\vv{OB}=(-2,k)$,则实数$k=$\xz
      \xx{$4\sqrt{3}$}{$3\sqrt{3}$}{$\dfrac{\sqrt{3}}2$}{$4$}
      \begin{answer}
        D
      \end{answer}
    \item%《习题化知识清单》P82知识2-4【向量的投影】
      已知点$A(-1,1)$,$B(1,2)$,$C(-2,-1)$,$D(3,4)$,则向量$\vv{CD}$在$\vv{AB}$方向上的投影为\tk.
      \begin{answer}
        $3\sqrt{5}$
      \end{answer}
    \item%《习题化知识清单》P84知识3-3【数量积的运算律】
      已知不共线向量$\bm a$,$\bm b$,$\abs{\bm a}=2$,$\abs{\bm b}=3$,$\bm a\cdot(\bm b-\bm a)=1$,则$\abs{\bm b-\bm a}=$\tk.
      \begin{answer}
        $\sqrt{3}$
      \end{answer}
    \item%《习题化知识清单》P84知识5-3【数量积的应用,运算律】
      已知$\abs{\bm a}=\abs{\bm b}=1$,$\bm a$,$\bm b$的夹角是直角,
      $\bm c=2\bm a+3\bm b$,$\bm d=k\bm a-4\bm b$,$\bm c\perp\bm d$,则$k=$\tk.
      \begin{answer}
        6
      \end{answer}
  \end{exercise}
\newpage
\section{课后练习}
  \begin{exercise}
    \item%《习题化知识清单》P82方法1【平面向量基本定理应用】
      在平行四边形$ABCD$中,$M$,$N$分别为$DC$,$BC$的中点,已知$\vv{AM}=\bm c$,$\vv{AN}=\bm d$,试用$\bm c$,$\bm d$表示$\vv{AB}=$\tk,$\vv{AD}=$\tk.
      \begin{answer}
        $\vv{AB}=\dfrac{2}3(2\bm d-\bm c)$,$\vv{AD}=\dfrac{2}3(2\bm c-\bm d)$
      \end{answer}
    \item%《习题化知识清单》P83方法2【向量共线条件应用】
      平面内给定三个向量$\bm a=(3,2)$,$\bm b=(-1,2)$,$\bm c=(4,1)$,则\\
      (1)若$(\bm a+k\bm c)\varparallel(2\bm b-\bm a)$,则实数$k=$\tk;\\
      (2)设$\bm d=(x,y)$满足$(\bm d-\bm c) \varparallel (\bm a+\bm b)$且$\abs{\bm d-\bm c}=1$,则$\bm d=$\tk.
      \begin{answer}
        (1)$k=-\dfrac{16}{13}$;(2)$\bm d=\Bigl(\dfrac{20+\sqrt{5}}5,\dfrac{5+2\sqrt{5}}5\Bigr)$或$\Bigl(\dfrac{20-\sqrt{5}}5,\dfrac{5-2\sqrt{5}}5\Bigr)$
      \end{answer}
    \item%《习题化知识清单》P83方法2-2【向量共线条件应用】
      若平面向量$\bm a$,$\bm b$满足$\abs{\bm a+\bm b}=1$,$\bm a+\bm b$平行于$x$轴,$\bm b=(2,-1)$,则$\bm a=$\tk.
      \begin{answer}
        $(-1,1)$或$(-3,1)$
      \end{answer}
    \item%《习题化知识清单》P84方法1【求向量夹角基本方法】
      已知$\abs{\bm a}=1$,$\abs{\bm b}=2$,$\bm a$与$\bm b$的夹角为120\degree,则使$\bm a+k\bm b$与$k\bm a+\bm b$的夹角为锐角的实数$k$的取值范围是\tk.
      \begin{answer}
        $\Bigl(\dfrac{5-\sqrt{21}}2,1\Bigr) \bigcup \Bigl(1,\dfrac{5-\sqrt{21}}2 \Bigr)$
      \end{answer}
    \item%《习题化知识清单》P84方法2【求向量模的基本方法】
      已知向量$\bm a$,$\bm b$夹角为45\degree,且$\abs{\bm a}=1$,$\abs{2\bm a-\bm b}=\sqrt{10}$,则$\abs{\bm b}=$\tk.
      \begin{answer}
        $3\sqrt{2}$
      \end{answer}
    \item%《习题化知识清单》P84方法2-4【求向量模的基本方法】
      已知向量$\bm a=(x.y)$,$\bm b=(-1,2)$,且$\bm a+\bm b=(1,3)$,则$\abs{\bm a-2\bm b}$等于\tk.
      \begin{answer}
        $8\sqrt{2}$
      \end{answer}
    \item%《习题化知识清单》P85方法4
      在矩形$ABCD$中,$AB=\sqrt{2}$,$BC=2$,点$E$为$BC$的中点,点$F$在边$CD$上,若$\vv{AB}\cdot\vv{AF}=\sqrt{2}$,则$\vv{AE}\cdot\vv{BF}$的值是\tk.
      \begin{answer}
        $\sqrt{2}$
      \end{answer}
    \item%《习题化知识清单》P85方法4-2
      已知正方形$ABCD$的边长为1,点$E$是$AB$边上的动点,则$\vv{DE}\cdot\vv{CB}$的值为\tk;$\vv{DE}\cdot\vv{DC}$的最大值为\tk.
      \begin{answer}
        1;1
      \end{answer}
    \item%《高中数学竞赛培优教程+一试(李名德 主编)》.pdf P121-例5.19
      已知$x^2+y^2=25$,函数$z=\sqrt{8y-6x+50}+\sqrt{8y+6x+50}$的最大值为\tk.
      \begin{answer}
        $z_{\max}=6\sqrt{10}$(当且仅当$x=0$,$y=5$时取得)
      \end{answer}
    \item%《高中数学竞赛培优教程+一试(李名德 主编)》.pdf P115-例5.10
      已知向量$\bm a=(\cos\alpha,\sin\alpha)$,$\bm b=(\cos\beta,\sin\beta)$,且$\bm a$,$\bm b$满足关系$\abs{k\bm a+\bm b}=\sqrt{3}\abs{\bm a-k\bm b}$($k>0$).\\
      (1)求将$\bm a$与$\bm b$的数量积用$k$表示的解析式$f(k)$;\\
      (2)\hspace{5pt}$\bm a$能否和$\bm b$垂直?$\bm a$能否和$\bm b$平行?若不能,则说明理由;若能,则求出对应的$k$值;\\
      (3)求$\bm a$与$\bm b$夹角的最大值.
      \begin{answer}
        (1)$f(k)=\dfrac{k^2+1}{4k} (k>0)$;
        (2)$\bm a$与$\bm b$不可能垂直;当$k=2\pm\sqrt{3}$时,$\bm a\varparallel \bm b$;
        (3)60\degree
      \end{answer}
    \vspace{7cm}
    \item%《高中数学竞赛培优教程+一试(李名德 主编)》.pdf P114-例5.8
      已知$\bm a=(\cos\alpha,\sin\alpha)$,$\bm b=(\cos\beta,\sin\beta)$($0<\alpha<\beta<\piup$).\\
      (1)求证:$\bm a+\bm b$与$\bm a-\bm b$相互垂直;\\
      (2)若$k\bm a+\bm b$与$\bm a-k\bm b$大小相等,求$\beta-\alpha$(其中$k$为非零实数).
      \begin{answer}
        (1)略;(2)$\dfrac{\piup}2$
      \end{answer}
    \vspace{4cm}
    \item%《习题化知识清单》P91单元检测16【三点共线,向量共线】
      设两个非零向量$\bm a$与$\bm b$不共线\\
      (1)若$\vv{AB}=\bm a+\bm b$,$\vv{BC}=2\bm a+8\bm b$,$\vv{CD}=3(\bm a-\bm b)$,求证:$A$,$B$,$D$三点共线;\\
      (2)试确定实数$k$,使$k\bm a+\bm b$与$\bm a+k\bm b$共线.
      \begin{answer}
        (1)略;(2)$k=\pm1$
      \end{answer}
    \vspace{5cm}
    \item%《高中数学竞赛培优教程+一试(李名德 主编)》.pdf P118-例5.14【2004年湖北高考题】
      (2004年湖北高考题)在$\mathrm{Rt}\triangle{ABC}$中,已知$BC=a$,若长为$2a$的线段$PQ$以点$A$为中点,问$\vv{PQ}$与$\vv{BC}$的夹角$\theta$取何值时$\vv{BP}\cdot\vv{CQ}$的值最大?并求出这个最大值.\\
      \begin{answer}
        $\theta=0$时,$\vv{BP}\cdot\vv{CQ}$取最大值0.
      \end{answer}
  \end{exercise}
\stopexercise
\newpage
\section{参考答案}
\begin{multicols}{2}
  \printanswer
\end{multicols}

\Topic{}
  \Teach{}
  \Grade{高一}
  % \Name{郑皓天}\FirstTime{20181207}\CurrentTime{20181207}
  % \Name{林叶}\FirstTime{20180908}\CurrentTime{20181125}
  %\Name{1v2}\FirstTime{20181028}\CurrentTime{20181117}
  % \Name{林叶}\FirstTime{20180908}\CurrentTime{20181125}
  % \Name{郭文镔}\FirstTime{20181111}\CurrentTime{20181117}
  % \Name{马灿威}\FirstTime{20181111}\CurrentTime{20181111}
  % \Name{黄亭燏}\FirstTime{20181231}\CurrentTime{20181231}
  % \Name{王睿妍}\FirstTime{20190129}\CurrentTime{}

  \newtheorem*{Theorem}{定理}
  \makefront
\vspace{-1.5em}
\startexercise
% \begin{exercise}{\heiti 课前检测}\\
%   和角公式、差角公式:\\
%   二倍角公式:\\
%   半角公式:\\
%   尺规作图
% \end{exercise}
\section{距离的测量}
  \section{数学起源与区分、数感}
  \section{长度的量化}武器的大小,身材的高矮,距离的远近……如何比较?
    放到一起,固定一端,观察另一端的位置即可比较进行比较;并可由此得到长短的概念。\\
    如果不能放到一起?选定一个参照物,分别比较大小即可。需要比较多个物体时,选定一个固定的共同标准即可,此即为单位长度。再次基础上,两个的单位长度,一半的单位长度都可得到定义。并可据此制作量尺。此选定的共同标准标准应具有恒定性、稳定性。\\
    % 古代长度单位1:(来源网络)
    %   尺:汉 刘向 《说苑·辨物》:“度量衡以黍生之为一分,十分为一寸,十寸为一尺。”
    %   寸:《备急千金要方》卷八十九《针灸方》:“其尺寸之法……仍取病者,男左女右,手中指上第一节为一寸。亦有长短不定者,即取手大拇指第一节,横度为一寸,以意消息,巧拙在人。”
    %   《汉书·律历志上》:“十分为寸,十寸为尺”
    %   《后汉书·律历志》:“(金星)日行一度二十二分”。一度的六十分之一为一分
    %   步长:古文里,迈出一足为跬,迈出两足才是步,古代的跬就是现在的步,古代的1步实际上是现代2步.古代的1步应该是1.3米左右.古制的三百步大约是现在的500米.当时的一步等于五尺,一尺等于22厘米=0.22米.一定要注意是“当时”。一步等于五尺.历代不一,秦代一步为六尺,周代一步为八尺.) 唐太宗李世民把自己的双步,也就是左右脚各走一步,定为长度单位“步”;还规定步的五分之一为1尺,300步为1里。据研究,唐代的一步为1.514米,1唐里折合454.2米。
    %   各个朝代的尺寸代表长度不尽相同。
    %   夏
    %   1尺 = 10寸(1尺=24.9厘米)
    %   商
    %   1尺 = 10寸,1寸 = 10分(1尺=31.1厘米)
    %   周
    %   1丈=10尺, 1尺=10寸, 1寸=10分(1尺=19.9厘米)
    %   秦
    %   1引=10丈, 1丈=10尺, 1尺=10寸, 1寸=10分(1尺=27.7厘米)
    %   汉
    %   1引=10丈, 1丈=10尺, 1尺=10寸, 1寸=10分(1尺=27.7厘米)
    %   新莽;后汉
    %   1引=10丈, 1丈=10尺, 1尺=10寸, 1寸=10分(1尺=23厘米)
    %   三国
    %   1丈=10尺, 1尺=10寸, 1寸=10分(1尺=24.1厘米)
    %   西晋
    %   1丈=10尺, 1尺=10寸, 1寸=10分(1尺=23厘米)
    %   东晋
    %   1丈=10尺, 1尺=10寸, 1寸=10分(1尺=24.5厘米)
    %   十六国
    %   1丈=10尺, 1尺=10寸, 1寸=10分(1尺=24.5厘米)
    %   南北朝
    %   1丈=10尺, 1尺=10寸, 1寸=10分(1尺=24.5厘米;1尺=29.6厘米)
    %   隋
    %   1丈=10尺, 1尺=10寸, 1寸=10分(前期:1尺=29.6厘米;后期:1尺=23.5厘米)
    %   唐;五代
    %   1丈=10尺, 1尺=10寸, 1寸=10分(小尺:1尺=31.1厘米;大尺:1尺=36厘米)
    %   宋
    %   1丈=10尺, 1尺=10寸, 1寸=10分(1尺=30.7厘米)
    %   元
    %   1丈=10尺, 1尺=10寸, 1寸=10分(1尺=30.7厘米)
    %   明
    %   1丈=10尺, 1尺=10寸, 1寸=10分(裁衣尺:1尺=34厘米;量地尺:1尺=32.7厘米;营造尺:1尺=31.1厘米)
    %   清
    %   1丈=10尺, 1尺=10寸, 1寸=10分(裁衣尺:1尺=35.5厘米;量地尺:1尺=34.5厘米;营造尺:1尺=32厘米)
    %   现代
    %   1丈=10尺, 1尺=10寸, 1寸=10分(1尺=33.33厘米)
    %   换成现在长度:关羽身高216.90cm,,张飞192.80cm,刘备168.70cm.
    % 古代长度单位2:(来源网络)
    %       国里的长度及其演变,无论今人关于度量衡的著作和古籍记载,都很少谈及。
    %   著名中国经济史专家梁方仲先生在所著《中国历代度量衡之变迁及其时代特征》中说:“自汉代以后,历代计算长度,都是自尺以上,到丈为止。至清光绪34年(1908年)重定度量衡制时,……才明文规定于尺之外,另立里制。①
    %   所以,中国里作为计算道路等的长度单位在制度上确立得是很晚的。但这决不是说中国古代没有里这种长度的概念,相反这种概念当时应用很多。如《汉书·西域传》所载数十个方国中,大多数方国到长安有多少里的记载:“鄯善国,本名楼兰,……去阳关千六百里,去长安六千一百里。”“且末国,……去长安六千八百二十里”等等。这说明里这种长度概念在当时是常常用到的。虽然如此,正史讲度量衡时却没有里的长度。如《汉书·律历志》谈到长度时说:“度者,分、寸、尺、丈、引也,所以度长短也。……一为一分,十分为寸,十寸为尺,十尺为丈,十丈为引,而五度审矣。”此处讲的五个长度单位,是分、寸、尺、丈、引,就是没有里。这里没有讲里的长度,不是没有里的长度,而是由于种种原因缺少记载造成的。
    %     探讨中国里的长度演变,不仅要探讨里本身的长度,而且还要牵连到其他长度的演变问题,如尺的长度。有时,同一个朝代就有几十个不同长度的尺,与此相应,可以计算出几十种里的长度。但这些里的长度,并不一定是法权认可的现实应用的里的长度。因此,这里仅计算社会公认的里长度的演变,供关注此问题的人们参考。
    %     一、周、秦、两汉里的长度:周代里的长度在计算井田面积时常常提到。《春秋·谷梁传》宣公十五年(前594年)载“古者,三百步一里,名曰井田。井田者,九百亩,公田居一。”周代井田制下,方一里,耕地九百亩,四边都是三百步一里的长度。这一点在古代是公认的。《说文解字段注》引《谷梁传》曰:“古者,三百步为里。”然而,仅仅知道一里三百步,不知一里多少尺还是无法求得一里的具体长度。
    %     《汉书·食货志上》说:“理民之道,地著为本。故必建步立亩,正其经界。六尺为步,步百为亩,亩百为夫,夫三为屋,屋三为升,井方一里,是为九夫。”从这一记载可知,井方一里,为九夫耕种的九百亩耕地,每一边的边长为一里三百步。一步六尺,则一里三百步为l800尺。这里需要说明的是,据《续文献通考》卷108《乐8》载“周以八尺为步”,“秦以六尺为步”,同时又引《律学新说》指出,二者是相等的。所以《汉书·食货志上》在这里是用秦的步尺制度代替了周的步尺制度。由于秦汉尺的长度如商鞅量尺、新莽铜斛尺、后汉建武铜尺都是一尺等于0.231米。②由此可以算出一里等于1800尺为415.8米。现今的市里一里为500米。则知周代一里为今市里的83.16%。由于周代一里三百步的里制到秦汉并没有发生什么变化,所以这一里制可视为周秦汉三代的里制。
    %     ————————
    %     ① 梁方仲:《中国历代户口、田地、田赋统计》,第527页。
    %     ② 梁方仲:《中国历代户口、田地、天赋统计》,第540页:古今尺度的比较表。
    %     二、以商尺(营造尺)计算里的长度是里长度的一次重要演变。商尺,传说为商朝的尺,唐以后历代为工部用的营造尺,也称部尺,俗名叫鲁班尺,也叫大尺。这种尺的一个重要特点是一步为五尺。一尺的长度为秦尺的1.25尺。《续文献通考》卷108《乐8·度量衡》:“商尺者,即今木匠所用曲尺。盖自鲁般传至于唐,唐人谓之大尺。由唐至今用之,名曰今尺,又名营造尺。古所谓车工尺。”由于营造尺是历代工部用的尺度,公信力强,应用广泛。随着社会发展,以营造尺计算里的长度是一种合理的选择。然而这一点来得毕竟太迟了。清光绪34年(1908年)重定度量衡时明确规定里制为:“五尺为一步,二步为一丈,十丈为一引,十八引为一里。”在“新制说略”中指出:“长短度分为两种:一曰尺度,以尺为单位,所以度寻之长短也。一曰里制,以一千八百尺为一里,用以计道路之长短也。里制即积尺制而成。盖道里甚长,若仅以尺计,则诸多不便,故必别为里制。”①这里把尺制、里制作为基本长度单位列出,在当时是有新意的。
    %     据上述清光绪末年所立里制可知:一里为营造尺1800尺。营造尺一尺等于0.32米,所以1800尺,等于576米。因今市里一里为500米,所以以营造尺计里则一里为今市里的115.2%。
    %     三、第三次中国里制的变化,发生在民国时期,“公元1929年制定一市里为150丈,合公制为500米。这次制定的里制一直沿用至今,既继承了中国传统里制的特点,又吸收了西方米制,并与其结合。所谓继承中国里制特点表现在:从周代开始中国传统里制为一里300步,这次里制为一里150丈,而以营造尺的五尺为步、二步为一丈,则150丈恰为三百步。所谓吸收西方米制与之结合表现在:“一市里为150丈合公制为500米”,则是以西方的米制表示中国的里制,而中国的市尺则变为西方一米的三分之一。这样二者融为一体。由于这次制定的里制,适合了社会发展的需要,方便了与西方长度单位的换算,所以沿用至今。
    %     上述三种里制,是中国历史上里长度变化的三次演变,今列表如下:
    %     中国历史上里长度的三次演变
    %     从上述三种里长度的演变来看,里的长度演变是很慢的。如与尺的演变相比较,这点尤为明显。从周代开始历代都有长度不同的尺存在,其发展的趋势是尺越来越长,这明显是由于统治者为多剥削民众造成的。里所以演变得慢,可能是由于里的长度与人们的利害关系牵连少的缘故。
    % 古代长度单位3:秦度量衡(来源网络)
    %   我国陕西、山西、江苏、山东、辽宁、河北、甘肃等地都发现了秦权,上面都刻有秦始皇的诏文,有的还加刻了秦二世的诏文,毫无疑问,它们都是秦政府颁发的标准量器。“权”即是今天所称的“砝码”,在有刻度的等臂衡杆上,利用杠杆原理测重。根据秦权上自铭所示量值实测折算,秦一斤应为 250 克。秦的量器也已发现多件,经实测一升为 200 毫升。至今尚未发现秦尺,但可以通过商鞅铜方升计算出秦的度值。商鞅方升是秦孝公十八年(前 344)颁发的标准量器,秦统一后,在商鞅方升底部加刻了秦始皇二十六年诏书,继续做标准器使用。商鞅方升的边上有一段铭文“爰积十六尊(寸)五分尊(寸)壹为升”,即以 161/5立方寸的容积为一升,近年来经反复实测,得出此升容积为 202.15 立方厘米,将 161/5立方寸和 202.15 立方厘米进行换算,得出 1 立方寸=12.478 立方厘米,进而算出 1 尺=23.2 厘米,这个数值,既是商鞅时的度值,也是秦统一后的度值。由于度量衡在使用中受到磨损,产生偏差,为此秦明令规定,每年都要对度量衡进行检验,校正。
    \section{长度的测量}
      测量长度、距离时,一般指的是两点间的直线距离,也即连接两点的直线段长度.(更准确说法是:测量测地线的长度)
      两点#$A$、$B$间距离的测量
      \begin{description}
        \item[直接用量尺测量]
          将量尺一刻度线对准一点,记下刻度值;与此同时,量尺记下另一点对准的刻度值;两刻度值的差即为两点的距离。
        \item[多次测量累计]
          量尺量程不足时,从一点出发,沿着两点连线方向,逐次测量累加,直到令一点为止;两点间可用直杆或绳相连时,将更简单些.
        \item[应用正弦定理]选定参考点,测量参考点与其中一点距离及这三点间夹角;使用余弦定理。
        \item[应用余弦定理]选定参考点,当两点相互看不见故而测不了与参考点夹角时。测量参考点与两点间距离及张角。使用余弦定理。
        \item[多次选取参考点]当无法直测量参考点与两点间距离时.选定第二参考点,则只要知道两参考点距离与待测位置间夹角即可运用正弦定理算出参考点与待测位置距离。情况复杂时,可能需要选定更多的参考点以得出参考点间的距离.
        \item[作出地图后得到两点的坐标位置]事先选定多个参考点作为地标,画出地图,则将大大降低工作量.选取一个固定方向为标准方向,比如某两地标的连线,比如正北方向.
      \end{description}


      \begin{exercise}
        \item
      \end{exercise}

\newpage
\section{课后作业}
  \begin{exercise}

  \end{exercise}
\stopexercise

\newpage
\section{参考答案}
\begin{multicols}{2}
  \printanswer
\end{multicols}

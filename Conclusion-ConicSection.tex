\Topic{等比数列与其前n项和}
  \Teach{}
  \Grade{高一}
  % \Name{郑皓天}\FirstTime{20181207}\CurrentTime{20181207}
  % \Name{林叶}\FirstTime{20180908}\CurrentTime{20181125}
  %\Name{1v2}\FirstTime{20181028}\CurrentTime{20181117}
  % \Name{林叶}\FirstTime{20180908}\CurrentTime{20181125}
  % \Name{郭文镔}\FirstTime{20181111}\CurrentTime{20181117}
  % \Name{马灿威}\FirstTime{20181111}\CurrentTime{20181111}
  % \Name{黄亭燏}\FirstTime{20181231}\CurrentTime{20181231}
  % \Name{王睿妍}\FirstTime{20190129}\CurrentTime{}

  \newtheorem*{Theorem}{定理}
  \makefront
\vspace{-1.5em}

\section{圆锥曲线结论}
\[\frac{\sin A}{\sin B}+\frac{\sin B}{\sin A} \leqslant \frac{A}{B}+\frac{B}{A}
    \]
$f^{\partial \prime}(x)f'(x)$
$\lim_{n\to +\infty}\frac{\ln{n!}}{n^2}$
$\frac{\ln{2}+\ln{3}+\ldots+\ln{(n-1)}+\ln{n}}{n^2}$

    \subsection{圆锥曲线统一结论}
        \begin{enumerate}
            \item 定义:到定点$F$距离与到定直线$l$的距离之比为定值$e(e>0)$的点的集合.其中定点$F$称为焦点,定直线$l$称为准线.定值$e$称为离心率.
            \item 圆锥曲线为轴对称图形,其中一条对称轴过焦点.
            \item 焦准距$p$:焦点$F$到准线$l$的距离.
            \item 弦:以圆锥曲线上(相异)两点为两端点的线段.
            \item 焦点弦:过焦点的弦.
            \item 焦半径:以焦点与圆锥曲线上一点为两端点的线段.
            \item 通径:过焦点且垂直于焦点所在对称轴的弦.
        \end{enumerate}

    \subsection{直线与圆锥曲线联立}
    联立方程:
        \begin{equation}
            \left\{\begin{aligned}
                Ax+By+C=0,\\
                \dfrac{x^2}{m}+\dfrac{y^2}{n}=1.
            \end{aligned}\right.
        \end{equation}\label{eq:eqs}
    整理并化简为$x$的二次方程,得:
        \begin{equation}
            \Bigl(\mfrac{B^2}{m}+\mfrac{A^2}{n}\Bigr)x^2+\mfrac{2AC}{n}x+\Bigl(\mfrac{C^2}{n}-B^2\Bigr)=0
        \end{equation}\label{eq:eq0}
    于是判别式:
        \begin{equation}
            \Delta=\Bigl(\mfrac{2B}{mn}\Bigr)^2\cdot mn\bigl(A^2m+B^2n-C^2\bigr)
        \end{equation}\label{eq:Delta}
    于是得到:
        \begin{equation}
            \Delta>0 \Longleftrightarrow mn\bigl(A^2m+B^2n-C^2\bigr)>0
        \end{equation}\label{eq:Delta}


        进行参数代换$A\leftrightarrow B$、$m\leftrightarrow n$,以上一系列结论即可转为关于$y$的二次方程结论;






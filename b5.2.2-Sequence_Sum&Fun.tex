\Topic{}
  \Teach{}
  \Grade{高一}
  % \Name{郑皓天}\FirstTime{20181207}\CurrentTime{20181207}
  % \Name{林叶}\FirstTime{20180908}\CurrentTime{20181125}
  %\Name{1v2}\FirstTime{20181028}\CurrentTime{20181117}
  % \Name{林叶}\FirstTime{20180908}\CurrentTime{20181125}
  % \Name{郭文镔}\FirstTime{20181111}\CurrentTime{20181117}
  % \Name{马灿威}\FirstTime{20181111}\CurrentTime{20181111}
  % \Name{黄亭燏}\FirstTime{20181231}\CurrentTime{20181231}
  % \Name{王睿妍}\FirstTime{20190129}\CurrentTime{}

  \newtheorem*{Theorem}{定理}
  \makefront
\vspace{-1.5em}
\startexercise
\begin{exercise}{\heiti 课前检测}\\

\end{exercise}

\section{数列求和相关问题}
  \subsection{求前$ n $项和的方法}
  \begin{enumerate}
  \item 公式法\begin{enumerate}
  \item 等差数列的前$ n $项和公式:$S_n=\dfrac{n(a_1+a_n)}{2}=na_1+\dfrac{n(n-1)}{2}d$.
  \item 等比数列的前$ n $项和公式:$S_n=\Bigg\{\begin{aligned}
  &na_1&\left(q=1\right)\\
  &\dfrac{a_1\left(1-q^n\right)}{1-q}&\left(q\ne1\right)
  \end{aligned}$
  \end{enumerate}
  \item 分组求和:把一个数列分成几个可以直接求和的数列;
  \item 拆项相消:有时把一个数列的通项公式分成两项差的形式,相加过程中消去中间项,只剩下有限项再求和;
  \item 错位相减:适用于一个等差数列和一个等比数列对应项相乘构成的数列求和;
  \item 倒序相加:把数列正着写和倒着写再相加,例如等差数列前$ n $项和公式的推导方法.
  \end{enumerate}
  \subsection{裂项相消法}
  \begin{enumerate}
  \item 对于裂项后明显有能够相消的项的一类数列,在求和时常用“裂项相消法”,分式数列的求和多用此法;
  \item 利用裂项相消法求和时,应注意抵消后并不一定只剩下第一项和最后一项,也可能有前面两相和最后两项,有些情况下,裂项时需要调整前面的系数,使裂开的两项之差和系数之积与原通项相等.
  \item 常用的拆项公式:\begin{enumerate}
  \item $ \dfrac{1}{n(n+1)}=\dfrac{1}{n}-\dfrac{1}{n+1}; $
  \item $ \dfrac{1}{n(n+d)}=\dfrac{1}{d}\left(\dfrac{1}{n}-\dfrac{1}{n+d}\right) $;
  \item $\dfrac{1}{\sqrt{n}+\sqrt{n+1}}=\sqrt{n+1}-\sqrt{n}$;
  \item $\dfrac{1}{n(n+1)(n+2)}=\dfrac{1}{2}\left[\dfrac{1}{n(n+1)}-\dfrac{1}{(n+1)(n+2)}\right]$
  \item 若数列$\{a_n\}$为等差数列,公差为$ d (d\ne0)$,则$ \dfrac{1}{a_n\bm{\cdot}a_{n+1}}=\dfrac{1}{d}\left(\dfrac{1}{a_n}-\dfrac{1}{a_{n+1}}\right). $
  \end{enumerate}
  \end{enumerate}
  \begin{proof}
  对于分式数列,通常会考虑裂项相消法进行消项,对于$ \dfrac{1}{n(n+d)} $式数列,可以使用待定系数法得到展开式,
  假设:$$ \dfrac{1}{n(n+d)}=\dfrac{k}{n}-\dfrac{k}{n+d} ~(k\text{为待定系数})$$
  右边通分有$$\dfrac{1}{n(n+d)}= \dfrac{k}{n}-\dfrac{k}{n+d}=\dfrac{kd}{n(n+d)} $$
  即有$ kd=1 $,算得$ k=\dfrac{1}{d} $.即得证
  \end{proof}
  \subsection{错位相减法}
  \begin{enumerate}
  \item 一般地,如果数列$\{a_n\}$是等差数列,数列$\{b_n\}$是等比数列,求数列$ \left\{a_n\bm{\cdot}b_n\right\} $的前$ n $项和时,可以采用错位相减法.
  \item 应用等比数列求和公式时,必须注意公比$ q\ne1 $这一前提条件,如果不能确定公比$ q $是否为$ 1 $,应分两种情况进行讨论.
  \end{enumerate}
  \begin{proof}
  设数列$ \left\{a_n\right\} $为等差数列,公差为$ d $,数列$ \left\{b_n\right\} $为等比数列,公比为$ q ~(q\ne1)$,数列$ \left\{c_n\right\}$满足$ c_n=a_n\bm{\cdot}b_n  $,则数列$ \left\{c_n\right\} $有:\begin{equation*}
  \begin{aligned}
  S_n=&c_1+c_2+c_3+\cdots+c_n\\
  =&a_1b_1+a_2b_2+a_3b_3+\cdots+a_nb_n\\
  qS_n=&a_1b_1q+a_2b_2q+\cdots+a_nb_nq\\
  =&a_1b_2+a_2b_3+a_3b_4+\cdots+a_nb_{n+1}\\
  S_n-qS_n=&a_1b_1+b_2\left(a_2-a_1\right)+b_3\left(a_3-a_2\right)+\cdots++b_n\left(a_n-a_{n-1}\right)-a_nb_{n+1}\\
  =&a_1b_1+db_2+db_3+\cdots+db_n-a_nb_{n+1}\\
  =&a_1b_1+d\left(b_2+b_3+\cdots+b_n\right)-a_nb_{n+1}\\
  =&a_1b_1-a_nb_{n+1}+\dfrac{b_2\left(1-q^{n-1}\right)}{1-q}d.
  \end{aligned}
  \end{equation*}
  故而有:$$S_n=\dfrac{a_1b_1-a_nb_{n+1}+\dfrac{b_2\left(1-q^{n-1}\right)}{1-q}d}{1-q}$$
  \end{proof}
\section{数列通项求法}
  \subsubsection{数列的前$ n $项和与通项公式的关系}
  \begin{enumerate}[1)]
  \item $ S_n=a_1+a_2+a_3+\cdots+a_n $;
  \item $ a_n=\Bigg\{\begin{aligned}
  &S_1&\left(n=1\right)\\
  &S_n-S_{n-1}&\left(n\ge2\right)
  \end{aligned} $
  \end{enumerate}
  注意:一定要验证$ n=1 $的情况.
  \subsubsection{利用递推关系通项}
  已知数列$\{a_n\}$的递推关系求通项时,通常用累加法、累乘法和构造法求解.
  \begin{enumerate}
  \item 形如$ a_n=a_{n-1}+m~ (n\ge 2,n\in\mathbf{N^*})$时,构造等差数列求解,形如$ a_n=xa_{n-1}+y~(n\ge2,n\in\mathbf{N^*}) $时,构造等比数列求解;
  \item 形如$ a_n=a_{n-1} +f(n)~(n\ge 2,n\in\mathbf{N^*})$时,用累加法;
  \item 形如$ \dfrac{a_n}{a_{n-1}}=f(n) ~(n\ge 2,n\in\mathbf{N^*})$时,用累乘法求解.
  \end{enumerate}
\newpage
\section{课后作业}
  \begin{exercise}

  \end{exercise}
\stopexercise

\newpage
\section{参考答案}
\begin{multicols}{2}
  \printanswer
\end{multicols}

\section{恒成立与存在性命题总结}
“对任意的$x_1\in D_1$,存在$x_2\in D_2$,使得$f(x_1)<g(x_2)$”,\\
记$f(x)$在$x\in D_1$上的值域为$F$,$g(x)$在$x\in D_2$上的值域为$G$,则以上命题可以简化理解为:\\
“对任意的$f(x_1)\in F$,存在$g(x_2)\in G$,使得$f(x_1)<g(x_2)$”\\
或者更简单地:“对于任意的$f(x_1)$,存在$g(x_2)$,使得$f(x_1)<g(x_2)$”,\\
更易于理解地:“对于任意的某个$f(x)$值,存在大于它的$g(x)$值”。
\newline
记$F$的最左端点与最右端点分别为$f_m$、$f_M$,$G$的最左端点与最右端点分别为$g_m$、$g_M$.\\
\newline
“对任意的$x_1\in D_1$,任取$x_2\in D_2$,均有$f(x_1)=g(x_2)$”,则$f(x)$与$g(x)$为值域相同的两常数函数.\\%%%%%%%%%%%%%%%
“对任意的$x_1\in D_1$,任取$x_2\in D_2$,均有$f(x_1)=g(x_2)$”,则$f(x)$与$g(x)$为值域相同的两常数函数.\\%%%%%%%%%%
“对于任意的$x_1\in D_1$,任取$x_2\in D_2$,均有$f(x_1)=g(x_2)$”,则$f(x)$与$g(x)$为值域相同的两常数函数.\\%%%%%%%%%%
“对任意的$x_1\in D_1$,任取$x_2\in D_2$,均有$f(x_1)<g(x_2)$”,则:
\begin{itemize}
    \item 当$f_M$处与$g_m$处都为实心端点时,有$g_m>f_M$;
    \item 否则,有$g_m\geqslant f_M$;
\end{itemize}
“对任意的$x_1\in D_1$,任取$x_2\in D_2$,均有$f(x_1)\leqslant g(x_2)$”,则有:$g_m\geqslant f_M$;\\
\newline
“对任意的$x_1\in D_1$,存在$x_2\in D_2$,使得$f(x_1)=g(x_2)$”,则有$F\subseteq G$.\\%%%%%%%%%%%%%%
“对任意的$x_1\in D_1$,存在$x_2\in D_2$,使得$f(x_1)=g(x_2)$”,则有$F\subseteq G$.\\
“对任意的$x_1\in D_1$,存在$x_2\in D_2$,使得$f(x_1)<g(x_2)$”,则:
\begin{itemize}
    \item 当$f_M$处为实心端点时,有$g_M>f_M$;
    \item 当$f_M$处为空心端点时,有$g_M\geqslant f_M$;
\end{itemize}
“对任意的$x_1\in D_1$,存在$x_2\in D_2$,使得$f(x_1) \leqslant g(x_2)$”,则:
\begin{itemize}
    \item 当$f_M$处为实心端点,且$g_M$处为空心端点时,有$g_M>f_M$;
    \item 否则,有$g_M\geqslant f_M$;
\end{itemize}
“存在$x_2\in D_2$,使得对任意的$x_1\in D_1$,均有$f(x_1) \leqslant g(x_2)$”,则:
\begin{itemize}
    \item 当$f_M$处为实心端点,且$g_M$处为空心端点时,有$g_M>f_M$;
    \item 否则,有$g_M\geqslant f_M$;
\end{itemize}
\newline
“对任意的$x_1\in D_1$,任取$x_2\in D_2$,均有$f(x_1)=g(x_2)$”,则$f(x)$与$g(x)$为值域相同的两常数函数.\\%%%%%%%%%%%%%%%
“对任意的$x_1\in D_1$,任取$x_2\in D_2$,均有$f(x_1)=g(x_2)$”,则$f(x)$与$g(x)$为值域相同的两常数函数.\\%%%%%%%%%%
“存在$x_1\in D_1$,使得对任意的$x_2\in D_2$,均有$f(x_1)=g(x_2)$”,则有:$g(x)$为常数函数,且$G\subseteq F$.\\
“存在$x_1\in D_1$,使得对任意的$x_2\in D_2$,均有$f(x_1)<g(x_2)$”,则:
\begin{itemize}
    \item 当$g_m$处为空心端点,且$f_m$处为实心端点时,有$f_m\leqslant g_m$;
    \item 否则,有$f_m<g_m$;
\end{itemize}
“存在$x_1\in D_1$,使得对任意的$x_2\in D_2$,均有$f(x_1)\leqslant g(x_2)$”,则:
\begin{itemize}
    \item 当$f_m$处为实心端点时,有$f_m\leqslant g_m$;
    \item 否则,有$f_m<g_m$;
\end{itemize}
\newline
“存在$x_1\in D_1$,$x_2\in D_2$,使得$f(x_1)=g(x_2)$”,则有$G\cap F\neq \varnothing$.\\
“存在$x_1\in D_1$,$x_2\in D_2$,使得$f(x_1)<g(x_2)$”,则有$f_m<g_M$;\\
“存在$x_1\in D_1$,$x_2\in D_2$,使得$f(x_1) \leqslant g(x_2)$”,则:
\begin{itemize}
    \item 当$g_M$处为实心端点,且$f_m$处为实心端点时,有$g_M \geqslant f_m$;
    \item 否则($g_M$处与$g(1)$处至少有一个为空心端点时),有$g_M>f_m$;
\end{itemize}有$f_m\leqslant g_m$;








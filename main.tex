\documentclass[12pt,UTF8]{ctexart}
\usepackage{QingDa}
\usepackage{multirow}
\usepackage[demo]{graphicx}
\usepackage{caption}
\usepackage{subcaption}
%\usepackage{subfig}
\usepackage[hypcap=true,labelsep=period,font=small]{caption}% 图的标题设置Fig.
\usepackage[hypcap=true]{subcaption}%用于画子图 可以适配hyperref包
\usepackage{float}
\usepackage[colorlinks,dvipdfm,
            pdfpagemode=UseOutlines,
            pdfstartview=FitH,
            anchorcolor=green,
            citecolor=blue,
            linkcolor=red,
            hyperindex=true,
            pagebackref,
            bookmarksnumbered,
            bookmarksopenlevel=2,
            colorlinks]{hyperref}
\pgfplotsset{width=6cm,compat=1.15}

\newcommand\putfig[2]{\begin{tabular}[t]{@{}l@{}}#1\\#2\end{tabular}}
\begin{document}
%\title{2012年某某中学高三数学测试题}
%\author{总分:150分and 时间120分钟}
%\date{命题人: 某某某}
%\maketitle
\begin{table}[h]
\small
\arrayrulewidth=1pt
    \centering
      \begin{tabular}{|m{2.25cm}<{\centering}|p{2.25cm}<{\centering}|p{2.25cm}<{\centering}|p{2.25cm}<{\centering}|p{1.5cm}<{\centering}|p{1.5cm}<{\centering}|p{1.5cm}<{\centering}|p{1.5cm}<{\centering}|}
    \hline
    {\hei 教师姓名}  & 沈炜炜&{\hei 学生姓名}&1v2  & {\hei \wuhao 首课时间}&181028&{\hei \wuhao 本课时间}  & 181103  \\ \hline
   {\hei 学习科目}  & 数学&{\hei 上课年级}&高一 & \multicolumn{2}{c|}{\hei 教材版本}&\multicolumn{2}{c|}{人教A版}  \\ \hline
   {\hei 课题名称} & \multicolumn{7}{l|}{必修一综合专题复习}\\\hline
   {\hei 重点难点}&\multicolumn{7}{l|}{分类讨论;}\\\hline
  \end{tabular}
\end{table}

%--------------------- 
\vspace{-1.7em}
%在一般我们写书籍的时候,每本书后面的习题就不太好用了,下面的处理方式选自ctex上的milksea帖子,对于大家写作书籍有很大帮助
\documentclass[openany]{ctexbook}
\usepackage{latexexercise}
\begin{document}
\startexercise
\chapter{例子}
\begin{exercise}
\item 题目
\begin{answer}
这有答案。
\end{answer}
\item 又一道
\begin{answer}
这还有答案。
\end{answer}
\end{exercise}
%\stopexercise
%\startexercise
\begin{exercise}
  \item 是的
  \begin{answer}
这有答案。
\end{answer}
  \item 哦也
  \begin{answer}
这有答案。
\end{answer}
\end{exercise}
\stopexercise

\chapter{答案}
\printanswer
\end{document} 
\vspace{1em}

\vspace{2em}

\vspace{2em}

\end{document}
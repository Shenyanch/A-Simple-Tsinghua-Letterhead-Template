\documentclass[12pt,UTF8]{ctexart}
\usepackage{QingDa}
\usepackage{multirow}
\begin{document}
%\title{2012年某某中学高三数学测试题}
%\author{总分:150分and 时间120分钟}
%\date{命题人: 某某某}
%\maketitle
\begin{table}[h]
\small
\arrayrulewidth=1pt
    \centering
      \begin{tabular}{|m{2.25cm}<{\centering}|p{2.25cm}<{\centering}|p{2.25cm}<{\centering}|p{2.25cm}<{\centering}|p{1.5cm}<{\centering}|p{1.5cm}<{\centering}|p{1.5cm}<{\centering}|p{1.5cm}<{\centering}|}
    \hline
    {\hei 教师姓名}  & 沈炜炜&{\hei 学生姓名}&1v3  & {\hei \wuhao 首课时间}&181028&{\hei \wuhao 本课时间}  & 181103  \\ \hline
   {\hei 学习科目}  & 数学&{\hei 上课年级}&高一 & \multicolumn{2}{c|}{\hei 教材版本}&\multicolumn{2}{c|}{人教A版}  \\ \hline
   {\hei 课题名称} & \multicolumn{7}{l|}{必修一复习}\\\hline
   {\hei 重点难点}&\multicolumn{7}{l|}{参数问题;函数图像}\\\hline
  \end{tabular}
\end{table}

%--------------------- 

%\setlength{\parskip} {5em}
%\setlength\parindent{2em}
\part{部分标题}
%\chapter{章标题}这一章我们介绍这些内容。
互惠 呼呼 u呼呼鬍惠普ih共和国
\section{节标题}
这是一个CTEX的utf-8编码例子,{\kai 这里是楷体显示},{\songti 这里是\wuhao 宋体显示},{\heiti 这里是黑体显示},{\fangsong 这里是仿宋显示},{\lishu 这里是隶书显示},{\youyuan 这里是幼圆显示}。
\item (17-18福州八中高一期中4)函数$f(x)=a^{-x^2+3x+2}(0<a<1)$的单调递增区间是\xz
        \xx{$(-\infty,\frac32)$}
        {$(\frac32,+\infty)$}
        {$(-\infty,-\frac32)$}
        {$(\frac32,+\infty)$}\\
这是一个CTEX的utf-8编码例子,{\kai 这里是楷体显示},{\songti 这里是宋体显示},{\heiti 这里是黑体显示},{\fangsong 这里是仿宋显示},{\lishu 这里是隶书显示},{\youyuan 这里是幼圆显示}。
\paragraph{这一段}
$X+Y\over c^t$\\
$\frac{X+Y}{c^t}$
\startexercise
\begin{exercise}
\item (17-18福州八中高一期中4)函数$f(x)=a^{-x^2+3x+2}(0<a<1)$的单调递增区间是\xz
        \xx{$(-\infty,\frac32)$}
        {$(\frac32,+\infty)$}
        {$(-\infty,-\frac32)$}
        {$(\frac32,+\infty)$}
\begin{answer}
不行吗。
\end{answer}
\item  (17-18福州八中高一期中5) 下列四组函数中,表示同一函数的是\xz
        \xx{$f(x)=\lg{x^4},g(x)=4\lg x$}
        {$f(x)=\sqrt{x+1}\cdot\sqrt{x-1},g(x)=\sqrt{x^2-1}$}
        {$f(x)=\frac{x^2-1}{x-1},g(x)=x+1$}
        {$f(x)=\begin{cases}x,x\geq0\\-x,x<0\end{cases},g(x)=\sqrt{x^2}$}
\begin{answer}
D
\end{answer}
\end{exercise}
\subsection{小节标题}这一小节我们介绍这些内容。\\
这一小节我们介绍这些内容。
\subsection{小节标题}这一小节我们介绍这些内容。\\
这一小节我们介绍这些内容。
\subsection{小节标题}这一小节我们介绍这些内容。
\subsection{小节标题}这一小节我们介绍这些内容。
\subsection{小节标题}这一小节我们介绍这些内容。
\part{部分标题}
\begin{exercise}
\item (17-18福州八中高一期中4)函数$f(x)=a^{-x^2+3x+2}(0<a<1)$的单调递增区间是\xz
        \xx{$(-\infty,\frac32)$}
        {$(\frac32,+\infty)$}
        {$(-\infty,-\frac32)$}
        {$(\frac32,+\infty)$}
\begin{answer}
不行吗。
\end{answer}
\item  (17-18福州八中高一期中5) 下列四组函数中,表示同一函数的是\xz
        \xx{$f(x)=\lg{x^4},g(x)=4\lg x$}
        {$f(x)=\sqrt{x+1}\cdot\sqrt{x-1},g(x)=\sqrt{x^2-1}$}
        {$f(x)=\frac{x^2-1}{x-1},g(x)=x+1$}
        {$f(x)=\begin{cases}x,x\geq0\\-x,x<0\end{cases},g(x)=\sqrt{x^2}$}
\begin{answer}
D
\end{answer}
\end{exercise}
\subsubsection{子节标题}这一子节我们介绍这些内容。\\这一子节我们介绍这些内容。
\subsubsection{子节标题}这一子节我们介绍这些内容。\\这一子节我们介绍这些内容。\\
\subsubsection{子节标题}这一子节我们介绍这些内容。\\这一子节我们介绍这些内容。\\
\begin{exercise}
\item (17-18福州八中高一期中4)函数$f(x)=a^{-x^2+3x+2}(0<a<1)$的单调递增区间是\xz
        \xx{$(-\infty,\frac32)$}
        {$(\frac32,+\infty)$}
        {$(-\infty,-\frac32)$}
        {$(\frac32,+\infty)$}
\begin{answer}
不行吗。
\end{answer}
\item  (17-18福州八中高一期中5) 下列四组函数中,表示同一函数的是\xz
        \xx{$f(x)=\lg{x^4},g(x)=4\lg x$}
        {$f(x)=\sqrt{x+1}\cdot\sqrt{x-1},g(x)=\sqrt{x^2-1}$}
        {$f(x)=\frac{x^2-1}{x-1},g(x)=x+1$}
        {$f(x)=\begin{cases}x,x\geq0\\-x,x<0\end{cases},g(x)=\sqrt{x^2}$}
\begin{answer}
D
\end{answer}
\end{exercise}
\section{节标题}这一节我们介绍这些内容。
\begin{exercise}
\item (17-18福州八中高一期中4)函数$f(x)=a^{-x^2+3x+2}(0<a<1)$的单调递增区间是\xz
        \xx{$(-\infty,\frac32)$}
        {$(\frac32,+\infty)$}
        {$(-\infty,-\frac32)$}
        {$(\frac32,+\infty)$}
\begin{answer}
不行吗。
\end{answer}
\item  (17-18福州八中高一期中5) 下列四组函数中,表示同一函数的是\xz
        \xx{$f(x)=\lg{x^4},g(x)=4\lg x$}
        {$f(x)=\sqrt{x+1}\cdot\sqrt{x-1},g(x)=\sqrt{x^2-1}$}
        {$f(x)=\frac{x^2-1}{x-1},g(x)=x+1$}
        {$f(x)=\begin{cases}x,x\geq0\\-x,x<0\end{cases},g(x)=\sqrt{x^2}$}
\begin{answer}
D
\end{answer}
\end{exercise}
\paragraph{段标题}这一段我们介绍这些内容。
\begin{exercise}
\item (17-18福州八中高一期中4)函数$f(x)=a^{-x^2+3x+2}(0<a<1)$的单调递增区间是\xz
        \xx{$(-\infty,\frac32)$}
        {$(\frac32,+\infty)$}
        {$(-\infty,-\frac32)$}
        {$(\frac32,+\infty)$}
\begin{answer}
不行吗。
\end{answer}
\item  (17-18福州八中高一期中5) 下列四组函数中,表示同一函数的是\xz
        \xx{$f(x)=\lg{x^4},g(x)=4\lg x$}
        {$f(x)=\sqrt{x+1}\cdot\sqrt{x-1},g(x)=\sqrt{x^2-1}$}
        {$f(x)=\frac{x^2-1}{x-1},g(x)=x+1$}
        {$f(x)=\begin{cases}x,x\geq0\\-x,x<0\end{cases},g(x)=\sqrt{x^2}$}
\begin{answer}
D
\end{answer}
\end{exercise}
\startexercise

%在一般我们写书籍的时候,每本书后面的习题就不太好用了,下面的处理方式选自ctex上的milksea帖子,对于大家写作书籍有很大帮助
\documentclass[openany]{ctexbook}
\usepackage{latexexercise}
\begin{document}
\startexercise
\chapter{例子}
\begin{exercise}
\item 题目
\begin{answer}
这有答案。
\end{answer}
\item 又一道
\begin{answer}
这还有答案。
\end{answer}
\end{exercise}
%\stopexercise
%\startexercise
\begin{exercise}
  \item 是的
  \begin{answer}
这有答案。
\end{answer}
  \item 哦也
  \begin{answer}
这有答案。
\end{answer}
\end{exercise}
\stopexercise

\chapter{答案}
\printanswer
\end{document} 
\vspace{1em}

\vspace{2em}

\vspace{2em}

\end{document}
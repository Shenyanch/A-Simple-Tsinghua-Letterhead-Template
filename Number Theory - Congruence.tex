\Topic{数学拓展数论专题 - 莫比乌斯函数}
  \Teach{}
  \Grade{高二}
  \Name{张一男}

  \newtheorem*{Theorem}{定理}
  \makefront
\vspace{-1.5em}

\subsection{乘性函数}
    算术函数:定义在所有正整数上的函数.\\
    算术函数$f$如果满足对任意两个互素的正整数$n$ 和$m$,均有$f(mn)=f(m)f(n)$,称为乘性函数(或积性函数).
    如果对任意两个正整数$n$和$m$,均有$f(mn)=f(m)f(n)$,就称为完全乘性(或完全积性)函数.

    【性质1】如果$f$是一个乘性函数,对任意正整数$n$有素数幂分解$n=p_1^{\alpha_1}p_2^{\alpha_2}\ldots p_s^{\alpha_s}$,
    那么\[f(n)=f(p_1^{\alpha_1})f(p_2^{\alpha_2})\ldots f(p_s^{\alpha_s})\].
\subsection{和函数}
    设$f$是一个算术函数,那么
    \[F(n) = \sum_{d\mid n}f(d)\]
    代表$f$在$n$中所有正因子处的值之和.函数$F$称为$f$的和函数.

    如果$f$是乘性函数,那么$f$的和函数,即$F(n) = \sum \limits_{d\mid n} f(d)$也是乘性函数.

    为了证明$F$是一个乘性函数,我们必须证明:如果$m$和$n$是互素的正整数,那么$F(mn)=F(m)F(n)$.
    所以首先假设$(m,n)=1$.
    有\[F(mn) = \sum_{d\mid mn}f(d).\]
        因为$(m,n)=1$,每个$mn$的因子可以唯一地写成$m$的因子$d_1$和$n$的因子$d_2$之
        积,并且这两个因子互素.即$d=d_1d_2$,所以有
        \[F(mn) = \sum_{\substack{d_1\mid m \\d_2\mid n}}f(d_1d_2)\]
        因为$f$是乘性的,且$(d_1,d_2)=1$,则
        \[
            \begin{aligned} 
                F(mn) &= \sum_{\substack{d_1\mid m \\d_2\mid n}}f(d_1)f(d_2)\\
                &=\sum_{d_1\mid m }f(d_1)\sum_{d_2\mid n }f(d_2)\\
                &=F(m)F(n)
            \end{aligned}
        \]

        【推论】因子和函数$\sigma$与因子个数函数$\tau$是乘性函数.\\
        设$f(n)=n$和$g(n)=1$,则$f$和$g$均是乘性的.
        于是$\sigma(n)=\sum \limits_{d\mid n} f(d)$和$\tau(n)=\sum \limits_{d\mid n} g(d)$是乘性的.
  
\clearpage
    【欧拉函数的和函数】设$n$为一个正整数,那么\[F(n) = \sum_{d\mid n}\phi(d)=n\]

    我们将从1到$n$的整数构成的集合分类.整数$m$如果与$n$的最大公因子为$d$,则$m$
    属于$C_d$类. 就是说,如果$m$属于$C_d$,那么$(m,n)=d$,当且仅当$(m/d,n/d)=1$.
    所以,$C_d$类中所含整数的个数是所有不超过$n/d$且和$n/d$互素的正整数的个数. 从上面的分析,我们可
    以看到$C_d$类中存在中$(n/d)$个整数.因为我们将1到$n$的所有整数分成互不相交的类,且每个
    整数只厲于其中一个类. 那么这些不同的类所含的所有整数的个数之和就是$n$,所以
    \[n = \sum_{d\mid n}\phi(n/d)\].
    因为$d$取遍所有整除$n$的正整数,$n/d$也取遍它的所有因子,从而
    \[n = \sum_{d\mid n}\phi(n/d)= \sum_{d\mid n}\phi(d)\].

    \subsection{莫比乌斯反演}
        莫比乌斯函数$\mu(n)$定义为
        \[\mu(n)=
        \left\{\begin{aligned}
            &1,&& n=1; \\
            &(-1)^s, &&n=p_1p_2\ldots p_s,\text{素数 }p_1<p_2<\ldots <p_s;\\
            &0, && \text{others}.
        \end{aligned}\right.
        \]\\
\vspace{2em}
        【性质1】莫比乌斯函数$\mu(n)$是乘性函数.\\
\vspace{2em}
        【性质2】莫比乌斯函数的和函数在$n$处的值$F(n)=\sum \limits_{d\mid n} \mu(d)$满足
        \[
            \sum \limits_{d\mid n} \mu(d)=
            \left\{\begin{aligned}
                &1,&& \text{if}\quad n=1; \\
                &0, && \text{if}\quad n>1.
            \end{aligned}\right.
        \]\\
        \vspace{3em}

        【莫比乌斯反演公式】若$f$是算术函数,$F$为$f$的和函数,满足
        \[F(n) = \sum_{d\mid n}f(d).\]
        则对任意正整数$n$,
        \[f(n) = \sum_{d\mid n}\mu(d)F(n/d).\]
        证明如下:
        \[
            \begin{aligned} 
                \sum_{d\mid n}\mu(d)F(n/d)&=\sum_{d\mid n}\biggl(\mu(d)\sum_{e\mid (n/d)}f(e)\biggr)\\
                                          &=\sum_{d\mid n}\biggl(\sum_{e\mid (n/d)}\mu(d)f(e)\biggr)\\
            \end{aligned}            
        \]

        注意到整数对$(d,e)$满足$d\mid n$和$e\mid (n/d)$时,同样必有$e\mid n$和$d\mid (n/e)$.于是
        \[
            \begin{aligned} 
                \sum_{d\mid n}\biggl(\sum_{e\mid (n/d)}\mu(d)f(e)\biggr)&=\sum_{e\mid n}\biggl(\sum_{d\mid (n/e)}\mu(d)f(e)\biggr)\\
                                          &=\sum_{e\mid n}\biggl(f(e)\sum_{d\mid (n/e)}\mu(d)\biggr)\\
            \end{aligned}            
        \]
        
        又和式$\sum \limits_{d\mid (n/e)} \mu(d)=1$仅在$n/e=1$是成立,其他情况下该和式都等于0.因此有:
        \[\sum_{e\mid n}\biggl(f(e)\sum_{d\mid (n/e)}\mu(d)\biggr)=f(n)\cdot1=f(n).\]\\
        \vspace{5em}

        【定理】设$f$为算术函数,它的和函数为$F(n) = \sum \limits_{d\mid n}f(d)$,那么如果$F$是乘性函数,则$f$也是乘性函数.








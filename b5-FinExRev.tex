\Topic{高一上学期期末复习}
  \Teach{三角函数}
  % \Teach{平面向量}
  % ====================
  % \Topic{平面向量复习}
  % \Teach{}
  % ====================
  \Grade{高一}
  % \Name{郑皓天}\FirstTime{20181207}\CurrentTime{20181207}
  % \Name{林叶}\FirstTime{20180908}\CurrentTime{20181125}
  %\Name{1v2}\FirstTime{20181028}\CurrentTime{20181117}
  % \Name{林叶}\FirstTime{20180908}\CurrentTime{20181125}
  % \Name{郭文镔}\FirstTime{20181111}\CurrentTime{20181117}
  % \Name{马灿威}\FirstTime{20181111}\CurrentTime{20181111}
  % \Name{黄亭燏}\FirstTime{20181231}\CurrentTime{20181231}
  \newtheorem*{Theorem}{定理}
  \makefront
\vspace{-1.5em}
\startexercise
% \section{三角函数}
\section{要点复习}
  \subsection{任意角与弧度制}
    \begin{enumerate}[label=\arabic*)]
      \item 与角$ \alpha $终边相同的角的集合为
        $ S=\bigl\{\beta \bigm| \beta =\alpha+2k\piup,k\inZ\bigr\} $.
      \item 角度与弧度对应关系:$2\piup\rad=360\degree$
        $\Rightarrow$
        $\begin{cases}
          1\degree=\dfrac{\piup}{180}\rad\\
          1\rad=\Bp{\dfrac{180}{\piup}}\degree\approx57.30\degree
        \end{cases}$
      \item
        \begin{itemize}
          \item 弧长公式:$l=\abs{\alpha}r$;
          \item 扇形面积公式:$S=\dfrac12 lr=\dfrac12 \abs{\alpha}r^2$.
        \end{itemize}
    \end{enumerate}
    \begin{exercise}
      \item%福州三中2016-2017学年第二学期高一数学期末考试…….doc-1【弧度制与角度制】
        (2017 \textbullet {\kaishu 福州三中} 1)
        关于角度制与弧度制的等式,正确的是\xz
        \xx{$\piup=1\rm{rad}$}
          {$\piup=180$}
          {$1^{\degree}=\dfrac{180}{\piup}\rm{rad}$}
          {$1\rm{rad}=\Bigl(\dfrac{180}{\piup}\Bigr)^\degree$}
        \begin{answer}
          D
        \end{answer}
      \item%福建师大附中2015-2016学年高一数学第二学期期末检测.doc-2【象限角】
        (2016 \textbullet {\kaishu 师大附中} 2)
        若点$P(\sin\theta\cos\theta,2\cos\theta)$位于第三象限,那么角$\theta$终边落在\xz
        \xx{第一象限}{第二象限}{第三象限}{第四象限}
        \begin{answer}
          B
        \end{answer}
      \item%福州第三中中学2015-2016学年高一数学第二学期期末检测.doc-3【弧度制、扇形面积】
        (2016 \textbullet {\kaishu 福州三中} 3)
        若3弧度的圆心角所对的弧长为$\SI{6}{\cm}$,则这个圆心角所夹的扇形面积是\xz
        \xx{$3\,\mathrm{cm}^{\mathrm2}$}
         {$6\,\mathrm{cm}^{\mathrm2}$}
         {$6\piup\,\mathrm{cm}^{\mathrm2}$}
         {$3\piup\,\mathrm{cm}^{\mathrm2}$}
        \begin{answer}
          B
        \end{answer}
    \end{exercise}
  \vspace{1.5cm}
  \subsection{任意角的三角函数与诱导公式}
    \begin{enumerate}[label=\arabic*)]
      \item 点$ P(x,y) $是角$ \alpha $终边上异于原点的一点,
        $ \abs{OP} =r=\sqrt{x^2+y^2}$,则\[\sin\alpha=\dfrac{y}{r},\quad \cos\alpha=\dfrac{x}{r},\quad \tan\alpha=\dfrac{y}{x}.\]
      \item 同角三角函数基本关系式:
        \circled{1} $\sin^2\alpha+\cos^2\alpha=1;$\qquad
        \circled{2} $ \tan\alpha=\dfrac{\sin\alpha}{\cos\alpha}.$
      \item 诱导公式:
        $ k\!\cdot\!\dfrac{\piup}{2}\pm\alpha $($k\inZ$)的三角函数值化为$\alpha$的三角函数值---{\kaishu 奇变偶不变,符号看象限}.
    \end{enumerate}
    \begin{exercise}
      \item%福建师大附中2016-2017高一下期末考试数学试题…….doc-1【任意角三角函数定义】
        (2017 \textbullet {\kaishu 师大附中} 1)
        角$\theta$的终边与单位圆交于点$P(\dfrac12,y)$,则$\sin\theta=$\xz
        \xx{$\sqrt3$}
         {$\pm\sqrt3$}
         {$\dfrac{\sqrt3}2$}
         {$\pm\dfrac{\sqrt3}2$}
        \begin{answer}
          D
        \end{answer}
      \item%《2018天利38套:高考真题单元专题训练(文)》专题13三角函数的概念……P41p2【2015文•福建】【同角三角函数基本关系式】
        {\kaishu (2015文 \textbullet 福建)}
        若$\sin\alpha=-\dfrac5{13}$,且$\alpha$为第四象限角,则$\tan\alpha$的值等于\xz
        \xx{$\dfrac{12}5$}
         {$-\dfrac{12}5$}
         {$\dfrac5{12}$}
         {$-\dfrac5{12}$}
        \begin{answer}
          D
        \end{answer}
      \item%《2018天利38套:高考真题单元专题训练(文)》专题13三角函数的概念……P41p2【2016文•全国新课标】【同角三角函数基本关系式、诱导公式】
        {\kaishu (2016文 \textbullet 全国新课标)}
        已知$\theta$是第四象限角,且$\sin\Bp{\theta+\dfracp{}4}=\dfrac35$,则$\tan\Bp{\theta-\dfracp{}4}=$\tk.
        \begin{answer}
          $-\dfrac43$
        \end{answer}
      \item%福建师大附中2015-2016学年高一数学第二学期期末检测.doc-20【诱导公式,化简】
        (2016 \textbullet {\kaishu 师大附中} 20)
        已知$\cos\alpha=-\dfrac{\sqrt5}5$,$\alpha\in\Bp{\piup,\dfracp{3}2}$.\\
        (I)求$\sin\alpha$的值;
        (II)求$\dfrac{\sin(\piup+\alpha)+2\sin\Bp{\dfracp{3}2+\alpha}}{\cos(3\piup-\alpha)+1}$的值.
        \begin{answer}
          (I)$\sqrt5-1$
          (II)原式$=\dfrac{-\sin\alpha-2\cos\alpha}{-\cos\alpha+1}=\dfrac{\sqrt5}5+1$
        \end{answer}
      \vspace{4cm}
    \end{exercise}
  \subsection{三角函数的图像和性质}
    \subsubsection{正弦函数}
      \begin{center}
        \begin{tikzpicture}[scale=0.7]
          \coordinate[label=below right:$O$] (O) at(0,0);
          \coordinate[label=below :\small$-\piup$] (t1) at(-pi,0);
          \coordinate[label=below :\small$\piup$] (t2) at(pi,0);
          \draw[->,>=latex](-3.5*pi,0)--(3.5*pi,0)node[below](x) {$x$};
          \draw[->,>=latex](0,-1.5)--(0,1.5)node[right](y) {\small $y=\sin(x)$};
          \draw [domain=-3*pi:3*pi,samples=1000] plot(\x,{sin(\x r)});
          \draw[densely dashed](pi/2,0)node[below](pi){$\dfrac{\piup}{2}$}--++(0,1);
          \draw[densely dashed](-pi/2,0)node[above](-pi){$-\dfrac{\piup}{2}$}--++(0,-1);
          \draw[densely dashed](0,1)node[left](max){$1$}--++(pi/2,0);
          \draw[densely dashed](0,-1)node[right](min){$-1$}--++(-pi/2,0);
        \end{tikzpicture}
      \end{center}
      \vspace{-0.9cm}
      \begin{enumerate}[label=\circled{\arabic*}]
        \item 定义域:$x\inR$;\quad 值域:$ \left[-1,1\right] $ ;\quad 奇偶性:奇函数;
        \item 对称轴:$x=k\piup+\dfrac{\piup}{2}\left(k\inZ\right)$;\quad 对称中心:$\left(k\piup,0\right)\left(k\inZ\right)$;\quad 最小正周期:$T=2\piup$;
        \item 单调递增区间:$ \left[2k\piup-\dfrac{\piup}{2},2k\piup+\dfrac{\piup}{2}\right]\left(k\inZ\right) $;\quad
              单调递减区间:$ \left[2k\piup+\dfrac{\piup}{2},2k\piup+\dfrac{3\piup}{2}\right] \left(k\inZ\right)$.
      \end{enumerate}
    \subsubsection{余弦函数}
      \begin{center}
        \begin{tikzpicture}[scale=0.7]
          \coordinate[label=below right:\small$O$] (O) at(0,0);
          \coordinate[label=below :\small $\frac{\piup}{2}$] (t1) at(pi/2,0);
          \coordinate[label=below :\small $-\frac{\piup}{2}$] (t2) at(-pi/2,0);
          \coordinate[label=below left :\small $1$] (t3) at(0,1);
          \draw[->,>=latex](-3.5*pi,0)--(3.5*pi,0)node[below](x) {$x$};
          \draw[->,>=latex](0,-1.5)--(0,1.5)node[right](y) {\small $y=\cos(x)$};
          \draw [domain=-3*pi:3*pi,samples=1000] plot(\x,{cos(\x r)});
          \draw[densely dashed](pi,0)node[below left](pi){\small $\piup$}--++(0,-1);
          \draw[densely dashed](-pi,0)node[below left](-pi){\small $-\piup$}--++(0,-1);
          \draw[densely dashed](-pi,-1)--(0,-1)node[below left](min){$-1$}--++(pi,0);
        \end{tikzpicture}
      \end{center}
      \vspace{-0.7cm}
      \begin{enumerate}[label=\circled{\arabic*}]
        \item 定义域:$x\inR$;\quad 值域:$ \left[-1,1\right] $;\quad 奇偶性:偶函数;
        \item 对称轴:$ x=k\piup \left(k\inZ\right) $;\quad 对称中心:$\left(k\piup+\dfrac{\piup}{2},0\right)\left(k\inZ\right)$;\quad 最小正周期:$ T=2\piup  $;
        \item 单调递增区间:$ \left[2k\piup-\piup,2k\piup\right] \left(k\inZ\right)$;\quad
              单调递减区间:$ \left[2k\piup,2k\piup+\piup\right]\left(k\inZ\right) $.
      \end{enumerate}
    \subsubsection{正切函数}
      \vspace{-0.5cm}
      \begin{center}
        \begin{tikzpicture}[scale=0.7]
          \coordinate[label=below right:$O$] (O) at(0,0);
          %\coordinate[label=below :$\dfrac{\pi}{2}$] (t1) at(pi/2,0);
          %\coordinate[label=below :$2\pi$] (t2) at(2*pi,0);
          \draw[->,>=latex](-pi,0)--(pi,0)node[below](x) {$x$};
          \draw[->,>=latex](0,-1.5)--(0,2)node[right](y) {\small $y=\tan(x)$};
          \draw [domain=-pi/3:1/3*pi,samples=1000] plot(\x,{tan(\x r)});
          \draw[densely dashed](2*pi/5,1.5)--++(0,-1.5)node[below right](pi){$\frac{\pi}{2}$}--++(0,-1.5);
          \draw[densely dashed](-2*pi/5,1.5)--++(0,-1.5)node[below left](pi){$-\frac{\pi}{2}$}--++(0,-1.5);
        \end{tikzpicture}
      \end{center}
      \vspace{-0.7cm}
      \begin{enumerate}[label=\circled{\arabic*}]
        \item 定义域:$\Bigl\{x\Bigm|x\ne k\piup+\dfrac{\piup}2,k\inZ\Bigr\}$;\quad 值域:$ \mathbf{R} $;\quad 奇偶性:奇函数;
        \item 对称中心:$\left(k\piup,0\right)\left(k\inZ\right)$;\quad 最小正周期:$ T=\piup  $;
        \item 单调递增区间:$ \left(k\piup-\dfrac{\piup}{2},k\piup+\dfrac{\piup}{2}\right) \left(k\inZ\right)$;
        \end{enumerate}
    \begin{exercise}
      \item%福州一中学2016-2017学年高一下学期期末考试数学…….doc-1【三角函数,比大小】
        (2017 \textbullet {\kaishu 福州一中} 1)
        设$a=\sin36\degree$,$b=\cos(-52\degree)$,$c=\tan218\degree$,则\xz
        \xx{$a<b<c$}
         {$a<c<b$}
         {$b<c<a$}
         {$b<a<c$}
        \begin{answer}
          A
        \end{answer}
      \item%《2018天利38套:高考真题单元专题训练(理)ISBN978-7-223-03393-0》专题14三角函数的图像与性质P53p4【2017•全国新课标】【正弦曲线性质】
          {\kaishu (2017 \textbullet 全国新课标)}
          设函数$f(x)=\cos\Bp{x+\dfrac{\piup}3}$,则下列结论错误的是\xz
          \xx{$f(x)$的一个周期为$-2\piup$}
           {$y=f(x)$的图像关于直线$x=\dfrac{8\piup}3$对称}
           {$f(x+\piup)$的一个零点为$x=\dfrac{\piup}6$}
           {$f(x)$在$\Bp{\dfrac{\piup}2,\piup}$单调递减}
          \begin{answer}
            D
          \end{answer}
      \item%福建师大附中2016-2017高一下期末考试数学试题…….doc-14【三角函数取值范围】
        (2017 \textbullet {\kaishu 师大附中} 14)
        函数$y=\sqrt{\cos x-\dfrac12}$的定义域为\tk.
        \begin{answer}
          $\Bigl[-\dfrac{\piup}3+2k\piup,\dfrac{\piup}3+2k\piup\Bigr]$,$k\inZ$
        \end{answer}
      \item%福州第三中中学2015-2016学年高一数学第二学期期末检测.doc-7【三角函数 复合函数单调性】
        (2016 \textbullet {\kaishu 福州三中} 7)
        函数$y=\sin\Bigl(-2x+\dfrac{\piup}4 \Bigr)$的单调增区间为\xz
        \xx{$\Bigl[k\piup-\dfrac{\piup}8,k\piup+\dfrac{3\piup}8 \Bigr]$,$k\in\mathbb{Z}$}
         {$\Bigl[k\piup+\dfrac{3\piup}8,k\piup+\dfrac{7\piup}8 \Bigr]$,$k\in\mathbb{Z}$}
         {$\Bigl[k\piup-\dfrac{3\piup}8,k\piup+\dfrac{\piup}8 \Bigr]$,$k\in\mathbb{Z}$}
         {$\Bigl[k\piup+\dfrac{\piup}8,k\piup+\dfrac{5\piup}8 \Bigr]$,$k\in\mathbb{Z}$}
        \begin{answer}
          B
        \end{answer}
      \item%福州第三中中学2015-2016学年高一数学第二学期期末检测.doc-14【三角函数性质 综合判断】
        (2016 \textbullet {\kaishu 福州三中} 14)
        关于函数$f(x)=2\sin\Bp{2x+\dfrac{\piup}3}$($x\inR$),有下列说法:\\
        \circled{1}由$f(x_1)=f(x_2)=0$可得$x_1-x_2$必是$\piup$的整数倍;
        \circled{2}$y=f(x)$的表达式可改写为$f(x)=2\cos\Bp{2x-\dfrac{\piup}6}$;
        \circled{3}$y=f(x)$的图像关于点$\Bp{-\dfrac{\piup}6,0}$对称;
        \circled{4}$y=f(x)$的图像关于直线$x=\dfrac{7\piup}{12}$对称.\\
        其中说法正确的序号是\tk.
        \begin{answer}
          \circled{2}\circled{3}\circled{4}
        \end{answer}
      \item%福建师大附中2016-2017高一下期末考试数学试题…….doc-10【三角函数取值范围,方程解】
        (2017 \textbullet {\kaishu 师大附中} 10)
        若方程$\cos\Bp{2x+\dfracp{}4}=m$在区间$\Bigl[0,\dfracp{}2\Bigr]$上有两个实根,则实数$m$取值范围是\xz
        \xx{$\Bigl[-1,-\dfrac{\sqrt2}2\Bigr]$}
         {$\Bigl(-1,-\dfrac{\sqrt2}2\Bigr]$}
         {$\Bigl[\dfrac{\sqrt2}2,1\Bigr]$}
         {$\Bigl[\dfrac{\sqrt2}2,1\Bigr)$}
        \begin{answer}
          B
        \end{answer}
      \item%福建师大附中2015-2016学年高一数学第二学期期末检测.doc-22【三角函数性质】
        (2016 \textbullet {\kaishu 师大附中} 22)
        已知函数$f(x)=3\sin\Bp{\dfrac{x}2+\dfrac{\piup}6}+3$,$x\inR$.\\
        (I)求函数$f(x)$的单调增区间;\\
        (II)若$x\in\Bigl[\dfrac{\piup}3,\dfrac{4\piup}3\Bigr]$,求$f(x)$的最大值和最小值,
        并指出$f(x)$取得最值时相
        应$x$的值.
        \begin{answer}
          (I)$\Bigl[-\dfrac{4\piup}3+4k\piup,\dfrac{2\piup}3+4k\piup\Bigr]$,$k\inZ$;
          (II)当$x=\dfrac{4\piup}3$时,取最小值$f(x)_{\min}=\dfrac92$;当$x=\dfrac{2\piup}3$时,取最大值$f(x)_{\max}=6$.
        \end{answer}
      \vspace{5cm}
      \item%福建师大附中2015-2016学年高一数学第二学期期末检测.doc-24【三角函数 复合函数最值】
        (2016 \textbullet {\kaishu 师大附中} 24)
        求函数$f(x)=3-2a\sin x-\cos^2x$的最小值.
        \begin{answer}
          $y_{\min}=\begin{cases}
            2a+3,\quad a\leqslant-1,\\
            a^2+2,\quad -1<a<1,\\
            -2a+3,\quad a\geqslant+3.
          \end{cases}$
        \end{answer}
      \vspace{5cm}
    \end{exercise}
  \subsection{三角函数图像的(线性)变换}
    \begin{description}
      \item 函数$y=f(x)$图像经平移或伸缩变换后的图像解析式:{\kaishu 坐标变量的变化与图像相反}
        \[\begin{aligned}
          y=f(x)&\xrightarrow[\text{平移}\abs{a}\text{个单位}]{\text{向左}(a>0)\text{或向右}(a<0)}y=f(x+a)\hspace{3em}
          y=f(x)\xrightarrow[\text{纵坐标不变}]{\text{横坐标变为原来的}k\text{倍}}y=f\Bp{\dfrac{x}{k}}\\
          y=f(x)&\xrightarrow[\text{平移}\abs{a}\text{个单位}]{\text{向下}(a>0)\text{或向上}(a<0)}y+a=f(x)\hspace{3em}
          y=f(x)\xrightarrow[\text{横坐标不变}]{\text{纵坐标变为原来的}A\text{倍}}\dfrac{y}{A}=f(x)
        \end{aligned}\]
      \item 由函数$y=\sin(x)$的图象经过变换得到$y=A\sin\left(\omega x+\varphi\right)$的图象方法\\
        \begin{minipage}[h]{0.45\linewidth}
          \hspace{-2em}\circled{1} 先平移后伸缩
            \[\begin{aligned}
              y=\sin x&\xrightarrow[\text{平移}\abs{\varphi}\text{个单位}]{\text{向左}(\varphi>0)\text{或向右}(\varphi<0)}y=\sin\left(x+\varphi\right)\\
              &\xrightarrow[\text{纵坐标不变}]{\text{横坐标变为原来的}\tfrac{1}{\omega}}y=\sin\left(\omega x+\varphi\right)\\
              &\xrightarrow[\text{横坐标不变}]{\text{纵坐标变为原来的}A\text{倍}}y=A\sin\left(\omega x+\varphi\right)
            \end{aligned}\]
        \end{minipage}\hfill
        \begin{minipage}[h]{0.55\linewidth}
          \circled{2} 先伸缩后平移
            \[\begin{aligned}
              y=\sin x&\xrightarrow[\text{纵坐标不变}]{\text{横坐标变为原来的}\tfrac{1}{\omega}}y=\sin\omega x\\
              &\xrightarrow[\text{平移}\abs{\tfrac{\varphi}{\omega}}\text{个单位}]{\text{向左}(\varphi>0)\text{或向右}(\varphi<0)}y=\sin\biggl[\omega\Bp{x+\dfrac{\varphi}{\omega}}\biggr]\\
              &\xrightarrow[\text{横坐标不变}]{\text{纵坐标变为原来的}A\text{倍}}y=A\sin\left(\omega x+\varphi\right)
            \end{aligned}\]
        \end{minipage}
      \item 由图象求函数$y=A\sin\left(\omega x+\varphi\right)$的解析式一般步骤:
        \begin{enumerate}[label=\arabic*\degree]
          \item 由函数的最值确定$ A $的取值;
          \item 由函数的周期确定$ \omega $的值, 周期:$ T=\dfrac{2\pi}{\abs{\omega}} $;
          \item 由函数图象最高点(最低点)的坐标得到关于$ \varphi $的方程,再由$ \varphi $的范围确定$ \varphi $的值.
        \end{enumerate}
    \end{description}
    \begin{exercise}
      \item%《2018天利38套:高考真题单元专题训练(理)ISBN978-7-223-03393-0》专题14三角函数的图像与性质P53p4【2015•全国新课标】【正弦曲线图像】
           %LaTeX-master/sanjiaohanshu/sanjiaohanshu-gaokao.tex 4
        {\kaishu (2015 \textbullet 全国新课标)}
        函数$f(x)=\cos(\omega x+\varphi)$的部分图象如图所示,则$f(x)$的单调递减区间为\xz
        \begin{minipage}[b]{0.8\linewidth}
          \vspace{2.5em}
          \xx{$\Bigl(k\piup-\dfrac{1}{4},k\piup+\dfrac{3}{4}\Bigr),k\in\mathbb{Z}$}
            {$ \Bigl(2k\piup-\dfrac{1}{4},2k\piup+\dfrac{3}{4}\Bigr),k\in\mathbb{Z}$}
            {$ \Bigl(k-\dfrac{1}{4},k+\dfrac{3}{4}\Bigr),k\in\mathbb{Z}$}
            {$\Bigl(2k-\dfrac{1}{4},2k+\dfrac{3}{4}\Bigr),k\in\mathbb{Z} $}
        \end{minipage}\hfill
        \begin{minipage}[h]{0.2\linewidth}
          \vspace{-3cm}
          \begin{tikzpicture}[scale=0.9]
            \node[below left](O) at(0,0) {\small$\bm{O}$};
            \draw(0,1)node[right]{\tiny$1$}--(0.1,1);
            \clip(-1.2,-1.2) rectangle (2,1.5);
            \draw[->,>=stealth](-1.2,0)--(2,0) node[below left] (x){$x$};
            \draw[->,>=stealth](0,-1.2)--(0,1.5) node[below right] (y){$y$};
            \draw[domain=-1.2:2,samples=1000] plot(\x,{cos((pi*(\x)+1/4*pi) r)});
            \node[below] (A)at (0.25,0){$\frac{1}{4}$};
            \node[below] (B)at (1.25,0){$\frac{5}{4}$};
          \end{tikzpicture}
        \end{minipage}
        \begin{answer}
          D
        \end{answer}
      \item%福州三中中学2015-2016学年高一数学第二学期期末检测.docx-9【正弦曲线图像】
        (2016 \textbullet {\kaishu 福州三中} 9)
        将函数$y=\sin\Bigl(x-\dfrac{\piup}3\Bigr)$的图像上所有点的横坐标伸长到原来的2倍(纵坐标不变),再将所得的图像向左平移$\dfrac{\piup}3$个单位,得到的函数图像对应的解析式是\xz
        \xx{$y=\sin\dfrac x2$}
          {$y=\sin\Bigl(\dfrac x2-\dfrac{\piup}2\Bigr)$}
          {$y=\sin\Bigl(\dfrac{x}2-\dfrac{\piup}6\Bigr)$}
          {$y=\sin\Bigl(2x-\dfrac{\piup}6\Bigr)$}
        \begin{answer}
          C
        \end{answer}
      \item%福州一中2015-2016学年高一数学第二学期期末检测.docx-5【正弦曲线图像】
        (2016 \textbullet {\kaishu 福州一中} 5)
        函数$y=\sin\Bp{2x+\dfracp{}3}$的图像向右平移$\dfrac{\piup}6$个单位,所得的图像对应的函数\xz
        \xx{为非奇非偶函数}
         {图像的对称中心为$(2k\piup,0)$($k\inZ$)}
         {为奇函数}
         {在$\Bigl[-\dfrac{\piup}3,\dfrac{\piup}6\Bigr]$上单调递增}
        \begin{answer}
          C
        \end{answer}
      \item%福州屏东中学2016-2017学年高一下学期期末考试数学试题.doc-10【正弦曲线图像】
        (2017 \textbullet {\kaishu 屏东中学} 10)
        函数$f(x)=\sin(\omega x+\phi)$(其中$\abs{\phi}<\dfrac{\piup}2$)的图像如图所示,为了得到$y=\sin\omega x$的图像,只需把$y=f(x)$的图像上所有点(\hspace{2.5em})个长度单位.\\
        \begin{minipage}[b]{0.8\linewidth}
          \vspace{2.5em}
          \xx{向右平移$\dfrac{\piup}6$}
           {向右平移$\dfrac{\piup}{12}$}
           {向左平移$\dfrac{\piup}6$}
           {向左平移$\dfrac{\piup}{12}$}
        \end{minipage}\hfill
        \begin{minipage}[h]{0.2\linewidth}
          \vspace{-2.7cm}
          \begin{center}
            \begin{tikzpicture}[scale=0.8]
              \coordinate[label=below left:$O$] (O) at(0,0);
              \coordinate[label=above :\small$\tfrac{\piup}3$] (t1) at(pi/3,0);
              \draw[->,>=latex](-0.2*pi,0)--(1.2*pi,0)node[below](x) {$x$};
              \draw[->,>=latex](0,-1.5)--(0,1.5)node[right](y) {\small $y$};
              \draw [domain=-0.1*pi:0.9*pi,samples=100] plot(\x,{sin((2*\x+pi/3) r)});
              \draw[densely dashed](7*pi/12,0)node[above](pi){$\tfrac{7\piup}{12}$}--++(0,-1);
              \draw[densely dashed](0,-1)node[left](min){$-1$}--++(7*pi/12,0);
            \end{tikzpicture}
          \end{center}
        \end{minipage}
         \begin{answer}
           A
         \end{answer}
        \vspace{-0.9cm}
      \item%《2018天利38套:高考真题单元专题训练(理)ISBN978-7-223-03393-0》专题14三角函数的图像与性质P53p8【2017•天津】【正弦曲线解析式】
            {\kaishu (2017 \textbullet 天津)}
            设函数$f(x)=2\sin(\omega x+\varphi)$,$x\inR$,其中$\omega>0$,$\abs{\varphi}<\piup$,若$f\Bp{\dfrac{5\piup}8}=2$,$f\Bp{\dfrac{11\piup}8}=0$,且$f(x)$的最小正周期大于$\piup$,则\xz
            \xx{$\omega=\dfrac23$,$\varphi=\dfrac{\piup}{12}$}
             {$\omega=\dfrac23$,$\varphi=-\dfrac{11\piup}{12}$}
             {$\omega=\dfrac13$,$\varphi=-\dfrac{11\piup}{24}$}
             {$\omega=\dfrac13$,$\varphi=\dfrac{7\piup}{24}$}
            \begin{answer}
              A
            \end{answer}
      \newpage
      % \item%高中数学习题解法辞典.pdf 2-2-1
      %   已知函数$f(x)=A\sin(\omega x+\varphi)(A,\omega,\varphi\text{为常数},\omega>0)$的图像上相邻两个最高点的坐标分别是$\Bigl(\dfrac{\piup}{12},2\Bigr)$,$\Bigl(\dfrac{13\piup}{12},2\Bigr)$.\\
      %   (1) 求函数$f(x)$的一个表达式;\\
      %   (2)画出函数$f(x)$在长度为一个周期的闭区间上的简图;\\
      %   (3)说明经过怎样的变换,可以由$y=\sin x$的图像得到$y=f(x)$的图像.
      %   \begin{answer}
      %     (1)$y=2\sin\Bigl(2x+\dfrac{\piup}3\Bigr)(\varphi=k\piup-\dfrac{2\piup}3$即可);(2)略;(3)将$y=\sin x$图像上所有点向左平移$\dfrac{\piup}3$个单位得到$y=\sin \Bigl(x+\dfrac{\piup}3\Bigr)$的图像;再把$y=\sin \Bigl(x+\dfrac{\piup}3\Bigr)$的图像上所有点的横坐标缩短到原来的$\dfrac12$(纵坐标不变),得到$y=\sin \Bigl(2x+\dfrac{\piup}3\Bigr)$的图像;最后把$y=\sin \Bigl(2x+\dfrac{\piup}3\Bigr)$的图像上所有点的纵坐标伸长到原来的2倍(横坐标不变),即可得到函数$y=f(x)$的图像.
      %   \end{answer}
      \item%《2018天利38套:高考真题单元专题训练(理)ISBN978-7-223-03393-0》专题14三角函数的图像与性质P54p18【2017•山东】【正弦曲线解析式,三角恒等变换】
            {\kaishu (2017 \textbullet 山东)}
            设函数$f(x)=\sin\Bp{\omega x-\dfrac{\piup}6}+\sin\Bp{\omega x-\dfrac{\piup}2}$,其中$0<\omega<3$.已知$f\Bp{\dfrac{\piup}6}=0$.\\
            (I)求$\omega$;\\
            (II)将函数$y=f(x)$的图像上各点的横坐标伸长为原来的2倍(纵坐标不变),再将得到的图像向左平移$\dfrac{\piup}4$个单位,得到函数$y=g(x)$的图像,求$g(x)$在$\Bigl[-\dfrac{\piup}4,\dfrac{3\piup}4\Bigr]$上的最小值.
            \begin{answer}
              (I)$f(x)=\sqrt3\sin\Bp{\omega x-\dfrac{\piup}3}$,$\omega=2$.
              (II)$g(x)=\sqrt3\sin\Bp{x-\dfrac{\piup}{12}}$,当$x=-\dfrac{\piup}4$时,$g(x)$取得最小值$-\dfrac32$
            \end{answer}
      \vspace{5cm}
    \end{exercise}

\section{课后作业}
  \begin{exercise}
    \item%福州屏东中学2016-2017学年高一下学期期末考试数学试题.doc-11【扇形面积】
      (2017 \textbullet {\kaishu 屏东中学} 11)
      若一个扇形的周长与面积的数值相等,则该扇形所在圆的半径不可能等于\xz
      \xx{5}{2}{3}{4}
      \begin{answer}
        C
      \end{answer}
    \item%《2018天利38套:高考真题单元专题训练(文)》专题13三角函数的概念……P41p5【2009文•重庆】【三角函数 比大小】
      {\kaishu (2009文 \textbullet 重庆)}
      %福建师大附中2015-2016学年高一数学第二学期期末检测.doc-8【三角函数 比大小】
      (2016 \textbullet {\kaishu 师大附中} 8)
      下列关系式中正确的是\xz
      \xx{$\sin11\degree<\cos10\degree<\sin168\degree$}
       {$\sin168\degree\sin11\degree<\cos10\degree$}
       {$\sin11\degree<\sin168\degree<\cos10\degree$}
       {$\sin168\degree<\cos10\degree<\sin11\degree$}
      \begin{answer}
        C
      \end{answer}
    \item%福州屏东中学2016-2017学年高一下学期期末考试数学试题.doc-9【三角函数取值范围】
      (2017 \textbullet {\kaishu 屏东中学} 9)
      函数$y=\sqrt{2\cos x+1}$的定义域是\xz
      \xx{$\Bigl[2k\piup-\dfrac{\piup}3,2k\piup+\dfrac{\piup}3 \Bigr]$,$k\in\mathbb{Z}$}
       {$\Bigl[2k\piup-\dfrac{\piup}6,2k\piup+\dfrac{\piup}6 \Bigr]$,$k\in\mathbb{Z}$}
       {$\Bigl[2k\piup+\dfrac{\piup}3,2k\piup+\dfrac{2\piup}3 \Bigr]$,$k\in\mathbb{Z}$}
       {$\Bigl[2k\piup-\dfrac{2\piup}3,2k\piup+\dfrac{2\piup}3 \Bigr]$,$k\in\mathbb{Z}$}
      \begin{answer}
        D
      \end{answer}
    \item%福建师大附中2015-2016学年高一数学第二学期期末检测(实验班).doc-6【三角函数变换、诱导公式】
      (2016 \textbullet {\kaishu 师大附中实验班} 6)
      已知函数$f(x)=\sin(\omega x+\phi)$(其中$\abs{\phi}<\dfrac{\piup}2$)图像相邻对称轴的距离为$\dfrac{\piup}2$,一个对称中心为$\Bp{-\dfrac{\piup}6}$,为了得到$y=\cos\omega x$的图像,则只要将$y=f(x)$的图像\xz\\
      \xx{向右平移$\dfrac{\piup}6$个单位}
       {向右平移$\dfrac{\piup}{12}$个单位}
       {向左平移$\dfrac{\piup}6$个单位}
       {向左平移$\dfrac{\piup}{12}$个单位}
      \begin{answer}
        D
      \end{answer}
    \item%福州一中学2016-2017学年高一下学期期末考试数学…….doc-10【三角函数与二次函数复合值域,同名三角函数关系】【中上难度】
      (2017 \textbullet {\kaishu 福州一中} 10)
      关于$x$的方程$\sin x-\cos^2x+a=0$在$x\in[0,2\piup)$内恰有4解,则实数$a$的取值范围是\xz
      \xx{$\Bp{-1,\dfrac54}$}
       {$\Bp{1,\dfrac54}$}
       {$\Bigl[-1,\dfrac54\Bigr)$}
       {$\Bigl[1,\dfrac54\Bigr)$}
      \begin{answer}
        B
      \end{answer}
    %填空题
    \item%福建师大附中2015-2016学年高一数学第二学期期末检测.doc-17【诱导公式】
      (2016 \textbullet {\kaishu 师大附中} 17)
      已知$\sin\Bp{\theta-\dfracp{}4}=\dfrac13$,则$\cos\Bp{\dfracp{}4+\theta}$的值等于\tk.
      \begin{answer}
        $-\dfrac13$
      \end{answer}
    %暂用
    \item%《习题化知识清单》P73易混清单例
      函数$y=2\sin\Bigl(\dfrac{\piup}3-2x \Bigr)$的单调增区间为\tk.
      \begin{answer}
        $\Bigl[k\piup+\dfrac{5\piup}{12},k\piup+\dfrac{11\piup}{12} \Bigr],k\in\mathbb{Z}$
      \end{answer}
    % \item%函数y=Asin(ωx+φ)的图象及简单应用P11.14
    %   已知曲线$y=A\sin(\omega x+\varphi)$$(A>0,\omega>0,\abs{\varphi}\leqslant\dfrac{\piup}2)$上最高点为$(2,\sqrt{2})$,该最高点与相邻的最低点间的曲线与$x$轴交于点$(6,0)$.\\
    %   (1)该函数的解析式;\\
    %   (2)该函数在$x\in[-6,0]$上的值域.
    %   \begin{answer}
    %     (1)$y=\sqrt{2}\sin\Bigl(\dfrac{\piup}8x+\dfrac{\piup}4\Bigr)$;
    %     (2)$[-\sqrt{2},0]$
    %   \end{answer}
    % \vspace{5cm}
    \item%福州屏东中学2016-2017学年高一下学期期末考试数学试题.doc-20【正弦曲线图像】
      (2017 \textbullet {\kaishu 屏东中学} 20)
      已知角$\alpha$的终边过点$P(-3,4)$.\\
      (1)求$\dfrac{\tan\alpha}{\sin(\piup-\alpha)-\cos\Bp{\dfracp12+\alpha}}$的值;\quad
      (2)若$\beta$为第三象限角,且$\tan\beta=\dfrac34$,求$\cos(2\alpha-\beta)$的值.
      \begin{answer}
        (1)$-\dfrac56$;
        (2)$\dfrac45$.
      \end{answer}
    \vspace{3cm}
    \item%《2018天利38套:高考真题单元专题训练(理)ISBN978-7-223-03438-8》专题15三角恒等变换 P59p20【2014•广东】
      (2014 \textbullet {\kaishu 广东})已知函数$f(x)=A\sin{\Bigl(x+\dfrac{\piup}4\Bigr)}$,$x\in\mathbb{R}$,且$f\Bigl(\dfrac{5\piup}{12}\Bigr)=\dfrac{3}2$.\\
      (I)求$A$的值;\\
      (II)若$f(\theta)+f(-\theta)=\dfrac{3}2$,$\theta\in \Bigl(0,\dfrac{\piup}2\Bigr)$,求$f\Bigl(\dfrac{3\piup}4-\theta\Bigr)$.
      \begin{answer}
        (I)$A=\sqrt{3}$;
        (II)$f\Bigl(\dfrac{3\piup}4-\theta\Bigr)=\dfrac{\sqrt{30}}4$
      \end{answer}
    \vspace{4cm}
    % \item%福州格致中学2015-2016学年高一数学第二学期期末检测.docx-22
    %   % (附加题:本小题满分15分)\\
    %   % (福州格致中学2015-2016学年高一数学第二学期期末检测22)
    %   (2016 \textbullet {\kaishu 格致中学} 22)
    %   已知函数$f(x)=A\sin(\omega x+\varphi)+B (A>0,\omega>0)$的一系列对应值如下表:
    %   \begin{center}
    %     \renewcommand{\arraystretch}{1.4}
    %     \begin{tabular}{|*{8}{c|}}
    %       \hline
    %         $x$
    %         &$\dfrac{\piup}6$
    %         &$-\dfrac{\piup}3$
    %         &$-\dfrac{5\piup}6$
    %         &$-\dfrac{4\piup}3$
    %         &$-\dfrac{11\piup}6$
    %         &$-\dfrac{7\piup}3$
    %         &$-\dfrac{17\piup}6$\\
    %       \hline
    %         $y$
    %         &$-1$
    %         &$1$
    %         &$3$
    %         &$1$
    %         &$-1$
    %         &$1$
    %         &$3$\\
    %       \hline
    %     \end{tabular}\\
    %   \end{center}
    %   (1)根据表格提供的数据求函数$f(x)$的一个解析式;\\
    %   (2)根据(1)的结果:\\
    %   \;(i)当$x\in\Bigl[0,\dfrac{\piup}3\Bigr]$时,方程$f(3x)=m$恰有两个不同的解,求实数$m$的取值范围;\\
    %   \;(ii)若是$\alpha,\beta$是锐角三角形的两个内角,试比较$f(\sin \alpha)$与$f(\cos \beta)$的大小.
    %   \begin{answer}
    %     (1)$f(x)=2\sin\Bigl(x-\dfrac{\piup}3\Bigr)+1$;(2)(i)$[\sqrt{3}+1,3)$;(ii)易得$f(x)$在$[-\dfrac{\piup}6,\dfrac{5\piup}6]$上单调递增,故$f(x)$在$[0,1]$上单调递增;又$0<\dfrac{\piup}2-\beta<\alpha<\dfrac{\piup}2$,从而$\sin\alpha>\sin(\dfrac{\piup}2-\beta)=\cos\beta$,于是$f(\sin \alpha)>f(\cos \beta)$
    %   \end{answer}
    \item%福州一中学2016-2017学年高一下学期期末考试数学…….doc-10【正弦曲线图像】
      \begin{minipage}[t]{0.7\linewidth}
        (2017 \textbullet {\kaishu 福州一中} 16)
        已知函数$f(x)=A\sin(\omega x+\phi)$($A>0$,$\omega>0$,$\abs{\phi}<\dfrac{\piup}2$)的部分图像如图所示,\\
        (I)求函数$f(x)$的单调递增区间;\\
        (II)将$f(x)$的图像向右平移$\dfrac{\piup}3$个单位长度,再将所得的图像上各店的横坐标缩短到原来的$\dfrac12$倍(纵坐标不变),
        得到$g(x)$的图像;当$x\in\Bp{0,\dfrac{\piup}4}$时,求$g(x)$的值域.
      \end{minipage}\hfill
      \begin{minipage}[h]{0.3\linewidth}
        \vspace{2.7cm}
        \begin{center}
          \begin{tikzpicture}[scale=0.8]
            \coordinate[label=below left:$O$] (O) at(0,0);
            \coordinate[label=above :\small$\tfrac{\piup}3$] (t1) at(pi/3,0);
            \draw[->,>=latex](-0.2*pi,0)--(1.2*pi,0)node[below](x) {$x$};
            \draw[->,>=latex](0,-1.5)--(0,1.5)node[right](y) {\small $y$};
            \draw [domain=-0.1*pi:0.9*pi,samples=100] plot(\x,{sin((2*\x+pi/3) r)});
            \draw[densely dashed](7*pi/12,0)node[above](pi){$\tfrac{7\piup}{12}$}--++(0,-1);
            \draw[densely dashed](0,-1)node[left](min){$-\sqrt2$}--++(7*pi/12,0);
          \end{tikzpicture}
        \end{center}
      \end{minipage}
      \begin{answer}
        (I)$f(x)=\sqrt2\sin\Bp{2x+\dfrac{\piup}3}$,单调递增区间为$\Bigl[-\dfrac{5\piup}{12}+k\piup,\dfrac{\piup}{12}+k\piup\Bigr]$,$k\inZ$.
        (II)$g(x)=\sqrt2\sin\Bp{4x-\dfrac{\piup}3}$,值域:$\Bigl[-\dfrac{\sqrt6}2,\sqrt2\Bigr]$
      \end{answer}
    \vspace{4cm}

  \end{exercise}

% \section{三角恒等变换}
%   \begin{exercise}
%     \item%福州第三中中学2015-2016学年高一数学第二学期期末检测.doc-15【诱导公式,和角】
%     (2016 \textbullet {\kaishu 福州三中} 15)
%     已知$\sin\BP{\alpha+\dfrac{\piup}4}=-\dfrac35$,且$0<\alpha<\dfrac{5\piup}4$,求$\cos\BP{\alpha+\dfrac{\piup}2}$的值.
%     \begin{answer}
%       $-\dfrac{\sqrt2}{10}$
%     \end{answer}
%     \vspace{5cm}
%     \item%福州三中2017高一下数学期末卷…….doc-15【辅助角公式灵活应用】
%          %《2018天利38套:高考真题单元专题训练(理)ISBN978-7-223-03438-8》专题14三角函数的图像与性质 P54p16【2013•全国新课标】
%       (2017 \textbullet {\kaishu 福州三中} 15)(2013 \textbullet {\kaishu 全国新课标})
%       设当$x=\theta$时,函数$f(x)=\sin x-2\cos x$取得最大值,则$\cos\theta=$\tk.
%       \begin{answer}
%         $-\dfrac{2\sqrt{5}}5$
%       \end{answer}
%     \item%《2018天利38套:高考真题单元专题训练(理)ISBN978-7-223-03393-0》专题14三角函数的图像与性质P54p13【2016•全国新课标】【三角函数变换、辅助角】
%         {\kaishu (2016 \textbullet 全国新课标)}
%         函数$y=\sin x-\sqrt3\cos x$的图像可由函数$y=\sin x+\sqrt3\cos x$的图像至少向右平移\tk个单位长度得到.
%         \begin{answer}
%           $\dfrac{2\piup}3$
%         \end{answer}
%     \item%福建师大附中2016-2017高一下期末考试数学试题…….doc-9【二倍角、诱导公式】
%      (2017 \textbullet {\kaishu 师大附中} 9)
%      已知$\sin\Bp{\dfracp{}6-\alpha}=\dfrac13$,则$\cos\Bp{\dfracp{2}3+2\alpha}=$\xz
%      \xx{$-\dfrac79$}
%       {$-\dfrac13$}
%       {$\dfrac13$}
%       {$\dfrac79$}
%      \begin{answer}
%        A
%      \end{answer}
%     \item%福州三中2017高一下数学期末卷…….doc-6【二倍角、诱导公式】
%       (2017 \textbullet {\kaishu 福州三中} 6)
%       已知$\sin\Bp{\dfracp{3}2+\theta}+2\cos(\piup+\theta)=\sin(-\theta)$,则$\sin\theta\cos\theta+\cos^2\theta=$\xz
%       \xx{$-\dfrac15$}
%        {$\dfrac25$}
%        {$\dfrac35$}
%        {$1$}
%       \begin{answer}
%         B
%       \end{answer}
%     \item%福州一中学2016-2017学年高一下学期期末考试数学…….doc-7【和角公式,韦达定理】
%       (2017 \textbullet {\kaishu 福州一中} 7)
%       已知$\tan\alpha$,$\tan\beta$是方程$x^2+3\sqrt3x+4=0$的两根,$\alpha,\beta\in(0,\piup)$,则$\alpha+\beta=$\xz
%       \xx{$\dfracp{}3$}
%        {$\dfracp{2}3$}
%        {$\dfracp{4}3$}
%        {$\dfracp{}3$或$\dfracp{4}3$}
%       \begin{answer}
%         A
%       \end{answer}
%     \item%福州一中学2016-2017学年高一下学期期末考试数学…….doc-5【二倍角、半角、奇偶性、周期性】
%         (2017 \textbullet {\kaishu 福州一中} 5)
%         函数$f(x)=\dfrac12(1-\cos{2x})\cos^2x$,$x\in\mathbb{R}$是\xz
%         \xx{最小正周期为$\piup$的偶函数}
%          {最小正周期为$\dfracp{}2$的偶函数}
%          {最小正周期为$\piup$的奇函数}
%          {最小正周期为$\dfracp{}2$的奇函数}
%         \begin{answer}
%           B
%         \end{answer}
%     \item%福州三中2017高一下数学期末卷…….doc-17【三角函数化简、二倍角、诱导公式】
%         (2017 \textbullet {\kaishu 福州三中} 17)
%         已知$f(x)=2\tan x+\dfrac{1-2\sin^2{\dfrac{x}2}}{\sin\dfrac{x}2\cdot\cos\dfrac{x}2}$.\\
%         (I)求$f(\dfrac{\piup}6)$的值;\\
%         (II)若$f(\alpha)=5$,求$f\Bp{\alpha+\dfrac{\piup}4}$的值.
%         \begin{answer}
%           (I)$\dfrac{8\sqrt3}3$;
%           (II)$\pm\dfrac{20}3$.
%         \end{answer}
%     \vspace{5cm}
%   \end{exercise}
% \item%福建师大附中2016-2017高一下期末考试数学试题…….doc-20【三角函数性质,向量数量积计算】
%   (2017 \textbullet {\kaishu 师大附中} 20)
%   已知向量$\bm a=(\cos x,\sin x)$,$\bm b=(3,-\sqrt3)$,记$f(x)=\bm a\cdot\bm b$.\\
%   (I)求$f(x)$的单调增区间;\\
%   (II)若$x\in[0,\piup]$,求$f(x)$的值域.
%   \begin{answer}
%     (I)$\Bigl[-\dfrac{5\piup}6+2k\piup,\dfrac{11\piup}6+2k\piup\Bigr]$,$k\inZ$;
%     (II)$[-2\sqrt3,3]$.
%   \end{answer}
% \vspace{5cm}
% \section{平面向量}
% \section{要点归纳}
%
%   \subsection{五种常见向量}
%     \begin{enumerate}[label=\arabic*)]
%       \item 单位向量:模为1的向量.
%       \item 零向量:模为0的向量.
%       \item 平行(共线向量):方向相同或相反或其一为零向量的两个向量.
%       \item 相等向量:模相等,方向相同的向量.
%       \item 相反向量:模相等,方向相反的向量.
%     \end{enumerate}
%   \subsection{平面向量运算律}
%     \begin{enumerate}[label=\arabic*)]
%       \item 交换律:
%         $\bm a+\bm b=\bm b+\bm a$,\quad
%         $\bm a\cdot\bm b=\bm b\cdot\bm a$
%       \item 结合律:
%         $(\bm{a}+\bm{b})+\bm{c}=\bm{a}+(\bm{b}+\bm{c})$,\quad
%         $(\lambda \bm a)\cdot\bm{b}=\lambda(\bm a\cdot\bm b)=\bm{a}\cdot(\lambda\bm{b})$
%       \item 分配律:
%         $(\lambda+\mu)\bm{a}=\lambda\bm{a}+\mu\bm{a}$,\quad
%         $\lambda(\bm{a}+\bm{b})=\lambda\bm{a}+\lambda\bm{b}$,\quad
%         $(\bm a+\bm b)\cdot \bm c=\bm a\cdot\bm c+\bm b\cdot \bm c$
%       \item 重要公式:(记号$\bm a^2=\bm a\cdot\bm a$)
%         $(\bm a+\bm b)(\bm a-\bm b)=\bm a^2-\bm b^2$,\quad
%         $(\bm a\pm\bm b)^2=\bm a^2\pm2\bm a\cdot\bm b+\bm b^2$.
%     \end{enumerate}
%   \subsection{两个重要定理}
%     \begin{enumerate}[label=\arabic*)]
%       \item 向量共线定理:
%         向量$\bm{a}~(\bm{a}\ne\bm{0})$与向量$\bm{b}$共线,当且仅当存在唯一的实数$ \lambda $,使得$\bm{b}=\lambda\bm{a}$.\\
%         {\kaishu
%          证明三点共线的方法:\circled{1}$\vv{AB}=\lambda\vv{AC}$,则$A$,$B$,$C$三点共线;\circled{2}$\vv{OA}=\lambda\vv{OB}+\mu\vv{OC}$,若$\lambda+\mu=1$,则$A$,$B$,$C$三点共线.
%         }
%       \item 平面向量基本定理:
%         如果$ \bm{e}_1,\bm{e}_2 $是同一平面内的两个\CJKunderdot{不共线}的向量,
%         则那么对于这一平面内的任意向量$ \bm{a} $,有且只有一对实数$ \lambda_1,~\lambda_2 $,使$\bm{a}=\lambda_1\bm{e}_1+\lambda_2\bm{e}_2$.
%         其中,不共线的向量$\bm{e}_1, \bm{e}_2$叫做表示这一平面内所有向量的一组\CJKunderdot{基底}.\\
%         {\kaishu 平面向量基本定理应用技巧:
%           \begin{enumerate}[label=\circled{\arabic*}]
%             \item 构造某一向量在同一基底下的两种不同表达形式,
%               根据向量分解的唯一性求解.即:\\
%               {\kaishu 以$\bm e_1$,$\bm e_2$为基底,且$\bm a=x_1\bm e_1+y_1\bm e_2=x_2\bm e_1+y_2\bm e_2$,则$\begin{cases}x_1=x_2\\y_1=y_2\end{cases}$}
%             \item 构造两个共线向量在同一基底下的表达形式,
%               根据向量共线定理求解.即:\\
%               {\kaishu 以$\bm e_1$,$\bm e_2$为基底,且$\bm a=x_1\bm e_1+y_1\bm e_2$,$\bm b=x_2\bm e_1+y_2\bm e_2$,且$\bm a\varparallel\bm b$,则$x_1y_2-x_2y_1=0$}
%             \item 将题目中的已知条件转化成
%               $\lambda_1\bm e_1+\lambda_2\bm e_2=\bm 0$的形式($\bm e_1$,$\bm e_2$不共线),根据$\lambda_1=\lambda_2=0$求解.
%           \end{enumerate}}
%     \end{enumerate}
%   \subsection{平面向量平行、垂直的等价条件}
%     设$\bm a=(x_1,y_1)$,$\bm b=(x_2,y_2)$,则:
%     \begin{enumerate}[label=\arabic*)]
%       \item $\bm a\varparallel\bm b$$\Leftrightarrow$$x_1y_2-x_2y_1=0$.
%       \item $\bm a\perp\bm b$
%             $\Leftrightarrow$$\bm a\cdot\bm b=0$
%             $\Leftrightarrow$$x_1x_2+y_1y_2=0$.
%     \end{enumerate}
%   \subsection{平面向量数量积相关量求解}
%     \begin{enumerate}[label=\arabic*)]
%       \item 向量夹角:设$\bm a=(x_1,y_1)$,$\bm b=(x_2,y_2)$,则
%         $\cos\vangle{\bm a}{\bm b}=\dfrac{\bm{a}\bm{\cdot}\bm{b}}{\abs{\bm{a}}\abs{\bm{b}}}=\dfrac{x_1x_2+y_1y_2}{\sqrt{x_1^2+y_1^2}\sqrt{x_2^2+y_2^2}} \quad \left(\vangle{\bm a}{\bm b}\in\left[0,\piup\right]$
%       \item 向量模长:若$\bm a=(x,y)$,则$\abs{\bm a}=\sqrt{\bm a\cdot\bm a}=\sqrt{x^2+y^2}$
%       \item 向量投影:向量$\bm a$在$\bm b$方向上的投影为  $\abs{\bm{a}}\cos\theta=\dfrac{\bm a\cdot\bm b}{\abs{\bm b}}$
%     \end{enumerate}
% \begin{exercise}{\textbf{习题}}
%   \item%【向量的线性运算】
%     (2017 \textbullet {\kaishu 广东深圳二模})如图所示,正方形$ABCD$中,$M$是$BC$的中点,若$\vv{AC}=\lambda\vv{AM}+\mu\vv{BD}$,则$\lambda+\mu$等于\xz
%     \xx{$\dfrac{4}3$}
%      {$\dfrac{5}3$}
%      {$\dfrac{15}8$}
%      {$2$}
%     \begin{center}
%       \begin{tikzpicture}
%         \coordinate[label=left:$A$](A)at(0,0);
%         \coordinate[label=right:$B$](B)at(3.5,0);
%         \coordinate[label=left:$D$](D)at(0,3.5);
%         \coordinate[label=right:$C$](C)at(3.5,3.5);
%         \coordinate[label=right:$M$](M)at($(B)!0.5!(C)$);
%         \draw (A)--(B)--(C)--(D)--cycle;
%         \draw[->,>=latex] (A)--(C);
%         \draw[->,>=latex] (A)--(M);
%         \draw[->,>=latex] (B)--(D);
%       \end{tikzpicture}
%     \end{center}
%     \begin{answer}
%       B
%     \end{answer}
%   \item%福州一中学2016-2017学年高一下学期期末考试数学…….doc-3【向量共线】
%     (2017 \textbullet {\kaishu 福州一中} 3)
%     已知向量$\bm a$,$\bm b$不共线,且$\bm c=\lambda\bm a+\bm b$,$\bm d=\bm a+(2\lambda-1)\bm b$,若$\bm c$与$\bm d$方向相反,则实数$\lambda$的值为\xz
%     \xx{$1$}
%      {$-\dfrac12$}
%      {$1$或$-\dfrac12$}
%      {$-1$或$-\dfrac12$}
%     \begin{answer}
%       B
%     \end{answer}
%   \item%福州三中2017高一下数学期末卷…….doc-5【向量投影,基底表示】
%     (2017 \textbullet {\kaishu 福州三中} 5)
%     设$\bm e_1$,$\bm e_2$为单位向量,且$\bm e_1$,$\bm e_2$的夹角为$\dfrac{\piup}3$,若$\bm a=\bm e_1-3\bm e_2$,$\bm b=\bm e_1+\bm e_2$,则向量$\bm a$在$\bm b$方向上的射影为\xz
%     \xx{$-\sqrt3$}
%      {$\sqrt3$}
%      {$-\dfrac{\sqrt{10}}5$}
%      {$\dfrac{\sqrt{10}}5$}
%     \begin{answer}
%       A
%     \end{answer}
%   \item%【向量坐标法在平面几何的应用,三角函数定义】
%     %如图,
%     半径为$\sqrt{3}$的扇形$AOB$的圆心角为120\degree,点$C$在$\arc{AB}$上,且$\angle{COB}=30\degree$,若$\vv{OC}=\lambda\vv{OA}+\mu\vv{OB}$,则$\lambda+\mu$等于\xz
%     \xx{$\sqrt{3}$}
%      {$\dfrac{\sqrt{3}}3$}
%      {$\dfrac{4\sqrt{3}}3$}
%      {$2\sqrt{3}$}
%     \begin{answer}
%       A
%     \end{answer}
%   \item%【平面向量几何应用:垂直问题】
%     直角坐标系$xOy$中,$\vv{AB}=(2,1)$,$\vv{AC}=(3,k)$,若$\triangle{ABC}$是直角三角形,则$k$的可能值个数是\xz
%     \xx{1}{2}{3}{4}
%     \begin{answer}
%       B
%     \end{answer}
%   \item%福建师大附中2016-2017高一下期末考试数学试题…….doc-6【数量积,三角形形状】
%     (2017 \textbullet {\kaishu 师大附中} 6)
%     若点$O$是$\triangle{ABC}$平面内一点,且满足$(\vv{OB}-\vv{OC})\cdot(\vv{OB}+\vv{OC}-2\vv{OA})=0$,则$\triangle{ABC}$形状为\xz
%     \xx{钝角三角形}{等腰三角形}{直角三角形}{锐角三角形}
%     \begin{answer}
%       B
%     \end{answer}
%   \item%福州三中2017高一下数学期末卷…….doc-16【向量投影,基底表示】
%     (2017 \textbullet {\kaishu 福州三中} 16)
%     已知$\bm a$,$\bm b$是平面内两个相互垂直的单位向量,若向量$\bm c$满足$(\bm a-\bm c)\cdot(\bm b-\bm c)=0$,则$|\bm c|$的最大值是\tk.
%     \begin{answer}
%       $\sqrt2$
%     \end{answer}
%   \item%%福建师大附中2016-2017高一下期末考试数学试题…….doc-15【向量夹角,线性运算模长】
%     (2017 \textbullet {\kaishu 师大附中} 15)
%     已知单位向量$\bm a$,$\bm b$的夹角为$\dfracp{}3$,那么$|\bm a-2\bm b|=$\tk.
%     \begin{answer}
%       $\sqrt3$
%     \end{answer}
%   \item%【平面向量的模与夹角】
%     已知$\triangle{ABC}$是正三角形,若$\vv{AC}-\lambda\vv{AB}$与向量$\vv{AC}$的夹角大于90\degree,则实数$\lambda$的取值范围是\tk.
%     \begin{answer}
%       $(2,+\infty)$
%     \end{answer}
%   \item%福建师大附中2016-2017高一下期末考试数学试题…….doc-17【数量积,几何】
%     (2017 \textbullet {\kaishu 师大附中} 17)
%     在$\triangle{ABC}$中,$|\vv{AD}|=|\vv{BD}|=|\vv{CD}|$,$|\vv{AB}|=3$,则$\vv{AB}\cdot\vv{AD}=$\tk.
%     \begin{answer}
%       $\dfrac92$
%     \end{answer}
%   \item%【向量坐标法在平面几何的应用,直线方程】
%     在$\mathrm{Rt}\triangle{ABC}$中,$CA=CB=2$,$M$,$N$是斜边$AB$上的两个动点,且$MN=\sqrt{2}$,则$\vv{CM}\cdot\vv{CN}$的取值范围是\tk.
%     \begin{answer}
%       $\Bigl[\dfrac{3}2,2\Bigr]$
%     \end{answer}
%   \item%福州一中学2016-2017学年高一下学期期末考试数学…….doc-14【数量积,外心】
%     (2017 \textbullet {\kaishu 福州一中} 14)
%     $\triangle{ABC}$中,$CA=4$,$CB=6$,点$O$为$\triangle{ABC}$的外心,则$\vv{CO}\cdot\vv{AB}=$\tk.
%     \begin{answer}
%       5
%     \end{answer}
%   \item%福建师大附中2016-2017高一下期末考试数学试题…….doc-19【向量共线、夹角、模长】
%     (2017 \textbullet {\kaishu 师大附中} 19)
%     已知$\bm a$,$\bm b$为两个不共线向量,$\abs{\bm a}=2$,$\abs{\bm b}=1$,$\bm c=2\bm a-\bm b$,$\bm d=\bm a+k\bm b$.\\
%     (I)若$\bm c\varparallel\bm d$,求实数$k$;\\
%     (II)若$k=-7$,且$\bm c\perp\bm d$,求$\bm a$与$\bm b$的夹角.
%     \begin{answer}
%       (I)$k=-\dfrac12$
%       (II)$\vangle{\bm a}{\bm b}=\dfrac{\piup}3$
%     \end{answer}
%   \vspace{3cm}
%   \item%福州一中学2016-2017学年高一下学期期末考试数学…….doc-15【数量积,垂直】
%     (2017 \textbullet {\kaishu 福州一中} 15)
%     已知$\bm a=(\cos\alpha,k\sin\alpha)$,$\bm b=(\cos\beta,\sin\beta)$($k>0$,$0<\alpha<\beta<\dfrac{\piup}2$),且$\bm a+\bm b$与$\bm a-\bm b$相互垂直.\\
%     (1)求$k$的值;\\
%     (2)若$\bm a\cdot\bm b=\dfrac45$且$\cos\beta=\dfrac35$,求$\sin\alpha$的值.
%     \begin{answer}
%       (1)$k=1$;
%       (2)$\sin\alpha=\dfrac7{25}$
%     \end{answer}
%     \vspace{4cm}
%   \vspace{3.5cm}
%   \item%【平面向量基本定理】
%     在$\triangle{OAB}$的边$OA$,$OB$上分别取点$M$,$N$,使得$\vv{OA}=3\vv{OM}$,$\vv{OB}=4\vv{ON}$,设线段$AN$与$BM$交于点$P$,
%     记$\vv{OA}=\bm a$,$\vv{OB}=\bm b$,用$\bm a$,$\bm b$表示向量$\vv{OP}$.
%     \begin{answer}
%       $\vv{OP}=\dfrac3{11}\bm a+\dfrac3{11}\bm b$
%     \end{answer}
%   \vspace{9cm}
%   \item%【平面向量基本定理】
%     在$\triangle{OAB}$中,$\vv{OA}=4\vv{OC}$,$\vv{OB}=2\vv{OD}$,设线段$AD$与$BC$交于点$M$,
%     记$\vv{OA}=\bm a$,$\vv{OB}=\bm b$.\\
%     (1)用$\bm a$,$\bm b$表示向量$\vv{OP}$.\\
%     (2)已知在线段$AC$上取一点$E$,在线段$BD$上取一点$F$,使$EF$过点$M$,设$\vv{OE}=p\vv{OA}$,$\vv{OF}=q\vv{OA}$,求证$\dfrac1{7p}+\dfrac3{7q}=1$
%     \begin{answer}
%       (1)$\vv{OP}=\dfrac17\bm a+\dfrac37\bm b$
%       (2)略
%     \end{answer}
%   \vspace{9cm}
% \end{exercise}

% \newpage
% \section{课后作业}
%   \begin{exercise}
%     \item%【向量的线性运算】
%       若点$D$在$\triangle{ABC}$的边$BC$上,且$\vv{CD}=4\vv{DB}=r\vv{AB}+s\vv{AC}$,则$3r+s$的值为\xz
%       \xx{$\dfrac{16}5$}
%        {$\dfrac{12}5$}
%        {$\dfrac{8}5$}
%        {$\dfrac{4}5$}
%       \begin{answer}
%         C
%       \end{answer}
%     \item%福州屏东中学2016-2017学年高一下学期期末考试数学试题.doc-4【向量共线】
%       (2017 \textbullet {\kaishu 屏东中学} 4)
%       若$A(-1,1)$,$B(1,3)$,$C(x,5)$,且$\vv{AB}=\lambda\vv{BC}$,则实数$\lambda$等于\xz
%       \xx{1}{2}{3}{4}
%       \begin{answer}
%         1
%       \end{answer}
%     \item%福建师大附中2016-2017高一下期末考试数学试题…….doc-3【向量投影,坐标表示】
%       (2017 \textbullet {\kaishu 师大附中} 3)
%       若$\bm a=(2,1)$,$\bm b=(3,4)$,则向量$\bm b$在向量$\bm a$方向上的投影为\xz
%       \xx{$2\sqrt5$}
%        {$2$}
%        {$\sqrt5$}
%        {$10$}
%       \begin{answer}
%         A
%       \end{answer}
%     \item%【向量表示】
%       设$D$为$\triangle{ABC}$所在平面内一点,$\vv{BD}=3\vv{CD}$,则\xz
%       \xx{$\vv{AD}=-\dfrac13\vv{AB}+\dfrac43\vv{AC}$}
%        {$\vv{AD}=\dfrac43\vv{AB}-\dfrac13\vv{AC}$}
%        {$\vv{AD}=\dfrac23\vv{AB}-\dfrac12\vv{AC}$}
%        {$\vv{AD}=-\dfrac12\vv{AB}+\dfrac32\vv{AC}$}
%       \begin{answer}
%         D
%       \end{answer}
%     \item%【向量夹角、模长】
%       已知$|\bm a|=1$,$\bm a\cdot\bm b=\dfrac12$,$|\bm a-\bm b|^2=1$,则$\bm a$与$\bm b$的夹角等于\xz
%       \xx{30\degree}{45\degree}{60\degree}{120\degree}
%       \begin{answer}
%         C
%       \end{answer}
%     \item%【向量共线】
%       已知向量$\bm a=(2,3)$,$\bm b=(-1,2)$,若$m\bm a+4\bm b$与$\bm a-2\bm b$共线,则$m$的值为\xz
%       \xx{$\dfrac12$}{$2$}{$-\dfrac12$}{$-2$}
%       \begin{answer}
%         D
%       \end{answer}
%     \item%【向量几何应用】
%       在平面四边形$ABCD$中,若$AC=3$,$BD=2$,则$(\vv{AB}+\vv{DC})\cdot(\vv{AC}+\vv{BD})=$\tk.
%       \begin{answer}
%         5
%       \end{answer}
%     \item%【向量垂直】
%       平面向量$\bm a=(\sqrt{3},-1)$,$\bm=\Bp{\dfrac12,\dfrac{\sqrt3}2}$,若存在不同时为0的实数$k$和$t$,使$\bm x=\bm a+(t^2-3)\bm b$,$\bm y=-k\bm a+t\bm b$,且$\bm x\perp\bm y$,试求函数关系式$k=f(t)$.
%       \begin{answer}
%         $k=f(t)=\dfrac14(t^3-3t)$
%       \end{answer}
%     \vspace{2.5cm}
%     \item%福州三中2017高一下数学期末卷…….doc-18【向量垂直,模长,共线】
%       (2017 \textbullet {\kaishu 福州三中} 18)
%       平面内的向量$\bm a=(3,2)$,$\bm b=(-1,2)$,$\bm c=(4,1)$.\\
%       (I)若$(\bm a+k\bm c)\perp(2\bm b-\bm a)$,求实数$k$的值;\\
%       (II)若向量$\bm d$满足$\bm d\varparallel\bm c$,且$\abs{\bm d}=\sqrt{34}$,求向量$\bm d$的坐标.
%       \begin{answer}
%         (I)$k=-\dfrac{11}{18}$
%         (II)$\bm d=(4\sqrt2,\sqrt2)$或$\bm d=(-4\sqrt2,-\sqrt2)$
%       \end{answer}
%     \vspace{5cm}
%     \item
%       已知点$P$是$\triangle{ABC}$内一点,且满足条件$\vv{AP}+\vv{AP}+\vv{AP}=\bm 0$,设点$Q$为$CP$的延长线与$AB$的交点,令$\vv{CP}=\bm p$,试用向量$\bm p$表示$\vv{CQ}$.
%       \begin{answer}
%         $\vv{CQ}=2\bm p$
%       \end{answer}
%     \vspace{6cm}
%     \item%《2018天利38套:高考真题单元专题训练(文)》专题18平面向量的概念与运算 P63p31【2010•江苏】
%       (2010 \textbullet {\kaishu 江苏})
%       在平面直角坐标系$xOy$中,已知点$A(-1,-2)$,$B(2,3)$,$C(-2,-1)$.\\
%       (I)求以线段$AB$,$AC$为邻边的平行四边形的两条对角线的长;\\
%       (II)设实数$t$满足$(\vv{AB}-t\vv{OC})\cdot\vv{OC}=0$,求$t$的值.
%       \begin{answer}
%         (I)两条对角线长分别为$4\sqrt2$,$2\sqrt{10}$;
%         (II)$t=-\dfrac{11}5$
%       \end{answer}
%     % \item%《2018天利38套:高考真题单元专题训练(理)ISBN978-7-223-03438-8》专题18平面向量的应用 P72p17【2014•陕西】
%     %   (2014 \textbullet {\kaishu 陕西})
%     %   在直角坐标系$xOy$中,已知点$A(1,1)$,$B(2,3)$,$C(3,2)$,点$P(x,y)$在$\triangle{ABC}$三边围成的区域(含边界)上.\\
%     %   (I)若$\vv{PA}+\vv{PB}+\vv{PC}=\bm 0$,求$\abs{\vv{OP}}$;\\
%     %   (II)设$\vv{OP}=m\vv{AB}+n\vv{AC}$($m,n\inR$),用$x$,$y$表示$m-n$,并求$m-n$的最大值.
%     %   \begin{answer}
%     %     (I)$\abs{\vv{OP}}=2\sqrt2$;
%     %     (II)$(x,y)=(m+2n,2m+n)$,$m-n$最大值为1.
%     %   \end{answer}
%   \end{exercise}
\stopexercise

\newpage
\section{参考答案}
\begin{multicols}{2}
  \printanswer
\end{multicols}

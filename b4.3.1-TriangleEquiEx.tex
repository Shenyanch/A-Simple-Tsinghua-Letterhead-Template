\Topic{三角恒等变换练习}
  \Teach{三角恒等变换的应用}
  \Grade{高一}
  % \Name{郑皓天}\FirstTime{20181207}\CurrentTime{20181207}
  % \Name{林叶}\FirstTime{20180908}\CurrentTime{20181125}
  %\Name{1v2}\FirstTime{20181028}\CurrentTime{20181117}
  % \Name{林叶}\FirstTime{20180908}\CurrentTime{20181125}
  % \Name{郭文镔}\FirstTime{20181111}\CurrentTime{20181231}
  % \Name{马灿威}\FirstTime{20181111}\CurrentTime{20181111}
  \newtheorem*{Theorem}{定理}
  \makefront
\vspace{-1.5em}
\startexercise
% \begin{exercise}{\heiti 课前检测}\\
%   表格实例:
%   \begin{center}
%     \renewcommand{\arraystretch}{1.4}
%     \begin{tabular}{|*{8}{c|}}
%       \hline
%         $x$
%         &$-\dfrac{\piup}6$
%         &$-\dfrac{\piup}3$
%         &$-\dfrac{5\piup}6$
%         &$-\dfrac{4\piup}3$
%         &$-\dfrac{11\piup}6$
%         &$-\dfrac{7\piup}3$
%         &$-\dfrac{17\piup}6$\\
%       \hline
%         $y$
%         &$-1$
%         &$1$
%         &$3$
%         &$1$
%         &$-1$
%         &$1$
%         &$3$\\
%       \hline
%     \end{tabular}\\
%   \end{center}
% \end{exercise}
\section{知识点总结}
  \begin{description}[leftmargin=0pt,labelsep=0pt]
    \item%[两角的和与差]
      \begin{itemizeMy}[两角的和与差\hspace{2em}]
        \item $\mathrm{C}_{\alpha\pm\beta}$:
        $\cos(\alpha\pm\beta)=\cos\alpha\cos\beta \mp \sin\alpha\sin\beta$
        \item $\mathrm{S}_{\alpha\pm\beta}$:
        $\sin(\alpha\pm\beta)=\sin\alpha\cos\beta \pm \cos\alpha\sin\beta$
        \item $\mathrm{T}_{\alpha\pm\beta}$:
        $\tan(\alpha\pm\beta)=\dfrac{\tan\alpha\pm \tan\beta}{1\mp\tan\alpha\tan\beta}$
      \end{itemizeMy}
    \item%[二倍角公式]
      \begin{itemizeMy}[二倍角公式\hspace{3em}]
        \item $\mathrm{S}_{2\alpha}$:
        $\sin{2\alpha}=2\sin\alpha\cos\alpha$
        \item $\mathrm{C}_{2\alpha}$:
        $\cos{2\alpha}=\cos^2{\alpha}-\sin^2{\alpha}=2\cos^2\alpha-1=1-2\sin^2\alpha$
        \item $\mathrm{T}_{2\alpha}$:
        $\tan{2\alpha}=\dfrac{2\tan\alpha}{1-\tan^2\alpha}$
      \end{itemizeMy}
      \item%[半角公式]
        \begin{itemizeMy}[半角公式\hspace{4em}]
          \item
          $\sin{\dfrac{\alpha}2}=\pm\sqrt{\dfrac{1-\cos\alpha}2}$
          \item $\cos{\dfrac{\alpha}2}=\pm\sqrt{\dfrac{1+\cos\alpha}2}$
          \item $\tan{\dfrac{\alpha}2}=\dfrac{\sin\alpha}{1+\cos\alpha}=\dfrac{1-\cos\alpha}{\sin\alpha}$
        \end{itemizeMy}
      \item%[万能公式]
        \begin{itemizeMy}[万能公式\hspace{4em}]
          \item $\sin{\alpha}=\dfrac{2\tan{\dfrac{\alpha}2}}{1+\tan^2{\dfrac{\alpha}2}}}$
          \item $\cos{\alpha}=\dfrac{1-\tan^2{\dfrac{\alpha}2}}{1+\tan^2{\dfrac{\alpha}2}}}$
          \item $\tan{\alpha}=\dfrac{2\tan{\dfrac{\alpha}2}}{1-\tan^2{\dfrac{\alpha}2}}}$
        \end{itemizeMy}
      \item%[万能公式]
        \begin{itemizeMy}[辅助角公式\hspace{3em}]
          \item $a\sin x+b\cos x=\sqrt{a^2+b^2}\sin(x+\varphi)$\\
          其中$\sin\varphi=\dfrac{b}{\sqrt{a^2+b^2}}$,$\cos\varphi=\dfrac{a}{\sqrt{a^2+b^2}}$\\
          $a>0$时,
          \item $a\sin x+b\cos x=\sqrt{a^2+b^2}\sin(x+\varphi)$\\
          其中$\tan\varphi=\dfrac{b}a$,$
          \varphi\in\Bigl(-\dfrac{\piup}2,\dfrac{\piup}2\Bigr)$
        \end{itemizeMy}
  \end{description}
\clearpage
\section{习题}
  \begin{exercise}
    \item%《2018天利38套:高考真题单元专题训练(理)ISBN978-7-223-03438-8》专题15三角恒等变换 P57p4【2008•山东】
      (2008 \textbullet {\kaishu 山东})已知$\cos{\Bigl(\alpha-\dfrac{\piup}6\Bigr)}+\sin\alpha=\dfrac{4}5\sqrt{3}}$,则$\sin{\Bigl(\alpha+\dfrac{7\piup}6\Bigr)}$的值是\xz
      \xx{$-\dfrac{2\sqrt{3}}5$}
       {$\dfrac{2\sqrt{3}}5$}
       {$-\dfrac{4}5$}
       {$\dfrac{4}5$}
      \begin{answer}
        C
      \end{answer}
    \item%《2018天利38套:高考真题单元专题训练(理)ISBN978-7-223-03438-8》专题15三角恒等变换 P57p5【2014•全国新课标】
      (2014 \textbullet {\kaishu 全国新课标})设$\alpha\in\Bigl(0,\dfrac{\piup}2\Bigr)$,$\beta\in\Bigl(0,\dfrac{\piup}2\Bigr)$,且$\tan\alpha=\dfrac{1+\sin\beta}{\cos\beta}$,则\xz
      \xx{$3\alpha-\beta=\dfrac{\piup}2$}
       {$3\alpha+\beta=\dfrac{\piup}2$}
       {$2\alpha-\beta=\dfrac{\piup}2$}
       {$2\alpha+\beta=\dfrac{\piup}2$}
      \begin{answer}
        C
      \end{answer}
    \item%《2018天利38套:高考真题单元专题训练(理)ISBN978-7-223-03438-8》专题15三角恒等变换 P57p7【2013•浙江】
      (2013 \textbullet {\kaishu 浙江})
      已知$\alpha\in\mathbb{R}$,$\sin\alpha+2\cos\alpha=\dfrac{\sqrt{10}}2$,则$\tan{2\alpha}=$\xz
      \xx{$\dfrac{4}3$}{$\dfrac{3}4$}{$-\dfrac{3}4$}{$-\dfrac{4}3$}
      \begin{answer}
        C
      \end{answer}
    \item%《2018天利38套:高考真题单元专题训练(理)ISBN978-7-223-03438-8》专题13三角函数的概念、... P49p7【2011•福建】
      (2011 \textbullet {\kaishu 福建})
      若$\tan\alpha=3$,则$\dfrac{\sin{2\alpha}}{\cos^2\alpha}$的值等于\xz
      \xx{2}{3}{4}{6}
      \begin{answer}
        D
      \end{answer}
    \item%《习题化知识清单》P87方法1【三角函数式化简】
      化简:
      $\sin{\Bigl(3x+\dfrac{\piup}3\Bigr)}\cos{\Bigl(x-\dfrac{\piup}6\Bigr)}+\cos{\Bigl(3x+\dfrac{\piup}3\Bigr)}\cos{\Bigl(x+\dfrac{\piup}3\Bigr)}=$\tk.
      \begin{answer}
        \cos{2x}
      \end{answer}
    \item%《2018天利38套:高考真题单元专题训练(理)ISBN978-7-223-03438-8》专题14三角函数的图像与性质 P54p16【2013•全国新课标】
      (2013 \textbullet {\kaishu 全国新课标})
      设当$x=\theta$时,函数$f(x)=\sin x-2\cos x$取得最大值,则$\cos\theta=$\tk.
      \begin{answer}
        $-\dfrac{2\sqrt{5}}5$
      \end{answer}
    \item%《习题化知识清单》P87方法1【三角函数式化简】
      函数$y=\sin{\Bigl(\dfrac{\piup}2+x\Bigr)\cos{\Bigl(\dfrac{\piup}6-x\Bigr)}}$的最大值为\tk.
      \begin{answer}
        $\dfrac{2+\sqrt{3}}4$
      \end{answer}
    \item%《2018天利38套:全国卷高考常考基础题(理)ISBN978-7-223-03393-0》练习8 三角恒等变换 P22p15
      已知$\cos(x+2\theta)+2\sin\theta\sin(x+\theta)=\dfrac{1}3$,则$\cos{2x}$的值为\tk.
      \begin{answer}
        $-\dfrac{7}9$
      \end{answer}
    \item%《2018天利38套:高考真题单元专题训练(理)ISBN978-7-223-03438-8》专题15三角恒等变换 P58p11【2017•江苏】
      (2017 \textbullet {\kaishu 江苏})
      若$\tan{\Bigl(\alpha-\dfrac{\piup}4\Bigr)}=\dfrac{1}6$,则$\tan\alpha=$\tk.
      \begin{answer}
        $\dfrac{7}5$
      \end{answer}
    \item%《2018天利38套:全国卷高考常考基础题(理)ISBN978-7-223-03393-0》练习8 三角恒等变换 P22p20
      已知$\sin{2\alpha}-2=2\cos{2\alpha}$,则$\sin^2\alpha+\sin{2\alpha}=$\tk.
      \begin{answer}
        $1$或$\dfrac{8}5$
      \end{answer}
    \item%《2018天利38套:高考真题单元专题训练(理)ISBN978-7-223-03438-8》专题15三角恒等变换 P58p16【2016•上海】
      (2016 \textbullet {\kaishu 上海})方程$3\sin x=1+\cos{2x}$在区间$[0,2\piup]$上的解为\tk.
      \begin{answer}
        $\dfrac{\piup}6$,$\dfrac{5\piup}6$
      \end{answer}
    % \item%《2018天利38套:高考真题单元专题训练(理)ISBN978-7-223-03438-8》专题15三角恒等变换 P58p17【2016•江苏】
    %   (2016 \textbullet {\kaishu 江苏})
    %   在锐角三角形$ABC$中,若$\sin{A}=2\sin{B}\sin{C}$,则$\tan{A}\tan{B}\tan{C}$的最小值是\tk.
    %   \begin{answer}
    %     8
    %   \end{answer}
    \item%《2018天利38套:高考真题单元专题训练(理)ISBN978-7-223-03438-8》专题15三角恒等变换 P59p20【2014•广东】
      (2014 \textbullet {\kaishu 广东})已知函数$f(x)=A\sin{\Bigl(x+\dfrac{\piup}4\Bigr)}$,$x\in\mathbb{R}$,且$f\Bigl(\dfrac{5\piup}{12}\Bigr)=\dfrac{3}2$.\\
      (I)求$A$的值;\\
      (II)若$f(\theta)+f(-\theta)=\dfrac{3}2$,$\theta\in \Bigl(0,\dfrac{\piup}2\Bigr)$,求$f\Bigl(\dfrac{3\piup}4-\theta\Bigr)$.
      \begin{answer}
        (I)$A=\sqrt{3}$;
        (II)$f\Bigl(\dfrac{3\piup}4-\theta\Bigr)=\dfrac{\sqrt{30}}4$
      \end{answer}
    \vspace{7cm}
    \item%《2018天利38套:高考真题单元专题训练(理)ISBN978-7-223-03438-8》专题15三角恒等变换 P59p19【2010•上海】
      (2010 \textbullet {\kaishu 上海})已知$0<x<\dfrac{\piup}2$,化简:\\
      $\lg\Bigl(\cos x\tan x+1-2\sin^2{\dfrac{x}2}\Bigr)+\lg\biggl[\sqrt{2}\cos {\Bigl(x-\dfrac{\piup}4\Bigr)}\biggr]-\lg(1+\sin{2x})$.
      \begin{answer}
        0
      \end{answer}
    \vspace{4cm}
    \item%《2018天利38套:高考真题单元专题训练(理)ISBN978-7-223-03438-8》专题14三角函数的图像与性质 P55p19【2016•天津】
      (2016 \textbullet {\kaishu 天津})
      已知函数$f(x)=4\tan x\sin{\Bigl(\dfrac{\piup}2-x\Bigr)}\cos{\Bigl(x-\dfrac{\piup}3\Bigr)}-\sqrt{3}$.\\
      (I)求$f(x)$的定义域与最小正周期;\\
      (II)讨论$f(x)$在区间$\Bigl[-\dfrac{\piup}4,\dfrac{\piup}4\Bigr]$上的单调性.
      \begin{answer}
        (I)$f(x)=2\sin{\Bigl(2x-\dfrac{\piup}3\Bigr)}$,定义域:$\Bigl\{x\Bigm|x\neq \dfrac{\piup}2+k\piup,k\in\mathbb{Z}\Bigr\}$;
        最小正周期:$T=\piup$.
        (II)$f(x)$在区间$\Bigl[-\dfrac{\piup}{12},\dfrac{\piup}4\Bigr]$上单调递增,在区间$\Bigl[-\dfrac{\piup}4,-\dfrac{\piup}{12}\Bigr]$上单调递减.
      \end{answer}
    \vspace{5.5cm}
    \item%《2018天利38套:高考真题单元专题训练(理)ISBN978-7-223-03438-8》专题13三角函数的概念、...  P52p13【2012•广东】
      (2012 \textbullet {\kaishu 广东})
      已知函数$f(x)=2\cos{\Bigl(\omega x+\dfrac{\piup}6\Bigr)}$(其中$\omega>0$,$x\in\mathbb{R}$)的最小正周期为$10\piup$.\\
      (I)求$\omega$的值\\
      (II)设$\alpha,\beta\in\Bigl[0,\dfrac{\piup}2\Bigr]$,$f{\Bigl(5\alpha+\dfrac{5\piup}3\Bigr)}=-\dfrac{6}5$,$f{\Bigl(5\beta-\dfrac{5\piup}6\Bigr)}=\dfrac{16}{17}$,求$\cos{(\alpha+\beta)}$的值.
      \begin{answer}
        (I)$\omega=\dfrac{1}5$.
        (II)$\sin\alpha=\dfrac{3}5$,$\cos\beta=\dfrac{8}{17}$,$\cos\alpha=\dfrac{4}5$,$\sin\beta=\dfrac{15}{17}$,$\cos{(\alpha+\beta)}=-\dfrac{13}{85}$.
      \end{answer}
\end{exercise}
\stopexercise

\newpage
\section{参考答案}
\begin{multicols}{2}
  \printanswer
\end{multicols}

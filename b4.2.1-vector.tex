\Topic{向量基本概念与线性运算}
  \Teach{向量共线定理及其运用}
  \Grade{高一}
  %\Name{1v2}\FirstTime{20181028}\CurrentTime{20181117}
  % \Name{林叶}\FirstTime{20180908}\CurrentTime{20181125}
  % \Name{郭文镔}\FirstTime{20181111}\CurrentTime{20181117}
  % \Name{马灿威}\FirstTime{20181111}\CurrentTime{20181111}
  \newtheorem*{Theorem}{定理}
  \makefront
\vspace{-1.5em}
\startexercise
\section{向量的基本相关概念}
  \begin{description}
    \item[有向线段]带有方向的线段.用$\vv{AB}$表示;线段AB的长度也叫做有向线段$\vv{AB}$的长度,记作$\abs{\vv{AB}}$.\par
      {\kaishu 有向线段包含三要素:{\textbf 起点、方向、长度}}
    \item[向量] 既有大小,又有方向的量,用$ \vv{a},~\vv{AB},~\bm{a} $表示;向量的大小即向量的长度或向量的模,用$ \abs{\bm{a}} $表示.\par
      \begin{itemize}[leftmargin=*]
        \kaishu
        \item 不同于有向线段,平面向量是自由向量(无源向量);
        \item 只有大小,没有方向的量称为数量;(物理学中通常称数量为标量,并把向量称为矢量)
      \end{itemize}
    \item[零向量] 长度为零的向量,其方向是任意的,记作$ \vv{0} $或$ \bm{0} $;
    \item[相等向量] 长度相等且方向相同的向量;\par
      {\kaishu 两个向量只能相等或者不相等,不能比较大小.}
    \item[相反向量] 长度相等且方向相反的向量\par
      {\kaishu 规定:$\bm{0}$的相反向量为$\bm{0}$}
    \item[单位向量] 长度等于一个单位长度的向量;\par
      {\kaishu 与向量$\bm a$方向相同的向量通常记为$\hat{\bm a}(=\dfrac{\bm a}{\abs{\bm a}})$}(一般手写为$\hat a$即可).
    \item[平行向量(共线向量)]方向相同或相反的非零向量叫做平行向量或共线向量;规定零向量与任一向量平行共线.
      向量$\bm{a}$、$\bm{b}$平行记作$\bm{a}\varparallel \bm{b}$.\par
      {\kaishu 向量平行不具有传递性}
    \clearpage
    \item[向量的夹角] 已知两个非零向量$\bm a$和$\bm b$,如图,做$\vv{OA}=\bm a$,$\vv{OB}=\bm b$,
      则$\angle{AOB}=\theta$叫做向量$\bm a$和$\bm b$的夹角.记作$\vangle{\bm a}{\bm b}$或$\vangle{\bm b}{\bm a}$.\par
      \begin{minipage}[b]{0.8\linewidth}
        \begin{itemize}
          \kaishu
          \item 向量夹角的取值范围:$[0,\piup]$ ;
          \item 当$\theta=0\degree$时,向量$\bm a,\bm b$共线且同向;
          \item 当$\theta=90\degree$时,向量$\bm a,\bm b$相互垂直,记作$\bm a\perp \bm b$;
          \item 当$\theta=180\degree$时,向量$\bm a,\bm b$共线且反向.
        \end{itemize}
      \end{minipage}\hfill
      \begin{minipage}[h]{0.2\linewidth}
        \vspace{-3.5cm}
        \begin{tikzpicture}[scale=1.5]
          \coordinate[lable=below:$O$](O) at (0,0);
          \coordinate[label=below:$A$](A) at (2,0);
          \coordinate[label=left:$B$](B) at(1,2);
          \draw[->,>=latex] (O)--(A)node[midway ,below](a){\small$\bm{a}$};
          \draw[->,>=latex] (O)--(B)node[midway, left](b){\small$\bm{b}$};
          \path (A)--(O)--(B) pic [draw,"$\theta$",angle eccentricity=1.5] {angle=A--O--B};
        \end{tikzpicture}
      \end{minipage}
  \end{description}
  \vspace{3cm}
  \begin{exercise}{\textbf{基础测试}}
    \item
      判断下列结论是否正确(请在括号中打“\checkmark”或“\XSolidBrush”)\\
      (1)向量与有向线段是一样的,因此可以用有向线段来表示向量.(  )\\
      (2)$\abs{\bm{a}}$与$\abs{\bm{b}}$是否相等与$\bm{a}$,$\bm{b}$的方向无关.(  )\\
      (3)若$\bm{a}\varparallel\bm{b}$,$\bm{b}\varparallel\bm{c}$,则$\bm{a}\varparallel\bm{c}$.(  )\\
      (4)若向量$\vv{AB}$与向量$\vv{CD}$是共线向量,则$A,B,C,D$四点在一条直线上.(  )\\
      (5)若向量$\vv{AB}$与向量$\vv{CD}$平行,则直线$AB$与$CD$平行.(  )\\
      (6)若向量$\bm a$与任一向量$\bm b$平行,则$\bm a=\bm 0$.(  )\\
      (7)若两个向量共线,则其方向必定相同或相反.(  )
      \begin{answer}
        (2)(5)正确
      \end{answer}
    \item
      有下列命题:
      \circled{1}两个相等向量,它们的起点相同,终点也相同;
      \circled{2}若$\abs{\bm{a}}=\abs{\bm{b}}$,则$\bm{a}=\bm{b}$;
      \circled{3}若$\abs{\vv{AB}}=\abs{\vv{CD}}$,则四边形$ABCD$是平行四边形;
      \circled{4}若$\bm{m}=\bm{n}$,$\bm{n}=\bm{k}$,则$\bm{m}=\bm{k}$;
      \circled{5}位移、速率、重力加速度都是向量;
      \circled{6}共线的向量,若起点不同,则终点一定不同.其中,错误的个数是\xz
      \xx{2}{3}{4}{5}
    \item
      正方形$ABCD$中,向量$\vv{AC}$与$\vv{BC}$的夹角为\tk,向量$\vv{AC}$与$\vv{CD}$的夹角为\tk.
    \item
      在平面内,若将所有单位向量的起点平移到同一点,则它们的终点构成的图形为\tk.
  \end{exercise}
\newpage
\section{向量的线性运算}
  向量的线性运算包括向量的加、减、数乘运算.
  \subsection{加法}
    \begin{description}
      \item[定义] 两个向量和的运算;
      \item[法则] 平行四边形法则或三角形法则
        \begin{center}
        \begin{tikzpicture}
          \coordinate(O) at (0,0);
          \coordinate(A) at (2,0);
          \coordinate(B) at(1,2);
          \coordinate(C) at ($(A)+(B)$);
          \draw[->,>=latex] (O)--(A)node[midway,below](a){\small$\bm{a}$};
          \draw[->,>=latex] (O)--(C)node[midway,above,sloped](c){\small$\bm{a}+\bm{b}$};
          \draw[->,>=latex] (A)--(C)node[midway, below](b){\small$\bm{b}$};
          \begin{scope}[xshift=4cm]
            \coordinate(O) at (0,0);
            \coordinate(A) at (2,0);
            \coordinate(B) at(1,2);
            \coordinate(C) at ($(A)+(B)$);
            \draw[->,>=latex] (O)--(A)node[midway ,below](a){\small$\bm{a}$};
            \draw[->,>=latex] (O)--(C)node[midway ,below,sloped](c){\small$\bm{a}+\bm{b}$};
            \draw[->,>=latex] (O)--(B)node[midway, left](b){\small$\bm{b}$};
            \draw[dashed](B)--(C);
            \draw[dashed](A)--(C);
          \end{scope}
        \end{tikzpicture}
        \end{center}
        {\kaishu 对于零向量与任一向量$\bm{a}$,规定$$\bm{a}+\bm{0}=\bm{0}+\bm{a}=\bm{a}$$}\par
        由三角形法则,可得向量不等式(有时称作“三角形不等式”):
        \[\bigm|{\abs{\bm{a}}-\abs{\bm{b}}\bigm|}\leqslant \abs{\bm{a}+\bm{b}}\leqslant \abs{\bm{a}}+\abs{\bm{b}}\]
        若$\bm a$和$\bm b$为非零向量,则:当$\bm a$与$\bm b$反向时, 左边等式成立;当$\bm a$与$\bm b$同向时, 右边等式成立;\par
      \item[运算律]
        \begin{itemize}%[leftmargin=*]
          \item 交换律:$\bm{a}+\bm{b}=\bm{b}+\bm{a}$\
          \item 结合律:$(\bm{a}+\bm{b})+\bm{c}=\bm{a}+(\bm{b}+\bm{c})$
        \end{itemize}
    \end{description}
  \subsection{减法}
    \begin{description}
      \item[定义]减去一个向量相当于加上这个向量的相反向量,即$$\bm{a}-\bm{b}=\bm{a}+(\bm{-b})$$
      \item[运算法则]三角形法则、平行四边形法则.%$\vv{AB}-\vv{AC}=\vv{CB}$.
      \begin{center}
      \begin{tikzpicture}
        \coordinate(O) at (0,0);
        \coordinate(A) at (2,0);
        \coordinate(B) at(1.5,1.5);
        \draw[->,>=latex] (O)--(A)node[midway,below](a){\small$\bm{b}$};
        \draw[->,>=latex] (O)--(B)node[midway, left](a){\small$\bm{a}$};
        \draw[->,>=latex](A)--(B)node[midway, above,sloped](a){\small$\bm{a}-\bm{b}$};
        \begin{scope}[xshift=6cm]
          \coordinate(O) at (0,0);
          \coordinate(B) at (2,0);
          \coordinate(A) at(1.5,1.5);
          \coordinate(B1) at (-2,0);
          \coordinate(C)at($(B1)+(A)$);
          \draw[->,>=latex] (O)--(B)node[midway,below](a){\small$\bm{b}$};
          \draw[->,>=latex] (O)--(A)node[midway, left](a){\small$\bm{a}$};
          \draw[->,>=latex](B)--(A)node[midway, above,sloped](a){\small$\bm{a}-\bm{b}$};
          \draw[->,>=latex](O)--(B1)node[midway,below](b1){\small$\bm{-b}$};
          \draw[->,>=latex](O)--(C)node[midway,below,sloped](b1){\small$\bm{a-b}$};
          \draw[dashed] (B1)--(C) (A)--(C);
        \end{scope}
      \end{tikzpicture}
      \end{center}
      {\kaishu
        对于任意一点$P$,$\vv{AB}=\vv{PB}-\vv{PA}$,
      }
    \end{description}
  \subsection{数乘}
    \begin{description}
      \item[定义] 求实数$ \lambda $与向量$\bm{a}$的积是一个向量,记作$\lambda\bm{a}$,长度与方向由以下法则规定:
      \item[法则]
        \begin{enumerate}[label=\arabic*)]
          \item $\abs{\lambda \bm{a}}=\abs{\lambda}\abs{\bm{a}} $;
          \item
            \begin{itemize}
              \item 当$ \lambda>0 $时,$ \lambda\bm{a} $的方向与$\bm{a}$的方向相同;
              \item 当$ \lambda<0 $时,$ \lambda\bm{a} $的方向与$\bm{a}$的方向相反;
              \item 当$ \lambda=0 $时,$ \lambda\bm{a}=\bm 0 $.
            \end{itemize}
        \end{enumerate}
      \item[运算律]
        设$\lambda,\mu\in\mathbb{R}$,则:\par
        \begin{itemize}
            \item $\lambda(\mu\bm{a})=(\lambda\mu)\bm{a}$;
            \item $(\lambda+\mu)\bm{a}=\lambda\bm{a}+\mu\bm{a}$;
            \item $\lambda(\bm{a}+\bm{b})=\lambda\bm{a}+\lambda\bm{b}$.
        \end{itemize}
      对于任意向量$\bm a,\bm b$以及任意实数$\lambda$,$\mu_1$,$\mu_2$,恒有:
      \[\lambda({\mu_1\bm a}\pm{\mu_2\bm b})={\lambda\mu_1\bm a}\pm{\lambda\mu_2\bm b}\]
    \end{description}
    \begin{Theorem}[向量共线定理]
      向量$\bm{a}~(\bm{a}\ne\bm{0})$与向量$\bm{b}$共线,当且仅当存在唯一的实数$ \lambda $,使得$\bm{b}=\lambda\bm{a}$.
    \end{Theorem}
    {\kaishu
     证明三点共线的方法:\circled{1}$\vv{AB}=\lambda\vv{AC}$,则$A$,$B$,$C$三点共线;\circled{2}$\vv{OA}=\lambda\vv{OB}+\mu\vv{OC}$,若$\lambda+\mu=1$,则$A$,$B$,$C$三点共线.
    }
  \clearpage
  \begin{exercise}{\textbf{基础测试}}
    \item
      如图,$\vv{AB}+\vv{BC}-\vv{AD}$等于\xz
      \begin{minipage}[b]{0.7\linewidth}
      	\xx{$\vv{AD}$}{$\vv{DC}$}{$\vv{DB}$}{$\vv{AB}$}
      \end{minipage}\hfill
      \begin{minipage}[h]{0.3\linewidth}
        \begin{tikzpicture}
          \coordinate[label=left:$B$](B)at(0,0);
          \coordinate[label=right:$C$](C)at(3,0);
          \coordinate[label=above:$A$](A)at(1.5,2);
          \coordinate[label=below:$D$](D)at($(B)!0.4!(C)$);
          \draw (A)--(B)--(C)--cycle;
          \draw (A)--(D);
        \end{tikzpicture}
      \end{minipage}
      \begin{answer}
        B
      \end{answer}
    \item%1平面向量的基本概念.pdf P10-训练1
      判断下列结论是否正确(请在括号中打“\checkmark”或“\XSolidBrush”)\\
      (1)若向量$\bm b$与向量$\bm{a}$共线,则存在唯一的实数$ \lambda $,使得$\bm{b}=\lambda\bm{a}$.(\hspace{2em})\\
      (2)若$\bm{b}=\lambda\bm{a}$,则$\bm a$与$\bm b$共线.(\hspace{2em})\\
      (3)若$\lambda\bm a=\bm 0$,则$\bm a=\bm 0$.(\hspace{2em})\\
    \item
      如图所示,在五边形$ABCDE$中,若四边形$ACDE$是平行四边形,且$\vv{AB}=\bm{a}$,$\vv{AC}=\bm{b}$,$\vv{AE}=\bm{c}$,试用向量$\bm a$,$\bm b$,$\bm c$表示向量$\vv{BD}$,$\vv{BC}$,$\vv{BE}$,$\vv{CD}$及$\vv{CE}$.
      \begin{flushright}
        \begin{tikzpicture}
          \coordinate[label=left:$D$](D)at(0,0);
          \coordinate[label=right:$E$](E)at(3,0);
          \coordinate[label=above:$C$](C)at(1,1.5);
          \coordinate[label=below:$A$](A)at($(C)+(E)$);
          \coordinate[label=below:$B$](B)at(3.5,2.5);
          \draw (A)--(B)--(C)--(D)--(E) --cycle;
          \draw (B)--(D) (B)--(E) (C)--(E);
          \draw[->,>=latex] (A)--(B);
          \draw[->,>=latex] (A)--(C);
          \draw[->,>=latex] (A)--(E);
        \end{tikzpicture}
      \end{flushright}
      \begin{answer}
        $\vv{BD}=-\bm a+\bm c+\bm b$;$\vv{BC}=\bm b-\bm a$;$\vv{BE}=\bm a-\bm a$;$\vv{CD}=\bm c$;$\vv{CE}=\bm c-\bm b$.
      \end{answer}
    \item
      \begin{enumerate}[label=\arabic*)]
        \item $3(6\bm{a}+\bm{b})-9(\bm{a}+\dfrac13\bm{b})=$\tk;
        \item 若$2(\bm{y}-\dfrac13\bm{a})-\dfrac12(\bm c+\bm b-3\bm y)+\bm b=0$其中$\bm a$,$\bm b$,$\bm c$为已知向量,则未知向量$\bm y=$\tk.
        \item 若$\bm a=\bm b+\bm c$,化简$3(\bm a+2\bm b)-2(3\bm b+\bm c)-2(\bm a+\bm b)=$\tk.
      \end{enumerate}
      \begin{answer}
        (1)$9\bm a$;(2)$\dfrac4{21}\bm a-\dfrac17\bm b+\dfrac17\bm c$;$-\bm a$.
      \end{answer}
    \item%《习题化知识清单》P81知识4-3【向量共线】
      已知向量$\bm a$、$\bm b$,且$\vv{AB}=\bm a+2\bm b$,$\vv{BC}=-5\bm a+6\bm b$,$\vv{CD}=7\bm a-2\bm b$,则一定共线的三点是\xz
      \xx{$A$、$B$、$D$}
        {$A$、$B$、$C$}
        {$B$、$C$、$D$}
        {$A$、$C$、$D$}
      \begin{answer}
        A
      \end{answer}
    \item%《习题化知识清单》P81知识4-4【向量线性运算、向量共线】
      已知向量$\bm a=\bm e_1+2\bm e_2$,$\bm b=2\bm e_1-\bm e_2$,则$\bm a+2\bm b$与$2\bm a-\bm b$\xz
      \xx{一定共线}
        {一定不共线}
        {当且仅当$\bm e_1$与$\bm e_2$共线时共线}
        {当且仅当$\bm e_1=\bm e_2$时共线}
      \begin{answer}
        C
      \end{answer}
  \end{exercise}
\newpage
\section{习题}
  \begin{exercise}
    \item%1平面向量的基本概念.pdf P3训练3
      一辆汽车从$A $点出发向西行驶了100 km 到达$B $点,然后又改变方向向西偏北$50\degree$走了200 km到达$C$ 点,最后又改变方向,向东行驶了100 km 到达$D$ 点.\\
      (1)作出向量$\vv{AB}$,$\vv{BC}$,$\vv{CD}$;\\
      (2)求$\abs{\vv{AD}}$.\\
    \vspace{4cm}\\
    \begin{minipage}[b]{0.65\linewidth}
    \item%LaTeX-master/xiangliang/xiangliangsorting.tex P10-p48
      在$\triangle ABC$中,点$ M$,$N $满足$ \vv{AM}=2\vv{MC}$,$\vv{BN}=\vv{NC}$.若$\vv{MN}=x\vv{AB}+y\vv{AC}$,则$ x= $\tk;$ y= ~$ \tk.
      \begin{answer}
        $x=\dfrac12$;$y=-\dfrac16$
      \end{answer}
    \end{minipage}
    \begin{minipage}[htbp!]{0.3\linewidth}
      \begin{center}
      \begin{tikzpicture}
        \draw(0,0)node[below](B){\small$B$}--(1,0)node[below](N){\small$N$}--(2,0)node[below](C){\small$C$};
        \draw (0,0)--(1.1,2.1)node[above](A){\small$A$}--(2,0);
        \draw (1,0)--(1.1,2.1);
        \draw(1,0)--($(1.1,2.1)!0.7!(2,0)$)node[right](M){\small$M$};
      \end{tikzpicture}
      \end{center}
    \end{minipage}
    \item
      (2018届贵州遵义航天高级中学一模)如图所示,向量$\vv{OA}=\bm{a}$,$\vv{OB}=\bm{b}$,$\vv{OC}=\bm{c}$,$A$,$B$,$C$在一条直线上,且$\vv{AC}=3\vv{BC}$,则\xz
      \begin{minipage}[b]{0.7\linewidth}
        \xx{$\bm{c}=\dfrac32\bm{b}-\dfrac12\bm{a}$}
          {$\bm{c}=\dfrac32\bm{a}-\dfrac12\bm{b}$}
          {$\bm{c}=-\bm{a}+2\bm{b}$}
          {$\bm{c}=\bm{a}+2\bm{b}$}
      \end{minipage}\hfill
      \begin{minipage}[htbp!]{0.3\linewidth}
        \begin{center}
        \begin{tikzpicture}
          \coordinate[label=left:$O$](O)at(0,0);
          \coordinate[label=right:$C$](C)at(3,0);
          \coordinate[label=left:$A$](A)at(-1,2.5);
          \coordinate[label=right:$B$](B)at($(A)!0.66!(C)$);
          \draw (A)--(B)--(C)--cycle;
          \draw[->,>=latex] (O)--(C);
          \draw[->,>=latex] (O)--(A);
          \draw[->,>=latex] (O)--(B);
        \end{tikzpicture}
        \end{center}
      \end{minipage}
      \begin{answer}
        A
      \end{answer}
    \item
      设向量$\bm a$,$\bm b$不共线,向量$\lambda\bm a+\bm b$与$\bm a+2\bm b$共线,则实数$\lambda=$\tk.
      \begin{answer}
        $\dfrac12$
      \end{answer}
    \vspace{2em}
    \item
      如图,在$\triangle ABC$中,$D$,$E$为边$AB$的两个三等分点,$\vv{CA}=3\bm a$,$\vv{CB}=2\bm b$,求$\vv{CD}$,$\vv{CE}$(用$\bm a$,$\bm b$表示).
      \begin{flushright}
        \begin{tikzpicture}
          \coordinate[label=left:$B$](B)at(0,0);
          \coordinate[label=right:$C$](C)at(2.5,0);
          \coordinate[label=above:$A$](A)at(3,3);
          \coordinate[label=above:$E$](E)at($(B)!0.33!(A)$);
          \coordinate[label=above:$D$](D)at($(B)!0.66!(A)$);
          \draw (A)--(B)--(C) --cycle;
          \draw (C)--(D) (C)--(E);
          \draw[->,>=latex] (C)--(A);
          \draw[->,>=latex] (C)--(B);
        \end{tikzpicture}
      \end{flushright}
      \begin{answer}
        $\vv{CD}=2\bm a+\dfrac23\bm b$;$\vv{CE}=\bm a+\dfrac43\bm b$
      \end{answer}
    \item%1平面向量的基本概念.pdf P10例2
      设$\bm a$,$\bm b$是不共线的两个非零向量.\\
      (1)若$\vv{OA}=2\bm a-\bm b$,$\vv{OB}=3\bm a+\bm b$,$\vv{OC}=\bm a-3\bm b$,求证:$A$,$B$,$C$三点共线;\\
      (2)若$8\bm a+k\bm b$与$k\bm a+\2\bm b$共线,求实数$k$的值;\\
      (3)若$\vv{OM}=m\bm a$,$\vv{ON}=n\bm b$,$\vv{OP}=\alpha\bm a+\beta\bm b$,其中$m$,$n$,$\alpha$,$\beta$均为实数,且$m,n\neq 0$,若$M$,$P$,$N$三点共线,求证:$\dfrac{\alpha}m+\dfrac{\beta}n=1$
      \begin{answer}
        (1)$\because \vv{AB}=\bm a+2\bm b$,$\vv{CB}=2\bm a+4\bm b$;$\therefore \vv{CB}=2\vv{AB}$;
        (2)$k=2\sqrt2$;
      \end{answer}
      \vspace{5cm}
    \item
      设点$G$为$\triangle ABC$重心,$D$,$E$,$F$分别为各边中点.试用向量证明:$AG=\dfrac23 AD$.
      \begin{flushright}
        \begin{tikzpicture}
          \coordinate[label=left:$B$](B)at(0,0);
          \coordinate[label=right:$C$](C)at(3,0);
          \coordinate[label=above:$A$](A)at(2,2);
          \coordinate[label=below:$D$](D)at($(B)!0.5!(C)$);
          \coordinate[label=right:$E$](E)at($(A)!0.5!(C)$);
          \coordinate[label=above:$F$](F)at($(B)!0.5!(A)$);
          \coordinate[label=above:$G$](G)at($(D)!0.33!(A)$);
          \draw (A)--(B)--(C) --cycle;
          \draw (A)--(D) (B)--(E) (C)--(F);
        \end{tikzpicture}
      \end{flushright}
    \vspace{2cm}
  \end{exercise}
\clearpage
\section{课后作业}
  \begin{exercise}
    \item%1平面向量的基本概念.pdf P2-训练1
      判断下列结论是否正确(请在括号中打“\checkmark”或“\XSolidBrush”)\\
      (1)向量就是有向线段.(  )\\
      (2)如果$\abs{\vv{AB}}>\abs{\vv{CD}}$,那么$\vv{AB}>\vv{CD}$.(  )\\
      (3)力、速度和质量都是向量.(  )\\
      (4)若$\bm a$,$\bm b$都是单位向量,则$\bm a=\bm b$.(  )\\
      (5)若$\bm a=\bm b$,且$\bm a$与$\bm b$的起点相同,则终点也相同.(  )\\
      (6)零向量的大小为0,没有方向.(  )
      \begin{answer}
        (5)正确,其余皆误.
      \end{answer}
    \item
      给出下列命题:
      \ding{192}两个具有公共终点的向量,一定是共线向量;
      \ding{193}两个向量不能比较大小,但它们的模能比较大小;
      \ding{194}$\lambda\bm{a}=\bm{0}$($\lambda$为实数),则$\lambda$必为零;
      \ding{195}$\lambda$,$\mu$为实数,若$\lambda\bm{a}=\mu\bm{b}$,则$\bm{a}$与$\bm{b}$共线.
      其中正确的命题的个数为\xz
      \xx{1}{2}{3}{4}
      \begin{answer}
        A
      \end{answer}
    \item
      (2018·安徽淮北第一中学最后一卷)设$\bm{a}$,$\bm{b}$都是非零向量,下列四个条件,使$\dfrac{\bm{a}}{\abs{\bm{a}}}=\dfrac{\bm{b}}{\abs{\bm{b}}}$成立当且仅当\xz
      \xx{$\bm a=\bm b$}
      {$\bm a=2\bm b$}
      {$\bm a\varparallel\bm b$且$\abs{\bm a}=\abs{\bm b}$}
      {$\bm a\varparallel\bm b$且方向相同}
      \begin{answer}
        D
      \end{answer}
    \item
      已知四边形$ABCD$ 是菱形,则下列等式中成立的是\xz
      \xx{$\vv{AB}+\vv{BC}=\vv{CA}$}
        {$\vv{AB}+\vv{AC}=\vv{BC}$}
        {$\vv{AC}+\vv{BA}=\vv{AD}$}
        {$\vv{AC}+\vv{AD}=\vv{DC}$}
      \begin{answer}
        C
      \end{answer}
    \item%1平面向量的基本概念.pdf P13-4
      已知$AM$ 是$\triangle ABC$ 的边$BC$ 上的中线,若$\vv{AB}=\bm a$,$\vv{AC}=\bm b$,则$\vv{AM}$等于\xz
      \xx{$\dfrac12(\bm a-\bm b)$}
        {$-\dfrac12(\bm a-\bm b)$}
        {$\dfrac12(\bm a+\bm b)$}
        {$-\dfrac12(\bm a+\bm b)$}
      \begin{answer}
        C
      \end{answer}
    \item%《习题化知识清单》P81知识4-1【向量共线】
      已知向量$\bm a$、$\bm b$不共线,$\bm c=k\bm a+\bm b({k\in\mathbb{R}})$,$\bm d=\bm a-\bm b$。如果$\bm c\varparallel \bm d$,那么\xz
      \xx{$k=1$且$\bm c$与$\bm d$同向}
        {$k=1$且$\bm c$与$\bm d$反向}
        {$k=-1$且$\bm c$与$\bm d$同向}
        {$k=-1$且$\bm c$与$\bm d$反向}
      \begin{answer}
        D
      \end{answer}
    \item%1平面向量的基本概念.pdf P13-10
      化简:\\
      \circled{1} $\vv{BC}+\vv{AB}$; \hspace{2em} \circled{2} $\vv{DB}+\vv{CD}+\vv{BC}$;\\
      \circled{3} $\vv{AB}-\vv{FD}+\vv{CD}-\vv{CB}+\vv{FA}$;\hspace{2em} \circled{4} $(\vv{AC}+\vv{BO}+\vv{OA})-(\vv{DC}-\vv{DO}-\vv{OB})$;\\
      \begin{answer}
        \circled{1}$\vv{AC}$;\circled{2}$\bm 0$;\circled{3}$\bm 0$;\circled{4}$\bm 0$
      \end{answer}
    \vspace{1.5cm}
    \clearpage
    \item%1平面向量的基本概念.pdf P8训练5
      一架飞机从$A$ 地按北偏东$35\degree$的方向飞行800 km 到达$B$ 地接到受伤人员,然后又从$B$ 地按南偏东$55\degree$的方向飞行600 km 送往$C $地医院,求这架飞机飞行的路程及两次位移的和.
      \begin{answer}
        路程1400km,位移1000km.
      \end{answer}
    \vspace{3cm}
    \item
      设点$G$为$\triangle ABC$重心,$D$,$E$,$F$分别为各边中点.\\
      (1)试用向量证明:三角形三条中线共点;
      (2)求$\vv{AD}+\vv{BE}+\vv{CF}$.
      \begin{flushright}
        \begin{tikzpicture}
          \coordinate[label=left:$B$](B)at(0,0);
          \coordinate[label=right:$C$](C)at(3,0);
          \coordinate[label=above:$A$](A)at(2,2);
          \coordinate[label=below:$D$](D)at($(B)!0.5!(C)$);
          \coordinate[label=right:$E$](E)at($(A)!0.5!(C)$);
          \coordinate[label=above:$F$](F)at($(B)!0.5!(A)$);
          \coordinate[label=above:$G$](G)at($(D)!0.33!(A)$);
          \draw (A)--(B)--(C) --cycle;
          \draw (A)--(D) (B)--(E) (C)--(F);
        \end{tikzpicture}
      \end{flushright}
      \vspace{1cm}
    \item
      已知$\vv{OA}=\lambda\vv{OB}+\mu\vv{OC}$($\lambda,\mu\in\mathbb{R}$),若$\lambda+\mu=1$,求证:点$A$,$B$,$C$三点共线.\\
      \vspace{3.5cm}
    \item
      【定比分点坐标公式】如图,设$P$为$\triangle ABO$边$AB$上一点.设$\vv{OA}=\bm a$,$\vv{OB}=\bm b$\vspace{8pt}\\
      (1)求证:$\vv{OP}=\dfrac{\abs{\vv{PB}}}{\abs{\bm b-\bm a}}\bm a+\dfrac{\abs{\vv{PA}}}{\abs{\bm b-\bm a}}\bm b$;\vspace{8pt}\\
      (2)设$\vv{AP}=\lambda\vv{PB}$,求证:$\vv{OP}=\dfrac{\bm a+\lambda\bm b}{1+\lambda}$
      \begin{flushright}\vspace{-3cm}
        \begin{tikzpicture}
          \coordinate[label=left:$O$](O)at(0,0);
          \coordinate[label=right:$A$](A)at(3,0);
          \coordinate[label=above:$B$](B)at(2,2);
          \coordinate[label=right:$P$](P)at($(B)!0.4!(A)$);
          \draw (A)--(B)--(O) --cycle;
          \draw[->,>=latex] (O)--(A);
          \draw[->,>=latex] (O)--(B);
          \draw[->,>=latex] (O)--(P);
        \end{tikzpicture}
      \end{flushright}
  \end{exercise}
\stopexercise
\newpage
\section{部分参考答案}
\printanswer

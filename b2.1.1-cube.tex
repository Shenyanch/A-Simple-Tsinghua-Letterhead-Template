\Topic{}
  \Teach{}
  \Grade{高一}
  % \Name{郑皓天}\FirstTime{20181207}\CurrentTime{20181207}
  % \Name{林叶}\FirstTime{20180908}\CurrentTime{20181125}
  %\Name{1v2}\FirstTime{20181028}\CurrentTime{20181117}
  % \Name{林叶}\FirstTime{20180908}\CurrentTime{20181125}
  % \Name{郭文镔}\FirstTime{20181111}\CurrentTime{20181117}
  % \Name{马灿威}\FirstTime{20181111}\CurrentTime{20181111}
  % \Name{黄亭燏}\FirstTime{20181231}\CurrentTime{20181231}
  \newtheorem*{Theorem}{定理}
  \makefront
\vspace{-1.5em}
\startexercise
% \begin{exercise}{\heiti 课前检测}\\
%   表格实例:
%   \begin{center}
%     \renewcommand{\arraystretch}{1.4}
%     \begin{tabular}{|*{8}{c|}}
%       \hline
%         $x$
%         &$-\dfrac{\piup}6$
%         &$-\dfrac{\piup}3$
%         &$-\dfrac{5\piup}6$
%         &$-\dfrac{4\piup}3$
%         &$-\dfrac{11\piup}6$
%         &$-\dfrac{7\piup}3$
%         &$-\dfrac{17\piup}6$\\
%       \hline
%         $y$
%         &$-1$
%         &$1$
%         &$3$
%         &$1$
%         &$-1$
%         &$1$
%         &$3$\\
%       \hline
%     \end{tabular}\\
%   \end{center}
% \end{exercise}
\section{空间几何体的结构}
  \begin{description}
    \item[空间几何体] 空间中的物体都占据着空间的一部分,如果我们只考虑物体的形状和大小,而不考虑其他因素,那么由这些物体抽象出来的空间图形叫做空间几何体.
    \item[多面体] 由若干个平面多边形围成的集合体叫做多面体.
      围成多面体的各个多边形叫做多面体的面;相邻两个面的公共边叫做多面体的棱;棱与棱的公共点叫做多面体的顶点.
    \item[旋转体] 由一个平面图形绕着它所在平面内的一条定直线旋转所形成的封闭几何体叫做旋转体.该定直线叫做旋转体的轴.
    \item[曲面] 曲面可看作直线或曲线沿着一定曲线运动所形成的轨迹.
      此运动的直线或曲线,称为曲面的母线;而定曲线则称为曲面的准线;由直线运动形成的曲面称为直线面(或直纹曲面).
      \begin{itemize}[leftmargin=*]
        \kaishu
        \item 柱面:平行于某定方向且与定曲线相交的所有直线构成的曲面(始终平行于一定直线的直母线沿曲线运动的轨迹).
        \item 锥面:由定曲线上各点与不在其上的定点确定的所有直线所构成的曲面(始终通过定点(导点)的直母线沿着曲线运动的轨迹).
      \end{itemize}
    \item[柱体] 由封闭准线的柱面被不平行与母线的两个平行平面所截得到的封闭几何体叫做柱体(简称柱).
      平行截面称为柱体的底面(简称底);两底间的柱面部分称为柱体的侧面,两底间的距离称为柱体的高,柱面母线被截得的线段称为柱体的母线.
      % 平面上一封闭图形S沿直线方向(不与原图形所在平面平行)平移一定距离得到的封闭几何体叫做柱体(简称柱).
      % 封闭图形S在始末位置的图形称为柱体的底面(简称底);其余外表部分称为柱体的侧面,两底间的距离称为柱体的高,S边界上一点的运动轨迹(一段线段)称为柱体的母线.
      \begin{itemize}[leftmargin=*]
        \kaishu
        \item 棱柱:底面为多边形的柱体称为棱柱,底面为n边形时即称作n棱柱.母线垂直于底面时称为直棱柱;否则称为斜棱柱;底面为正多边形的直棱柱称为正棱柱.
        \item 圆柱:底面为圆的柱体称为圆柱.
      \end{itemize}
    \item[椎体] 由不过锥面顶点且与所有母线相交的平面截得的封闭几何体叫做椎体(简称椎).
      截面称为椎体的底面(简称底);位于顶点和底面之间的锥面部分称为椎体的侧面,锥面母线在锥顶与底面间的部分称为椎体的母线;顶点到底面的垂线段称为椎体的高.
      % 设平面封闭图形S以及平面外一定点P,当点A在S边界上运动时,线段PA的运动轨迹与平面图形S构成的封闭几何体叫做椎体(简称椎).
      % 平面封闭图形S称为椎体的底面(简称底);其余外表部分称为椎体的侧面,线段PA称为椎体的母线;定点P称为椎体的顶点,顶点到底面的垂线段称为椎体的高.
      \begin{itemize}[leftmargin=*]
        \kaishu
        \item 棱椎:底面为多边形的椎体称为棱椎,底面为n边形时即称作n棱椎.底面为正多边形且底面中心与棱锥顶点连线垂直于底面时称为正棱椎.
        \item 圆椎:底面为圆的椎体称为圆椎. 一般默认为正圆锥(直圆锥):
          底面中心与圆锥顶点连线垂直于底面.用一个过圆锥对称轴的平面截圆锥,所得图像称为圆锥的轴截面;轴截面的顶角,即轴截面上两条母线间的夹角;轴与母线的夹角称为半顶角.
      \end{itemize}
    \item[台体] 用不过椎体顶点且平行于椎体底面的平面截去椎体的椎尖部分,则剩下的部分称为台体.
      截面和原椎底称为台体的底面,通常将大的底面称为下底面,小的底面称为上底面;被截锥体侧面余下的部分称为台体的侧面;两底面间的距离称为台体的高.
      \begin{itemize}[leftmargin=*]
        \kaishu
        \item 棱台:由棱锥截得的台体称为棱台;正棱锥截得正棱台.
        \item 圆台:由圆椎截得的台体称为圆台.
      \end{itemize}
    \item[万能求积公式](牛顿-辛普森公式)
      \[V=\dfrac{h}6\Bigl(S_{\text{上底}}+S_{\text{下底}}+4S_{\text{中截面}}\Bigr) \]
  \end{description}
  \begin{exercise}
    \item
  \end{exercise}
\section{空间几何体的直观图}
  所谓“直观图”,是指用以表示几何体的平面图能够形成所画物体的空间概念:富有立体感,且能表达出图形各主要部分的位置关系和度量关系的图形.%一般采用下述的“斜二测画法”作出物体在平行投影下的直观图.\par
  {\kaishu 可以近似的认为,人的眼睛是根据中心投影的原理工作的.因此在中心投影下的直观图与实际看到的画面比较接近;
    而当物体不很大时,中心投影可以近似看作平行投影}
  \subsection{平行投影和中心投影}
    由于光的照射,在不透明物体后面的屏幕上会留下这个物体的影子,这种现象叫做投影.其中,形成投影的光线叫做投影线,留下物体影子的屏幕叫做投影面.\par
    {\kaishu 数学上的投影概念是实际投影现象的抽象.}
    \begin{description}
      \item[中心投影] 光由一点向外散射形成的投影.
        \begin{itemize}[leftmargin=*]
          \kaishu
          \item 中心投影的投影线交于一点(投影中心).
          \item 一般情况下,直线的中心投影是直线;当直线本身即为投射线时,中心投影为一点.
        \end{itemize}
      \item[平行投影] 在一束平行光线照射下形成的投影.当投影线与投影面垂直时,又称为正投影.
        \begin{itemize}[leftmargin=*]
          \kaishu
          \item 平行投影的投影线相互平行.
          \item 一般情况下,直线的平行投影是直线;当直线本身即为投射线时,投影为一点.
          \item 互相平行的直线投影为平行或重合的直线;且位于平行直线(或同一直线)上的两线段的比例在平行投影中保持不变.
          \item 位于平行直线(或同一直线)上任何线段的投影与原线段的比例是一个常数;特别地,平行于投影面的线段与其投影线段平行且相等.
        \end{itemize}
    \end{description}
  \subsection{空间向量基本定理与空间直角坐标系}
    在平面直角坐标系$xOy$的基础上,加上表示高度的$z$轴,即可构成空间直角坐标系,记作$O-xyz$;$z$轴以$O$为原点,方向与$x$轴,$y$轴方向垂直,且遵守右手定则:当右手的四指从$x$轴以不超过180度的转角转向$y$轴时,竖起的大拇指指向即为$z$轴的方向。\par
    设点M为空间的一个定点,过点$M$分别作平行于面$yOz$、面$xOz$、面$xOy$轴的平面,依次交$x$轴、$y$轴、$z$轴于$P$,$Q$,$R$,设三点在坐标轴上的坐标分别为$x$、$y$、$z$,则点$M$就对应于唯一确定的有序实数组$(x,y,z)$,称为点$M$在此空间直角坐标系中的坐标,记作$M(x,y,z)$.$x$、$y$、$z$分别叫做点$M$的横坐标、纵坐标、竖坐标.
  \subsection{斜二测画法}用斜二测画法画空间几何体的直观图方法:
    \begin{enumerate}[label=\arabic*)]
      \item 在已知图形中建立空间直角坐标系$O-xyz$;
      \item 画直观图时,把上述三坐标轴保持单位长度不变分别画成交于点$O'$的$x'$轴、$y'$轴与$z'$轴,且使$\angle{x'O'y'}=45\degree$(或135\degree),$\angle{x'O'z'}=90\degree$(或$\angle{y'O'z'}=90\degree$),$x'O'y"$所确定的平面表示水平面;
      \item 已知图形中直角坐标系$O-xyz$下坐标为$(x,y,z)$的点,在直观图中对应于非直角坐标系$x'O'y'$下坐标为$(x,y/2)$的点再往$z'$轴方向平移z个单位长度.
    \end{enumerate}
    {\kaishu
      由于绘制比较复杂,水平放置的圆的直观图一般不采用斜二测画法,常用“正等测画法”绘制.
    }
  \subsection{几种常见几何图形的直观图}


\section{空间几何体的表面积与体积}


\newpage
\section{课后作业}
  \begin{exercise}

  \end{exercise}
\stopexercise

\newpage
\section{参考答案}
\begin{multicols}{2}
  \printanswer
\end{multicols}

\Topic{函数的平移与伸缩变换&函数$y=A\sin (\omega x+\varphi)$的图像及简单应用}
  \Teach{函数$y=A\sin (\omega x+\varphi)$的图像变换}
  \Grade{高一}
  % \Name{郑皓天}\FirstTime{20181207}\CurrentTime{20181207}
  % \Name{林叶}\FirstTime{20180908}\CurrentTime{20181125}
  %\Name{1v2}\FirstTime{20181028}\CurrentTime{20181117}
  % \Name{林叶}\FirstTime{20180908}\CurrentTime{20181125}
  % \Name{郭文镔}\FirstTime{20181111}\CurrentTime{20181117}
  % \Name{马灿威}\FirstTime{20181111}\CurrentTime{20181111}
  \newtheorem*{Theorem}{定理}
  \makefront
\vspace{-1.5em}
\startexercise
\begin{exercise}{\heiti 课前检测}\\
\end{exercise}
\section{函数的(线性)变换}
  点$(x_0,y_0)$在图像$y=f(x)$上,则 $y_0=f(x_0)$\\
  点$(x_0-1,y_0)$在图像$y=f(x+1)$上\\
  点$(x_0,y_0-1)$在图像$y+1=f(x)$上\\
  点$(2x_0,y_0)$在图像$y=f(x/2)$上\\
  点$(x_0,2y_0)$在图像$y/2=f(x)$上\\
  点$(-x_0,y_0)$在图像$y=f(-x)$上\\
  点$(x_0,-y_0)$在图像$-y=f(x)$上\\
  当$\begin{cases}
      x_0=P(x')\\
      y_0=Q(y')
    \end{cases}$时,点$(x',y')$在图像$Q(y)=f(P(x))$上;也即\\
  在图像$Q(y)=f(P(x))$上每一点可由图像$y=f(x)$上的点$(x_0,y_0)$
  经变换$\begin{cases}
          x‘=P^{-1}(x_0)\\
          y’=Q^{-1}(y_0)
        \end{cases}$得到.
\section{函数$y=A\sin (\omega x+\varphi)$的图像}
  \begin{description}
    \item [label]
  \end{description}
  \begin{exercise}
    \qs
      函数$f(x)=\cos\left(\omega x+\varphi\right)$的部分图象如图所示,则$f(x)$的单调递减区间为\xx
      \begin{center}
      \begin{tikzpicture}
        \node[below left](O) at(0,0) {\small$\bm{O}$};
        \draw(0,1)node[right]{\tiny$1$}--(0.1,1);
        \clip(-1.2,-1.2) rectangle (2,1.5);
        \draw[->,>=stealth](-1.2,0)--(2,0) node[below left] (x){$x$};
        \draw[->,>=stealth](0,-1.2)--(0,1.5) node[below right] (y){$y$};
        \draw[domain=-1.2:2,samples=1000] plot(\x,{cos((pi*(\x)+1/4*pi) r)});
        \node[below] (A)at (0.25,0){$\frac{1}{4}$};
        \node[below] (B)at (1.25,0){$\frac{5}{4}$};
      \end{tikzpicture}
      \end{center}
      \twochx{$ \left(k\pi-\dfrac{1}{4},k\pi+\dfrac{3}{4}\right),k\inZ$}
        {$ \left(2k\pi-\dfrac{1}{4},2k\pi+\dfrac{3}{4}\right),k\inZ$}
        {$ \left(k-\dfrac{1}{4},k+\dfrac{3}{4}\right),k\inZ$}
        {$\left(2k-\dfrac{1}{4},2k+\dfrac{3}{4}\right),k\inZ $}
  \end{exercise}
\section{三角函数模型的简单应用-}
 \fz[12]
$f(x0)=d d$\fz[4]
\section{课后作业}
  \begin{exercise}

  \end{exercise}
\stopexercise
\section{参考答案}
\printanswer

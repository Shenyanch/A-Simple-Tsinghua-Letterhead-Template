% Generated by GrindEQ Word-to-LaTeX 
\documentclass[a4paper,11pt,UTF8,twoside]{ctexart} % use \documentstyle for old LaTeX compilers

% \usepackage[english]{babel} % 'french', 'german', 'spanish', 'danish', etc.
\usepackage{amssymb}
\usepackage{amsmath}
\usepackage{txfonts}
% \usepackage{mathdots}
% \usepackage[classicReIm]{kpfonts}
\usepackage{graphicx}
% \graphicspath{{Fig/}} 
% You can include more LaTeX packages here 


\begin{document}


\section{2020-2021学年福建省福州一中高二(上)期中数学试卷}

\textbf{一、选择题:本题共8小题,每小题5分,共40分在每小题给出的四个选项中,只有一项是符合题目要求的。}

1.(5分)命题\textit{p}:$\mathrm{\forall }$\textit{x}$\mathrm{\in }$\textbf{R},\textit{x}${}^{2}$﹣2\textit{x}$\mathrm{\ge}$1的否定是(  )

A.\includegraphics*[width=1.75in, height=0.28in, keepaspectratio=false]{image2} B.$\mathrm{\forall }$\textit{x}$\mathrm{\in }$\textbf{R},\textit{x}${}^{2}$﹣2\textit{x}<1 

C.\includegraphics*[width=1.75in, height=0.28in, keepaspectratio=false]{image3} D.\includegraphics*[width=1.71in, height=0.28in, keepaspectratio=false]{image4}

【分析】根据含有量词的命题的否定即可得到结论.

【解答】解:命题为全称命题,则命题的否定为:\includegraphics*[width=1.75in, height=0.28in, keepaspectratio=false]{image5}.,

故选:\textit{A}.

【点评】本题主要考查含有量词的命题的否定,比较基础.

2.(5分)如果命题``\textit{p}且\textit{q}''是假命题,``¬\textit{q}''也是假命题,则(  )

A.命题``¬\textit{p}或\textit{q}''是假命题 B.命题``\textit{p}或\textit{q}''是假命题 

C.命题``¬\textit{p}且\textit{q}''是真命题 D.命题``\textit{p}且¬\textit{q}''是真命题

【分析】因为命题``\textit{p}且\textit{q}''是假命题,可得\textit{p}和\textit{q}至少有一个是假命题,因为``¬\textit{q}''也是假命题,所以\textit{q}是真命题,根据此信息进行判断;

【解答】解:命题``\textit{p}且\textit{q}''是假命题,可得\textit{p}和\textit{q}至少有一个为假命题,

因为``¬\textit{q}''也是假命题,可得\textit{q}是真命题,可得\textit{p}是假命题,

\textit{A}、命题``¬\textit{p}是真命题,可得命题``¬\textit{p}或\textit{q}''是真命题,故\textit{A}错误;

\textit{B}、因为\textit{q}是真命题,故命题``\textit{p}或\textit{q}''是真命题,故\textit{B}错误;

\textit{C}、\textit{p}是假命题,\textit{q}为真命题,命题``¬\textit{p}且\textit{q}''是真命题,故\textit{C}正确;

\textit{D}、\textit{p}是假命题,命题``\textit{p}且¬\textit{q}''是假命题,故\textit{D}错误;

故选:\textit{C}.

【点评】本题主要考查了非\textit{P}命题与\textit{p}或\textit{q}命题的真假的应用,注意``或''``且''``非''的含义,是一道基础题;

3.(5分)如图:在平行六面体\textit{ABCD}﹣\textit{A}${}_{1}$\textit{B}${}_{1}$\textit{C}${}_{1}$\textit{D}${}_{1}$中,\textit{M}为\textit{A}${}_{1}$\textit{C}${}_{1}$与\textit{B}${}_{1}$\textit{D}${}_{1}$的交点.若\includegraphics*[width=0.39in, height=0.21in, keepaspectratio=false]{image6},\includegraphics*[width=0.39in, height=0.21in, keepaspectratio=false]{image7},\includegraphics*[width=0.51in, height=0.27in, keepaspectratio=false]{image8},则下列向量中与\includegraphics*[width=0.19in, height=0.21in, keepaspectratio=false]{image9}相等的向量是(  )

\includegraphics*[width=1.68in, height=1.28in, keepaspectratio=false]{image10}

A.\includegraphics*[width=0.90in, height=0.37in, keepaspectratio=false]{image11} B.\includegraphics*[width=0.79in, height=0.37in, keepaspectratio=false]{image12} C.\includegraphics*[width=0.90in, height=0.37in, keepaspectratio=false]{image13} D.\includegraphics*[width=0.79in, height=0.37in, keepaspectratio=false]{image14}

【分析】利用向量的运算法则:三角形法则、平行四边形法则表示出\includegraphics*[width=0.19in, height=0.21in, keepaspectratio=false]{image15}.

【解答】解:$\mathrm{\because}$\includegraphics*[width=0.99in, height=0.27in, keepaspectratio=false]{image16}

=\includegraphics*[width=0.53in, height=0.37in, keepaspectratio=false]{image17}

=\includegraphics*[width=1.11in, height=0.37in, keepaspectratio=false]{image18}

=\includegraphics*[width=0.94in, height=0.37in, keepaspectratio=false]{image19}

=\includegraphics*[width=0.90in, height=0.37in, keepaspectratio=false]{image20}

故选:\textit{A}.

【点评】本题考查利用向量的运算法则将未知的向量用已知的基底表示从而能将未知向量间的问题转化为基底间的关系解决.

4.(5分)《易系辞上》有``河出图,洛出书''之说,河图、洛书是中国古代流传下来的两幅神秘图案,被誉为``宇宙魔方''.河图的排列结构如图所示,一与六共宗居下,二与七为朋居上,三与八同道居左,四与九为友居右,五与十相守居中,其中白圈数为阳数,黑点数为阴数,若从阳数和阴数中各取一数,则其差的绝对值为3的概率为(  )

\includegraphics*[width=1.32in, height=1.15in, keepaspectratio=false]{image21}

A.\includegraphics*[width=0.13in, height=0.37in, keepaspectratio=false]{image22} B.\includegraphics*[width=0.22in, height=0.37in, keepaspectratio=false]{image23} C.\includegraphics*[width=0.22in, height=0.37in, keepaspectratio=false]{image24} D.\includegraphics*[width=0.13in, height=0.37in, keepaspectratio=false]{image25}

【分析】从阳数和阴数中各取一数,基本事件总数\textit{n}=5$\mathrm{\times}$5=25,利用列举法求出其差的绝对值为3包含的基本事件有7个,由此能求出从阳数和阴数中各取一数,则其差的绝对值为3的概率.

【解答】解:阳数是1,3,5,7,9,阴数是2,4,6,8,10,

从阳数和阴数中各取一数,

基本事件总数\textit{n}=5$\mathrm{\times}$5=25,

其差的绝对值为3包含的基本事件有7个,分别为:

(1,4),(3,6),(5,2),(5,8),(7,4),(7,10),(9,6),

$\mathrm{\therefore}$从阳数和阴数中各取一数,则其差的绝对值为3的概率为\textit{P}=\includegraphics*[width=0.22in, height=0.37in, keepaspectratio=false]{image26}.

故选:\textit{B}.

【点评】本题考查概率的求法,考查古典概型、列举法等基础知识,考查运算求解能力,是基础题.

5.(5分)椭圆4\textit{x}${}^{2}$+9\textit{y}${}^{2}$=144内有一点\textit{P}(3,2),则以\textit{P}为中点的弦所在直线的斜率为(  )

A.\includegraphics*[width=0.24in, height=0.37in, keepaspectratio=false]{image27} B.\includegraphics*[width=0.24in, height=0.37in, keepaspectratio=false]{image28} C.\includegraphics*[width=0.24in, height=0.37in, keepaspectratio=false]{image29} D.\includegraphics*[width=0.24in, height=0.37in, keepaspectratio=false]{image30}

【分析】利用中点坐标公式、斜率计算公式、``点差法''即可得出.

【解答】解:设以点\textit{P}为中点的弦所在直线与椭圆相交于点\textit{A}(\textit{x}${}_{1}$,\textit{y}${}_{1}$),\textit{B}(\textit{x}${}_{2}$,\textit{y}${}_{2}$),斜率为\textit{k}.

则\includegraphics*[width=1.21in, height=0.29in, keepaspectratio=false]{image31},\includegraphics*[width=1.21in, height=0.29in, keepaspectratio=false]{image32},两式相减得4(\textit{x}${}_{1}$+\textit{x}${}_{2}$)(\textit{x}${}_{1}$﹣\textit{x}${}_{2}$)+9(\textit{y}${}_{1}$+\textit{y}${}_{2}$)(\textit{y}${}_{1}$﹣\textit{y}${}_{2}$)=0,

又\textit{x}${}_{1}$+\textit{x}${}_{2}$=6,\textit{y}${}_{1}$+\textit{y}${}_{2}$=4,\includegraphics*[width=0.52in, height=0.49in, keepaspectratio=false]{image33}=\textit{k},

代入解得\textit{k}=﹣\includegraphics*[width=0.46in, height=0.37in, keepaspectratio=false]{image34}=\includegraphics*[width=0.24in, height=0.37in, keepaspectratio=false]{image35}.

故选:\textit{A}.

【点评】熟练掌握中点坐标公式、斜率计算公式、``点差法''是解题的关键.

6.(5分)在发生某公共卫生事件期间,有专业机构认为该事件在一段时间内没有发生大规模群体感染的标志为:``连续10天,每天新增疑似病例不超过7人'',根据过去10天甲、乙、丙、丁四地新增疑似病例据,一定符合该标志的是(  )

A.甲地:总体平均值为3,中位数为4 

B.乙地;总体平均值为1,总体方差大于0 

C.丙地:中位数为2,众数为3 

D.丁地:总体均值为2,总体方差为2

【分析】平均数与中位数不能限制极端值的出现,当总体方差大于0,不知道总体方差的具体数值,因此不能确定数据的波动大小,中位数和众数也不能确定,当总体平均数是2,若有一个数据超过7,则方差就接大于2,符合要求.

【解答】解:$\mathrm{\because}$总体平均数为3,中位数为4,平均数与中位数不能限制极端值的出现,因而有可能出现超过7人的情况,故\textit{A}不正确,

当总体方差大于0,不知道总体方差的具体数值,因此不能确定数据的波动大小,

故\textit{B}不正确,

中位数和众数也不能确定,

故\textit{C}不正确,

当总体平均数是2,若有一个数据超过7,则方差就大于\includegraphics*[width=0.22in, height=0.37in, keepaspectratio=false]{image36}(7﹣2)${}^{2}$=2.5>2,

$\mathrm{\therefore}$总体均值为2,总体方差为2时,没有数据超过7.

故\textit{D}正确.

故选:\textit{D}.

【点评】本题考查数据的几个特征量,这几个量各自表示数据的一个方面,有时候一个或两个量不能说明这组数据的特点,若要掌握这组数据则要全面掌握.

7.(5分)已知\textit{F}${}_{1}$,\textit{F}${}_{2}$是双曲线\includegraphics*[width=0.23in, height=0.48in, keepaspectratio=false]{image37}﹣\includegraphics*[width=0.23in, height=0.49in, keepaspectratio=false]{image38}=1(\textit{a}>0,\textit{b}>0)的左、右焦点,若点\textit{F}${}_{2}$关于直线\textit{y}=\includegraphics*[width=0.13in, height=0.37in, keepaspectratio=false]{image39}\textit{x}的对称点\textit{M}也在双曲线上,则该双曲线的离心率为(  )

A.\includegraphics*[width=0.24in, height=0.38in, keepaspectratio=false]{image40} B.\includegraphics*[width=0.21in, height=0.19in, keepaspectratio=false]{image41} C.\includegraphics*[width=0.21in, height=0.19in, keepaspectratio=false]{image42} D.2

【分析】求出过焦点\textit{F}${}_{2}$且垂直渐近线的直线方程,联立渐近线方程,解方程组可得对称中心的点的坐标,再由中点坐标公式可得对称点的坐标,代入双曲线的方程结合\textit{a}${}^{2}$+\textit{b}${}^{2}$=\textit{c}${}^{2}$,解出\textit{e}即得.

【解答】解:过焦点\textit{F}${}_{2}$且垂直渐近线的直线方程为:\textit{y}﹣0=﹣\includegraphics*[width=0.13in, height=0.37in, keepaspectratio=false]{image43}(\textit{x}﹣\textit{c}),

联立渐近线方程\textit{y}=\includegraphics*[width=0.13in, height=0.37in, keepaspectratio=false]{image44}\textit{x}与\textit{y}﹣0=﹣\includegraphics*[width=0.13in, height=0.37in, keepaspectratio=false]{image45}(\textit{x}﹣\textit{c}),

解之可得\textit{x}=\includegraphics*[width=0.23in, height=0.43in, keepaspectratio=false]{image46},\textit{y}=\includegraphics*[width=0.22in, height=0.37in, keepaspectratio=false]{image47},

故对称中心的点坐标为(\includegraphics*[width=0.23in, height=0.43in, keepaspectratio=false]{image48},\includegraphics*[width=0.22in, height=0.37in, keepaspectratio=false]{image49}),

由中点坐标公式可得对称点的坐标为(\includegraphics*[width=0.34in, height=0.43in, keepaspectratio=false]{image50}﹣\textit{c},\includegraphics*[width=0.30in, height=0.37in, keepaspectratio=false]{image51}),

将其代入双曲线的方程可得\includegraphics*[width=0.91in, height=0.48in, keepaspectratio=false]{image52}﹣\includegraphics*[width=0.34in, height=0.48in, keepaspectratio=false]{image53}=1,结合\textit{a}${}^{2}$+\textit{b}${}^{2}$=\textit{c}${}^{2}$,

化简可得\textit{c}${}^{2}$=5\textit{a}${}^{2}$,故可得\textit{e}=\includegraphics*[width=0.13in, height=0.37in, keepaspectratio=false]{image54}=\includegraphics*[width=0.21in, height=0.19in, keepaspectratio=false]{image55}.

故选:\textit{C}.

【点评】本题考查双曲线的简单性质,涉及离心率的求解和对称问题,考查运算能力,属中档题.

8.(5分)已知\textit{F}${}_{1}$、\textit{F}${}_{2}$是双曲线或椭圆的左、右焦点,若椭圆或双曲线上存在点\textit{P},使得点{\textbar}\textit{PF}${}_{1}${\textbar}=2{\textbar}\textit{PF}${}_{2}${\textbar},且存在$\mathrm{\triangle}$\textit{PF}${}_{1}$\textit{F}${}_{2}$,则称此椭圆或双曲线存在``$\mathrm{\Omega }$点'',下列曲线中存在``$\mathrm{\Omega }$点''的是(  )

A.\includegraphics*[width=0.23in, height=0.43in, keepaspectratio=false]{image56}+\includegraphics*[width=0.23in, height=0.44in, keepaspectratio=false]{image57}=1 B.\includegraphics*[width=0.23in, height=0.43in, keepaspectratio=false]{image58}+\includegraphics*[width=0.23in, height=0.44in, keepaspectratio=false]{image59}=1 

C.\textit{x}${}^{2}$﹣\textit{y}${}^{2}$=\includegraphics*[width=0.13in, height=0.37in, keepaspectratio=false]{image60} D.16\textit{x}${}^{2}$﹣\includegraphics*[width=0.42in, height=0.44in, keepaspectratio=false]{image61}=1

【分析】验证四个答案中哪一个符合题干中的条件:存在点\textit{P},使得点\textit{P}到两个焦点的距离之比为2:1即可.

【解答】解:\includegraphics*[width=0.23in, height=0.43in, keepaspectratio=false]{image62}+\includegraphics*[width=0.23in, height=0.44in, keepaspectratio=false]{image63}=1,\textit{F}${}_{1}$(﹣1,0)、\textit{F}${}_{2}$(1,0),\textit{a}=6,所以\textit{a}+\textit{c}=7,\textit{a}﹣\textit{c}=5,

\textit{P}到焦点距离的最小值为5,最大值为7,

所以不存在点\textit{P},满足{\textbar}\textit{PF}${}_{1}${\textbar}=2{\textbar}\textit{PF}${}_{2}${\textbar},

所以\textit{A}不存在``$\mathrm{\Omega }$点'';

\includegraphics*[width=0.23in, height=0.43in, keepaspectratio=false]{image64}+\includegraphics*[width=0.23in, height=0.44in, keepaspectratio=false]{image65}=1,\textit{F}${}_{1}$(﹣1,0)、\textit{F}${}_{2}$(1,0),\textit{a}=4,所以\textit{a}+\textit{c}=5,\textit{a}﹣\textit{c}=3,

\textit{P}到焦点距离的最小值为3,最大值为5,

所以不存在点\textit{P},满足{\textbar}\textit{PF}${}_{1}${\textbar}=2{\textbar}\textit{PF}${}_{2}${\textbar},

所以\textit{B}不存在``$\mathrm{\Omega }$点'';

若双曲线的方程为\textit{x}${}^{2}$﹣\textit{y}${}^{2}$=\includegraphics*[width=0.13in, height=0.37in, keepaspectratio=false]{image66}.

则双曲线的两个焦点为\textit{F}${}_{1}$(﹣1,0)、\textit{F}${}_{2}$(1,0),

若双曲线上存在点\textit{P},使得点\textit{P}到两个焦点\textit{F}${}_{1}$,\textit{F}${}_{2}$的距离之比为2:1,

{\textbar}\textit{PF}${}_{1}${\textbar}=2{\textbar}\textit{PF}${}_{2}${\textbar},{\textbar}\textit{PF}${}_{1}${\textbar}﹣{\textbar}\textit{PF}${}_{2}${\textbar}=\includegraphics*[width=0.21in, height=0.19in, keepaspectratio=false]{image67},可得{\textbar}\textit{PF}${}_{2}${\textbar}=\includegraphics*[width=0.21in, height=0.19in, keepaspectratio=false]{image68}>\includegraphics*[width=0.24in, height=0.38in, keepaspectratio=false]{image69}>1﹣\includegraphics*[width=0.24in, height=0.38in, keepaspectratio=false]{image70},

即双曲线\textit{x}${}^{2}$﹣\textit{y}${}^{2}$=\includegraphics*[width=0.13in, height=0.37in, keepaspectratio=false]{image71}存在``$\mathrm{\Omega }$点'';

16\textit{x}${}^{2}$﹣\includegraphics*[width=0.42in, height=0.44in, keepaspectratio=false]{image72}=1,\textit{a}=\includegraphics*[width=0.13in, height=0.37in, keepaspectratio=false]{image73},\textit{c}=1,\textit{F}${}_{1}$(﹣1,0)、\textit{F}${}_{2}$(1,0),

所以\textit{a}+\textit{c}=\includegraphics*[width=0.13in, height=0.37in, keepaspectratio=false]{image74},\textit{c}﹣\textit{a}=\includegraphics*[width=0.13in, height=0.37in, keepaspectratio=false]{image75},由双曲线的性质可知,\textit{P}到焦点的距离,随{\textbar}\textit{x}{\textbar}增大,到两个焦点的距离接近相等,

\textit{P}在顶点处,\textit{P}到两个焦点的距离的差最大,

所以不存在点\textit{P},满足{\textbar}\textit{PF}${}_{1}${\textbar}=2{\textbar}\textit{PF}${}_{2}${\textbar},

所以\textit{D}不存在``$\mathrm{\Omega }$点'';

故选:\textit{C}.

【点评】本题考查新定义,考查学生分析解决问题的能力,属于中档题.

\textbf{二、选择题:本题共4小题,每小题5分,共20分在每小题给出的选项中,有多项符合题目要求全部选对的得5分,有选错的得0分,部分选对的得3分。}

9.(5分)下列说法中正确的是(  )

A.在频率分布直方图中,中位数左边和右边的直方图的面积相等. 

B.若\textit{A}、\textit{B}为对立事件,则\textit{A}的对立事件与\textit{B}的对立事件一定互斥. 

C.若回归直线\includegraphics*[width=0.10in, height=0.34in, keepaspectratio=false]{image76}=\includegraphics*[width=0.10in, height=0.33in, keepaspectratio=false]{image77}\textit{x}+\includegraphics*[width=0.10in, height=0.33in, keepaspectratio=false]{image78}的斜率\includegraphics*[width=0.10in, height=0.33in, keepaspectratio=false]{image79}>0,则变量\textit{x}与\textit{y}正相关. 

D.某个班级内有40名学生,抽10名同学去参加某项活动,则每4人中必有1人抽中.

【分析】直接利用频率分布直方图和回归直线的应用,互斥事件和对立事件的应用,随机事件的应用求出结果.

【解答】解:对于\textit{A},在频率分布直方图中,中位数左边和右边的直方图的面积相等,均为0.5,故\textit{A}正确;

对于\textit{B},若\textit{A}、\textit{B}为互斥事件,则\textit{A}的对立事件与\textit{B}的对立事件一定互斥,不一定互斥,可以有重叠关系,故错误,故\textit{B}不正确.

对于\textit{C}回归直线 \includegraphics*[width=0.10in, height=0.34in, keepaspectratio=false]{image80}=\includegraphics*[width=0.10in, height=0.33in, keepaspectratio=false]{image81}\textit{x}+\includegraphics*[width=0.10in, height=0.33in, keepaspectratio=false]{image82}的斜率\includegraphics*[width=0.10in, height=0.33in, keepaspectratio=false]{image83}>0,说明直线的斜率大于0,则变量\textit{x}与\textit{y}正相关.故\textit{C}正确;

对于\textit{D}:某个班级内有40名学生,抽10名同学去参加某项活动,则每4人中必有1人抽中,可以全部抽到别的学生,这几个人不一定抽到.故\textit{D}错误.

故选:\textit{ACD}.

【点评】本题考查的知识要点:频率分布直方图和回归直线的应用,互斥事件和对立事件的应用,随机事件的应用,主要考查学生的运算能力和转换能力及思维能力,属于基础题.

10.(5分)已知双曲线\textit{E}的中心在原点,对称轴为坐标轴,离心率为\includegraphics*[width=0.21in, height=0.19in, keepaspectratio=false]{image84},则双曲线\textit{E}的渐近线方程为(  )

A.\textit{y}=\includegraphics*[width=0.43in, height=0.38in, keepaspectratio=false]{image85}\textit{x} B.\textit{y}=\includegraphics*[width=0.39in, height=0.19in, keepaspectratio=false]{image86}\textit{x} C.\textit{y}=\includegraphics*[width=0.33in, height=0.37in, keepaspectratio=false]{image87}\textit{x} D.\textit{y}=$\mathrm{\pm}$2\textit{x}

【分析】通过焦点所在的轴,结合离心率公式,推出\textit{a},\textit{b}关系,即可求解渐近线方程.

【解答】解:如焦点在\textit{x}轴上,离心率为\includegraphics*[width=0.21in, height=0.19in, keepaspectratio=false]{image88},可得\textit{c}${}^{2}$=5\textit{a}${}^{2}$,\textit{b}${}^{2}$=4\textit{a}${}^{2}$,则\includegraphics*[width=0.13in, height=0.37in, keepaspectratio=false]{image89}=2,所以双曲线的渐近线方程\textit{y}=$\mathrm{\pm}$2\textit{x},

如焦点在\textit{y}轴上,离心率为\includegraphics*[width=0.21in, height=0.19in, keepaspectratio=false]{image90},可得\textit{c}${}^{2}$=5\textit{a}${}^{2}$,\textit{b}${}^{2}$=4\textit{a}${}^{2}$,则\includegraphics*[width=0.13in, height=0.37in, keepaspectratio=false]{image91}=2,所以双曲线的渐近线方程\textit{y}=$\mathrm{\pm}$\includegraphics*[width=0.13in, height=0.37in, keepaspectratio=false]{image92}\textit{x},

故选:\textit{CD}.

【点评】本题考查双曲线的离心率的应用,注意运用双曲线方程和渐近线方程的关系,考查运算能力,属于基础题.

11.(5分)已知抛物线\textit{E}:\textit{y}${}^{2}$=4\textit{x}的焦点为\textit{F},准线为\textit{l},过\textit{F}的直线与\textit{E}交于\textit{A},\textit{B}两点,\textit{C},\textit{D}分别为\textit{A},\textit{B}在\textit{l}上的射影,且{\textbar}\textit{AF}{\textbar}=3{\textbar}\textit{BF}{\textbar},\textit{M}为\textit{AB}中点,则下列结论正确的是(  )

A.$\mathrm{\angle}$\textit{CFD}=90$\mathrm{{}^\circ}$ B.$\mathrm{\triangle}$\textit{CMD}为等腰直角三角形 

C.直线\textit{AB}的斜率为\includegraphics*[width=0.39in, height=0.19in, keepaspectratio=false]{image93} D.$\mathrm{\triangle}$\textit{AOB}的面积为4

【分析】由题意写出焦点\textit{F}的坐标及准线方程,设直线\textit{AB}的方程及\textit{A},\textit{B}的坐标,可得\textit{C},\textit{D}的坐标,再由{\textbar}\textit{AF}{\textbar}=3{\textbar}\textit{BF}{\textbar},求出直线\textit{AB}的参数,进而判断出所给命题的真假.

【解答】解:由题意由抛物线的对称性,焦点\textit{F}(1,0),准线方程为\textit{x}=﹣1,

由题意可得直线\textit{AB}的斜率不为0,由题意设直线\textit{AB}的方程为:\textit{x}=\textit{my}+1,

设\textit{A}(\textit{x}${}_{1}$,\textit{y}${}_{1}$),\textit{B}(\textit{x}${}_{2}$,\textit{y}${}_{2}$),由题意可知\textit{C}(﹣1,\textit{y}${}_{1}$),\textit{D}(﹣1,\textit{y}${}_{2}$),

将直线\textit{AB}与抛物线联立整理得:\textit{y}${}^{2}$﹣4\textit{my}﹣4=0,\textit{y}${}_{1}$+\textit{y}${}_{2}$=4\textit{m},\textit{y}${}_{1}$\textit{y}${}_{2}$=﹣4,

\textit{A}中,$\mathrm{\because}$\includegraphics*[width=0.48in, height=0.21in, keepaspectratio=false]{image94}=(﹣2,\textit{y}${}_{1}$)(﹣2,\textit{y}${}_{2}$)=(﹣2)(﹣2)+\textit{y}${}_{1}$\textit{y}${}_{2}$=4﹣4=0,$\mathrm{\therefore}$\includegraphics*[width=0.56in, height=0.21in, keepaspectratio=false]{image95},即$\mathrm{\angle}$\textit{CFD}=90$\mathrm{{}^\circ}$,所以\textit{A}正确;

\textit{B}中,由\textit{A}正确,不可能\textit{CM}$\mathrm{\bot}$\textit{DM},更不会$\mathrm{\angle}$\textit{C}或$\mathrm{\angle}$\textit{D}为直角,所以\textit{B}不正确;

\textit{C}中,因为{\textbar}\textit{AF}{\textbar}=3{\textbar}\textit{BF}{\textbar},所以\includegraphics*[width=0.19in, height=0.21in, keepaspectratio=false]{image96}=3\includegraphics*[width=0.19in, height=0.21in, keepaspectratio=false]{image97},即\textit{y}${}_{1}$=﹣3\textit{y}${}_{2}$,\textit{y}${}_{1}$+\textit{y}${}_{2}$=4\textit{m},\textit{y}${}_{1}$\textit{y}${}_{2}$=﹣4,$\mathrm{\therefore}$\includegraphics*[width=0.84in, height=0.58in, keepaspectratio=false]{image98},解得\textit{m}${}^{2}$=\includegraphics*[width=0.13in, height=0.37in, keepaspectratio=false]{image99},\textit{m}=\includegraphics*[width=0.43in, height=0.38in, keepaspectratio=false]{image100},所以直线\textit{AB}的斜率为\includegraphics*[width=0.39in, height=0.19in, keepaspectratio=false]{image101},所以\textit{C}正确;

\textit{D}中,由题意可得弦长{\textbar}\textit{AB}{\textbar}=\includegraphics*[width=1.95in, height=0.30in, keepaspectratio=false]{image102}=\includegraphics*[width=1.33in, height=0.25in, keepaspectratio=false]{image103}=\includegraphics*[width=0.43in, height=0.38in, keepaspectratio=false]{image104}\includegraphics*[width=0.89in, height=0.38in, keepaspectratio=false]{image105}=\includegraphics*[width=0.22in, height=0.37in, keepaspectratio=false]{image106},\textit{O}到直线\textit{AB}的距离\textit{d}=\includegraphics*[width=0.52in, height=0.45in, keepaspectratio=false]{image107}=\includegraphics*[width=0.46in, height=0.58in, keepaspectratio=false]{image108}=\includegraphics*[width=0.24in, height=0.38in, keepaspectratio=false]{image109},所以\textit{S}${}_{\triangle }$\textit{${}_{OAB}$}=\includegraphics*[width=0.74in, height=0.37in, keepaspectratio=false]{image110}=\includegraphics*[width=0.81in, height=0.38in, keepaspectratio=false]{image111}=\includegraphics*[width=0.34in, height=0.38in, keepaspectratio=false]{image112},所以\textit{D}不正确,

故选:\textit{AC}.

\includegraphics*[width=2.45in, height=2.27in, keepaspectratio=false]{image113}

【点评】考查抛物线的性质及命题真假的判断,属于中档题.

12.(5分)``曼哈顿距离''是十九世纪的赫尔曼闵可夫斯基所创辞汇,定义如下:在直角坐标平面上任意两点\textit{A}(\textit{x}${}_{1}$,\textit{y}${}_{1}$)\textit{B}(\textit{x}${}_{2}$,\textit{y}${}_{2}$)的曼哈顿距离为:\textit{d}(\textit{A},\textit{B})={\textbar}\textit{x}${}_{1}$﹣\textit{x}${}_{2}${\textbar}+{\textbar}\textit{y}${}_{1}$﹣\textit{y}${}_{2}${\textbar}.在此定义下以下结论正确的是(  )

A.已知点\textit{O}(0,0),满足\textit{d}(\textit{O},\textit{M})=1的点\textit{M}轨迹围成的图形面积为2 

B.已知点\textit{F}${}_{1}$(﹣1,0),\textit{F}${}_{2}$(1,0),满足\textit{d}(\textit{M},\textit{F}${}_{1}$)+(\textit{M},\textit{F}${}_{2}$)=4的点\textit{M}轨迹的形状为六边形 

C.已知点\textit{F}${}_{1}$(﹣1,0),\textit{F}${}_{2}$(1,0),不存在动点\textit{M}满足方程:{\textbar}\textit{d}(\textit{M},\textit{F}${}_{1}$)﹣\textit{d}(\textit{M},\textit{F}${}_{2}$){\textbar}=1 

D.已知点\textit{M}在圆\textit{O}:\textit{x}${}^{2}$+\textit{y}${}^{2}$=1上,点\textit{N}在直线\textit{l}:2\textit{x}+\textit{y}﹣6=0上,则\textit{d}(\textit{M}、\textit{N})的最小值为3﹣\includegraphics*[width=0.24in, height=0.38in, keepaspectratio=false]{image114}

【分析】\textit{A},\textit{B}直线去绝对值作图;\textit{C}解绝对值方程;\textit{D}先通过固定圆的点来\textit{d}(\textit{M},\textit{N})何时最小值,进而让圆上的点动起来再求出\textit{d}(\textit{M},\textit{N})的最小值.

【解答】解:对于 \textit{A},设 \textit{M}(\textit{x},\textit{y}),因为 \textit{d}(\textit{O},\textit{M})=1,所 以{\textbar}\textit{x}{\textbar}+{\textbar}\textit{y}{\textbar}=1,

①当 \textit{xy}$\mathrm{\neq}$0 时,

\includegraphics*[width=1.63in, height=0.99in, keepaspectratio=false]{image115}

②当\textit{x}=0 时,\textit{y}=$\mathrm{\pm}$1; 当 \textit{y}=0 时,\textit{x}=$\mathrm{\pm}$1;作出图象如下图所示,

\includegraphics*[width=3.24in, height=3.22in, keepaspectratio=false]{image116}

易知这是一个边长为 \includegraphics*[width=0.21in, height=0.19in, keepaspectratio=false]{image117} 的正方形,所以面积为 \includegraphics*[width=0.70in, height=0.25in, keepaspectratio=false]{image118},故 \textit{A} 正确

对于 \textit{B},设 \textit{M}(\textit{x},\textit{y}),因为 \textit{d}(\textit{M},\textit{F}${}_{1}$)+\textit{d}(\textit{M},\textit{F}${}_{2}$)=4,所以{\textbar}\textit{x}﹣1{\textbar}+{\textbar}\textit{x}+1{\textbar}+2{\textbar}\textit{y}{\textbar}=4,

①当 \textit{x}$\mathrm{\neq}$$\mathrm{\pm}$1 且 \textit{y}$\mathrm{\neq}$0 时,

\includegraphics*[width=2.04in, height=1.52in, keepaspectratio=false]{image119},

②当\textit{x}=1 时,\textit{y}=$\mathrm{\pm}$1; 当 \textit{x}=﹣1 时,\textit{y}=$\mathrm{\pm}$﹣1; 当 \textit{y}=0 时,\textit{x}=$\mathrm{\pm}$2;

作出图象如下图所示,

\includegraphics*[width=4.37in, height=3.26in, keepaspectratio=false]{image120}

所以 \textit{M} 点轨迹是一个六边形,故 \textit{B} 正确.

对于 \textit{C},设 \textit{M}(\textit{x},\textit{y}),因为{\textbar}\textit{d}(\textit{M},\textit{F}${}_{1}$)﹣\textit{d}(\textit{M},\textit{F}${}_{2}$){\textbar}=1,

所以{\textbar}\textit{x}+1{\textbar}﹣{\textbar}\textit{x}﹣1{\textbar}=1,解得 \includegraphics*[width=0.49in, height=0.37in, keepaspectratio=false]{image121},所以 \textit{M}点的轨迹为两条直线,故 \textit{C} 错误;

对于 \textit{D},如下图所示,\textit{M} 为圆 \textit{O} 上一点,\textit{N} 为直线 \textit{l} 上一点,过 \textit{M} 点作 \textit{x} 轴的平行线交直 线 \textit{l} 与点 \textit{D},过 \textit{N} 点作 \textit{x} 轴的垂线交 \textit{MD} 于点 \textit{C},

\includegraphics*[width=4.71in, height=3.15in, keepaspectratio=false]{image122}

当 \textit{M} 点固定时,显然当 \textit{N} 在 \textit{D},点上方时 \textit{d}(\textit{M},\textit{N})=\textit{CM}+\textit{CN} 最小,则

\textit{d}(\textit{M},\textit{N})=\textit{CM}+\textit{CN}=\textit{MD}﹣\textit{CD}+\textit{CN},

又因为 \textit{NC}$\mathrm{\geqslant}$\textit{CD},所以 \textit{d}(\textit{M},\textit{N})=\textit{CM}+\textit{CN}=\textit{MD}﹣\textit{CD}+\textit{CN}$\mathrm{\geqslant}$\textit{MD},

由几何关系易得当 \textit{OM}$\mathrm{\bot}$\textit{l} 时此时 \textit{MD} 取得最小值,如下图所示

\includegraphics*[width=4.94in, height=3.12in, keepaspectratio=false]{image123}

由点到直线的距离公式得 \includegraphics*[width=1.02in, height=0.40in, keepaspectratio=false]{image124},所以

\includegraphics*[width=2.92in, height=0.81in, keepaspectratio=false]{image125}

所以 \includegraphics*[width=2.65in, height=0.38in, keepaspectratio=false]{image126},故 \textit{D} 正确.

故选:\textit{ABD}.

【点评】本题考查新定义问题,考查分类讨论思想,考查绝对值方程的解法,考查点到直线的距离公式,综合性比较强,属于难题.

\textbf{三、填空题:本题共4小题,每小题5分,共20分。}

13.(5分)命题``若\textit{a}>\textit{b},则\textit{a}+\textit{c}>\textit{b}+\textit{c}''的逆否命题是\underbar{ 若}\textit{\underbar{a}}\underbar{+}\textit{\underbar{c}}\underbar{$\mathrm{\le}$}\textit{\underbar{b}}\underbar{+}\textit{\underbar{c}}\underbar{,则}\textit{\underbar{a}}\underbar{$\mathrm{\le}$}\textit{\underbar{b}}\underbar{ }.

【分析】根据命题``若\textit{p},则\textit{q}''的逆否命题是``若¬\textit{q},则¬\textit{p}'',写出它的逆否命题即可.

【解答】解:命题``\textit{a}>\textit{b},则\textit{a}+\textit{c}>\textit{b}+\textit{c}''的逆否命题``若\textit{a}+\textit{c}$\mathrm{\le}$\textit{b}+\textit{c},则\textit{a}$\mathrm{\le}$\textit{b}'',

故答案为:若\textit{a}+\textit{c}$\mathrm{\le}$\textit{b}+\textit{c},则\textit{a}$\mathrm{\le}$\textit{b}.

【点评】本题考查了四种命题之间的关系的应用问题,解题时应熟悉命题与逆否命题之间的关系,是容易题.

14.(5分)若方程\includegraphics*[width=0.23in, height=0.43in, keepaspectratio=false]{image127}+\includegraphics*[width=0.30in, height=0.44in, keepaspectratio=false]{image128}=1表示焦点在\textit{x}轴上的双曲线,则\textit{m}的取值范围是\underbar{ (0,2) }.

【分析】利用双曲线的性质,列出不等式求解即可.

【解答】解:方程\includegraphics*[width=0.23in, height=0.43in, keepaspectratio=false]{image129}+\includegraphics*[width=0.30in, height=0.44in, keepaspectratio=false]{image130}=1表示焦点在\textit{x}轴上的双曲线,

可得:\includegraphics*[width=0.63in, height=0.45in, keepaspectratio=false]{image131},解得\textit{m}$\mathrm{\in }$(0,2).

故答案为:(0,2).

【点评】本题考查双曲线的简单性质的应用,是基本知识的考查.

15.(5分)椭圆\includegraphics*[width=0.23in, height=0.48in, keepaspectratio=false]{image132}+\includegraphics*[width=0.23in, height=0.49in, keepaspectratio=false]{image133}=1(\textit{a}>\textit{b}>0)与圆\textit{x}${}^{2}$+\textit{y}${}^{2}$=(\includegraphics*[width=0.13in, height=0.37in, keepaspectratio=false]{image134}+\textit{c})${}^{2}$(\textit{c}为椭圆半焦距)有四个不同交点,则离心率的取值范围是\underbar{ }\includegraphics*[width=0.81in, height=0.38in, keepaspectratio=false]{image135}\underbar{ }.

【分析】由圆的方程求得圆的半径,要使椭圆与圆有四个不同交点,则圆的半径大于椭圆短半轴小于椭圆长半轴长,由此得到不等式求得椭圆离心率的范围.

【解答】解:由圆\textit{x}${}^{2}$+\textit{y}${}^{2}$=(\includegraphics*[width=0.13in, height=0.37in, keepaspectratio=false]{image136}+\textit{c})${}^{2}$是以原点为圆心,以\includegraphics*[width=0.33in, height=0.37in, keepaspectratio=false]{image137}为半径的圆,

$\mathrm{\therefore}$要使椭圆\includegraphics*[width=0.23in, height=0.48in, keepaspectratio=false]{image138}+\includegraphics*[width=0.23in, height=0.49in, keepaspectratio=false]{image139}=1(\textit{a}>\textit{b}>0)与圆\textit{x}${}^{2}$+\textit{y}${}^{2}$=(\includegraphics*[width=0.13in, height=0.37in, keepaspectratio=false]{image140}+\textit{c})${}^{2}$有四个不同交点,

则\includegraphics*[width=0.84in, height=0.37in, keepaspectratio=false]{image141},

由\includegraphics*[width=0.60in, height=0.37in, keepaspectratio=false]{image142},得\textit{b}<2\textit{c},即\textit{a}${}^{2}$﹣\textit{c}${}^{2}$<4\textit{c}${}^{2}$,即\includegraphics*[width=0.56in, height=0.38in, keepaspectratio=false]{image143};

联立\includegraphics*[width=0.90in, height=0.65in, keepaspectratio=false]{image144},解得\includegraphics*[width=0.46in, height=0.37in, keepaspectratio=false]{image145}或\textit{e}>1(舍).

$\mathrm{\therefore}$椭圆离心率的取值范围是\includegraphics*[width=0.81in, height=0.38in, keepaspectratio=false]{image146}.

故答案为:\includegraphics*[width=0.81in, height=0.38in, keepaspectratio=false]{image147}.

【点评】本题考查了椭圆的简单几何性质,考查了椭圆与圆的位置关系,是基础题.

16.(5分)``嫦娥四号''探测器实现历史上的首次月背着陆,如图是``嫦娥四号''运行轨道示意图,圆形轨道距月球表面100千米,椭圆形轨道的一个焦点是月球球心,一个长轴顶点位于两轨道相切的变轨处,另一个长轴顶点距月球表面15千米,则椭圆形轨道的焦距为\underbar{ 85 }千米.

\includegraphics*[width=1.65in, height=1.44in, keepaspectratio=false]{image148}

【分析】根据近月点和远月点与焦点的距离列方程组求解.

【解答】解:设椭圆长轴长为2\textit{a},焦距为2\textit{c},月球半径为\textit{R},

则\includegraphics*[width=0.88in, height=0.39in, keepaspectratio=false]{image149},两式作差,可得2\textit{c}=85.

$\mathrm{\therefore}$椭圆形轨道的焦距为85千米.

故答案为:85.

【点评】本题考查了椭圆的简单性质,是基础的计算题.

\textbf{四、解答题:本题共6小题,共70分。解答应写出文字说明、证明过程或演算步.}

17.(10分)设\textit{p}:\textit{x}$\mathrm{\in }$\textit{P}=$\mathrm{\{}$\textit{x}{\textbar}2\textit{m}﹣1$\mathrm{\le}$\textit{x}$\mathrm{\le}$2\textit{m}${}^{2}$﹣\textit{m}$\mathrm{\}}$$\mathrm{\neq}$$\mathrm{\emptyset }$,\textit{q}:\textit{x}$\mathrm{\in }$\textit{S}=$\mathrm{\{}$\textit{x}{\textbar}\textit{x}${}^{2}$﹣2\textit{x}﹣3$\mathrm{\le}$0$\mathrm{\}}$,且¬\textit{p}是¬\textit{q}的必要不充分条件,求实数\textit{m}的取值范围.

【分析】解不等式求出满足\textit{q}的\textit{x}的范围,根据集合的包含关系得到[2\textit{m}﹣1,2\textit{m}${}^{2}$﹣\textit{m}]$\mathrm{\subsetneqq}$[﹣1,3],得到关于\textit{m}的不等式组,解出即可.

【解答】解:由$\mathrm{\{}$\textit{x}{\textbar}2\textit{m}﹣1$\mathrm{\le}$\textit{x}$\mathrm{\le}$2\textit{m}${}^{2}$﹣\textit{m}$\mathrm{\}}$$\mathrm{\neq}$$\mathrm{\emptyset }$,得:2\textit{m}﹣1$\mathrm{\le}$2\textit{m}${}^{2}$﹣\textit{m},解得:\textit{m}$\mathrm{\ge}$1或\textit{m}$\mathrm{\le}$\includegraphics*[width=0.13in, height=0.37in, keepaspectratio=false]{image150},

由$\mathrm{\{}$\textit{x}{\textbar}\textit{x}${}^{2}$﹣2\textit{x}﹣3$\mathrm{\le}$0$\mathrm{\}}$,得:﹣1$\mathrm{\le}$\textit{x}$\mathrm{\le}$3,故满足\textit{q}的集合\textit{B}=$\mathrm{\{}$\textit{x}{\textbar}﹣1$\mathrm{\le}$\textit{x}$\mathrm{\le}$3$\mathrm{\}}$,

由¬\textit{p}是¬\textit{q}的必要不充分条件,即\textit{q}是\textit{p}的必要不充分条件,

故[2\textit{m}﹣1,2\textit{m}${}^{2}$﹣\textit{m}]$\mathrm{\subsetneqq}$[﹣1,3],即\includegraphics*[width=0.82in, height=0.46in, keepaspectratio=false]{image151},解得:0$\mathrm{\le}$\textit{m}$\mathrm{\le}$\includegraphics*[width=0.13in, height=0.37in, keepaspectratio=false]{image152},

而\textit{m}$\mathrm{\ge}$1或\textit{m}$\mathrm{\le}$\includegraphics*[width=0.13in, height=0.37in, keepaspectratio=false]{image153},

故\textit{m}的取值范围是[0,\includegraphics*[width=0.13in, height=0.37in, keepaspectratio=false]{image154}]$\mathrm{\cup}$[1,\includegraphics*[width=0.13in, height=0.37in, keepaspectratio=false]{image155}].

【点评】本题考查了集合的运算,考查不等式问题以及集合的包含关系,考查转化思想,是一道基础题.

18.(12分)如图,三棱柱\textit{ABC}﹣\textit{A}${}_{1}$\textit{B}${}_{1}$\textit{C}${}_{1}$中,\textit{A}${}_{1}$\textit{B}${}_{1}$$\mathrm{\bot}$平面\textit{ACC}${}_{1}$\textit{A}${}_{1}$,$\mathrm{\angle}$\textit{CAA}${}_{1}$=60$\mathrm{{}^\circ}$,\textit{AB}=\textit{AA}${}_{1}$=1,\textit{AC}=2.

(1)证明:\textit{A}${}_{1}$\textit{A}$\mathrm{\bot}$\textit{B}${}_{1}$\textit{C};

(2)求异面直线\textit{AB}${}_{1}$与\textit{A}${}_{1}$\textit{C}${}_{1}$所成角的余弦值.

\includegraphics*[width=2.32in, height=1.32in, keepaspectratio=false]{image156}

【分析】(1)求出\textit{AA}${}_{1}$$\mathrm{\bot}$\textit{A}${}_{1}$\textit{C},根据\textit{A}${}_{1}$\textit{B}${}_{1}$$\mathrm{\bot}$平面\textit{ACC}${}_{1}$\textit{A}${}_{1}$,求出\textit{A}${}_{1}$\textit{B}${}_{1}$$\mathrm{\bot}$\textit{AA}${}_{1}$,从而求出\textit{AA}${}_{1}$$\mathrm{\bot}$平面\textit{A}${}_{1}$\textit{B}${}_{1}$\textit{C},证明结论成立即可;

(2)问题转化为求\textit{AB}${}_{1}$和\textit{AC}所成角的余弦值,根据余弦定理求出即可.

【解答】(1)证明:如图示:

\includegraphics*[width=2.32in, height=1.29in, keepaspectratio=false]{image157},

连接\textit{A}${}_{1}$\textit{C},在$\mathrm{\triangle}$\textit{A}${}_{1}$\textit{AC}中,\includegraphics*[width=0.38in, height=0.28in, keepaspectratio=false]{image158}=\includegraphics*[width=0.37in, height=0.28in, keepaspectratio=false]{image159}+\textit{AC}${}^{2}$﹣2\textit{AA}${}_{1}$\textit{AC}cos$\mathrm{\angle}$\textit{A}${}_{1}$\textit{AC},

即\includegraphics*[width=0.38in, height=0.28in, keepaspectratio=false]{image160}=1+4﹣2$\mathrm{\times}$1$\mathrm{\times}$2cos60$\mathrm{{}^\circ}$=3,故\textit{A}${}_{1}$\textit{C}=\includegraphics*[width=0.21in, height=0.19in, keepaspectratio=false]{image161},故\includegraphics*[width=0.38in, height=0.28in, keepaspectratio=false]{image162}+\textit{A}\includegraphics*[width=0.28in, height=0.28in, keepaspectratio=false]{image163}=\textit{AC}${}^{2}$,

$\mathrm{\therefore}$\textit{AA}${}_{1}$$\mathrm{\bot}$\textit{A}${}_{1}$\textit{C},又\textit{A}${}_{1}$\textit{B}${}_{1}$$\mathrm{\bot}$平面\textit{ACC}${}_{1}$\textit{A}${}_{1}$,\textit{AA}${}_{1}$$\mathrm{\subset }$平面\textit{ACC}${}_{1}$\textit{A}${}_{1}$,$\mathrm{\therefore}$\textit{A}${}_{1}$\textit{B}${}_{1}$$\mathrm{\bot}$\textit{AA}${}_{1}$,

又\textit{A}${}_{1}$\textit{B}${}_{1}$$\mathrm{\cap}$\textit{A}${}_{1}$\textit{C}=\textit{A}${}_{1}$,$\mathrm{\therefore}$\textit{AA}${}_{1}$$\mathrm{\bot}$平面\textit{A}${}_{1}$\textit{B}${}_{1}$\textit{C},而\textit{B}${}_{1}$\textit{C}$\mathrm{\subset }$平面\textit{A}${}_{1}$\textit{B}${}_{1}$\textit{C},

$\mathrm{\therefore}$\textit{A}${}_{1}$\textit{A}$\mathrm{\bot}$\textit{B}${}_{1}$\textit{C};

(2)$\mathrm{\because}$\textit{A}${}_{1}$\textit{C}${}_{1}$$\mathrm{\parallel}$\textit{AC},$\mathrm{\therefore}$异面直线\textit{AB}${}_{1}$与\textit{A}${}_{1}$\textit{C}${}_{1}$所成角的余弦值,

即\textit{AB}${}_{1}$和\textit{AC}所成角的余弦值,

由\textit{AB}${}_{1}$=\includegraphics*[width=0.37in, height=0.19in, keepaspectratio=false]{image164}=\includegraphics*[width=0.21in, height=0.19in, keepaspectratio=false]{image165},\textit{AC}=2,\textit{B}${}_{1}$\textit{C}=\includegraphics*[width=1.04in, height=0.35in, keepaspectratio=false]{image166}=\includegraphics*[width=0.37in, height=0.19in, keepaspectratio=false]{image167}=2,

故cos$\mathrm{\angle}$\textit{B}${}_{1}$\textit{AC}=\includegraphics*[width=0.78in, height=0.38in, keepaspectratio=false]{image168}=\includegraphics*[width=0.24in, height=0.38in, keepaspectratio=false]{image169}.

故异面直线\textit{AB}${}_{1}$与\textit{A}${}_{1}$\textit{C}${}_{1}$所成角的余弦值是\includegraphics*[width=0.24in, height=0.38in, keepaspectratio=false]{image170}.

【点评】本题考查了线线,线面垂直,考查余弦定理的应用以及转化思想,数形结合思想,是一道中档题.

19.(12分)已知双曲线\textit{C}的焦距为4,点\textit{M}(\includegraphics*[width=0.21in, height=0.19in, keepaspectratio=false]{image171},1)在双曲线上,且抛物线\textit{y}${}^{2}$=2\textit{px}(\textit{p}>0)的焦点\textit{F}与双曲线的个焦点重合.

(1)求双曲线和抛物线的标准方程;

(2)过焦点\textit{F}作一条直线\textit{l}交抛物线\textit{A}、\textit{B}两点,当直线\textit{l}的斜率为1时,求线段\textit{AB}的长度.

【分析】(1)设双曲线的方程为\includegraphics*[width=0.23in, height=0.48in, keepaspectratio=false]{image172}﹣\includegraphics*[width=0.23in, height=0.49in, keepaspectratio=false]{image173}=1(\textit{a}>0,\textit{b}>0),根据题意可得2\textit{c}=4,\textit{a}${}^{2}$+\textit{b}${}^{2}$=4,\includegraphics*[width=0.23in, height=0.43in, keepaspectratio=false]{image174}﹣\includegraphics*[width=0.23in, height=0.43in, keepaspectratio=false]{image175}=1,解得\textit{a}${}^{2}$,\textit{b}${}^{2}$,进而可得双曲线标准方程,抛物线的标准方程.

(2)联立直线\textit{l}与抛物线方程,得到两根之和,根据抛物线的焦半径公式易求得线段\textit{AB}的长度.

【解答】解:(1)设双曲线的方程为\includegraphics*[width=0.23in, height=0.48in, keepaspectratio=false]{image176}﹣\includegraphics*[width=0.23in, height=0.49in, keepaspectratio=false]{image177}=1(\textit{a}>0,\textit{b}>0)

由题设2\textit{c}=4,\textit{c}=2,

所以\textit{a}${}^{2}$+\textit{b}${}^{2}$=4,①

又点(\includegraphics*[width=0.21in, height=0.19in, keepaspectratio=false]{image178},1)在双曲线上,所以\includegraphics*[width=0.23in, height=0.43in, keepaspectratio=false]{image179}﹣\includegraphics*[width=0.23in, height=0.43in, keepaspectratio=false]{image180}=1,②,

由①②解得\textit{a}${}^{2}$=3,\textit{b}${}^{2}$=1,

故双曲线标准方程为\includegraphics*[width=0.23in, height=0.43in, keepaspectratio=false]{image181}﹣\textit{y}${}^{2}$=1,

由抛物线焦点为\textit{F}(2,0),即\includegraphics*[width=0.13in, height=0.37in, keepaspectratio=false]{image182}=2,解得\textit{p}=4,

所以抛物线的标准方程为\textit{y}${}^{2}$=8\textit{x}.

(2)设直线\textit{y}=\textit{x}﹣2交抛物线于\textit{A}(\textit{x}${}_{1}$,\textit{y}${}_{1}$),\textit{B}(\textit{x}${}_{2}$,\textit{y}${}_{2}$),

联立\includegraphics*[width=0.57in, height=0.48in, keepaspectratio=false]{image183},得\textit{x}${}^{2}$﹣12\textit{x}+4=0,故\textit{x}${}_{1}$+\textit{x}${}_{2}$=12,

由抛物线定义知\textit{AF}=\textit{x}${}_{1}$+2,\textit{BF}=\textit{x}${}_{2}$+2,

所以{\textbar}\textit{AB}{\textbar}=\textit{x}${}_{1}$+\textit{x}${}_{2}$+4=16.

【点评】本题考查双曲线,抛物线方程,直线与抛物线相交问题,属于中档题.

20.(12分)某企业为了参加上海的进博会,大力研发新产品,为了对新研发的一批产品进行合理定价,将该产品按事先拟定的价格进行试销,得到一组销售数据(\textit{x${}_{i}$},\textit{y${}_{i}$})(\textit{i}=1,2,{\dots},6),如表所示:

\begin{tabular}{|p{0.9in}|p{0.2in}|p{0.3in}|p{0.3in}|p{0.3in}|p{0.3in}|p{0.3in}|} \hline 
试销单价\textit{x}/元 & 4 & 5 & 6 & 7 & 8 & 9 \\ \hline 
产品销量\textit{y}/件 & \textit{q} & 84 & 83 & 80 & 75 & 68 \\ \hline 
\end{tabular}

已知\includegraphics*[width=0.10in, height=0.19in, keepaspectratio=false]{image184}=\includegraphics*[width=0.13in, height=0.37in, keepaspectratio=false]{image185}\includegraphics*[width=0.23in, height=0.48in, keepaspectratio=false]{image186}\textit{y${}_{i}$}=80.

(1)求\textit{q}的值;

(2)已知变量\textit{x},\textit{y}具有线性相关关系,求产品销量\textit{y}(件)关于试销单价\textit{x}(元)的线性回归方程\includegraphics*[width=0.10in, height=0.34in, keepaspectratio=false]{image187}=\includegraphics*[width=0.10in, height=0.33in, keepaspectratio=false]{image188}\textit{x}+\includegraphics*[width=0.10in, height=0.33in, keepaspectratio=false]{image189};

(3)用\includegraphics*[width=0.19in, height=0.28in, keepaspectratio=false]{image190}表示用正确的线性回归方程得到的与\textit{x${}_{i}$}对应的产品销量的估计值,当{\textbar}\includegraphics*[width=0.19in, height=0.28in, keepaspectratio=false]{image191}﹣\textit{y${}_{i}$}{\textbar}$\mathrm{\le}$1时,将销售数据(\textit{x${}_{i}$},\textit{y${}_{i}$})称为一个``好数据'',现从6个销售数据中任取2个,求抽取的2个销售数据中至少有一个是``好数据''的概率.

参考公式:\includegraphics*[width=0.10in, height=0.33in, keepaspectratio=false]{image192}=\includegraphics*[width=1.03in, height=0.98in, keepaspectratio=false]{image193}=\includegraphics*[width=1.45in, height=0.98in, keepaspectratio=false]{image194},\includegraphics*[width=0.10in, height=0.33in, keepaspectratio=false]{image195}=\includegraphics*[width=0.10in, height=0.19in, keepaspectratio=false]{image196}﹣\includegraphics*[width=0.10in, height=0.33in, keepaspectratio=false]{image197}\includegraphics*[width=0.10in, height=0.19in, keepaspectratio=false]{image198}.

【分析】(1)由\includegraphics*[width=0.10in, height=0.19in, keepaspectratio=false]{image199}=\includegraphics*[width=0.13in, height=0.37in, keepaspectratio=false]{image200}\includegraphics*[width=0.23in, height=0.48in, keepaspectratio=false]{image201}\textit{y${}_{i}$}=80直接求得\textit{q}值;

(2)由已知数据求出回归系数,进一步求得\includegraphics*[width=0.10in, height=0.33in, keepaspectratio=false]{image202},可得线性回归方程;

(3)确定基本事件的个数,再由古典概型概率计算公式求解.

【解答】解:(1)由\includegraphics*[width=0.10in, height=0.19in, keepaspectratio=false]{image203}=\includegraphics*[width=0.13in, height=0.37in, keepaspectratio=false]{image204}\includegraphics*[width=0.23in, height=0.48in, keepaspectratio=false]{image205}\textit{y${}_{i}$}=80,求得\textit{q}=90;

(2)\includegraphics*[width=1.78in, height=0.43in, keepaspectratio=false]{image206},\includegraphics*[width=0.10in, height=0.33in, keepaspectratio=false]{image207}=80+4$\mathrm{\times}$6.5=106,

$\mathrm{\therefore}$所求的线性回归方程为\includegraphics*[width=0.10in, height=0.34in, keepaspectratio=false]{image208}=﹣4\textit{x}+106;

(3)当\textit{x}${}_{1}$=4时,\textit{y}${}_{1}$=90;当\textit{x}${}_{2}$=5时,\textit{y}${}_{2}$=86;当\textit{x}${}_{3}$=6时,\textit{y}${}_{3}$=82;

当\textit{x}${}_{4}$=7时,\textit{y}${}_{4}$=78;当\textit{x}${}_{5}$=8时,\textit{y}${}_{5}$=74;当\textit{x}${}_{6}$=9时,\textit{y}${}_{6}$=70.

与销售数据对比可知满足{\textbar}\includegraphics*[width=0.19in, height=0.28in, keepaspectratio=false]{image209}﹣\textit{y${}_{i}$}{\textbar}$\mathrm{\le}$1(\textit{i}=1,2,{\dots},6)的共有3个``好数据'':

(4,90)、(6,83)、(8,75).

从6个销售数据中任意抽取2个的所有可能结果有\includegraphics*[width=0.38in, height=0.37in, keepaspectratio=false]{image210}=15种,

其中2个数据中至少有一个是``好数据''的结果有3$\mathrm{\times}$3+3=12种,

于是从抽得2个数据中至少有一个销售数据中的产品销量不超过80的概率为\includegraphics*[width=0.46in, height=0.37in, keepaspectratio=false]{image211}.

【点评】本题考查线性回归方程,概率的计算,考查学生的计算能力,属于中档题.

21.(12分)动圆\textit{M}与圆\textit{F}:(\textit{x}+1)${}^{2}$+\textit{y}${}^{2}$=8相内切,且恒过点\textit{F}$\mathrm{\prime}$(1,0).

(1)求动圆圆心\textit{M}的轨迹\textit{E}的方程;

(2)已知垂直于\textit{x}轴的直线\textit{l}${}_{1}$交\textit{E}于\textit{A}、\textit{B}两点,垂直于\textit{y}轴的直线\textit{l}${}_{2}$交\textit{E}于\textit{C}、\textit{D}两点,\textit{l}${}_{1}$与\textit{l}${}_{2}$的交点为\textit{P},且{\textbar}\textit{AB}{\textbar}={\textbar}\textit{CD}{\textbar},证明:存在两定点\textit{M}、\textit{N},使得{\textbar}{\textbar}\textit{PM}{\textbar}﹣{\textbar}\textit{PN}{\textbar}{\textbar}为定值,求出\textit{M}、\textit{N}的坐标.

【分析】(1)设出\textit{M}的半径,依据题意列出关系\textit{MF}+\textit{MF}$\mathrm{\prime}$=2\includegraphics*[width=0.21in, height=0.19in, keepaspectratio=false]{image212},可求轨迹\textit{E}的方程;

(2)设\textit{A}(\textit{x}${}_{1}$,\textit{y}${}_{1}$),\textit{B}(\textit{x}${}_{1}$,﹣\textit{y}${}_{1}$),\textit{C}(\textit{x}${}_{2}$,\textit{y}${}_{2}$),\textit{D}(﹣\textit{x}${}_{2}$,\textit{y}${}_{2}$),则\textit{P}(\textit{x}${}_{1}$,\textit{y}${}_{2}$),\includegraphics*[width=0.23in, height=0.48in, keepaspectratio=false]{image213}+\includegraphics*[width=0.19in, height=0.29in, keepaspectratio=false]{image214}=1,\includegraphics*[width=0.23in, height=0.48in, keepaspectratio=false]{image215}+\includegraphics*[width=0.19in, height=0.29in, keepaspectratio=false]{image216}=1,{\textbar}\textit{y}${}_{1}${\textbar}={\textbar}\textit{x}${}_{2}${\textbar},消去\textit{x}${}_{2}$,\textit{y}${}_{1}$,得2\includegraphics*[width=0.19in, height=0.29in, keepaspectratio=false]{image217}﹣\includegraphics*[width=0.23in, height=0.48in, keepaspectratio=false]{image218}=1,从而得到点\textit{P}在双曲线\textit{T}:2\textit{y}${}^{2}$﹣\includegraphics*[width=0.23in, height=0.43in, keepaspectratio=false]{image219}=1上,所以当点\textit{M},\textit{N}为双曲线的焦点时,{\textbar}{\textbar}\textit{PM}{\textbar}﹣{\textbar}\textit{PN}{\textbar}{\textbar}为定值.

【解答】(1)解:设圆\textit{M}的半径为\textit{r}.

因为圆过点\textit{F}$\mathrm{\prime}$(1,0),且与圆\textit{F}相内切,所以\textit{MF}$\mathrm{\prime}$=\textit{r},

所以\textit{MF}=2\includegraphics*[width=0.21in, height=0.19in, keepaspectratio=false]{image220}﹣\textit{MF}$\mathrm{\prime}$,即:\textit{MF}+\textit{MF}$\mathrm{\prime}$=2\includegraphics*[width=0.21in, height=0.19in, keepaspectratio=false]{image221},

所以点\textit{M}的轨迹\textit{E}是以\textit{F},\textit{F}$\mathrm{\prime}$为焦点的椭圆,

其中2\textit{a}=2\includegraphics*[width=0.21in, height=0.19in, keepaspectratio=false]{image222},\textit{c}=1,所以\textit{a}=\includegraphics*[width=0.21in, height=0.19in, keepaspectratio=false]{image223},\textit{b}=1,

所以曲线\textit{C}的方程为\includegraphics*[width=0.23in, height=0.43in, keepaspectratio=false]{image224}+\textit{y}${}^{2}$=1.

(2)证明:设\textit{A}(\textit{x}${}_{1}$,\textit{y}${}_{1}$),\textit{B}(\textit{x}${}_{1}$,﹣\textit{y}${}_{1}$),\textit{C}(\textit{x}${}_{2}$,\textit{y}${}_{2}$),\textit{D}(﹣\textit{x}${}_{2}$,\textit{y}${}_{2}$),

则\textit{P}(\textit{x}${}_{1}$,\textit{y}${}_{2}$),\includegraphics*[width=0.23in, height=0.48in, keepaspectratio=false]{image225}+\includegraphics*[width=0.19in, height=0.29in, keepaspectratio=false]{image226}=1,\includegraphics*[width=0.23in, height=0.48in, keepaspectratio=false]{image227}+\includegraphics*[width=0.19in, height=0.29in, keepaspectratio=false]{image228}=1,{\textbar}\textit{y}${}_{1}${\textbar}={\textbar}\textit{x}${}_{2}${\textbar},

消去\textit{x}${}_{2}$,\textit{y}${}_{1}$,得2\includegraphics*[width=0.19in, height=0.29in, keepaspectratio=false]{image229}﹣\includegraphics*[width=0.23in, height=0.48in, keepaspectratio=false]{image230}=1,

所以点\textit{P}在双曲线\textit{T}:2\textit{y}${}^{2}$﹣\includegraphics*[width=0.23in, height=0.43in, keepaspectratio=false]{image231}=1上,

因为\textit{T}的两个焦点为\textit{M}(0,\includegraphics*[width=0.33in, height=0.38in, keepaspectratio=false]{image232}),\textit{N}(0,﹣\includegraphics*[width=0.33in, height=0.38in, keepaspectratio=false]{image233}),实轴长为\includegraphics*[width=0.21in, height=0.19in, keepaspectratio=false]{image234},

所以存在两定点\textit{M}(0,\includegraphics*[width=0.33in, height=0.38in, keepaspectratio=false]{image235}),\textit{N}(0,﹣\includegraphics*[width=0.33in, height=0.38in, keepaspectratio=false]{image236}),使得{\textbar}{\textbar}\textit{PM}{\textbar}﹣{\textbar}\textit{PN}{\textbar}{\textbar}为定值\includegraphics*[width=0.21in, height=0.19in, keepaspectratio=false]{image237}.

【点评】本题考查了圆与圆的位置关系,椭圆的定义,椭圆的方程,以及直线与椭圆的位置关系,属于中档题.

22.(12分)在①离心率\textit{e}=\includegraphics*[width=0.13in, height=0.37in, keepaspectratio=false]{image238},②椭圆\textit{C}过点(1,\includegraphics*[width=0.13in, height=0.37in, keepaspectratio=false]{image239}),③\textit{P}为椭圆上一点,$\mathrm{\triangle}$\textit{PF}${}_{1}$\textit{F}${}_{2}$面积的最大值为\includegraphics*[width=0.21in, height=0.19in, keepaspectratio=false]{image240},这三个条件中任选一个,补充在下面(横线处)问题中,解决下面两个问题.

设椭圆\textit{C}:\includegraphics*[width=0.23in, height=0.48in, keepaspectratio=false]{image241}+\includegraphics*[width=0.23in, height=0.49in, keepaspectratio=false]{image242}=1(\textit{a}>\textit{b}>0)的左、右焦点分别为\textit{F}${}_{1}$、\textit{F}${}_{2}$,已知椭圆\textit{C}的短轴长为2\includegraphics*[width=0.21in, height=0.19in, keepaspectratio=false]{image243},\_\_\_\_\_\_.

(1)求椭圆\textit{C}的方程;

(2)过\textit{F}${}_{1}$的直线\textit{l}交椭圆\textit{C}于\textit{A}、\textit{B}两点,请问$\mathrm{\triangle}$\textit{ABF}${}_{2}$的内切圆\textit{E}的面积是否存在最大值?若存在,求出这个最大值及直线\textit{l}的方程,若不存在,请说明理由.

【分析】(1)选①离心率\textit{e}=\includegraphics*[width=0.13in, height=0.37in, keepaspectratio=false]{image244},依题意\includegraphics*[width=0.90in, height=0.91in, keepaspectratio=false]{image245},解得\textit{a},\textit{b},进而可得椭圆\textit{C}的方程.

(2)设$\mathrm{\triangle}$\textit{ABF}${}_{2}$内切圆\textit{E}的半径为\textit{r},则\textit{S}\includegraphics*[width=0.46in, height=0.18in, keepaspectratio=false]{image246}=\includegraphics*[width=0.13in, height=0.37in, keepaspectratio=false]{image247}({\textbar}\textit{AB}{\textbar}+{\textbar}\textit{AF}${}_{2}${\textbar}+{\textbar}\textit{BF}${}_{2}${\textbar})\textit{r}=2\textit{ar}=4\textit{r},当\textit{S}\includegraphics*[width=0.46in, height=0.18in, keepaspectratio=false]{image248}最大时,\textit{r}也最大,$\mathrm{\triangle}$\textit{ABF}${}_{2}$内切圆的面积也最大,只需求出\textit{r}的最大值,即可得出答案.

【解答】解:(1)选①离心率\textit{e}=\includegraphics*[width=0.13in, height=0.37in, keepaspectratio=false]{image249},

依题意\includegraphics*[width=0.90in, height=0.91in, keepaspectratio=false]{image250},解得\includegraphics*[width=0.54in, height=0.45in, keepaspectratio=false]{image251},

所以椭圆\textit{C}的方程为\includegraphics*[width=0.23in, height=0.43in, keepaspectratio=false]{image252}+\includegraphics*[width=0.23in, height=0.44in, keepaspectratio=false]{image253}=1.

(2)设$\mathrm{\triangle}$\textit{ABF}${}_{2}$内切圆\textit{E}的半径为\textit{r},则$\mathrm{\triangle}$\textit{ABF}${}_{2}$的面积,

\textit{S}\includegraphics*[width=0.46in, height=0.18in, keepaspectratio=false]{image254}=\includegraphics*[width=0.13in, height=0.37in, keepaspectratio=false]{image255}({\textbar}\textit{AB}{\textbar}+{\textbar}\textit{AF}${}_{2}${\textbar}+{\textbar}\textit{BF}${}_{2}${\textbar})\textit{r}=\includegraphics*[width=0.13in, height=0.37in, keepaspectratio=false]{image256}[({\textbar}\textit{AF}${}_{1}${\textbar}+{\textbar}\textit{AF}${}_{2}${\textbar})+({\textbar}\textit{BF}${}_{1}${\textbar}+{\textbar}\textit{BF}${}_{2}${\textbar})]\textit{r}=2\textit{ar}=4\textit{r},

当\textit{S}\includegraphics*[width=0.46in, height=0.18in, keepaspectratio=false]{image257}最大时,\textit{r}也最大,$\mathrm{\triangle}$\textit{ABF}${}_{2}$内切圆的面积也最大,

设\textit{A}(\textit{x}${}_{1}$,\textit{y}${}_{1}$),\textit{B}(\textit{x}${}_{2}$,\textit{y}${}_{2}$),(\textit{y}${}_{1}$>0,\textit{y}${}_{2}$<0),

则\textit{S}\includegraphics*[width=0.46in, height=0.18in, keepaspectratio=false]{image258}=\includegraphics*[width=0.13in, height=0.37in, keepaspectratio=false]{image259}{\textbar}\textit{F}${}_{1}$\textit{F}${}_{2}${\textbar}{\textbar}\textit{y}${}_{1}${\textbar}+\includegraphics*[width=0.13in, height=0.37in, keepaspectratio=false]{image260}{\textbar}\textit{F}${}_{1}$\textit{F}${}_{2}${\textbar}{\textbar}\textit{y}${}_{2}${\textbar}=\textit{y}${}_{1}$﹣\textit{y}${}_{2}$,

由\includegraphics*[width=0.84in, height=0.68in, keepaspectratio=false]{image261},得(3\textit{m}${}^{2}$+4)\textit{y}${}^{2}$﹣6\textit{my}﹣9=0,

$\mathrm{\triangle}$=(﹣6\textit{m})${}^{2}$+36(3\textit{m}${}^{2}$+4)>0,

\textit{y}${}_{1}$+\textit{y}${}_{2}$=\includegraphics*[width=0.52in, height=0.43in, keepaspectratio=false]{image262},\textit{y}${}_{1}$\textit{y}${}_{2}$=\includegraphics*[width=0.52in, height=0.43in, keepaspectratio=false]{image263},

所以\textit{S}\includegraphics*[width=0.46in, height=0.18in, keepaspectratio=false]{image264}=\includegraphics*[width=0.71in, height=0.50in, keepaspectratio=false]{image265},

令\textit{t}=\includegraphics*[width=0.49in, height=0.25in, keepaspectratio=false]{image266},则\textit{t}$\mathrm{\ge}$1,且\textit{m}${}^{2}$=\textit{t}${}^{2}$﹣1,

有\textit{S}\includegraphics*[width=0.46in, height=0.18in, keepaspectratio=false]{image267}=\includegraphics*[width=0.85in, height=0.43in, keepaspectratio=false]{image268}=\includegraphics*[width=0.52in, height=0.43in, keepaspectratio=false]{image269}=\includegraphics*[width=0.44in, height=0.56in, keepaspectratio=false]{image270},

令\textit{f}(\textit{t})=3\textit{t}+\includegraphics*[width=0.13in, height=0.37in, keepaspectratio=false]{image271}(\textit{t}$\mathrm{\ge}$1),

当\textit{t}$\mathrm{\ge}$1时,\textit{f}(\textit{t})在[1,+$\mathrm{\infty}$)上单调递增,有\textit{f}(\textit{t})$\mathrm{\ge}$\textit{f}(1)=4,

\textit{S}\includegraphics*[width=0.46in, height=0.18in, keepaspectratio=false]{image272}$\mathrm{\le}$\includegraphics*[width=0.22in, height=0.37in, keepaspectratio=false]{image273}=3,即当\textit{t}=1,\textit{m}=0时,4\textit{r}有最大值3,得\textit{r${}_{max}$}=\includegraphics*[width=0.13in, height=0.37in, keepaspectratio=false]{image274},

这时所求内切圆的面积为\includegraphics*[width=0.40in, height=0.37in, keepaspectratio=false]{image275},

所以存在直线\textit{l}:\textit{x}=﹣1,$\mathrm{\triangle}$\textit{ABF}${}_{2}$的内切圆\textit{M}的面积最大值为\includegraphics*[width=0.40in, height=0.37in, keepaspectratio=false]{image276}.

【点评】本题考查直线与椭圆的相交问题,解题中需要一定的运算能力,属于中档题.

声明:试题解析著作权属菁优网所有,未经书面同意,不得复制发布

日期:2021/1/12 17:37:58;用户:15210223722;邮箱:15210223722;学号:38732274


\end{document}


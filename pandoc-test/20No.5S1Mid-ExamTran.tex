  \item
    已知集合$A=\{x\in\mathbb{N}\mid -1\leq x \leq 4\}$,$B\{x\mid -2\leq x \leq 3\}$,则$A\cap B=$\xz
    \xx{$[-1,3]]$}
      {$[-2,4]]$}
      {$\{0,1,2,3\}$}
      {$\{1,2,3\}$}
    \begin{answer}
      
    \end{answer}
  \item
    命题“$\forall x\geq0,x^2-1\geq-1$”的否定是\xz
      \xx{$\forall x\geq0,x^2-1<-1$}
      {$\forall x<0,x^2-1<-1$}
      {$\exists x\geq0,x^2-1<-1$}
      {$\exists x<0,x^2-1<-1$}
    \begin{answer}
      
    \end{answer}
  \item
    设函数$f(x)=\begin{cases}
      2^x+m,x\leqslant0,\\
      g(x),x>0.
    \end{cases}$是奇函数,则$f(2)=$\xz
      \xx{$\frac34$}{$-\frac34$}{$4$}{$-4$}
    \begin{answer}
      
    \end{answer}
    \item
    函数$y=\frac{xa^x}{|x|}(a>1)$的图像的大致形状是\xz
      \xx{\includegraphics{}}
       {$1$}
       {$\dfrac{24}{25}$}
       {$-\dfrac{24}{25}$}

         \item
      已知函数$f(x)=1-ax+\log_2{\frac{1-x}{1+x}}$\\
      (1)若$f\left(\frac35\right)=-\frac85$,则实数$a$的值为\tk;  
      (2)$f\left(\frac1{2019}\right)+f\left(-\frac1{2019}\right)=$\tk. 
      \item
      酒驾
      \begin{answer}
        $x\geqslant\log_{\frac34}0.3=\frac{\lg0.3}{\lg\frac34}=\frac{\lg3-1}{\lg3-\lg4}\approx\frac{0.48-1}{0.48-0.60}\approx4.3$
      \end{answer}
   \item %《2019金考卷双测20套(文)ISBN978-7-5371-9890-5》题型9不等式P9p4【2018•大连双基测试】【线性规划】\\
      \source{2018文}{大连双基测试}
      设实数$x$,$y$满足约束条件
      $\left\{\begin{aligned}
        x-y+1\geqslant0\\
        x+y-1\leqslant0\\
        x-2y-1\leqslant0
      \end{aligned}\right.$
      则目标函数$z=2x+y$的取值范围为\xz
      \xx{$[1,+\infty)$}{$[2,+\infty)$}{$[-8,1]$}{$[-8,2]$}
 
      \item
    设$f(x)=\begin{cases}
      x-2,x\geq10\\f[f(x+6)],x<10
    \end{cases}$,则$f(9)$的值为\xz
      \xx{10}{11}{12}{13}
    \begin{answer}
      B
    \end{answer}
  \item  
      设集合~$A=\{x\mid -1< x <2\}$~,$B=\{x\mid x^2-3x<0\}$,
       \item
      $y=\frac1{f(x)}$
    \item
      $f(x)=\frac{ax+b}{x^2+1}$\\
       
  \item %《2019金考卷双测20套(文)ISBN978-7-5371-9890-5》题型9不等式P9p4【2018•大连双基测试】【线性规划】\\
    \source{2018文}{大连双基测试}
    设实数$x$,$y$满足约束条件
    $\left\{\begin{aligned}
      x-y+1\geqslant0\\
      x+y-1\leqslant0\\
      x-2y-1\leqslant0
    \end{aligned}\right.$
    $\begin{cases}
      x-y+1\geqslant0\\
      x+y-1\leqslant0\\
      x-2y-1\leqslant0
    \end{cases}$
    则目标函数$z=2x+y$的取值范围为\xz
    \xx{$[1,+\infty)$}{$[2,+\infty)$}{$[-8,1]$}{$[-8,2]$}
  \item %《2019金考卷双测20套(文)ISBN978-7-5371-9890-5》题型9不等式P9p4【2018•大连双基测试】【线性规划】\\
    \source{2018文}{大连双基测试}
    设实数$x$,$y$满足约束条件
    $\begin{cases}
      x-y+1\geqslant0\\
      x+y-1\leqslant0\\
      x-2y-1\leqslant0
    \end{cases}$
    则目标函数$z=2x+y$的取值范围为\xz
    \xx{$[1,+\infty)$}{$[2,+\infty)$}{$[-8,1]$}{$[-8,2]$}
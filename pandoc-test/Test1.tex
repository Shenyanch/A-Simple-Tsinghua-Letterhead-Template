% 首先,我们回顾一下极限的定义:\par
% 设$A$为给定常数,如果对任给的$\varepsilon>0$,存在正数$\delta$,使得当$0<|x-x_0|<\delta$时,有$|f(x)-A|<\varepsilon$,
% 则称{\fangsong 函数$f$当$x$趋于$x_0$时以$A$为极限}.\par
首先,要证明的结论是:\par
任取$\varepsilon>0$,存在$\delta>0$,使得当$0<|x-x_0|<\delta$时,有
    \begin{equation}
        \left|\frac{1}{g(x)}-\frac{1}{B}\right|<\varepsilon\label{eq:conclusion}
    \end{equation}
式\eqref{eq:conclusion}可化为:
\begin{equation}
    \frac{|g(x)-B|}{|B|\,|g(x)|}<\varepsilon\label{eq:2}
\end{equation}
注意到上式左侧由以下三项组成:$\mfrac{1}{|B|}$、$|g(x)-B|$与$\mfrac1{|g(x)|}$.
现在要做的,就是在已经给定的正数$\varepsilon$下,找到合适的$\delta$,然后分别判断这三项的取值范围,最后论证这三项在给定的$\delta$下乘积能够小于$\varepsilon$.\par
第一项$\frac1{|B|}$是常数;\par
对第二项$|g(x)-B|$,与第三项$\frac1{|g(x)|}$,由
\begin{equation}
    \lim_{x \to x_0} g(x)=B\label{eq:lim}
\end{equation}
可得:对于任意$\varepsilon_0>0$,能够找到$\delta_0$,使得当$0<|x-x_0|<\delta_0$时,有$|g(x)-B|<\varepsilon_0$.\\
于是:
\begin{enumerate}[label=\circled{\arabic*}]
    \item 取$\varepsilon_0=\varepsilon$,则必定存在正数$\delta_{10}$,使得当$0<|x-x_0|<\delta_{10}$时,有
        \begin{equation}
            |g(x)-B|<\varepsilon\label{eq:lim1}
        \end{equation}
    \item 取$\varepsilon_0=\frac{|B|}{2}$,则必定存在正数$\delta_2$,使得当$0<|x-x_0|<\delta_2$时,就有
        \begin{equation}
            |g(x)-B|<\frac{|B|}{2}\label{eq:lim2}
        \end{equation}
\end{enumerate}
由\circled{2},可由绝对值不等式得到
    \begin{equation}
        \frac1{|g(x)|}<\frac{2}{|B|}\label{eq:lim2-1}
    \end{equation}
取$\delta=\min\{\delta_{10},\delta_2\}$,那么,当$0<|x-x_0|<\delta$时,就同时有$0<|x-x_0|<\delta_{10}$与$0<|x-x_0|<\delta_2$成立.\\
当$0<|x-x_0|<\delta_{10}$成立时,式\eqref{eq:lim1}成立,\\
当$0<|x-x_0|<\delta_2$成立时,式\eqref{eq:lim2}成立,那么式\eqref{eq:lim2-1}也就能够成立.\\
于是,当$0<|x-x_0|<\delta$时,\eqref{eq:lim1}、\eqref{eq:lim2-1}两式就能同时成立.
这时,就有:
        \begin{equation}
            \left|\frac{1}{g(x)}-\frac{1}{B}\right|=\frac{1}{|B|}\times|g(x)-B|\times\frac{1}{|g(x)|}
            <\frac{1}{|B|}\times \varepsilon \times \frac{2}{|B|}=\frac{2\varepsilon}{|B|^2}
        \end{equation}
这与我们最终想要证明的式\eqref{eq:conclusion}还差了那么一点,所以需要稍作调整.\\
\circled{1}中,改取$\varepsilon_0=\frac{|B|^2}{2}\varepsilon$,那么,则必定存在正数$\delta_1$,使得当$0<|x-x_0|<\delta_1$时,有
\begin{equation}
    |g(x)-B|<\frac{|B|^2}{2}\varepsilon.\label{eq:lim1-1}
\end{equation}
这时,改取$\delta=\min\{\delta_{1},\delta_2\}$,那么当$0<|x-x_0|<\delta$时,可知\eqref{eq:lim1-1}、\eqref{eq:lim2-1}两式同时成立.
于是
\begin{equation}
    \left|\frac{1}{g(x)}-\frac{1}{B}\right|=\frac{1}{|B|}\times|g(x)-B|\times\frac{1}{|g(x)|}
    <\frac{1}{|B|}\times \frac{|B|^2}{2}\varepsilon \times \frac{2}{|B|}=\varepsilon
\end{equation}
这就是我们要证明的式\eqref{eq:conclusion}.




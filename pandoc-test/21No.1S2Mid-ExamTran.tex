  \item
  (5分)命题$p$:$\forall x\in \mathbf{R}$,$x^2-2\rm x\geqslant 1$的否定是(  )\xz
    \xx{$[-1,3]]$}
      {$[-2,4]]$}
      {$\exists x_0 \in \mathbb{R}$,$x_0^2-2x_0\geqslant 1$}
      {$\{1,2,3\}$}
    \begin{answer}
      A
    \end{answer}
  \item
    命题``若\textit{a}>\textit{b},则\textit{a}+\textit{c}>\textit{b}+\textit{c}''的逆否命题是:\tk.
    \begin{answer}
      若$a+c\leqslant b+c$,则$a\leqslant b$
    \end{answer}
  \item
    设p:$x\in P=\{ x \mid 2m-1\leqslant x\leqslant 2m^2-m\} \neq \emptyset$,q:$x\in S=\{ x \mid x^2-2x-3\leqslant0 \}$ ,
    \begin{answer}
      【分析】解不等式求出满足\textit{q}的\textit{x}的范围,根据集合的包含关系得到[2\textit{m}﹣1,2\textit{m}${}^{2}$﹣\textit{m}]$\mathrm{\subsetneqq}$[﹣1,3],得到关于\textit{m}的不等式组,解出即可.
      【解答】解:由$\{ x \mid 2m-1\leqslant x\leqslant 2m^2-m \neq \emptyset \}$,得:$2m-1\leqslant 2m^2-m$,解得:$ m\geqslant 1$或$m\leqslant \frac12$,
            由$\{ x \mid x^2-2x-3\leqslant0 \}$,得:$-1\leqslant x\leqslant 3$,故满足\textit{q}的集合\textit{B}=$\{ x \mid -1\leqslant x\leqslant 3\}$,
            由¬\textit{p}是¬\textit{q}的必要不充分条件,即\textit{q}是\textit{p}的必要不充分条件,
            故[2\textit{m}﹣1,2\textit{m}${}^{2}$﹣\textit{m}]$\mathrm{\subsetneqq}$[﹣1,3],即
            $\left\{\begin{aligned}
              2m-1 \geqslant -1\\
              2m^2-m \leqslant 3
            \end{aligned}\right.$
            ,解得:$0\leqslant m\leqslant \frac32$ ,
            而$m\geqslant1$或$m\leqslant\frac12$,
            故\textit{m}的取值范围是$[0,\frac12]\bigcup [1,\frac32]$.
      【点评】本题考查了集合的运算,考查不等式问题以及集合的包含关系,考查转化思想,是一道基础题.
    \end{answer}
  \item
    第22题
    \begin{answer}
      $t\geqslant 1$\\
      $f(t)=3t+\frac1t$( $t\geqslant 1$)\\
      $f(t)\geqslant f(1)=4$\\
      $S_{\triangle ABF_2}\leqslant \frac{12}4$

    \end{answer}










 
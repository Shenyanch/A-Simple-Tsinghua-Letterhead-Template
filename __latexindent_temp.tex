\Topic{中考代数基础练习}
  \Teach{实数、因式分解、方程、不等式}
  \Grade{初三}
  % \Name{郑皓天}\FirstTime{20181207}\CurrentTime{20181207}
  % \Name{林叶}\FirstTime{20180908}\CurrentTime{20181125}
  %\Name{1v2}\FirstTime{20181028}\CurrentTime{20181117}
  % \Name{林叶}\FirstTime{20180908}\CurrentTime{20181125}
  % \Name{郭文镔}\FirstTime{20181111}\CurrentTime{20181231}
  % \Name{马灿威}\FirstTime{20181111}\CurrentTime{20181111}
  \newtheorem*{Theorem}{定理}
  \makefront
\vspace{-1.5em}
\startexercise
  % \begin{exercise}{\heiti 课前检测}\\
  %   表格实例:
  %   \begin{center}
  %     \renewcommand{\arraystretch}{1.4}
  %     \begin{tabular}{|*{8}{c|}}
  %       \hline
  %         $x$
  %         &$-\dfrac{\piup}6$
  %         &$-\dfrac{\piup}3$
  %         &$-\dfrac{5\piup}6$
  %         &$-\dfrac{4\piup}3$
  %         &$-\dfrac{11\piup}6$
  %         &$-\dfrac{7\piup}3$
  %         &$-\dfrac{17\piup}6$\\
  %       \hline
  %         $y$
  %         &$-1$
  %         &$1$
  %         &$3$
  %         &$1$
  %         &$-1$
  %         &$1$
  %         &$3$\\
  %       \hline
  %     \end{tabular}\\
  %   \end{center}
  % \end{exercise}
  % \section{知识点总结}
  %   \begin{description}[leftmargin=0pt,labelsep=0pt]
  %     \item%[两角的和与差]
  %       \begin{itemizeMy}[两角的和与差\hspace{2em}]
  %         \item $\mathrm{C}_{\alpha\pm\beta}$:
  %         $\cos(\alpha\pm\beta)=\cos\alpha\cos\beta \mp \sin\alpha\sin\beta$
  %         \item $\mathrm{S}_{\alpha\pm\beta}$:
  %         $\sin(\alpha\pm\beta)=\sin\alpha\cos\beta \pm \cos\alpha\sin\beta$
  %         \item $\mathrm{T}_{\alpha\pm\beta}$:
  %         $\tan(\alpha\pm\beta)=\dfrac{\tan\alpha\pm \tan\beta}{1\mp\tan\alpha\tan\beta}$
  %       \end{itemizeMy}
  %     \item%[二倍角公式]
  %       \begin{itemizeMy}[二倍角公式\hspace{3em}]
  %         \item $\mathrm{S}_{2\alpha}$:
  %         $\sin{2\alpha}=2\sin\alpha\cos\alpha$
  %         \item $\mathrm{C}_{2\alpha}$:
  %         $\cos{2\alpha}=\cos^2{\alpha}-\sin^2{\alpha}=2\cos^2\alpha-1=1-2\sin^2\alpha$
  %         \item $\mathrm{T}_{2\alpha}$:
  %         $\tan{2\alpha}=\dfrac{2\tan\alpha}{1-\tan^2\alpha}$
  %       \end{itemizeMy}
  %     \item%[半角公式]
  %       \begin{itemizeMy}[半角公式\hspace{4em}]
  %         \item
  %         $\sin{\dfrac{\alpha}2}=\pm\sqrt{\dfrac{1-\cos\alpha}2}$
  %         \item $\cos{\dfrac{\alpha}2}=\pm\sqrt{\dfrac{1+\cos\alpha}2}$
  %         \item $\tan{\dfrac{\alpha}2}=\dfrac{\sin\alpha}{1+\cos\alpha}=\dfrac{1-\cos\alpha}{\sin\alpha}$
  %       \end{itemizeMy}
  %     \item%[万能公式]
  %       \begin{itemizeMy}[万能公式\hspace{4em}]
  %         \item $\sin{\alpha}=\dfrac{2\tan{\dfrac{\alpha}2}}{1+\tan^2{\dfrac{\alpha}2}}}$
  %         \item $\cos{\alpha}=\dfrac{1-\tan^2{\dfrac{\alpha}2}}{1+\tan^2{\dfrac{\alpha}2}}}$
  %         \item $\tan{\alpha}=\dfrac{2\tan{\dfrac{\alpha}2}}{1-\tan^2{\dfrac{\alpha}2}}}$
  %       \end{itemizeMy}
  %     \item%[辅助角公式]
  %       \begin{itemizeMy}[辅助角公式\hspace{3em}]
  %         \item $a\sin x+b\cos x=\sqrt{a^2+b^2}\sin(x+\varphi)$\\
  %         其中$\sin\varphi=\dfrac{b}{\sqrt{a^2+b^2}}$,$\cos\varphi=\dfrac{a}{\sqrt{a^2+b^2}}$\\
  %         $a>0$时,
  %         \item $a\sin x+b\cos x=\sqrt{a^2+b^2}\sin(x+\varphi)$\\
  %         其中$\tan\varphi=\dfrac{b}a$,$
  %         \varphi\in\Bigl(-\dfrac{\piup}2,\dfrac{\piup}2\Bigr)$
  %       \end{itemizeMy}
  %   \end{description}
  % \clearpage
  \section{习题}
    
    \begin{exercise}
      \item%《福建省2014-2018年中考数学试题分项解析 – 代数式与因式分解 - print》三、解答题 P7p12
        \source{2015}{莆田中考,第19题,8分}
        先化简,再求值:$\dfrac{a^2-2ab}{a-b}-\dfrac{b^2}{b-a}$,其中$a=1+\sqrt3$,$b=-1+\sqrt3$.
          \begin{answer}
            2
          \end{answer}
      \vspace{4cm}
      \item%《福建省2014-2018年中考数学试题分项解析 – 代数式与因式分解 - print》三、解答题 P9p24
        \source{2018}{福建中考}
        先化简,再求值:$\biggl(\dfrac{2m+1}{m}-1\biggr)\div\dfrac{m^2-1}m$,其中$m=\sqrt3+1$.
        \begin{answer}
          $\frac{\sqrt3}3$\\
          原式$=\frac1{m-1}$
        \end{answer}
      \vspace{4cm}\piec
      \item%《福建省2014-2018年中考数学试题分项解析 – 方程(组)与不等式(组) - print》三、解答题 P12p31
        \source{2018}{福建中考}
        解方程组:
          $\left\{\begin{aligned}
            x+y&=1\,\\
            4x+y&=10\,
          \end{aligned}\right.$.
        \begin{answer}
          $\left\{\begin{aligned}
            &x=3\,,\\
            &y=-2\,.
          \end{aligned}\right.$.
        \end{answer}
    \end{exercise}
\stopexercise

\newpage
\section{参考答案}
\begin{multicols}{2}
  \printanswer
\end{multicols}

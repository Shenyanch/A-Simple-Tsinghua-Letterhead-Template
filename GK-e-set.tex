\Topic{集合}
  \Teach{}
  \Grade{高一}
  % \Name{郑皓天}\FirstTime{20181207}\CurrentTime{20181207}
  % \Name{林叶}\FirstTime{20180908}\CurrentTime{20181125}
  %\Name{1v2}\FirstTime{20181028}\CurrentTime{20181117}
  % \Name{林叶}\FirstTime{20180908}\CurrentTime{20181125}
  % \Name{郭文镔}\FirstTime{20181111}\CurrentTime{20181117}
  % \Name{马灿威}\FirstTime{20181111}\CurrentTime{20181111}
  % \Name{黄亭燏}\FirstTime{20181231}\CurrentTime{20181231}
  % \Name{王睿妍}\FirstTime{20190129}\CurrentTime{}
  % \Name{郑旭晶}\FirstTime{20190423}\CurrentTime{20190514}
  \newtheorem*{Theorem}{定理}
  \makefront

\startexercise
\section{集合与元素}
  \begin{framed}\hspace{-25pt}{\heiti 知识点}
    \begin{enumerate}
      \item {\fangsong 元素}:研究对象的统称.
      \item {\fangsong 集合}:一些元素组成的总体.
      \item 如果$a$是集合$a$的元素,就说 $a${\fangsong 属于}$A$,记作 $a \in A$;\\
            如果$a$不是集合$a$的元素,就说$a${\fangsong 不属于}$A$,记作 $a \notin A$;
      \item 不包含任何元素的集合称为{\fangsong 空集},记为$\varnothing$.
      \item 几个特殊的集合:
        \begin{itemize}
          \item 全体{\fangsong 自然数}$0,1,2,3,\ldots ,$组成的集合,记作$\mathbb{N}$(natural);
          \item 全体正自然数组成的集合,记作$\mathbb{N}_+$或$\mathbb{N}^*$
          \item 全体{\fangsong 整数}$0,\pm1,\pm2,\pm3,\ldots ,$组成的集合,记作$\mathbb{Z}$;
          \item 全体{\fangsong 有理数}组成的集合,记作$\mathbb{Q}$(quotient);
          \item 全体{\fangsong 实数}组成的集合,记作$\mathbb{R}$(real).
          % \item 全体{\fangsong 复数}组成的集合,记作$\mathbb{C}$(complex)
        \end{itemize}
    \end{enumerate}
  \end{framed}

  \begin{exercise}
    \item 下列对象能组成集合的是\xz
          \xx{大于5的自然数}
            {一切很大的数}
            {某班个子高的学生}
            {某班考试得分很高的学生}
    \item 以下正确的是\xz
          \xx{$2.5 \in \mathbb{Z}$}
           {$\sqrt2 \notin \mathbb{Q}$}
           {$\mfrac23 \in \mathbb{N}$}
           {$0 \notin \mathbb{R}$}
    \item 下列对象不能组成集合的是\xz
          \xx{不大于8的有理数}
           {很接近1的数}
           {方程$x^2=-4$的解集}
           {不等式$2x+1\geqslant 0$}
    \item 集合$A$表示大于3并且小于6的自然数组成的集合,那么$A$中的元素有:\tk.
    \item 集合$B$表示不大于5的自然数组成的集合,那么$B$中的元素有:\tk.
    \item 用符号“$\in$”或“$\notin$”填空:
          \begin{align*}
            &10 \_\_\_\_\mathbb{N}, \qquad &-1.2 \_\_\_\_\mathbb{R}, \qquad &1.3 \_\_\_\_\mathbb{Z}, \qquad 
            &\pi \_\_\_\_\mathbb{Q}, \qquad &0 \_\_\_\_\varnothing, \qquad &-5 \_\_\_\_\mathbb{Z}, \qquad
            \\ 
            &-8 \_\_\_\mathbb{Q}, \qquad &3 \_\_\_\_\varnothing, \qquad &2.56 \_\_\_\_\mathbb{Q}, \qquad 
            &\mfrac37 \_\_\_\_\mathbb{Q}, \qquad &0 \_\_\_\_\mathbb{N}, \qquad &6 \_\_\_\_\mathbb{Z}. \qquad
          \end{align*}
    \item 判断对错:
          \begin{enumerate}[label=({\arabic*}),itemsep= -18 pt]
            \item 方程$x^2+3=0$的解集中的元素是$0$.\xz
            \item 由所有大于5的整数构成的集合中由无数个元素.\xz
            \item 若$A$为不等式$x^2\geqslant9$的解集,那么$-3\in A$.\xz
            \item 由所有大于2并且小于8的有理数组成的集合是个有限集.\xz\vspace{-18pt}
          \end{enumerate}}
    \item 将以下几个集合按照有限集、无限集和空集归类:
          \begin{enumerate}[label=\circled{\arabic*}]
            \item 所有大于0并且小于20的奇数组成的集合.
            \item 不等式$x-1\leqslant0$的解集.
            \item 方程$x^2+1=0$的解集.
            \item 所有大于3并且小于4的实数组成的集合.
            \item 方程$x+5=5$的解集.
            \item 在直角坐标系中,由第一象限所有点组成的集合.
          \end{enumerate}}
    \item d
    \item 
  \end{exercise}
\section{集合的表示法}
  \begin{framed}\hspace{-25pt}{\heiti 知识点}\par
    \hspace{-10pt}常见的集合的表示法由列举法和描述法.\par
    \begin{enumerate}\vspace{-10pt}
      \item {\fangsong 列举法}:.
            \\\eg{由小于5的自然数组成的集合可以表示为$\{0,1,2,3,4\}$.}
      \item {\fangsong 描述法}:.
            \\\eg{由小于5的自然数组成的集合可以表示为$\{x\mid x<5,x\in\mathbb{N}\}$}.
    \end{enumerate}
  \end{framed}
  \begin{exercise}
    \item 用符号“$\in$”或“$\notin$”填空:
          \begin{align*}
            &0 \_\_\_\_\{0\}, \qquad &3 \_\_\_\_\{2,3,5\}, \qquad &4 \_\_\_\_\{2,3,5\}, \qquad &0 \_\_\_\_\varnothing, \qquad &5 \_\_\_\_\{x \mid x<3  \}, \qquad &3 \_\_\_\_\{x \mid -5\leqslant x<5\}, \qquad
            \\ 
            &-8 \_\_\_\mathbb{Q}, \qquad &3 \_\_\_\_\varnothing, \qquad &2.56 \_\_\_\_\mathbb{Q}, \qquad 
            &\mfrac37 \_\_\_\_\mathbb{Q}, \qquad &0 \_\_\_\_\mathbb{N}, \qquad &6 \_\_\_\_\mathbb{Z}. \qquad
          \end{align*}
  \end{exercise}
\section{集合之间的关系}
\section{集合的运算}
\section{}


% \newpage
% \section{课后作业}
%   \begin{exercise}{\heiti 练习}
%
%   \end{exercise}
\stopexercise

\newpage
\section{参考答案}
\begin{multicols}{2}
  \printanswer
\end{multicols}

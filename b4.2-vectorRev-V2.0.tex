% 节录自b5-FinExRev.tex
\Topic{平面向量与三角恒等变换复习}
  \Teach{}
  \Grade{高一}
  % \Name{郑皓天}\FirstTime{20181207}\CurrentTime{20181207}
  % \Name{林叶}\FirstTime{20180908}\CurrentTime{20181125}
  %\Name{1v2}\FirstTime{20181028}\CurrentTime{20181117}
  % \Name{林叶}\FirstTime{20180908}\CurrentTime{20181125}
  % \Name{郭文镔}\FirstTime{20181111}\CurrentTime{20181117}
  % \Name{马灿威}\FirstTime{20181111}\CurrentTime{20181111}
  % \Name{黄亭燏}\FirstTime{20181231}\CurrentTime{20190112}
  \newtheorem*{Theorem}{定理}
  \makefront
\vspace{-1.5em}
\startexercise
\section{平面向量要点归纳}
  \subsection{五种常见向量}
    \begin{enumerate}[label=\arabic*)]
      \item 单位向量:模为1的向量.
      \item 零向量:模为0的向量.
      \item 平行(共线向量):方向相同或相反或其一为零向量的两个向量.
      \item 相等向量:模相等,方向相同的向量.
      \item 相反向量:模相等,方向相反的向量.
    \end{enumerate}
  \subsection{平面向量运算律}
    \begin{enumerate}[label=\arabic*)]
      \item 交换律:
        $\bm a+\bm b=\bm b+\bm a$,\quad
        $\bm a\cdot\bm b=\bm b\cdot\bm a$
      \item 结合律:
        $(\bm{a}+\bm{b})+\bm{c}=\bm{a}+(\bm{b}+\bm{c})$,\quad
        $(\lambda \bm a)\cdot\bm{b}=\lambda(\bm a\cdot\bm b)=\bm{a}\cdot(\lambda\bm{b})$
      \item 分配律:
        $(\lambda+\mu)\bm{a}=\lambda\bm{a}+\mu\bm{a}$,\quad
        $\lambda(\bm{a}+\bm{b})=\lambda\bm{a}+\lambd+a\bm{b}$,\quad
        $(\bm a+\bm b)\cdot \bm c=\bm a\cdot\bm c+\bm b\cdot \bm c$
      \item 重要公式:(记号$\bm a^2=\bm a\cdot\bm a$)
        $(\bm a+\bm b)(\bm a-\bm b)=\bm a^2-\bm b^2$,\quad
        $(\bm a\pm\bm b)^2=\bm a^2\pm2\bm a\cdot\bm b+\bm b^2$.
    \end{enumerate}
  \subsection{两个重要定理}
    \begin{enumerate}[label=\arabic*)]
      \item 向量共线定理:
        向量$\bm{a}~(\bm{a}\ne\bm{0})$与向量$\bm{b}$共线,当且仅当存在唯一的实数$ \lambda $,使得$\bm{b}=\lambda\bm{a}$.\\
        {\kaishu
         证明三点共线的方法:\circled{1}$\vv{AB}=\lambda\vv{AC}$,则$A$,$B$,$C$三点共线;\circled{2}$\vv{OA}=\lambda\vv{OB}+\mu\vv{OC}$,若$\lambda+\mu=1$,则$A$,$B$,$C$三点共线.
        }
      \item 平面向量基本定理:
        如果$ \bm{e}_1,\bm{e}_2 $是同一平面内的两个\CJKunderdot{不共线}的向量,
        则那么对于这一平面内的任意向量$ \bm{a} $,有且只有一对实数$ \lambda_1,~\lambda_2 $,使$\bm{a}=\lambda_1\bm{e}_1+\lambda_2\bm{e}_2$.
        其中,不共线的向量$\bm{e}_1, \bm{e}_2$叫做表示这一平面内所有向量的一组\CJKunderdot{基底}.\\
        {\kaishu 平面向量基本定理应用技巧:
          \begin{enumerate}[label=\circled{\arabic*}]
            \item 构造某一向量在同一基底下的两种不同表达形式,
              根据向量分解的唯一性求解.即:\\
              {\kaishu 以$\bm e_1$,$\bm e_2$为基底,且$\bm a=x_1\bm e_1+y_1\bm e_2=x_2\bm e_1+y_2\bm e_2$,则$\begin{cases}x_1=x_2\\y_1=y_2\end{cases}$}
            \item 构造两个共线向量在同一基底下的表达形式,
              根据向量共线定理求解.即:\\
              {\kaishu 以$\bm e_1$,$\bm e_2$为基底,且$\bm a=x_1\bm e_1+y_1\bm e_2$,$\bm b=x_2\bm e_1+y_2\bm e_2$,且$\bm a\varparallel\bm b$,则$x_1y_2-x_2y_1=0$}
            \item 将题目中的已知条件转化成
              $\lambda_1\bm e_1+\lambda_2\bm e_2=\bm 0$的形式($\bm e_1$,$\bm e_2$不共线),根据$\lambda_1=\lambda_2=0$求解.
          \end{enumerate}}
    \end{enumerate}
  \subsection{平面向量平行、垂直的等价条件}
    设$\bm a=(x_1,y_1)$,$\bm b=(x_2,y_2)$,则:
    \begin{enumerate}[label=\arabic*)]
      \item $\bm a\varparallel\bm b$$\Leftrightarrow$$x_1y_2-x_2y_1=0$.
      \item $\bm a\perp\bm b$
            $\Leftrightarrow$$\bm a\cdot\bm b=0$
            $\Leftrightarrow$$x_1x_2+y_1y_2=0$.
    \end{enumerate}
  \subsection{平面向量数量积相关量求解}
    \begin{enumerate}[label=\arabic*)]
      \item 向量夹角:设$\bm a=(x_1,y_1)$,$\bm b=(x_2,y_2)$,则
        $\cos\vangle{\bm a}{\bm b}=\dfrac{\bm{a}\bm{\cdot}\bm{b}}{\abs{\bm{a}}\abs{\bm{b}}}=\dfrac{x_1x_2+y_1y_2}{\sqrt{x_1^2+y_1^2}\sqrt{x_2^2+y_2^2}} \quad (\vangle{\bm a}{\bm b}\in\left[0,\piup\right])$
      \item 向量模长:若$\bm a=(x,y)$,则$\abs{\bm a}=\sqrt{\bm a\cdot\bm a}=\sqrt{x^2+y^2}$
      \item 向量投影:向量$\bm a$在$\bm b$方向上的投影为  $\abs{\bm{a}}\cos\theta=\dfrac{\bm a\cdot\bm b}{\abs{\bm b}}$
    \end{enumerate}
  \begin{exercise}{\textbf{向量表示}}
    \item
      (2018届贵州遵义航天高级中学一模)如图所示,向量$\vv{OA}=\bm{a}$,$\vv{OB}=\bm{b}$,$\vv{OC}=\bm{c}$,$A$,$B$,$C$在一条直线上,且$\vv{AC}=3\vv{BC}$,则\xz
      \begin{minipage}[b]{0.7\linewidth}
        \xx{$\bm{c}=\dfrac32\bm{b}-\dfrac12\bm{a}$}
          {$\bm{c}=\dfrac32\bm{a}-\dfrac12\bm{b}$}
          {$\bm{c}=-\bm{a}+2\bm{b}$}
          {$\bm{c}=\bm{a}+2\bm{b}$}
      \end{minipage}\hfill
      \begin{minipage}[htbp!]{0.3\linewidth}
        \begin{center}
        \begin{tikzpicture}
          \coordinate[label=left:$O$](O)at(0,0);
          \coordinate[label=right:$C$](C)at(3,0);
          \coordinate[label=left:$A$](A)at(-1,2.5);
          \coordinate[label=right:$B$](B)at($(A)!0.66!(C)$);
          \draw (A)--(B)--(C)--cycle;
          \draw[->,>=latex] (O)--(C);
          \draw[->,>=latex] (O)--(A);
          \draw[->,>=latex] (O)--(B);
        \end{tikzpicture}
        \end{center}
      \end{minipage}
      \begin{answer}
        A
      \end{answer}
    {\begin{minipage}[b]{0.65\linewidth}
      \item%LaTeX-master/xiangliang/xiangliangsorting.tex P10-p48
        在$\triangle ABC$中,点$ M$,$N $满足$ \vv{AM}=2\vv{MC}$,$\vv{BN}=\vv{NC}$.若$\vv{MN}=x\vv{AB}+y\vv{AC}$,则$ x= $\tk;$ y= ~$ \tk.
        \begin{answer}
          $x=\dfrac12$;$y=-\dfrac16$
        \end{answer}
      \end{minipage}
      \begin{minipage}[htbp!]{0.3\linewidth}
        \begin{center}
        \begin{tikzpicture}
          \draw(0,0)node[below](B){\small$B$}--(1,0)node[below](N){\small$N$}--(2,0)node[below](C){\small$C$};
          \draw (0,0)--(1.1,2.1)node[above](A){\small$A$}--(2,0);
          \draw (1,0)--(1.1,2.1);
          \draw(1,0)--($(1.1,2.1)!0.7!(2,0)$)node[right](M){\small$M$};
        \end{tikzpicture}
        \end{center}
      \end{minipage}}
    \item%【向量的线性运算】
      (2017 \textbullet {\kaishu 广东深圳二模})如图所示,正方形$ABCD$中,$M$是$BC$的中点,若$\vv{AC}=\lambda\vv{AM}+\mu\vv{BD}$,则$\lambda+\mu$等于\xz
      \xx{$\dfrac{4}3$}
       {$\dfrac{5}3$}
       {$\dfrac{15}8$}
       {$2$}
      \begin{center}
        \begin{tikzpicture}
          \coordinate[label=left:$A$](A)at(0,0);
          \coordinate[label=right:$B$](B)at(3.5,0);
          \coordinate[label=left:$D$](D)at(0,3.5);
          \coordinate[label=right:$C$](C)at(3.5,3.5);
          \coordinate[label=right:$M$](M)at($(B)!0.5!(C)$);
          \draw (A)--(B)--(C)--(D)--cycle;
          \draw[->,>=latex] (A)--(C);
          \draw[->,>=latex] (A)--(M);
          \draw[->,>=latex] (B)--(D);
        \end{tikzpicture}
      \end{center}
      \begin{answer}
        B
      \end{answer}
    \item%【向量坐标法在平面几何的应用,三角函数定义】
      %如图,
      半径为$\sqrt{3}$的扇形$AOB$的圆心角为120\degree,点$C$在$\arc{AB}$上,且$\angle{COB}=30\degree$,若$\vv{OC}=\lambda\vv{OA}+\mu\vv{OB}$,则$\lambda+\mu$等于\xz
      \xx{$\sqrt{3}$}
       {$\dfrac{\sqrt{3}}3$}
       {$\dfrac{4\sqrt{3}}3$}
       {$2\sqrt{3}$}
      \begin{answer}
        A
      \end{answer}
  \end{exercise}
  \begin{exercise}{\textbf{向量共线}}
    \item%福州格致中学2015-2016学年高一数学第二学期期末检测.docx-8
      (2016 \textbullet {\kaishu 格致中学} 8)
      设$\bm e_1$,$\bm e_2$是两个不共线的向量,$\vv{AB}=2\bm e_1+k\bm e_2$,$\vv{CB}=2\bm e_1+3\bm e_2$,$\vv{CD}=2\bm e_1-\bm e_2$,若$A$,$B$,$D$三点共线,则$k=$\xz
      \xx{$\dfrac12$}
        {$-8$}
        {$-\dfrac18$}
        {$2$}
      \begin{answer}
        B
      \end{answer}
    \item%福州一中学2016-2017学年高一下学期期末考试数学…….doc-3【向量共线】
      (2017 \textbullet {\kaishu 福州一中} 3)
      已知向量$\bm a$,$\bm b$不共线,且$\bm c=\lambda\bm a+\bm b$,$\bm d=\bm a+(2\lambda-1)\bm b$,若$\bm c$与$\bm d$方向相反,则实数$\lambda$的值为\xz
      \xx{$1$}
       {$-\dfrac12$}
       {$1$或$-\dfrac12$}
       {$-1$或$-\dfrac12$}
      \begin{answer}
        B
      \end{answer}
    \item%《习题化知识清单》P83方法2【向量共线条件应用】
      平面内给定三个向量$\bm a=(3,2)$,$\bm b=(-1,2)$,$\bm c=(4,1)$,则\\
      (1)若$(\bm a+k\bm c)\varparallel(2\bm b-\bm a)$,则实数$k=$\tk;\\
      (2)设$\bm d=(x,y)$满足$(\bm d-\bm c) \varparallel (\bm a+\bm b)$且$\abs{\bm d-\bm c}=1$,则$\bm d=$\tk.
      \begin{answer}
        (1)$k=-\dfrac{16}{13}$;(2)$\bm d=\Bigl(\dfrac{20+\sqrt{5}}5,\dfrac{5+2\sqrt{5}}5\Bigr)$或$\Bigl(\dfrac{20-\sqrt{5}}5,\dfrac{5-2\sqrt{5}}5\Bigr)$
      \end{answer}
    \item%《习题化知识清单》P83方法2-2【向量共线条件应用】
      若平面向量$\bm a$,$\bm b$满足$\abs{\bm a+\bm b}=1$,$\bm a+\bm b$平行于$x$轴,$\bm b=(2,-1)$,则$\bm a=$\tk.
      \begin{answer}
        $(-1,1)$或$(-3,1)$
      \end{answer}
    \item%《习题化知识清单》P91单元检测16【三点共线,向量共线】
      设两个非零向量$\bm a$与$\bm b$不共线\\
      (1)若$\vv{AB}=\bm a+\bm b$,$\vv{BC}=2\bm a+8\bm b$,$\vv{CD}=3(\bm a-\bm b)$,求证:$A$,$B$,$D$三点共线;\\
      (2)试确定实数$k$,使$k\bm a+\bm b$与$\bm a+k\bm b$共线.
      \begin{answer}
        (1)略;(2)$k=\pm1$
      \end{answer}
    \vspace{5cm}
  \end{exercise}
  \begin{exercise}{\textbf{向量数量积、投影}}
    \item%福州三中2017高一下数学期末卷…….doc-5【向量投影,基底表示】
      (2017 \textbullet {\kaishu 福州三中} 5)
      设$\bm e_1$,$\bm e_2$为单位向量,且$\bm e_1$,$\bm e_2$的夹角为$\dfrac{\piup}3$,若$\bm a=\bm e_1-3\bm e_2$,$\bm b=\bm e_1+\bm e_2$,则向量$\bm a$在$\bm b$方向上的射影为\xz
      \xx{$-\sqrt3$}
       {$\sqrt3$}
       {$-\dfrac{\sqrt{10}}5$}
       {$\dfrac{\sqrt{10}}5$}
      \begin{answer}
        A
      \end{answer}
    \item%《习题化知识清单》P84知识5-3【数量积的应用,运算律】
      已知$\abs{\bm a}=\abs{\bm b}=1$,$\bm a$,$\bm b$的夹角是直角,
      $\bm c=2\bm a+3\bm b$,$\bm d=k\bm a-4\bm b$,$\bm c\perp\bm d$,则$k=$\tk.
      \begin{answer}
        6
      \end{answer}
      已知$\triangle{ABC}$是正三角形,若$\vv{AC}-\lambda\vv{AB}$与向量$\vv{AC}$的夹角大于90\degree,则实数$\lambda$的取值范围是\tk.
      \begin{answer}
        $(2,+\infty)$
      \end{answer}
    \item%福建师大附中2016-2017高一下期末考试数学试题…….doc-17【数量积,几何】
      (2017 \textbullet {\kaishu 师大附中} 17)
      在$\triangle{ABC}$中,$|\vv{AD}|=|\vv{BD}|=|\vv{CD}|$,$|\vv{AB}|=3$,则$\vv{AB}\cdot\vv{AD}=$\tk.
      \begin{answer}
        $\dfrac92$
      \end{answer}
    \item%《高中数学竞赛培优教程+一试(李名德 主编)》.pdf P114-例5.8
      已知$\bm a=(\cos\alpha,\sin\alpha)$,$\bm b=(\cos\beta,\sin\beta)$($0<\alpha<\beta<\piup$).\\
      (1)求证:$\bm a+\bm b$与$\bm a-\bm b$相互垂直;\\
      (2)若$k\bm a+\bm b$与$\bm a-k\bm b$大小相等,求$\beta-\alpha$(其中$k$为非零实数).
      \begin{answer}
        (1)略;(2)$\dfrac{\piup}2$
      \end{answer}
    \vspace{4cm}
  \end{exercise}
  \begin{exercise}{\textbf{向量求模、夹角}}
    \item%%福建师大附中2016-2017高一下期末考试数学试题…….doc-15【向量夹角,线性运算模长】
      (2017 \textbullet {\kaishu 师大附中} 15)
      已知单位向量$\bm a$,$\bm b$的夹角为$\dfracp{}3$,那么$|\bm a-2\bm b|=$\tk.
      \begin{answer}
        $\sqrt3$
      \end{answer}
    \item%福州三中2017高一下数学期末卷…….doc-16【向量投影,基底表示】
      (2017 \textbullet {\kaishu 福州三中} 16)
      已知$\bm a$,$\bm b$是平面内两个相互垂直的单位向量,若向量$\bm c$满足$(\bm a-\bm c)\cdot(\bm b-\bm c)=0$,则$|\bm c|$的最大值是\tk.
      \begin{answer}
        $\sqrt2$
      \end{answer}
    \item%《习题化知识清单》P84方法1【求向量夹角基本方法】
      已知$\abs{\bm a}=1$,$\abs{\bm b}=2$,$\bm a$与$\bm b$的夹角为120\degree,则使$\bm a+k\bm b$与$k\bm a+\bm b$的夹角为锐角的实数$k$的取值范围是\tk.
      \begin{answer}
        $\Bigl(\dfrac{5-\sqrt{21}}2,1\Bigr) \bigcup \Bigl(1,\dfrac{5-\sqrt{21}}2 \Bigr)$
      \end{answer}
    \item%福建师大附中2016-2017高一下期末考试数学试题…….doc-19【向量共线、夹角、模长】
      (2017 \textbullet {\kaishu 师大附中} 19)
      已知$\bm a$,$\bm b$为两个不共线向量,$\abs{\bm a}=2$,$\abs{\bm b}=1$,$\bm c=2\bm a-\bm b$,$\bm d=\bm a+k\bm b$.\\
      (I)若$\bm c\varparallel\bm d$,求实数$k$;\\
      (II)若$k=-7$,且$\bm c\perp\bm d$,求$\bm a$与$\bm b$的夹角.
      \begin{answer}
        (I)$k=-\dfrac12$
        (II)$\vangle{\bm a}{\bm b}=\dfrac{\piup}3$
      \end{answer}
    \vspace{4.5cm}
    \item%福州一中学2016-2017学年高一下学期期末考试数学…….doc-15【数量积,垂直】
      (2017 \textbullet {\kaishu 福州一中} 15)
      已知$\bm a=(\cos\alpha,k\sin\alpha)$,$\bm b=(\cos\beta,\sin\beta)$($k>0$,$0<\alpha<\beta<\dfrac{\piup}2$),且$\bm a+\bm b$与$\bm a-\bm b$相互垂直.\\
      (1)求$k$的值;\\
      (2)若$\bm a\cdot\bm b=\dfrac45$且$\cos\beta=\dfrac35$,求$\sin\alpha$的值.
      \begin{answer}
        (1)$k=1$;
        (2)$\sin\alpha=\dfrac7{25}$
      \end{answer}
    \vspace{5.5cm}
  \end{exercise}
  \begin{exercise}{\textbf{平面几何应用}}
    \item
      已知四边形$ABCD$ 是菱形,则下列等式中成立的是\xz
      \xx{$\vv{AB}+\vv{BC}=\vv{CA}$}
        {$\vv{AB}+\vv{AC}=\vv{BC}$}
        {$\vv{AC}+\vv{BA}=\vv{AD}$}
        {$\vv{AC}+\vv{AD}=\vv{DC}$}
      \begin{answer}
        C
      \end{answer}
    \item%【平面向量几何应用:垂直问题】
      直角坐标系$xOy$中,$\vv{AB}=(2,1)$,$\vv{AC}=(3,k)$,若$\triangle{ABC}$是直角三角形,则$k$的可能值个数是\xz
      \xx{1}{2}{3}{4}
      \begin{answer}
        B
      \end{answer}
    \item%福州一中学2016-2017学年高一下学期期末考试数学…….doc-14【数量积,外心】
      (2017 \textbullet {\kaishu 福州一中} 14)
      $\triangle{ABC}$中,$CA=4$,$CB=6$,点$O$为$\triangle{ABC}$的外心,则$\vv{CO}\cdot\vv{AB}=$\tk.
      \begin{answer}
        5
      \end{answer}
    \item%福建师大附中2016-2017高一下期末考试数学试题…….doc-6【数量积,三角形形状】
      (2017 \textbullet {\kaishu 师大附中} 6)
      若点$O$是$\triangle{ABC}$平面内一点,且满足$(\vv{OB}-\vv{OC})\cdot(\vv{OB}+\vv{OC}-2\vv{OA})=0$,则$\triangle{ABC}$形状为\xz
      \xx{钝角三角形}{等腰三角形}{直角三角形}{锐角三角形}
      \begin{answer}
        B
      \end{answer}
  \end{exercise}
\newpage

\section{三角恒等变换公式总结}
  \begin{description}[leftmargin=0pt,labelsep=0pt]
    \item%[两角的和与差]
      \begin{itemizeMy}[两角的和与差\hspace{2em}]
        \item $\mathrm{C}_{\alpha\pm\beta}$:
        $\cos(\alpha\pm\beta)=\cos\alpha\cos\beta \mp \sin\alpha\sin\beta$
        \item $\mathrm{S}_{\alpha\pm\beta}$:
        $\sin(\alpha\pm\beta)=\sin\alpha\cos\beta \pm \cos\alpha\sin\beta$
        \item $\mathrm{T}_{\alpha\pm\beta}$:
        $\tan(\alpha\pm\beta)=\dfrac{\tan\alpha\pm \tan\beta}{1\mp\tan\alpha\tan\beta}$
      \end{itemizeMy}
    \item%[二倍角公式]
      \begin{itemizeMy}[二倍角公式\hspace{3em}]
        \item $\mathrm{S}_{2\alpha}$:
        $\sin{2\alpha}=2\sin\alpha\cos\alpha$
        \item $\mathrm{C}_{2\alpha}$:
        $\cos{2\alpha}=\cos^2{\alpha}-\sin^2{\alpha}=2\cos^2\alpha-1=1-2\sin^2\alpha$
        \item $\mathrm{T}_{2\alpha}$:
        $\tan{2\alpha}=\dfrac{2\tan\alpha}{1-\tan^2\alpha}$
      \end{itemizeMy}
      \item%[半角公式]
        \begin{itemizeMy}[半角公式\hspace{4em}]
          \item $\sin{\dfrac{\alpha}2}=\pm\sqrt{\dfrac{1-\cos\alpha}2}$,
                $\cos{\dfrac{\alpha}2}=\pm\sqrt{\dfrac{1+\cos\alpha}2}$
          \item $\tan{\dfrac{\alpha}2}=\dfrac{\sin\alpha}{1+\cos\alpha}=\dfrac{1-\cos\alpha}{\sin\alpha}$
        \end{itemizeMy}
      % \item%[万能公式]
      %   \begin{itemizeMy}[万能公式\hspace{4em}]
      %     \item $\sin{\alpha}=\dfrac{2\tan{\dfrac{\alpha}2}}{1+\tan^2{\dfrac{\alpha}2}}}$
      %     \item $\cos{\alpha}=\dfrac{1-\tan^2{\dfrac{\alpha}2}}{1+\tan^2{\dfrac{\alpha}2}}}$
      %     \item $\tan{\alpha}=\dfrac{2\tan{\dfrac{\alpha}2}}{1-\tan^2{\dfrac{\alpha}2}}}$
      %   \end{itemizeMy}
      \item%[辅助角公式]
        \begin{itemizeMy}[辅助角公式\hspace{3em}]
          \item $a\sin x+b\cos x=\sqrt{a^2+b^2}\sin(x+\varphi)$,
            其中$\sin\varphi=\dfrac{b}{\sqrt{a^2+b^2}}$,$\cos\varphi=\dfrac{a}{\sqrt{a^2+b^2}}$\\
            $a>0$时,
            令$\tan\varphi=\dfrac{b}a$,$
            \varphi\in\Bigl(-\dfrac{\piup}2,\dfrac{\piup}2\Bigr)$
          \item 令$\bm p=(b,a)$,$\bm q=(\cos x,\sin x)$,则由$\bm p\cdot\bm q=\abs{\bm p}\abs{\bm q} \cos\vangle{\bm p}{\bm q}$,得:\\
          $a\sin x+b\cos x=\sqrt{a^2+b^2}\cos{(x-\phi)}$,其中角$\phi$终边经过点$(b,a)$.
        \end{itemizeMy}
  \end{description}
  \begin{exercise}{\textbf{习题}}
    \item%《2018天利38套:高考真题单元专题训练(理)ISBN978-7-223-03438-8》专题15三角恒等变换 P57p4【2008•山东】
      (2008 \textbullet {\kaishu 山东})
      已知$\cos{\Bigl(\alpha-\dfrac{\piup}6\Bigr)}+\sin\alpha=\dfrac{4}5\sqrt{3}}$,则$\sin{\Bigl(\alpha+\dfrac{7\piup}6\Bigr)}$的值是\xz
      \xx{$-\dfrac{2\sqrt{3}}5$}
       {$\dfrac{2\sqrt{3}}5$}
       {$-\dfrac{4}5$}
       {$\dfrac{4}5$}
      \begin{answer}
        C
      \end{answer}
    \item%《2018天利38套:高考真题单元专题训练(理)ISBN978-7-223-03438-8》专题15三角恒等变换 P57p7【2013•浙江】
      (2013 \textbullet {\kaishu 浙江})
      已知$\alpha\in\mathbb{R}$,$\sin\alpha+2\cos\alpha=\dfrac{\sqrt{10}}2$,则$\tan{2\alpha}=$\xz
      \xx{$\dfrac{4}3$}{$\dfrac{3}4$}{$-\dfrac{3}4$}{$-\dfrac{4}3$}
      \begin{answer}
        C
      \end{answer}
    \item%《2018天利38套:高考真题单元专题训练(理)ISBN978-7-223-03438-8》专题15三角恒等变换 P57p5【2014•全国新课标】
      (2014 \textbullet {\kaishu 全国新课标})
      设$\alpha\in\Bigl(0,\dfrac{\piup}2\Bigr)$,$\beta\in\Bigl(0,\dfrac{\piup}2\Bigr)$,且$\tan\alpha=\dfrac{1+\sin\beta}{\cos\beta}$,则\xz
      \xx{$3\alpha-\beta=\dfrac{\piup}2$}
       {$3\alpha+\beta=\dfrac{\piup}2$}
       {$2\alpha-\beta=\dfrac{\piup}2$}
       {$2\alpha+\beta=\dfrac{\piup}2$}
      \begin{answer}
        C
      \end{answer}
    \item%《2018天利38套:高考真题单元专题训练(理)ISBN978-7-223-03438-8》专题13三角函数的概念、... P49p7【2011•福建】
      (2011 \textbullet {\kaishu 福建})
      若$\tan\alpha=3$,则$\dfrac{\sin{2\alpha}}{\cos^2\alpha}$的值等于\xz
      \xx{2}{3}{4}{6}
      \begin{answer}
        D
      \end{answer}
    \item%《习题化知识清单》P87方法1【三角函数式化简】
      化简:
      $\sin{\Bigl(3x+\dfrac{\piup}3\Bigr)}\cos{\Bigl(x-\dfrac{\piup}6\Bigr)}+\cos{\Bigl(3x+\dfrac{\piup}3\Bigr)}\cos{\Bigl(x+\dfrac{\piup}3\Bigr)}=$\tk.
      \begin{answer}
        \cos{2x}
      \end{answer}
    \item%《2018天利38套:高考真题单元专题训练(理)ISBN978-7-223-03438-8》专题14三角函数的图像与性质 P54p16【2013•全国新课标】
      (2013 \textbullet {\kaishu 全国新课标})
      设当$x=\theta$时,函数$f(x)=\sin x-2\cos x$取得最大值,则$\cos\theta=$\tk.
      \begin{answer}
        $-\dfrac{2\sqrt{5}}5$
      \end{answer}
    \item%《习题化知识清单》P87方法1【三角函数式化简】
      函数$y=\sin{\Bigl(\dfrac{\piup}2+x\Bigr)\cos{\Bigl(\dfrac{\piup}6-x\Bigr)}}$的最大值为\tk.
      \begin{answer}
        $\dfrac{2+\sqrt{3}}4$
      \end{answer}
  \end{exercise}
\newpage

\section{课后作业}
\begin{exercise}
  \item%【向量的线性运算】
    若点$D$在$\triangle{ABC}$的边$BC$上,且$\vv{CD}=4\vv{DB}=r\vv{AB}+s\vv{AC}$,则$3r+s$的值为\xz
    \xx{$\dfrac{16}5$}
     {$\dfrac{12}5$}
     {$\dfrac{8}5$}
     {$\dfrac{4}5$}
    \begin{answer}
      C
    \end{answer}
  \item%福州屏东中学2016-2017学年高一下学期期末考试数学试题.doc-4【向量共线】
    (2017 \textbullet {\kaishu 屏东中学} 4)
    若$A(-1,1)$,$B(1,3)$,$C(x,5)$,且$\vv{AB}=\lambda\vv{BC}$,则实数$\lambda$等于\xz
    \xx{1}{2}{3}{4}
    \begin{answer}
      1
    \end{answer}
  \item%福建师大附中2016-2017高一下期末考试数学试题…….doc-3【向量投影,坐标表示】
    (2017 \textbullet {\kaishu 师大附中} 3)
    若$\bm a=(2,1)$,$\bm b=(3,4)$,则向量$\bm b$在向量$\bm a$方向上的投影为\xz
    \xx{$2\sqrt5$}
     {$2$}
     {$\sqrt5$}
     {$10$}
    \begin{answer}
      A
    \end{answer}
  \item%【向量表示】
    设$D$为$\triangle{ABC}$所在平面内一点,$\vv{BD}=3\vv{CD}$,则\xz
    \xx{$\vv{AD}=-\dfrac13\vv{AB}+\dfrac43\vv{AC}$}
     {$\vv{AD}=\dfrac43\vv{AB}-\dfrac13\vv{AC}$}
     {$\vv{AD}=\dfrac23\vv{AB}-\dfrac12\vv{AC}$}
     {$\vv{AD}=-\dfrac12\vv{AB}+\dfrac32\vv{AC}$}
    \begin{answer}
      D
    \end{answer}
  \item%【向量夹角、模长】
    已知$|\bm a|=1$,$\bm a\cdot\bm b=\dfrac12$,$|\bm a-\bm b|^2=1$,则$\bm a$与$\bm b$的夹角等于\xz
    \xx{30\degree}{45\degree}{60\degree}{120\degree}
    \begin{answer}
      C
    \end{answer}
  \item%【向量共线】
    已知向量$\bm a=(2,3)$,$\bm b=(-1,2)$,若$m\bm a+4\bm b$与$\bm a-2\bm b$共线,则$m$的值为\xz
    \xx{$\dfrac12$}{$2$}{$-\dfrac12$}{$-2$}
    \begin{answer}
      D
    \end{answer}
  \item%【向量几何应用】
    在平面四边形$ABCD$中,若$AC=3$,$BD=2$,则$(\vv{AB}+\vv{DC})\cdot(\vv{AC}+\vv{BD})=$\tk.
    \begin{answer}
      5
    \end{answer}
  \item%《习题化知识清单》P84知识3-3【数量积的运算律】
    已知不共线向量$\bm a$,$\bm b$,$\abs{\bm a}=2$,$\abs{\bm b}=3$,$\bm a\cdot(\bm b-\bm a)=1$,则$\abs{\bm b-\bm a}=$\tk.
    \begin{answer}
      $\sqrt{3}$
    \end{answer}
  \item%《习题化知识清单》P84方法2【求向量模的基本方法】
    已知向量$\bm a$,$\bm b$夹角为45\degree,且$\abs{\bm a}=1$,$\abs{2\bm a-\bm b}=\sqrt{10}$,则$\abs{\bm b}=$\tk.
    \begin{answer}
      $3\sqrt{2}$
    \end{answer}
  \item%《2018天利38套:全国卷高考常考基础题(理)ISBN978-7-223-03393-0》练习8 三角恒等变换 P22p15
    已知$\cos(x+2\theta)+2\sin\theta\sin(x+\theta)=\dfrac{1}3$,则$\cos{2x}$的值为\tk.
    \begin{answer}
      $-\dfrac{7}9$
    \end{answer}
  \item%《2018天利38套:高考真题单元专题训练(理)ISBN978-7-223-03438-8》专题15三角恒等变换 P58p11【2017•江苏】
    (2017 \textbullet {\kaishu 江苏})
    若$\tan{\Bigl(\alpha-\dfrac{\piup}4\Bigr)}=\dfrac{1}6$,则$\tan\alpha=$\tk.
    \begin{answer}
      $\dfrac{7}5$
    \end{answer}
  \item%《2018天利38套:全国卷高考常考基础题(理)ISBN978-7-223-03393-0》练习8 三角恒等变换 P22p20
    已知$\sin{2\alpha}-2=2\cos{2\alpha}$,则$\sin^2\alpha+\sin{2\alpha}=$\tk.
    \begin{answer}
      $1$或$\dfrac{8}5$
    \end{answer}
  % \item%【向量垂直】
  %   平面向量$\bm a=(\sqrt{3},-1)$,$\bm=\Bp{\dfrac12,\dfrac{\sqrt3}2}$,若存在不同时为0的实数$k$和$t$,
  %   使$\bm x=\bm a+(t^2-3)\bm b$,$\bm y=-k\bm a+t\bm b$,且$\bm x\perp\bm y$,试求函数关系式$k=f(t)$.
  %   \begin{answer}
  %     $k=f(t)=\dfrac14(t^3-3t)$
  %   \end{answer}
  % \vspace{2.5cm}
  \item%福州三中2017高一下数学期末卷…….doc-18【向量垂直,模长,共线】
    (2017 \textbullet {\kaishu 福州三中} 18)
    平面内的向量$\bm a=(3,2)$,$\bm b=(-1,2)$,$\bm c=(4,1)$.\\
    (I)若$(\bm a+k\bm c)\perp(2\bm b-\bm a)$,求实数$k$的值;\\
    (II)若向量$\bm d$满足$\bm d\varparallel\bm c$,且$\abs{\bm d}=\sqrt{34}$,求向量$\bm d$的坐标.
    \begin{answer}
      (I)$k=-\dfrac{11}{18}$
      (II)$\bm d=(4\sqrt2,\sqrt2)$或$\bm d=(-4\sqrt2,-\sqrt2)$
    \end{answer}
  \vspace{5cm}
  \item
    已知点$P$是$\triangle{ABC}$内一点,且满足条件$\vv{AP}+\vv{AP}+\vv{AP}=\bm 0$,
    设点$Q$为$CP$的延长线与$AB$的交点,令$\vv{CP}=\bm p$,试用向量$\bm p$表示$\vv{CQ}$.
    \begin{answer}
      $\vv{CQ}=2\bm p$
    \end{answer}
  \vspace{6cm}
  \item%《2018天利38套:高考真题单元专题训练(文)》专题18平面向量的概念与运算 P63p31【2010•江苏】
    (2010 \textbullet {\kaishu 江苏})
    在平面直角坐标系$xOy$中,已知点$A(-1,-2)$,$B(2,3)$,$C(-2,-1)$.\\
    (I)求以线段$AB$,$AC$为邻边的平行四边形的两条对角线的长;\\
    (II)设实数$t$满足$(\vv{AB}-t\vv{OC})\cdot\vv{OC}=0$,求$t$的值.
    \begin{answer}
      (I)两条对角线长分别为$4\sqrt2$,$2\sqrt{10}$;
      (II)$t=-\dfrac{11}5$
    \end{answer}
  \vspace{5.5cm}
  % \item%《2018天利38套:高考真题单元专题训练(理)ISBN978-7-223-03438-8》专题18平面向量的应用 P72p17【2014•陕西】
  %   (2014 \textbullet {\kaishu 陕西})
  %   在直角坐标系$xOy$中,已知点$A(1,1)$,$B(2,3)$,$C(3,2)$,点$P(x,y)$在$\triangle{ABC}$三边围成的区域(含边界)上.\\
  %   (I)若$\vv{PA}+\vv{PB}+\vv{PC}=\bm 0$,求$\abs{\vv{OP}}$;\\
  %   (II)设$\vv{OP}=m\vv{AB}+n\vv{AC}$($m,n\inR$),用$x$,$y$表示$m-n$,并求$m-n$的最大值.
  %   \begin{answer}
  %     (I)$\abs{\vv{OP}}=2\sqrt2$;
  %     (II)$(x,y)=(m+2n,2m+n)$,$m-n$最大值为1.
  %   \end{answer}
  \item%福州屏东中学2016-2017学年高一下学期期末考试数学试题.doc-17【向量共线】
    (2017 \textbullet {\kaishu 屏东中学} 17)
    已知向量$\bm m=(\cos\alpha,1-\sin\alpha)$,$\bm n=(-\cos\alpha,\sin\alpha)$($\alpha\inR$)\\
    (I)若$\bm m\perp\bm n$,求角$\alpha$的值;\qquad (II)若$\abs{\bm m-\bm n}=\sqrt3$,求$\cos{2\alpha}$的值.
    \begin{answer}
      (I)$\alpha=\dfrac{\piup}2$
      (II)$\cos{2\alpha}=\dfrac{\sqrt2}2$
    \end{answer}
\end{exercise}
\stopexercise

\newpage
\section{参考答案}
\begin{multicols}{2}
  \printanswer
\end{multicols}

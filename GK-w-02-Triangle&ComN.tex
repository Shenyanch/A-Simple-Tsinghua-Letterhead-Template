\Topic{三角函数基础、三角恒等变换、复数}
  \Teach{三角恒等变换}
  \Grade{高三}
  % \Name{郑皓天}\FirstTime{20181207}\CurrentTime{20181207}
  % \Name{林叶}\FirstTime{20180908}\CurrentTime{20181125}
  %\Name{1v2}\FirstTime{20181028}\CurrentTime{20181117}
  % \Name{林叶}\FirstTime{20180908}\CurrentTime{20181125}
  % \Name{郭文镔}\FirstTime{20181111}\CurrentTime{20181117}
  % \Name{马灿威}\FirstTime{20181111}\CurrentTime{20181111}
  % \Name{黄亭燏}\FirstTime{20181231}\CurrentTime{20181231}
  % \Name{王睿妍}\FirstTime{20190129}\CurrentTime{}
  \Name{郑旭晶}\FirstTime{20190423}\CurrentTime{20190425}
  \newtheorem*{Theorem}{定理}
  \makefront
\vspace{-1.5em}
\startexercise
% \begin{exercise}{\heiti 课前检测}\\
% \end{exercise}
\section{三角函数概念、同角关系与恒等变换}
  \begin{description}[leftmargin=0pt,labelsep=0pt]
    \item%[两角的和与差]
      \begin{itemizeMy}[两角的和与差\hspace{2em}]
        \item $\mathrm{C}_{\alpha\pm\beta}$:
        $\cos(\alpha\pm\beta)=\cos\alpha\cos\beta \mp \sin\alpha\sin\beta$
        \item $\mathrm{S}_{\alpha\pm\beta}$:
        $\sin(\alpha\pm\beta)=\sin\alpha\cos\beta \pm \cos\alpha\sin\beta$
        \item $\mathrm{T}_{\alpha\pm\beta}$:
        $\tan(\alpha\pm\beta)=\dfrac{\tan\alpha\pm \tan\beta}{1\mp\tan\alpha\tan\beta}$
      \end{itemizeMy}
    \item%[二倍角公式]
      \begin{itemizeMy}[二倍角公式\hspace{3em}]
        \item $\mathrm{S}_{2\alpha}$:
        $\sin{2\alpha}=2\sin\alpha\cos\alpha$
        \item $\mathrm{C}_{2\alpha}$:
        $\cos{2\alpha}=\cos^2{\alpha}-\sin^2{\alpha}=2\cos^2\alpha-1=1-2\sin^2\alpha$
        \item $\mathrm{T}_{2\alpha}$:
        $\tan{2\alpha}=\dfrac{2\tan\alpha}{1-\tan^2\alpha}$
      \end{itemizeMy}
      \item%[半角公式]
        \begin{itemizeMy}[半角公式\hspace{4em}]
          \item
          $\sin{\dfrac{\alpha}2}=\pm\sqrt{\dfrac{1-\cos\alpha}2}$
          \item $\cos{\dfrac{\alpha}2}=\pm\sqrt{\dfrac{1+\cos\alpha}2}$
          \item $\tan{\dfrac{\alpha}2}=\dfrac{\sin\alpha}{1+\cos\alpha}=\dfrac{1-\cos\alpha}{\sin\alpha}$
        \end{itemizeMy}
      % \item%[万能公式]
      %   \begin{itemizeMy}[万能公式\hspace{4em}]
      %     \item $\sin{\alpha}=\dfrac{2\tan{\dfrac{\alpha}2}}{1+\tan^2{\dfrac{\alpha}2}}}$
      %     \item $\cos{\alpha}=\dfrac{1-\tan^2{\dfrac{\alpha}2}}{1+\tan^2{\dfrac{\alpha}2}}}$
      %     \item $\tan{\alpha}=\dfrac{2\tan{\dfrac{\alpha}2}}{1-\tan^2{\dfrac{\alpha}2}}}$
      %   \end{itemizeMy}
      \item%[辅助角公式]
        \begin{itemizeMy}[辅助角公式\hspace{3em}]
          \item $a\sin x+b\cos x=\sqrt{a^2+b^2}\sin(x+\varphi)$\\
          其中$\sin\varphi=\dfrac{b}{\sqrt{a^2+b^2}}$,$\cos\varphi=\dfrac{a}{\sqrt{a^2+b^2}}$\\
          $a>0$时,
          \item $a\sin x+b\cos x=\sqrt{a^2+b^2}\sin(x+\varphi)$\\
          其中$\tan\varphi=\dfrac{b}a$,$
          \varphi\in\Bigl(-\dfrac{\piup}2,\dfrac{\piup}2\Bigr)$
        \end{itemizeMy}
  \end{description}
  \begin{exercise}
    \item%福建师大附中2015-2016学年高一数学第二学期期末检测.doc-2【象限角】
      (2016 \textbullet {\kaishu 师大附中} 2)
      若点$P(\sin\theta\cos\theta,2\cos\theta)$位于第三象限,那么角$\theta$终边落在\xz
      \xx{第一象限}{第二象限}{第三象限}{第四象限}
      \begin{answer}
        B
      \end{answer}
    \item%《2018天利38套:高考真题单元专题训练(文)》专题13三角函数的概念……P41p2【2015文•福建】【同角三角函数基本关系式】
      {\kaishu (2015文 \textbullet 福建)}
      若$\sin\alpha=-\dfrac5{13}$,且$\alpha$为第四象限角,则$\tan\alpha$的值等于\xz
      \xx{$\dfrac{12}5$}
       {$-\dfrac{12}5$}
       {$\dfrac5{12}$}
       {$-\dfrac5{12}$}
      \begin{answer}
        D
      \end{answer}
    \item%《2018天利38套:高考真题单元专题训练(文)》专题13三角函数的概念……P41p2【2016文•全国新课标】【同角三角函数基本关系式、诱导公式】
      {\kaishu (2016文 \textbullet 全国新课标)}
      已知$\theta$是第四象限角,且$\sin\Bp{\theta+\dfracp{}4}=\dfrac35$,则$\tan\Bp{\theta-\dfracp{}4}=$\tk.
      \begin{answer}
        $-\dfrac43$
      \end{answer}
    \item %《2019金考卷双测20套(文)ISBN978-7-5371-9890-5》题型5三角函数、三角恒等变换P15p3【2018•福州期末】【三角恒等变换】\\
        {\kaishu (2018 \textbullet 福州期末(文))}
        $\sqrt3\cos15\degree-4\sin^215\degree\cos15\degree=$\xz
        \xx{$\dfrac12$}
         {$\dfrac{\sqrt2}2$}
         {$1$}
         {$\sqrt2$}
        \begin{answer}
          D
        \end{answer}
    \item %《2019金考卷双测20套(文)ISBN978-7-5371-9890-5》题型5三角函数、三角恒等变换P15p4【2018•唐山五校联考】【三角恒等变换】\\
        {\kaishu (2018 \textbullet 唐山五校联考(文))}
        已知$\alpha$是第三象限角,且$\tan\alpha=2$,则$\sin\Bp{\alpha+\dfrac{\piup}4}=$\xz
        \xx{$-\dfrac{3\sqrt{10}}{10}$}
         {$\dfrac{3\sqrt{10}}{10}$}
         {$-\dfrac{\sqrt{10}}{10}$}
         {$\dfrac{\sqrt{10}}{10}$}
        \begin{answer}
          A
        \end{answer}

  \end{exercise}
\vspace{4em}
\section{复数}
  \begin{description}[leftmargin=0pt,labelsep=0pt]
    \item[]%[定义]
      \begin{itemizeMy}[定义\hspace{3em}]
        \item 形如$a+b\ii(a,b\inR)$的数叫做复数,通常用字母$z$来表示,即
              \[z=a+b\ii(a,b\inR).\]
              其中$a$叫做复数$z$的实部,$b$叫做复数$z$的虚部.
        \item 当且仅当两个复数的实部和虚部分别相等时,两个复数相等.即:如果$a,b,c,d\inR$,那么
              \[a+b\ii=c+d\ii \Leftrightarrow a=c\text{且}b=d\]
        \item %分类
          \[\text{复数}a+b\ii(a,b\inR)
            \begin{cases}
              \text{实数($b=0$)}\\
              \text{虚数($b\neq0$)}\begin{cases} \text{纯虚数}(a=0)\\\text{非纯虚数}(a\neq0)\end{cases}
            \end{cases}\]
        \item 只有两个数都为实数时,这两个数才可以比较大小.如:$2+3\ii$与$3$之间不存在大小关系.
      \end{itemizeMy}
    \item[]%[表示]
      \begin{itemizeMy}[表示\hspace{3em}]
        \item %任何一个复数$z=a+b\ii(a,b\inR)$,都可以由一个有序实数对$(a,b)$唯一确定.
          复数$z=a+b\ii(a,b\inR)$与坐标为$(a,b)$的点对应
        \item 复数$z=a+b\ii(a,b\inR)$对应的点$(a,b)$与原点的距离叫做复数的模,记为$|z|$.即:
          \[|z|=|a+b\ii|=\sqrt{a^2+b^2}\]
          {\kaishu 复数的模是一个不小于0的实数.}
      \end{itemizeMy}
    \item[]%[复数的运算]
      \begin{itemizeMy}[复数的运算\hspace{3em}]
        $a,b,c,d\inR$
        \item $(a+b\ii)+(c+d\ii)=(a+c)+(b+d)\ii$
        \item $(a+b\ii)(c+d\ii)=ac+ad\ii+bc\ii+bd\ii^2$
        \item 对于复数$z=a+b\ii$,称$\bar{z}=a-b\ii$为$z$的共轭复数.
      \end{itemizeMy}
  \end{description}
  \clearpage
  \begin{exercise}
    \item %《习题化知识清单》P283方法1-1【复数】\\
      已知$\dfrac{(1-\ii)^2}z=1+\ii$($\ii$为虚数单位),则复数$z=$\xz
      \xx{$1+\ii$}
       {$1-\ii$}
       {$-1+\ii$}
       {$-1-\ii$}
      \begin{answer}
        D
      \end{answer}
    \item %《2018天利38套:高考真题单元专题训练(文)ISBN978-7-223-03161-5》专题33算法、复数P117p1【2017•全国II新课标】【复数计算】\\
        {\kaishu (2017 \textbullet 全国II新课标(文))}
        $(1+\ii)(2+\ii)=$\xz
        \xx{$1-\ii$}
         {$1+3\ii$}
         {$3+\ii$}
         {$3+3\ii$}
        \begin{answer}
          B
        \end{answer}
    \item %《2018天利38套:高考真题单元专题训练(文)ISBN978-7-223-03161-5》专题33算法、复数P117p2【2016•全国新课标】【复数计算】\\
        {\kaishu (2016 \textbullet 全国新课标(文))}
        设$(1+2\ii)(a+\ii)$的实部与虚部相等,其中$a$为实数,则$a=$\xz
        \xx{$-3$}{$-2$}{$2$}{$3$}
        \begin{answer}
          A
        \end{answer}
    \item %《2018天利38套:高考真题单元专题训练(文)ISBN978-7-223-03161-5》专题33算法、复数P117p3【2017•北京】【复数计算】\\
        {\kaishu (2017 \textbullet 北京(文))}
        若复数$(1-\ii)(a+\ii)$在复平面内对应的点在第二象限,则实数$a$的取值范围是\xz
        \xx{$(-\infty,1)$}
         {$(-\infty,-1)$}
         {$(1,+\infty)$}
         {$(-1,+\infty)$}
        \begin{answer}
          B
        \end{answer}
    \item %《2018天利38套:高考真题单元专题训练(文)ISBN978-7-223-03161-5》专题33算法、复数P117p4【2016•山东】【复数计算】\\
        {\kaishu (2016 \textbullet 山东(文))}
        若复数$z=\dfrac2{1-\ii}$,其中$\ii$为虚数单位,则$\bar z=$\xz
        \xx{$1+\ii$}
         {$1-\ii$}
         {$-1+\ii$}
         {$-1-\ii$}
        \begin{answer}
          B
        \end{answer}
    \item %《习题化知识清单》P283方法1-2【复数】\\
      已知复数$z=1+\ii$,,则$\dfrac{z^2-2z}{z-1}=$\xz
      \xx{$-2\ii$}
       {$2\ii$}
       {$-2$}
       {$2$}
      \begin{answer}
        B
      \end{answer}
    \item %《2018天利38套:高考真题单元专题训练(文)ISBN978-7-223-03161-5》专题33算法、复数P117p5【2015•全国新课标】【复数计算】\\
        {\kaishu (2015 \textbullet 全国新课标(文))}
        已知复数$z$满足$(z-1)\ii=1+\ii$,则$z=$\xz
        \xx{$-2-\ii$}
         {$-2+\ii$}
         {$2-\ii$}
         {$2+3\ii$}
        \begin{answer}
          C
        \end{answer}
    \item %《2019金考卷双测20套(文)ISBN978-7-5371-9890-5》题型16复数、推理与证明P16p8【2018•开封定位考试】【复数计算】\\
        {\kaishu (2018 \textbullet 开封定位考试(文))}
        已知复数$z=\dfrac{2}{-1+\ii}$,则下列选项中说法正确的是\xz
        \xx{$z$的共轭复数为$1+\ii$}
         {$z$的实部为$1$}
         {$|z|=2$}
         {$z$的虚部为$-1$}
        \begin{answer}
          D
        \end{answer}
  \end{exercise}
\newpage
\section{课后作业}
  \begin{exercise}{\heiti 练习}\\
    \item 求下列复数$z$的值: \begin{multicols}{2}
          \begin{enumerate}[label=\arabic*)]
            \item $(2+\ii)z=(4-3\ii)$;
            \vspace{2cm}
            \item $(1-\ii)^2=(1+\ii)z$;
            \vspace{2cm}
            \item $2+\ii(1-z\ii)=2z+3\ii$;
            \vspace{2cm}
            \item $(1+z\ii)(2+\ii)=3z-4\ii$;
            \vspace{2cm}
          \end{enumerate}
          \begin{answer}
            \begin{enumerate}[itemindent=1em,listparindent=6em, label=\arabic*)]
              \item $z=1-2\ii$;
              \item $z=-1-\ii$;
              \item $z=2-2\ii$;
              \item $z=\dfrac{-1+12\ii}{10}$;
            \end{enumerate}
          \end{answer}
        \end{multicols}
  \end{exercise}
  \begin{exercise}
    \item %【2017•新课标全国卷I】【复数计算】\\
        {\kaishu (2017 \textbullet 全国I新课标(文))}
        下列各式的运算结果为纯虚数的是\xz
        \xx{$\ii(1+\ii)^2$}
         {$\ii^2(1-\ii)$}
         {$(1+\ii)^2$}
         {$\ii(1+\ii)$}
        \begin{answer}
          C
        \end{answer}
    \item %《2019金考卷双测20套(文)ISBN978-7-5371-9890-5》题型16复数、推理与证明P16p1【2018•全国I卷】【复数计算】\\
        {\kaishu (2018 \textbullet 全国I卷(文))}
        设$z=\dfrac{1-\ii}{1+\ii}+2\ii$,则$\bar z=$\xz
        \xx{$0$}
         {$\dfrac12$}
         {$1$}
         {$\sqrt2$}
        \begin{answer}
          C
        \end{answer}
    \item %《2019金考卷双测20套(文)ISBN978-7-5371-9890-5》题型16复数、推理与证明P16p5【2018•太原一模】【复数计算】\\
        {\kaishu (2018 \textbullet 太原一模(文))}
        设复数$z$满足$\dfrac{1-z}{1+z}=\ii$,则$z$的共轭复数为\xz
        \xx{$\ii$}{$-\ii$}{$2\ii$}{$-2\ii$}
        \begin{answer}
          A
        \end{answer}
    \item %《2019金考卷双测20套(文)ISBN978-7-5371-9890-5》题型16复数、推理与证明P16p7【2018•南昌二模】【复数计算】\\
        {\kaishu (2018 \textbullet 南昌二模(文))}
        若实数$x,y$满足$\dfrac{x}{1+\ii}+y=2+\ii$($\ii$为虚数单位),则$x+y\ii$在复平面内对应的点位于\xz
        \xx{第一象限}{第二象限}{第三象限}{第四象限}
        \begin{answer}
          B
        \end{answer}
    \item %《2019金考卷双测20套(文)ISBN978-7-5371-9890-5》题型16复数、推理与证明P16p13【2018•江苏卷】【复数计算】\\
        {\kaishu (2018 \textbullet 江苏卷(文))}
        若复数$z$满足$\ii\cdot z=1+2\ii$,其中$\ii$为虚数单位,则$z$的实部为\tk.
        \begin{answer}
          2
        \end{answer}
  \end{exercise}
\stopexercise

\newpage
\section{参考答案}
\begin{multicols}{2}
  \printanswer
\end{multicols}

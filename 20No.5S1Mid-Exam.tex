% sty文件使用 \RequirePackage{latexexercise0}
% 主文件使用 \documentclass[a3paper,twocolumn,2twoside,landscape,12pt,UTF8]{ctexart}
\hspace{3cm}\\
\vspace{0.5cm}
\centering{\heiti \xiaoer 福州格致中学2020-2021学年第一学段高一数学半期考(A)}\\
\vspace{0.5cm}
% \centering{\heiti \erhao 高一数学\quad 必修4}\\
\vspace{0.4cm}
\centering{\wuhao 命题:郑鹏宇\hspace{5em}审核:金声\hspace{5em}日期:2020年11月}\\
\vspace{0.4cm}
\centering{\wuhao 考试时间:120分钟\hspace{5em}全卷满分:150分}\\
\vspace{-1.6em}
% \part{第I卷(100分)}
% \vspace{-3em}
\startexercise
\begin{exercise}
\section{单项选择题:本题共8小题,每小题5分,共40分.在每小题给出的四个选项中,只有一项是符合题目要求的}
  \item
    已知集合$A=\{x\in\mathbb{N}\mid -1\leq x \leq 4\}$,$B\{x\mid -2\leq x \leq 3\}$,则$A\cap B=$\xz
    \xx{$[-1,3]]$}
      {$[-2,4]]$}
      {$\{0,1,2,3\}$}
      {$\{1,2,3\}$}
    \begin{answer}
      
    \end{answer}
  \item
    命题“$\forall x\geq0,x^2-1\geq-1$”的否定是\xz
      \xx{$\forall x\geq0,x^2-1<-1$}
      {$\forall x<0,x^2-1<-1$}
      {$\exists x\geq0,x^2-1<-1$}
      {$\exists x<0,x^2-1<-1$}
    \begin{answer}
      
    \end{answer}
  \item
    设函数$f(x)=\begin{cases}
      2^x+m,x\leqslant0,\\
      g(x),x>0.
    \end{cases}$是奇函数,则$f(2)=$\xz
      \xx{$\frac34$}{$-\frac34$}{$4$}{$-4$}
    \begin{answer}
      
    \end{answer}
  \item 
    “$0<a<4$”是“关于$x$的方程$ax^2+ax+1=0$无实根”的\xz
      \xx{充分不必要条件}
       {必要不充分条件}
       {充要条件}
       {既不充分也不必要条件}
    \begin{answer}
      
    \end{answer}
  \item
    已知$x=2^{0.6}$,$y=\log_{1.2}2.4$,$y=\log_{1.2}3.6$,则\xz
      \xx{$x<y<z$}
        {$x<z<y$}
        {$z<x<y$}
        {$y<x<z$}
    \begin{answer}
      
    \end{answer}
  \item
    函数$y=\tfrac{xa^x}{|x|}$的图像的大致形状是\xz
      \xx{\includegraphics{}}
       {$1$}
       {$\dfrac{24}{25}$}
       {$-\dfrac{24}{25}$}
    \begin{answer}
      C
    \end{answer}
    \item
      $f(x)=\Bigl(\frac12\Bigr)^x-\frac12$
    \item
      已知$f(x)=\begin{cases}
        \frac{a}{x},x\leqslant-1,\\
        (a-3)x+a-5,x>-1.
      \end{cases}$ 在$(-\infty,+\infty)$
    \item
      酒驾
      \begin{answer}
        $x\geqslant\log_{\frac34}0.3=\frac{\lg0.3}{\lg\frac34}=\frac{\lg3-1}{\lg3-\lg4}\approx\frac{0.48-1}{0.48-0.60}\approx4.3$
      \end{answer}
   
    \item
      $\log_3\sqrt{27}+\lg{25}+\lg4-7^{\log_7{2}}+\log_42$的值是\tk.
    \item
      已知函数$f(x)=1-ax+\log_2{\frac{1-x}{1+x}}$\\
      (1)若$f\left(\frac35\right)=-\frac85$,则实数$a$的值为\tk;  
      (2)$f\left(\frac1{2019}\right)+f\left(-\frac1{2019}\right)=$\tk. 
    \item  
      设集合~$A=\{x\mid -1< x <2\}$~,$B=\{x\mid x^2-3x<0\}$,
      $C=\left\{x\in\mathbb{N}\bigm\mid \frac{10}x\in\mathbf{N}\right\}$
      \\$\complement_{\mathbf R}B$
    \item
      $y=\frac1{f(x)}$
    \item
      $f(x)=\frac{ax+b}{x^2+1}$\\
    \\\begin{minipage}[htbp!]{0.7\linewidth}\item
    如图,正方形ABCD中,点E,F分别是DC,BC的中点,那么$\vv{EF}=$\xz
    \xx{$\dfrac12\vv{AB}-\dfrac12\vv{AD}$}
     {$-\dfrac12\vv{AB}+\dfrac12\vv{AD}$}
     {$-\dfrac12\vv{AB}-\dfrac12\vv{AD}$}
     {$\dfrac12\vv{AB}+\dfrac12\vv{AD}$}
    \end{minipage}
    \begin{minipage}[htbp!]{0.25\linewidth}
      \begin{flushright}
        \vspace{-1.5em}
        \begin{tikzpicture}
          \coordinate[label=left:$A$](A)at(0,0);
          \coordinate[label=right:$B$](B)at(2,0);
          \coordinate[label=left:$D$](D)at(0,2);
          \coordinate[label=right:$C$](C)at(2,2);
          \coordinate[label=right:$F$](F)at($(C)!0.5!(B)$);
          \coordinate[label=above:$E$](E)at($(C)!0.5!(D)$);
          \draw (A)--(B)--(C)--(D)--cycle;
          \draw[->,>=latex] (A)--(B);
          \draw[->,>=latex] (A)--(D);
          \draw[->,>=latex] (E)--(F);
        \end{tikzpicture}
      \end{center}
    \end{minipage}
    \begin{answer}
      A
    \end{answer}
  \item
    已知$a=\dfrac{\sqrt2}2(\sin{18\degree}+\cos{18\degree})$,$b=2\cos^2{16\degree}-1$,$c=\dfrac{\sqrt3}2$,则\xz
    \xx{$c<a<b$}{$b<c<a$}{$a<b<c$}{$b<a<c$}
    \\
    \begin{answer}
      B
    \end{answer}
\par
\section{填空题(本大题共4小题,每小题5分,共20分)}
  \item
     在平行四边形$ABCD$中,若$\vv{AB}=(2,4)$,$\vv{AD}=(-1,-1)$,则$\vv{BD}=$\tk
    \begin{answer}
      $(-3,-5)$
    \end{answer}
  \item
     若角$\alpha$的终边过点$P\bigl(2\cos{120\degree},\sqrt2\sin(-45\degree)\bigr)$,则$\sin\alpha=$\tk
     \begin{answer}
       $-\dfrac{\sqrt2}2$
     \end{answer}
  \item
     已知函数$y=A\sin(\omega x+\varphi)$($A>0$,$\omega>0$)的振幅是$3$,频率是$\dfrac5{2\piup}$,初相是$\dfrac{\piup}6$,则这个函数是$y=$\tk
    \begin{answer}
      $3\sin\Bp{5x+\dfrac{\piup}6}$
    \end{answer}
  \item
    设$\bm a$,$\bm b$,$\bm c$为任意非零向量,且相互不共线,则下列命题中是真命题的序号为\tk\\
    (1)$|\bm a|-|\bm b|<|\bm a-\bm b|$\hspace{4em}
    (2)$(\bm a\cdot\bm b)\cdot\bm c-(\bm c\cdot\bm a)\cdot\bm b=0$\\
    (3)$(\bm b\cdot\bm c)\cdot\bm a-(\bm c\cdot\bm a)\cdot\bm b$与$\bm c$垂直\hspace{2em}
    (4)$(3\bm a+\bm b)\cdot(3\bm a-\bm b)=9|\bm a|^2-|\bm b|^2$
    \begin{answer}
      (1)(3)(4)
    \end{answer}
\section{解答题(本大题共有4个小题,共40分. 解答应写出文字说明、演算步骤或证明过程)}
  \item
    (本小题满分8分)\\
    已知$\bm a=(1,2)$,$\bm b=(x,1)$,分别求实数$x$的值使得:\\
    \circled{1} $(2\bm a+\bm b)\varparallel(\bm a-2\bm b)$;
    \circled{2} $\bm a$与$\bm b$的夹角是60\degree.
    \begin{answer}
      \circled{1} $2\bm a+\bm b=(2+x,5)$,$\bm a-2\bm b=(1-2x,0)$;
        $\because (2\bm a+\bm b)\varparallel(\bm a-2\bm b)$,
        $\therefore 0+5(1-2x)=0$,$\therefore x=\dfrac12$;\par
      \circled{2} $\bm a\cdot\bm b
        =\abs{\bm a}\abs{\bm b}\cos{60\degree}
        =\sqrt5\cdot\sqrt{x^2+1}\cdot\dfrac12=x+2$,即
        $5(x^2+1)=4(x+2)^2$,
        解得$x=8\pm 5\sqrt3$
    \end{answer}
  \vspace{4em}
  \item
    (本小题满分10分)\\
    已知$\cos\alpha-\sin\alpha=\dfrac{3\sqrt2}5$,且$\piup<\alpha<\dfrac{3\piup}2$,化简$\dfrac{\sin{2\alpha}+2\sin^2\alpha}{1-\tan\alpha}$并求其值.
    \begin{answer}
      由$\cos\alpha-\sin\alpha=\dfrac{3\sqrt2}5$,
      $\cos^2\alpha+\sin^2\alpha-2\sin\alpha\cos\alpha=\dfrac{3\sqrt2}{5}$,\\
      $\therefore 2\sin\alpha\cos\alpha=1-\dfrac{18}{25}=\dfrac7{25}$;\\
      $\therefore (\sin\alpha+\cos\alpha)^2=\sin^2\alpha+\cos^2\alpha+2\sin\alpha\cos\alpha
      =1+\dfrac7{25}=\dfrac{32}{25}$;
      又$\because \piup<\alpha<\dfrac{3\piup}2$,
      $\therefore \sin\alpha<0$,$\cos\alpha<0$,\\
      $\therefore \sin\alpha+\cos\alpha=\dfrac{4\sqrt2}5$,\\
      $\therefore \dfrac{\sin{2\alpha}+2\sin^2\alpha}{1-\tan\alpha}
      =\dfrac{2\sin\alpha\cos\alpha+2\sin^2\alpha}{1-\dfrac{\sin\alpha}{\cos\alpha}}
      =\dfrac{2\sin\alpha\cos\alpha(\sin\alpha+\cos\alpha)}{\sin\alpha+\cos\alpha}
      =\dfrac{\dfrac7{25}\cdot(-\dfrac{4\sqrt2}{5})}{\dfrac{3\sqrt2}{5}}
      =-\dfrac{28}{75}$
    \end{answer}
  \vspace{4em}
  \item
    (本小题满分10分)\\
    已知向量$\bm m=(1,1)$,向量$\bm n$与$\bm m$的夹角为$\dfrac{3\piup}4$,且$\bm m\cdot\bm n=-1$.\\
    (1)求向量$\bm n$;\\
    (2)设向量$\bm a=(1,0)$,$\bm b=(\cos x,\sin x)$,其中$x\inR$,若$\bm n\cdot\bm a=0$,试求$|\bm n+\bm b|$的取值范围.
    \begin{answer}
      (1)由$\bm m=(1,1)$,$\bm m$与x轴正半轴夹角为45\degree,
        又$\bm n$与$\bm m$的夹角为$\dfrac{3\piup}4$,故$\bm n$在$x$轴或$y$轴负半轴上,\\
        设$\bm n=(n_1,0)$或$\bm n=(0,n_2)$,又$\bm m\cdot\bm n=-1$,
        $\therefore \bm n=(-1,0)$或$(0,-1)$.\par
      (2)由$\bm n\cdot\bm a=0$,$\bm n=(0,-1)$;
        $\therefore \abs{\bm n+\bm b}=\sqrt{\cos^2x+(\sin{x}-1)^2}
        =\sqrt{\cos^2x+\sin^2x-2\sin x+1}=\sqrt{2-2\sin x} \in[0,2]$;
    \end{answer}
  \vspace{5em}
  \item
    (本小题满分12分)\\
    已知函数$f(x)=\sqrt3\cos{\omega x}$,$g(x)=\sin\Bp{\omega x-\dfrac{\piup}3}$($\omega>0$),且$g(x)$的最小正周期为$\piup$.\\
    (1)若$f(\alpha)=\dfrac{\sqrt6}2$,$\alpha\in[0,\piup]$,求$\alpha$的值;\\
    (2)求函数$y=f(x)+g(x)$的单调递增区间.
    \begin{answer}
      (1)$g(x)$的最小正周期为$\piup$,$\therefore \dfrac{2\piup}{\omega}=\piup$,
        $\there \omega=2$\\
        $\therefore f(\alpha)=\sqrt3\cos{2\alpha}=\dfrac{\sqrt6}2$,
        $\therefore 2\alpha=\pm\dfrac{\piup}4+2k\pipu$,
        $\therefore \alpha=\pm\dfrac{\piup}8+k\pipu$,$k\inZ$;\\
        又$\alpha\in[0,\piup]$,
        $\therefore$ $\alpha=\dfrac{\piup}8$或$\dfrac{7\piup}8$\par
      (2)$y=f(x)+g(x)=\sqrt3\cos{2x}+\sin\Bp{2x-\dfrac{\piup}3}
          =\dfrac12\sin{2x}+\dfrac{\sqrt3}2\cos{2x}=\sin\Bp{2x+\dfrac{\piup}3}$,\\
          由$-\dfrac12\piup+2k\piup \leqslant 2x+\dfrac{\piup}3 \leqslant \dfrac12\piup+2k\piup$,得:
          $-\dfrac{5\piup}{12}+k\piup \leqslant x \leqslant \dfrac{\piup}{12}+k\piup$,$k\inZ$;\\
          故函数$y=f(x)+g(x)$的单调递增区间为
          $\Bigl[-\dfrac{5\piup}{12}+k\piup,\dfrac{\piup}{12}+k\piup\Bigr]$,$k\inZ$
    \end{answer}
  \vspace{4em}
\newpage
\vspace{-3em}
\part{第II卷(50分)}
\vspace{-1.5em}
\section{选择题(本大题共4小题,每小题4分,共16分.每题有且只有一个选项是正确的,请把答案填在答卷相应位置上)}
  \item
    若$\cos{165\degree}=a$,则$\tan{195\degree}=$\xz
      \xx
       {$\sqrt{1-a^2}$}
       {$\dfrac{\sqrt{1+a^2}}a$}
       {$\dfrac{\sqrt{1-a^2}}a$}
       {$-\dfrac{\sqrt{1-a^2}}a$}
    \begin{answer}
      D
    \end{answer}
  \item
    已知函数$f(x)=2\sin\Bp{\omega x+\dfrac{\piup}3}$图像的一个对称中心为$\Bp{\dfrac{\piup}3,0}$,其中$\omega$为常数,且$\omega\in(1,3)$,若对任意的实数$x$,恒有$f(x_1)\leq f(x)\leq f(x_2)$,则$|x_1-x_2|$的最小值是\xz
      \xx{$1$}{$\dfrac{\piup}2$}{$2$}{$\piup$}
    \begin{answer}
      B
    \end{answer}
  \item
    已知$O$是$\triangle{ABC}$所在平面上一点,满足$|\vv{OA}|^2+|\vv{BC}|^2=|\vv{OB}|^2+|\vv{CA}|^2$,则点$O$\xz
      \xx{在过点$C$与$AB$垂直的直线上}
       {在$\angle{A}$的平分线所在直线上}
       {在过点$C$边$AB$的中线所在直线上}
       {以上都不对}
    \begin{answer}
      A
    \end{answer}
  \item
    目前听说中国最高的摩天轮是“南昌之星”,它的最高点离地面160米,直径为156米,并以每30分钟一周的速度匀速旋转,若从最低点开始计时,则摩天轮运行5分钟后离地面的高度为\xz
      \xx{31米}{43米}{58米}{63米}
    \begin{answer}
      B
    \end{answer}
\section{填空题(本大题共2小题,每小题4分,共8分)}
  \item
    若函数$f(x)=\tan{\omega x}$($\omega>0$)的图像相邻的两支截直线$y=\dfrac{\piup}4$所得线段的长为$\dfrac{\piup}4$,则$f(\dfrac{\piup}4)$的值是\tk
    \begin{answer}
      $0$
    \end{answer}
  \item
    在矩形$ABCD$中,边$AB$、$AD$的长分别为2、1,若$M$、$N$分别是边$BC$、$CD$上的点,且满足$\dfrac{|\vv{BM}|}{|\vv{BC}|}=\dfrac{|\vv{CN}|}{|\vv{CD}|}$,则$\vv{AM}\cdot\vv{AN}$的取值范围是\tk
    \begin{answer}
      $[1,4]$
    \end{answer}
\newpage
\section{解答题(本大题共有2个小题,共26分. 解答应写出文字说明、演算步骤或证明过程)}
  \item
    (本小题满分12分)\\
    已知$O$为$\triangle{ABC}$的外心,以线段$OA$、$OB$为邻边作平行四边形,第四个顶点为$D$,再以$OC$、$OD$为邻边作平行四边形,它的第四个顶点为$H$.\\
    (1) 若$\vv{OA}=\bm a$,$\vv{OB}=\bm b$,$\vv{OC}=\bm c$,$\vv{OH}=\bm h$,试用$\bm a$,$\bm b$,$\bm c$表示$\bm h$;\\
    (2)证明:$\vv{AM}\perp\vv{BC}$;\\
    (3)若$\triangle{ABC}$的$\angle{A}=60\degree$,$\angle{B}=45\degree$,外接圆的半径为$R$,用$R$表示$|\bm h|$.
    \begin{answer}
      (1)$\bm h=\bm a+\bm b+\bm c$\par
      (2)\because $\vv{AH}=\vv{OH}-\vv{OA}=\bm b+\bm c$,$\vv{BC}=\bm c-\bm b$.\\
         \therefore $\vv{AH}\cdot\vv{BC}=|\bm c|^2-|\bm b|^2=0$,\\
         \therefore $\vv{AH}\perp\vv{BC}$.\par
      (3)\because$|\bm h|^2=(\bm a+\bm b+\bm c)^2=3R^2+2R^2(\cos120\degree+\cos90\degree+\cos150\degree)=(2-\sqrt3)R^2$,\\
         \therefore $|\bm h|=\dfrac{\sqrt6-\sqrt2}2 R$
    \end{answer}
  \vspace{12em}
  \item
    (本小题满分14分)\\
    已知函数$f(x)=2\cos x\sin\Bp{x+\dfrac{\piup}3}-\sqrt3\sin^2x+\sin x\cos x$\\
    (1) 当$x\in\Bigl[0,\dfrac{\piup}2\Bigr]$时,求$f(x)$的值域;\\
    (2) 解不等式:$f(x)+1\geq 0$;\\
    (3) 若$x\in\Bigl[0,\dfrac{\piup}3\Bigr]$时,方程$f\Bp{\dfrac32x-\dfrac{\piup}3}=m$恰有两个不同的解,求实数$m$的取值范围
    \begin{answer}
      (1) $f(x)=2\cos x\sin\Bp{x+\dfrac{\piup}3}-\sqrt3\sin^2x+\sin x\cos x
               =2\cos x(\dfrac12\sin x+\dfrac{\sqrt3}2\cos x)-\sqrt3\sin^2x+\sin x\cos x
               =\sin{2x}+\sqrt3\cos{2x}
               =2\sin\Bp{2x+\dfrac{\piup}3}$,\\
          故$x\in\Bigl[0,\dfrac{\piup}2\Bigr]$时,$2x+\dfrac{\piup}3 \in\Bigl[\dfrac{\piup}3,\dfrac{4\piup}4\Bigr]$,\\
          $f(x)$值域为$[-\sqrt3,2]$;\par
      (2) 由$f(x)+1\geq 0$,得$\sin\Bp{2x+\dfrac{\piup}3}\geq -\dfrac12$\\
          $\therefore -\dfrac{\piup}6+2k\piup \leqslant 2x+\dfrac{\piup}3 \leqslant \dfrac{7\piup}{6}+2k\piup$,$k\inZ$;\\
          $\therefore \Bigl[-\dfrac{\piup}4+k\piup,\dfrac{5\piup}{12}+k\piup \Bigr]$,$k\inZ$;\par
      (3) $x\in\Bigl[0,\dfrac{\piup}3\Bigr]$时,$f\Bp{\dfrac32x-\dfrac{\piup}3}=2\sin\Bp{3x-\dfrac{\piup}3}$,
          \begin{center}
            \begin{tikzpicture}[>=latex,,scale=1,declare function={f(\k)=2*sin(deg(3*\k-pi/3));}]
              \tikzmath{
                \xmin = 0;    \xmax = pi/3;
                \ymin = -2;   \ymax = 2;
                \xstep= pi/6; \ystep=1;
                \xl=\xmin-0.5*\xstep; \xr=\xmax+0.5*\xstep;
                \yl=\ymin-0.5*\ystep; \yr=\ymax+0.5*\ystep;
              }
              % \tikzset{elegant/.style={smooth,thick,samples=50,magenta}}
              \begin{axis}[axis x line=middle,
                     axis y line=middle,
                     xmin=\xl,xmax=\xr,
                     ymin=\yl,ymax=\yr,
                     % xstep=\xstep,ystep=\ystep,
                     xtick={0,1,2},
                     ytick distance=\ystep,
                     ylabel=$y$,
                     xlabel=$x$]
                    \addplot[elegant,orange,domain=0:1]{f(x)};
                    \addplot[elegant,dashed,domain=0:\xstep]{2};
                    \addplot[elegant,dashed,domain=\xl:\xr]{1.732};
                    % \draw[dashed] (0,2)node[left](a){$2$}-|(\a,0)node[below](a1){$\dfrac{5\pi}{12}$};
              \end{axis}
              % \draw[->](\xl,0)--(\xr,0) node[below](x){$x$};
              % \draw[->](0,\yl)--(0,\yr) node[left](y){$y$};
              % \node[below left](O) at(0,0){$\small O$};
              % \draw[domain=\xmin:\xmax,samples=300] plot (\x,{f(\x)});
              % \draw[dashed] (0,2)node[left](a){$2$}-|(\a,0)node[below](a1){$\dfrac{5\pi}{12}$} ;;
              % \draw[dashed](0,-2)node[left](b){$-2$}-|(\b,0)node[above](b1){$\dfrac{11\pi}{12}$} ;
            \end{tikzpicture}
          \end{center}
          $\therefore 3x-\dfrac{\piup}3\in\Bigl[-\dfrac{\piup}3,\dfrac{2\piup}3\Bigr]$,\\
          故方程$f\Bp{\dfrac32x-\dfrac{\piup}3}=m$恰有两个不同的解时,$m\in\Bigl[2\sin\dfrac{2\piup}3,2\Bigr]$,即\\
          $m\in[\sqrt3,2)$;
    \end{answer}
  \vspace{8em}
\end{exercise}
\stopexercise
\hspace{2em}
\clearpage
\hspace{3cm}\\
\vspace{-2.5em}
\part{\heiti \xiaoer 参考答案}
\begin{multicols}{2}
  \printanswer
\end{multicols}

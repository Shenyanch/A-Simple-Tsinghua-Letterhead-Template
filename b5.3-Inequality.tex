\Topic{}
  \Teach{}
  \Grade{高一}
  % \Name{郑皓天}\FirstTime{20181207}\CurrentTime{20181207}
  % \Name{林叶}\FirstTime{20180908}\CurrentTime{20181125}
  %\Name{1v2}\FirstTime{20181028}\CurrentTime{20181117}
  % \Name{林叶}\FirstTime{20180908}\CurrentTime{20181125}
  % \Name{郭文镔}\FirstTime{20181111}\CurrentTime{20181117}
  % \Name{马灿威}\FirstTime{20181111}\CurrentTime{20181111}
  % \Name{黄亭燏}\FirstTime{20181231}\CurrentTime{20181231}
  % \Name{王睿妍}\FirstTime{20190129}\CurrentTime{}

  \newtheorem*{Theorem}{定理}
  \makefront
\vspace{-1.5em}
\startexercise
% \begin{exercise}{\heiti 课前检测}\\
%   表格实例:
%   \begin{center}
%     \renewcommand{\arraystretch}{1.4}
%     \begin{tabular}{|*{8}{c|}}
%       \hline
%         $x$
%         &$-\dfrac{\piup}6$
%         &$-\dfrac{\piup}3$
%         &$-\dfrac{5\piup}6$
%         &$-\dfrac{4\piup}3$
%         &$-\dfrac{11\piup}6$
%         &$-\dfrac{7\piup}3$
%         &$-\dfrac{17\piup}6$\\
%       \hline
%         $y$
%         &$-1$
%         &$1$
%         &$3$
%         &$1$
%         &$-1$
%         &$1$
%         &$3$\\
%       \hline
%     \end{tabular}\\
%   \end{center}
% \end{exercise}
\section{不等关系与不等式}
  \begin{description}
    \item [符号说明]
      \begin{itemize}[leftmargin=*]
        % \kaishu
        \item $\leqslant$ 表示“小于或等于”、“不大于”;$a\leqslant b\Rightarrow a<b$;
        \item $\geqslant$ 表示“大于或等于”、“不小于”;$a\geqslant b\Rightarrow a>b$;
        \item 连续不等式$a<b<c$的含义为$\begin{cases}a<b\\b<c\end{cases}$;
        \item 式子$a<b=c$的含义为$\begin{cases}a<b\\b=c\end{cases}$;
      \end{itemize}
  \end{description}
  \subsection{定义与性质}
  \subsection{比较大小与不等式证明}
  \begin{exercise}
    \item
  \end{exercise}
\section{一元二次不等式及其解法}
  \subsection{基础解法}
  \subsection{含参问题}
  \begin{description}
    \item (连续)函数$f(x)$,函数定义域$I$.
      \begin{itemize}[leftmargin=*]
        % \kaishu
        \item 若$f(x)>0$在区间$I$上恒成立 $\Leftrightarrow$ $f(x)$在区间$I$上恒大于0,则$f(x)$在$I$上的最小值大于0;
        \item 若$f(x)>0$在区间$I$上有解 $\Leftrightarrow$ $\exists x_0\in I$,使得$f(x_0)>0$,则$f(x)$在$I$上的最大值大于0;
        \item 若$f(x)>0$在区间$I$上无解 $\Leftrightarrow$ $f(x)\leqslant 0$在区间$I$上恒成立,则$f(x)$在$I$上的最大值不大于0.
      \end{itemize}
    \item  若(连续)函数$f(x)$由参变量$\lambda$确定,记为$f_\lambda(x)$.设参数取值范围$J$,函数定义域$I$.\\
      则$f_\lambda(x)$亦可视为以$x$为参变量的函数$g_x(\lambda)$,函数定义域$J$,参数取值范围$I$;\\
      亦可视为以“$x$”、“$\lambda$”为自变量的二元函数$H(x,\lambda)$.
      \begin{itemize}[leftmargin=*]
        % \kaishu
        \item 当$\lambda\in J$时,$f_\lambda(x)>0$在$I$上恒成立;
              $\Leftrightarrow$ 当$x\in I$,$\lambda\in J$时,$H(x,\lambda)>0$恒成立;
              $\Leftrightarrow$ 当$x\in I$时,$g_x(\lambda)>0$在$J$上恒成立.
          % \begin{itemize}
          %   \kaishu
          %   \item
          % \end{itemize}
      \end{itemize}
  \end{description}
  \begin{exercise}{\textbf{例题}}
    \item 设函数$f(x)=mx^2-mx-1$.
          (1) 若对于一切实数$x$,$f(x)<0$恒成立,求$m$的取值范围;
          (2) 对于$x\in[1,3]$,$f(x)<-m+5$恒成立,求$m$的取值范围.
          \begin{answer}
            (1) [解法一] 依题意,有$\begin{cases}m<0\\ \Delta<0\end{cases}$,解得$-4<m\leqslant 0$;\\
                [解法二] 令$h(m)=f(x)=(x^2-x)m-1$,则$h(m)$截距为$-1$,斜率$(x^2-x)\in\Bigl[-\dfrac14,+\infty\Bigr)$;
                         于是$y=h(m)$在横轴的截距取值范围:$[-\infty,-4]\cup(0,+\infty)$,从而$-4<m\leqslant 0$时,$h(m)<0$恒成立.
            (2) $m<\dfrac67$
                \begin{itemize}
                  \item 【解法一】 
                  \item 【解法二】令$h(m)=f(x)+m-5=(x^2-x+1)m-6$,则$h(m)$截距为$-6$,斜率$(x^2-x+1)\in[2,7]$;
                        故要使$h(m)<0$恒成立,必须$m<\dfrac67$
                \end{itemize}
          \end{answer}
  \end{exercise}


\section{基本不等式}
  \subsection{比较大小与不等式证明}
  \subsection{求最值}




\newpage
\section{课后作业}
  \begin{exercise}

  \end{exercise}
\stopexercise

\newpage
\section{参考答案}
\begin{multicols}{2}
  \printanswer
\end{multicols}

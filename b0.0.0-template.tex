\Topic{}
  \Teach{}
  \Grade{高一}
  % \Name{郑皓天}\FirstTime{20181207}\CurrentTime{20181207}
  % \Name{林叶}\FirstTime{20180908}\CurrentTime{20181125}
  %\Name{1v2}\FirstTime{20181028}\CurrentTime{20181117}
  % \Name{林叶}\FirstTime{20180908}\CurrentTime{20181125}
  % \Name{郭文镔}\FirstTime{20181111}\CurrentTime{20181117}
  % \Name{马灿威}\FirstTime{20181111}\CurrentTime{20181111}
  \newtheorem*{Theorem}{定理}
  \makefront
\vspace{-1.5em}
\startexercise
\begin{exercise}{\heiti 课前检测}\\
  表格实例:
  \begin{center}
    \renewcommand{\arraystretch}{1.4}
    \begin{tabular}{|*{8}{c|}}
      \hline
        $x$
        &$-\dfrac{\piup}6$
        &$-\dfrac{\piup}3$
        &$-\dfrac{5\piup}6$
        &$-\dfrac{4\piup}3$
        &$-\dfrac{11\piup}6$
        &$-\dfrac{7\piup}3$
        &$-\dfrac{17\piup}6$\\
      \hline
        $y$
        &$-1$
        &$1$
        &$3$
        &$1$
        &$-1$
        &$1$
        &$3$\\
      \hline
    \end{tabular}\\
  \end{center}
\end{exercise}
\section{第一章节}
  \begin{description}
    \item [label]
  \end{description}
  \begin{exercise}
    \item%LaTeX-master/sanjiaohanshu/sanjiaohanshu-gaokao.tex 4
      函数$f(x)=\cos(\omega x+\varphi)$的部分图象如图所示,则$f(x)$的单调递减区间为\xz
      \begin{minipage}[b]{0.8\linewidth}
        \vspace{2.5em}
        \xx{$\Bigl(k\piup-\dfrac{1}{4},k\piup+\dfrac{3}{4}\Bigr),k\in\mathbb{Z}$}
          {$ \Bigl(2k\piup-\dfrac{1}{4},2k\piup+\dfrac{3}{4}\Bigr),k\in\mathbb{Z}$}
          {$ \Bigl(k-\dfrac{1}{4},k+\dfrac{3}{4}\Bigr),k\in\mathbb{Z}$}
          {$\Bigl(2k-\dfrac{1}{4},2k+\dfrac{3}{4}\Bigr),k\in\mathbb{Z} $}
      \end{minipage}\hfill
      \begin{minipage}[h]{0.2\linewidth}
        \vspace{-3cm}
        \begin{tikzpicture}
          \node[below left](O) at(0,0) {\small$\bm{O}$};
          \draw(0,1)node[right]{\tiny$1$}--(0.1,1);
          \clip(-1.2,-1.2) rectangle (2,1.5);
          \draw[->,>=stealth](-1.2,0)--(2,0) node[below left] (x){$x$};
          \draw[->,>=stealth](0,-1.2)--(0,1.5) node[below right] (y){$y$};
          \draw[domain=-1.2:2,samples=1000] plot(\x,{cos((pi*(\x)+1/4*pi) r)});
          \node[below] (A)at (0.25,0){$\frac{1}{4}$};
          \node[below] (B)at (1.25,0){$\frac{5}{4}$};
        \end{tikzpicture}
      \end{minipage}
      \begin{answer}
        D
      \end{answer}
  \end{exercise}

\newpage
\section{课后作业}
  \begin{exercise}

  \end{exercise}
\stopexercise

\newpage
\section{参考答案}
\begin{multicols}{2}
  \printanswer
\end{multicols}

%福州重点中学期中考真题分类汇编 2函数的相关性质.pdf P9
\item (福建省师大附中 2015-2016 高一上学期期中考试22)已知函数$f(x)=-1+\log_a{x+2}$($a>0$,且 $a \neq1$),$g(x)=(\frac12)^{x-1}$.\\
(1)函数$ y= f (x )$ 的图象恒过定点 $A$,求 $A$ 点坐标;\\
(2)若函数 $F ( x )= f ( x )- g ( x )$ 的图像过点$(2,\frac12)$, 证明:方程 $F ( x )= 0$ 在 $x\in(1,2)$上有唯一解.
\begin{answer}
(1)$(-1,-1)$;\\
\end{answer}

%福州重点中学期中考真题分类汇编 2函数的相关性质.pdf P10
\item (福建省师大附中 2015-2016 高一上学期期中考试23) 已知函数 $f ( x ) =\log_a ( x+ 1), g ( x )= 2 \log_a ( 2 x+ t )(t\in \mathbb{R})$, $a> 0$, 且$a\neq 1$.\\
(\Rmnum{1})若 1 是关于 $x$ 的方程 $f ( x) -g ( x) =0$ 的一个解,求 $t$ 的值;\\
(\Rmnum{2})当 $0< a< 1$且$t=-1$ 时,解不等式 $f ( x)\leq g ( x) $;\\
(\Rmnum{3})若函数 $F ( x)= a^{f ( x ) }+ tx^2- 2t+ 1 $在区间 $(-1,2]$上有零点,求 $t$ 的取值范围.
\begin{answer}
(\Rmnum{1})$t=\sqrt{2}-2$;
(\Rmnum{2})$x\in(\frac12,\frac54]$;
(\Rmnum{3})$t\in(-\infty,2]\cup [\frac{2+\sqrt{2}}{4},+\infty)$.
\end{answer}

\item
%福州重点中学期中考真题分类汇编 2函数的相关性质.pdf P11
(福州八中 2015—2016 高一上学期期中考试23)设 $f (x )$ 是定义在 $\mathbb{R}$ 上的奇函数,且对任意 $a,b\in \mathbb{R}$ ,当$a+b\neq0$时,都有 $\frac{f(a)+f(b)}{a+b}>0$\\
(1)若 $a> b$ ,试比较 $f (a ) $与 $f (b)$ 的大小关系;\\
(2)若 $f (9^x- 2\cdot 3^x )+ f ( 2\cdot 9^x-k )> 0 $对任意 $x\in[0,\infty )$ 恒成立,求实数 $k$ 的取值范围.
\begin{answer}
(1)$f(a)>f(b)$;\\
(2)$k<1$.\\
\end{answer}

%福州重点中学期中考真题分类汇编 2函数的相关性质.pdf P12
(福州八中 2015—2016 高一上学期期中考试24)已知函数 $y=x+\frac tx$ 有如下性质:如果常数 $t>0$,那么该函数在$(0,\sqrt t]$上是减函数, 在$[\sqrt t, +\infty)$上是增函数.\\
(1)已知 $f(x)=\frac{4x^2-12x-3}{2x+1} $,$x\in[0,1]$,利用上述性质,求函数 $f(x)$的单调区间和值域;\\
(2)对于(1)中的函数 $f(x)$和函数$g(x)=-x-2a$,若对任意 $x_1 \in[0,1]$,总存在 $x_2\in[0,1]$,使得 $g(x_2 )=f(x_1 ) $成立,求实数 $a$ 的值.


%福州重点中学期中考真题分类汇编 2函数的相关性质.pdf P14
(福州市第三中学 2016-2017 高一上期中考试23)设函数 $f (x)=a^x+ bx +c$ ( $a> 0$ , $b, c\in \mathbb{R})$ . \\
⑴若 $f (1)= c$ , $f (x)$在$( k,+\infty)$单调递增,求实数 $k$ 的取值范围;\\
 ⑵若 $f( 1)=-\frac a2$,求证:函数 $f (x) $在$( 0,2) $内至少有一个零点.


%福州重点中学期中考真题分类汇编 2函数的相关性质.pdf P16
(福建师大附属中学 2016-2017 高一年级期中考试22)已知二次函数 $f ( x )= ax^2+ bx+ c$ 的图像过点 $(-2,0)$ ,且不等式 $2 x\leq f ( x )\leq \frac12x^2+ 2$ 对一切实数 $x$ 都成立.\\
(\Rmnum{1})求函数 $f ( x ) $的解析式. \\
(\Rmnum{2}成立,求实数 $t$ 的取值范围.


%福州重点中学期中考真题分类汇编 2函数的相关性质.pdf P16
(福州市高级中学 2016-2017 高一上期中19)已知 $f(x)$是定义在$\mathbb{R}$上的奇函数,当$x\geq0$时,$f(x)=a^x-2$,其中 $a> 0$ 且 $a\neq 1$\\
(I)求 $f (x )$ 的解析式;\\
(II)解关于$x$ 的不等式 $-1< f(x)<4$ ,结果用集合或区间表示.


%福州重点中学期中考真题分类汇编 2函数的相关性质.pdf P17
(福州市高级中学 2016-2017 高一上期中21)记函数 $f (x )=a-\log_2{x}(1\leq x\leq 4)$,函数$y=[f(x)]^2-f(\frac x2)$,记函数$f(x)$ 的最小值为 $g( a)$.\\
(I)求 $g( a) $的表达式;\\
(II)作出函数$y=|g(a)|$的图像,并根据图像回答:当 $k$ 为何实数时,方程$|g( a)|-k=0$ 有两个解、有四个解、有无穷多个解?

%福州重点中学期中考真题分类汇编 2函数的相关性质.pdf P19
(福州市高级中学 2016-2017 高一上期中22)已知函数$f(x)=x^2-2ax+5(a>1)$\\
(I)若 $f (x )$ 的定义域和值域均是 $[1, a]$ ,求实数 $a$ 的值; \\
(II)若 $f (x ) $在区间 $[4,+\infty)$上是增函数,且对任意的$ x \in[1, a+ 2]$,都有 $f( x )\leq 0$ ,求实数$a$的取值范围;\\
(III)若 $g( x )=2^x+\log_2{x+ 1 }$ ,且对任意的 $x \in[0,1]$ ,都存在$f(x_0)=g(x)$ 成立,求实数$a$的取值范围.


%福州重点中学期中考真题分类汇编 2函数的相关性质.pdf P21
(福州市格致中学 2016-2017 高一上期中考试数学学科试卷22)已知二次函数 $f ( x )= ax^2+ bx+3$ 是偶函数,且 过点$(-1,4)$,$ g ( x )= x + 4$ .\\
(Ⅰ)求 $f (x) $的解析式;\\
()求函数 $F ( x )= f (2^x )+ g (2^{x+1} )$ 的值域; \\
(Ⅲ)若 $f ( x ) \geq g ( mx +m )$ 对 $x\in [2, 6] $恒成立,求实数 $m$ 的取值范围.

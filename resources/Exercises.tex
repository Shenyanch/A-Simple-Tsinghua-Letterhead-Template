\item
%30次课学完高中数学P3.7拓3
函数$y=\frac{9-x^2}{|x+4|+|x-3|}$ 的图像关于\xz
\xx{$x$轴对称}{$y$轴对称}{原点对称}{直线$x-y=0 $对称}
\begin{answer}
B
\end{answer}

\item
%30次课学完高中数学P3.7例6(4)
(2009四川卷文理12) 已知函数$f(x)$是定义在实数集$\mathbb{R}$上的不恒为零的偶函数,且对任意实数$x$都有$xf(x+1)=(1+x)f(x) $,则$f(\frac{5}{2}) $的值是\xz
\xx{0}
{$\frac12$}
{1}
{$\frac52$}
\begin{answer}
A
\end{answer}

\item
%30次课学完高中数学P4.11例5
已知函数$f(x-2)=ax^2-(a-3)x+a-2$($a$为负整数)的图像经过点 $(m-2,0) $,$m\in \mathbb{R}$,设$g(x)=f[f(x)] $,$F(x)=pg(x)+f(x) $.问是否存在实数$p$($p<0$)使得$F(x)$在区间$(-\infty,f(2)) $上是减函数,且区间$(f(2),0) $上是增函数?若有,求出相应的$p$,若无,说明理由.\\
\begin{answer}
$a=-1$;$p=-\frac{1}{16}$
\end{answer}

\item
%30次课学完高中数学P4.8
已知幂函数$y=x^{m^2-2m-3} $ ($m\in \mathbb{N}^* $)的图像关于$y$轴对称,且在$(0,+\infty) $上是减函数,求满足$(a+1)^{-m}<(3-2a)^{-m} $的$a$的范围
\begin{answer}
($m=1$)$a\in (-\infty,-1)\cup(\frac23<a<\frac32)$
\end{answer}

\item
%30次课学完高中数学P3.7(2)
已知函数$f(x)$是定义在$\mathbb{R}$上的奇函数,$g(x)$是定义在$\mathbb{R} $的偶函数,且$f(x)-g(x)=1-x^2-x^3 $,则$g(x) $的解析式为\xz
\xx{$1-x^2$}{$2-2x^2$}{$x^2-1 $}{$2x^2-2 $}
\begin{answer}
C
\end{answer}

\item
%30次课学完高中数学P6.11例7
已知函数$f(x)=\lg(ax^2+2x+1) $.\\
(1) 若$f(x)$的定义域为$\mathbb{R}$,求实数$a$的范围;
(2) 若$f(x)$的值域为$\mathbb{R}$,求实数$a$的范围.
\begin{answer}
(1) $a\in(1,+\infty) ;
(2) $a\in[0,1] $
\end{answer}

\item
(17-18八中高一期中19)已知$f(x)$是奇函数并且是$\mathbb{R}$上的单调函数,若函数$y=f(2x^2+1)+f(\lambda-x) $只有一个零点,则实数$\lambda $的值是\xz
\xx{$\frac14 $}{$\frac18 $}{$-\frac78 $}{$-\frac38 $}
\begin{answer}
C
\end{answer}

\item
(17-18三中高一上期中考12)已知$f(x)=\begin{cases}
 |\log_3x|,0<x\leq3\\13-4x,x>3   
\end{cases} $ ,若正数$a$ ,$b$ , $c$互不相等,且$f(a)=f(b)=f(c)$ ,则$abc$ 的取值范围是\xz
\xx{$(3,13)$}{$(\frac14,13) $}{$(1,\frac{13}{4}) $}{$(3,\frac{13}{4})$}
\begin{answer}
D
\end{answer}

\item
(17-18三中高一上期中考10)已知函数$f(x)$ 为$\mathbb{R}$ 上的奇函数,$f(x)$ 在$(0,+\infty)$ 内单调递减,且存在$x_0<0$ 使得 $f(x)=0$,则函数$f(x)$ 的零点个数为\xz\\
\xx{1个}{2个}{3个}{4个}
\begin{answer}
C
\end{answer}
\item
(17-18八中高一期中20)若函数$f(x)=|2^x-1|-b $有两个零点,则实数$b$的取值范围是\tk.\\


%福州重点中学期中考真题分类汇编 2函数的相关性质.pdf P5.7
\item(福建师大附中 16-17 高一期中考,7)已知定义在 $\mathbb{R}$上的函数$f(x)$在 $(-\infty,2)$内为减函数,且$f(x+2) $为偶函数,则$f(-1)$,$f(4)$,$f(\frac{11}2 )$的大小为\xz \\
\xx{$f(4)<f(-1)<f(\frac{11}2 )$}
	   {$f(-1)<f(4)<f(\frac{11}2 )$}
       {$f(\frac{11}2 )<f(4)<f(-1)$}
       {$f(-1)<f(\frac{11}2 )<f(4)$}
\begin{answer}A\end{answer}\\
     
\item
%福州重点中学期中考真题分类汇编 2函数的相关性质.pdf P6.9\\
(福州高级中学 16-17 高一期中考,13)已知函数$f(x)=\sqrt{x^2+ax+a} $ 在区间$[1,+\infty] $上单调递增,则实数$a$的取值范围是\xz\\
\xx{$[-2,-\frac12]$}
		{$[-\frac12,+\infty] $}
        {$[-\frac12,2] $}
        {$[-2,+\infty] $}
\begin{answer}B\end{answer}

\item
(17-18三中高一上期中考16)已知函数$f(x)=(\log_427)^2+2a\log_2x+1 $ 有大于 1的零点,则实数$a$ 的取值范围是\tk.
\begin{answer}
$(-\infty,-1]$
\end{answer}


\item
%福州重点中学期中考真题分类汇编 2函数的相关性质.pdf P8.9\\
(福建师大附中 16-17 高一期中考,16)函数$f(x)$ 是$(0,+\infty)$ 上的单调增函数,当 $n\in \mathbb{N}^*$ 时,$f(n)\in\mathbb{N}^*$ ,且 $f[f(n) ] =3n$,则$f(1)$ 的值为\tk.\\
\begin{answer}2\end{answer}


\item
%福州重点中学期中考真题分类汇编 2函数的相关性质.pdf P6.13\\
(福州格致中学 16-17 高一期中考,10)若$f(x)=-x^2+2ax $
 与$g(x)=\frac a{x+1} $在区间$[1,2] $上都是减函数,则实数$a $ 的取值范围\xz\\
 \xx{$(-1,0)\cup (0,1) $ }
 		{$(-1,0)\cup (0,1]$}
        {$(0,1) $}
        {(0,1]}
 \begin{answer}      D\end{answer}

\item
%福州重点中学期中考真题分类汇编 2函数的相关性质.pdf P7.7\\
(福建师大附中 16-17 高一期中考,14)已知函数 $f(x)$ 的反函数是 $y=\frac{1}{3^x}$,则函数 $f(2x-x^2) $的减区间为\tk

\item
%福州重点中学期中考真题分类汇编 2函数的相关性质.pdf P8.12\\
(福州格致中学 16-17 高一期中考,15)若函数$f(x)=\begin{cases}(a-2)x,x\leq2\\(\frac{1}{2})^x-1,x<2\end{cases}$是$\mathbb{R}$上的单调递减函数,则实数$a$的取值范围是\tk.
\begin{answer}
$a\leq\frac{13}{8}$
\end{answer}

\item
%福州三中高一上数学七中卷(2017-2018).doc
(17-18三中高一上期中考17)(本小题满分分)已知全集 $U=\mathbb{R}$,集合$A=\{x|(x-3)(x+2)\leq0\} $,$B=\{x|2a\leq x\leq a+2,a\in \mathbb{R} \} $.\\
(I)若$a=-2 $,求集合$(\complement_UA)\cup B $;
(II)若$B\subseteq A $,求实数$a $的取值范围.
\begin{answer}
 (I) $\{x|-4\leq x<-2 \} $
 (II) $[-1,1]\cup[2,+\infty] $
\end{answer}

%福\cup [2,+\infty][-1\end{answer},1]州三中高一上数学七中卷(2017-2018).doc
\item
(17-18三中高一上期中考19)(本小题满分 分)\\
已知函数$f(x)=\sqrt{ax+4}$ $(a\in\mathbb{R},a\neq0)$  .\\
(I)若$a=-1$ ,求函数$f(x)$ 的定义域和值域;\\  
(II)若$f(x)$ 在区间$[-1,2] $ 上为单调函数,求实数$a$ 的最大值和最小值.\\
\begin{answer}
(I)定义域$(-\infty,4]$,值域$[0,+\infty)$;
(II)$a_{\min}=-2,a_{\max}=4 $.
\end{answer}





\item
(17-18八中高一期中22)(本小题共12分)
已知函数$f(x)=x^2+4ax+2a+6 $.\\
(I)若函数$y=\log_2f(x)$的最小值为2,求$a$的值;\\
(II)若对任意$x\in\mathbb{R}$,都有$f(x)\geq0$成立,求函数$g(a)=2-a|a+3| $的值域.\\

\item
%福州重点中学期中考真题分类汇编 4函数方程及函数模型的应用.pdf P6
(16-17 附中21) 为了研究某种药物,用小白鼠进行试验,发现药物在血液内的浓度与时间的关系因使用方式的不同而不同。若使用注射方式给药,则在注射后的 3 小时内,药物在白鼠血液内的浓度$y_1$与时间$t$ 满足关系式:$y_1 =4-at$ ($0<a<\frac{4}{3})$,$a$为常数),若使用口服方式给药,则药物在白鼠血液内的浓度 $y_2$与时间 $t$满足关系 式:$y_2=\begin{cases}\sqrt{t},0<t<1,\\3-\frac{2}{t},1\leq t\leq3.\end{cases}$现对小白鼠同时进行注射和口服该种药物,且注射药物和口服药物的吸收与代谢互不干扰.\\
(\Rmnum{1} )若$a=1$,求 3 小时内,该小白鼠何时血液中药物的浓度最高,并求出最大值;\\
(\Rmnum{2})若使小白鼠在用药后 3 小时内血液中的药物浓度不低于 4,求正数 $a$ 的取值范围.\\
\begin{answer}
(1)当$t=\frac12$时,$y_{\max}=\frac{17}{4}$;\\
(2)$0<a\leq\frac{7}{9}$
\end{answer}


\item
(17-18八中高一期中23)(本小题共15分)
已知二次函数$f(x)$满足$f(x+1)-f(x)=2x $ $(x\in\mathbb{R})$,且$f(0)=1$.\\
(I)求$f(x)$的解析式;\\
(II)若函数$g(x)=f(x)-2tx$在区间$[-1,5]$上是单调函数,求实数$t$的取值范围;\\
(III)若关$x$的方程$f(x)=x+m $在区间$(-1,2)$上有唯一实数根,求实数$m$的取值范围.(注:相等的实数根算一个).\\

%福州重点中学期中考真题分类汇编 2函数的相关性质.pdf P9
\item (福建省师大附中 2015-2016 高一上学期期中考试22)已知函数$f(x)=-1+\log_a{x+2}$($a>0$,且 $a \neq1$),$g(x)=(\frac12)^{x-1}$.\\
(1)函数$ y= f (x )$ 的图象恒过定点 $A$,求 $A$ 点坐标;\\
(2)若函数 $F ( x )= f ( x )- g ( x )$ 的图像过点$(2,\frac12)$, 证明:方程 $F ( x )= 0$ 在 $x\in(1,2)$上有唯一解.\\
\begin{answer}
(1)$(-1,-1)$;\\
\end{answer}

%福州重点中学期中考真题分类汇编 2函数的相关性质.pdf P10
\item (福建省师大附中 2015-2016 高一上学期期中考试23) 已知函数 $f ( x ) =\log_a ( x+ 1), g ( x )= 2 \log_a ( 2 x+ t )(t\in \mathbb{R})$, $a> 0$, 且$a\neq 1$.\\
(\Rmnum{1})若 1 是关于 $x$ 的方程 $f ( x) -g ( x) =0$ 的一个解,求 $t$ 的值;\\
(\Rmnum{2})当 $0< a< 1$且$t=-1$ 时,解不等式 $f ( x)\leq g ( x) $;\\
(\Rmnum{3})若函数 $F ( x)= a^{f ( x ) }+ tx^2- 2t+ 1 $在区间 $(-1,2]$上有零点,求 $t$ 的取值范围.\\
\begin{answer}
(\Rmnum{1})$t=\sqrt{2}-2$;
(\Rmnum{2})$x\in(\frac12,\frac54]$;
(\Rmnum{3})$t\in(-\infty,2]\cup [\frac{2+\sqrt{2}}{4},+\infty)$.
\end{answer}

\item
%福州重点中学期中考真题分类汇编 2函数的相关性质.pdf P11
(福州八中 2015—2016 高一上学期期中考试23)设 $f (x )$ 是定义在 $\mathbb{R}$ 上的奇函数,且对任意 $a,b\in \mathbb{R}$ ,当$a+b\neq0$时,都有 $\frac{f(a)+f(b)}{a+b}>0$\\
(1)若 $a> b$ ,试比较 $f (a ) $与 $f (b)$ 的大小关系;\\
(2)若 $f (9^x- 2\cdot 3^x )+ f ( 2\cdot 9^x-k )> 0 $对任意 $x\in[0,\infty )$ 恒成立,求实数 $k$ 的取值范围.\\
\begin{answer}
(1)$f(a)>f(b)$;\\
(2)$k<1$.\\
\end{answer}

\item
%福州重点中学期中考真题分类汇编 2函数的相关性质.pdf P12
(福州八中 2015—2016 高一上学期期中考试24)已知函数 $y=x+\frac tx$ 有如下性质:如果常数 $t>0$,那么该函数在$(0,\sqrt t]$上是减函数, 在$[\sqrt t, +\infty)$上是增函数.\\
(1)已知 $f(x)=\frac{4x^2-12x-3}{2x+1} $,$x\in[0,1]$,利用上述性质,求函数 $f(x)$的单调区间和值域;\\
(2)对于(1)中的函数 $f(x)$和函数$g(x)=-x-2a$,若对任意 $x_1 \in[0,1]$,总存在 $x_2\in[0,1]$,使得 $g(x_2 )=f(x_1 ) $成立,求实数 $a$ 的值.\\
\begin{answer}
(1) 减区间为 $[0,\frac12] $,增区间为$[\frac12,1] $; 值域为$[-4,-3] $.
(2) $a=\frac32 $
\end{answer}

\item
%福州重点中学期中考真题分类汇编 2函数的相关性质.pdf P14
(福州市第三中学 2016-2017 高一上期中考试23)设函数 $f (x)=a^x+ bx +c$ ( $a> 0$ , $b, c\in \mathbb{R})$ . \\
⑴若 $f (1)= c$ , $f (x)$在$( k,+\infty)$单调递增,求实数 $k$ 的取值范围;\\
 ⑵若 $f( 1)=-\frac a2$,求证:函数 $f (x) $在$( 0,2) $内至少有一个零点.\\
 \begin{answer}
(1) $k\in [\frac12,+\infty) $
\end{answer} 

\item
%福州重点中学期中考真题分类汇编 2函数的相关性质.pdf P16
(福建师大附属中学 2016-2017 高一年级期中考试22)已知二次函数 $f ( x )= ax^2+ bx+ c$ 的图像过点 $(-2,0)$ ,且不等式 $2 x\leq f ( x )\leq \frac12x^2+ 2$ 对一切实数 $x$ 都成立.\\
(\Rmnum{1})求函数 $f ( x ) $的解析式. \\
(\Rmnum{2})对一切 $x\in[-1,1] $ ,不等式$f(x+t)<f(\frac x2) $恒成立,求实数 $t$ 的取值范围.\\
\begin{answer}
(I) $f(x)=\frac{x^2}4+x+1 $; (II) $t\in (-\frac52,-\frac12) $
\end{answer}

\item
%福州重点中学期中考真题分类汇编 2函数的相关性质.pdf P16
(福州市高级中学 2016-2017 高一上期中19)已知 $f(x)$是定义在$\mathbb{R}$上的奇函数,当$x\geq0$时,$f(x)=a^x-2$,其中 $a> 0$ 且 $a\neq 1$\\
(I)求 $f (x )$ 的解析式;\\
(II)解关于$x$ 的不等式 $-1< f(x)<4$ ,结果用集合或区间表示.\\

\item
%福州重点中学期中考真题分类汇编 2函数的相关性质.pdf P17
(福州市高级中学 2016-2017 高一上期中21)记函数 $f (x )=a-\log_2{x}(1\leq x\leq 4)$,函数$y=[f(x)]^2-f(\frac x2)$,记函数$f(x)$ 的最小值为 $g( a)$.\\
(I)求 $g( a) $的表达式;\\
(II)作出函数$y=|g(a)|$的图像,并根据图像回答:当 $k$ 为何实数时,方程$|g( a)|-k=0$ 有两个解、有四个解、有无穷多个解?\\
\begin{answer}
(I)$g(a)=\begin{cases}-\frac54,a\leq\frac52,\\a^2-5a+5,a>\frac52\end{cases} $
(II) $k=0$或 $k>\frac54 $: 一个解;  $0<k<\frac54$: 两个解;   $k=\frac54$: 无数解
\end{answer}

\item
%福州重点中学期中考真题分类汇编 2函数的相关性质.pdf P19
(福州市高级中学 2016-2017 高一上期中22)已知函数$f(x)=x^2-2ax+5(a>1)$\\
(I)若 $f (x )$ 的定义域和值域均是 $[1, a]$ ,求实数 $a$ 的值; \\
(II)若 $f (x ) $在区间 $[4,+\infty)$上是增函数,且对任意的$ x \in[1, a+ 2]$,都有 $f( x )\leq 0$ ,求实数$a$的取值范围;\\
(III)若 $g( x )=2^x+\log_2{x+ 1 }$ ,且对任意的 $x \in[0,1]$ ,都存在$f(x_0)=g(x)$ 成立,求实数$a$的取值范围.\\
\begin{answer}
(I) $a=2$; (II) $a\geq3$; (III) $a\geq\frac52$
\end{answer}

\item
%福州重点中学期中考真题分类汇编 2函数的相关性质.pdf P21
(福州市格致中学 2016-2017 高一上期中考试数学学科试卷22)已知二次函数 $f ( x )= ax^2+ bx+3$ 是偶函数,且 过点$(-1,4)$,$ g ( x )= x + 4$ .\\
(\Rmnum{1} )求 $f (x) $的解析式;\\
(\Rmnum{2} )求函数 $F ( x )= f (2^x )+ g (2^{x+1} )$ 的值域; \\
(\Rmnum{3} )若 $f ( x ) \geq g ( mx +m )$ 对 $x\in [2, 6] $恒成立,求实数 $m$ 的取值范围.\\
\begin{answer}
(I) $f(x)=ax^2+3$; (II) $(7,+\infty)$; (III) $m\leq1$
\end{answer}

\item
%福州重点中学期中考真题分类汇编 2函数的相关性质.pdf P23
(福州市屏东中学 2016-2017 高一上期中22)已知函数$f(x)=2^x-2^{-2} $,定义域为$\mathbb{R} $;函数 $g(x)=2^{x+1}-2^{2x} $,定义域为$[-1,1] $.\\
(1)判断函数$f(x) $的奇偶性,不用证明;\\
(2)求函数$g(x) $ 的最值;\\
 (3) 若不等式$f(g(x))\leq f(-3am+m^2+1) $对$x\in[-1,1] $,$a\in[-2,2] $ 上恒成立,求 $m$ 的取值范围.\\
 \begin{answer}
 (1) 增函数; (2) $g_{\max}(t)=g(1)=1 $; $g_{\min}(t)=g(2)=0 $; (3) $m\in (-\infty,-6)\cup[6,+\infty)\cup\{0\} $
\end{answer}
















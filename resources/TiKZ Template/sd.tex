% https://blog.csdn.net/robert_chen1988/article/details/73065719
% latex 绘图功能比预计地还要强大,甚至可以画地图,画圣诞树。外导给了一个网站链接,上面有各种 latex 画图的例子:
% 
% http://www.texample.net/tikz/examples/feature/remember-picture/
% 
% 用 latex 画图 还是挺麻烦的,它用一堆代码表示图形,并不像 excel 和 visio 那么形象。但是,对于写论文来说,它画图比较美观简洁。常用的画图宏包为 pdfplot,它的 manual 详细介绍了它的用法,下载地址:
% 
% http://www.bakoma-tex.com/doc/latex/pgfplots/pgfplots.pdf
% 
% 
% 
% manual 非常详细,非常全。下面是我画的图:我将一些数据放进 csv 文件里,然后从这个文件中提取横坐标,纵坐标画图。把数据放进 csv 文件用到了 filecontents 宏包。
% \usepackage{pgfplots}
% \usepackage{filecontents}
% --------------------- 
% 作者:心态与做事习惯决定人生高度 
% 来源:CSDN 
% 原文:https://blog.csdn.net/robert_chen1988/article/details/73065719 
% 版权声明:本文为博主原创文章,转载请附上博文链接!
\begin{filecontents*}{mydata.csv}a,    sQS-std-in, sS-std-in,  RS-std-in, RQ-std-in, sQS-rm-in, sS-rm-in,  RS-rm-in, RQ-rm-in,sQS-std-out, sS-std-out,  RS-std-out, RQ-std-out, sQS-rm-out, sS-rm-out,  RS-rm-out, RQ-rm-out100, 0.93, 1.19, 1.66, 2.40, 3.24, 2.60, 6.22, 11.65, 1.64, 1.29, 1.45, 1.34,   3.24,  2.99, 5.48, 11.21 500, 0.70, 0.87, 1.21, 1.49, 3.94, 2.77, 6.26, 11.74, 1.40, 1.06, 1.17, 1.15, 3.39, 2.80, 5.41, 11.14 1000, 0.56, 0.64, 1.21, 1.44, 3.92, 2.70, 6.20, 11.59, 0.65, 0.75, 1.19, 1.35, 2.53, 2.65, 5.36, 10.97 1500,0.55, 0.65, 1.32, 1.52, 4.09, 2.84, 6.24, 11.78, 0.52, 0.72, 1.31, 1.35, 2.35, 2.63, 5.37, 11.21\end{filecontents*} \begin{figure}[!ht]\centering\subfigure[Mean STD for different number of scenarios.]{\begin{tikzpicture}\pgfplotsset{every axis legend/.append style={at={(0.5,1.03)},anchor=south},every axis y label/.append style={at={(0.07,0.5)}}}\begin{axis}[title=(a) Mean STD for ,xlabel=Num of scenarios,    ylabel=Mean STD,xtick =data,legend columns=4,legend style={font=\tiny},font=\footnotesize,width=8cm]\addplot table [x=a, y=sQS-std-in,, col sep=comma] {mydata.csv};\addplot table [x=a, y=sS-std-in, col sep=comma] {mydata.csv};\addplot table [x=a, y=RS-std-in, col sep=comma] {mydata.csv};\addplot table [x=a, y=RQ-std-in, col sep=comma] {mydata.csv};\addplot table [x=a, y=sQS-std-out, col sep=comma] {mydata.csv};\addplot table [x=a, y=sS-std-out, col sep=comma] {mydata.csv};\addplot table [x=a, y=RS-std-out, col sep=comma] {mydata.csv};\addplot table [x=a, y=RQ-std-out, col sep=comma] {mydata.csv};\legend{s$\overline{Q}$S-in, sS-in, RS-in, RQ-in, s$\overline{Q}$S-out, sS-out, RS-out, RQ-out}\end{axis}\end{tikzpicture}}~~~~\subfigure[Mean RMSE for different number of scenarios.]{\begin{tikzpicture}\pgfplotsset{every axis legend/.append style={at={(0.5,1.03)},anchor=south},every axis y label/.append style={at={(0.07,0.5)}}}\begin{axis}[xlabel=Num of scenarios,    ylabel=Mean RMSE,xtick =data,legend columns=4,legend style={font=\tiny},font=\footnotesize,width=8cm]\addplot table [x=a, y=sQS-rm-in,, col sep=comma] {mydata.csv};\addplot table [x=a, y=sS-rm-in, col sep=comma] {mydata.csv};\addplot table [x=a, y=RS-rm-in, col sep=comma] {mydata.csv};\addplot table [x=a, y=RQ-rm-in, col sep=comma] {mydata.csv};\addplot table [x=a, y=sQS-rm-out, col sep=comma] {mydata.csv};\addplot table [x=a, y=sS-rm-out, col sep=comma] {mydata.csv};\addplot table [x=a, y=RS-rm-out, col sep=comma] {mydata.csv};\addplot table [x=a, y=RQ-rm-out, col sep=comma] {mydata.csv};\legend{s$\overline{Q}$S-in, sS-in, RS-in, RQ-in, s$\overline{Q}$S-out, sS-out, RS-out, RQ-out}\end{axis}\end{tikzpicture}}\caption{Stability test results for different number of scenarios.}\label{fig:InsampleOutsample}\end{figure}
--------------------- 
作者:心态与做事习惯决定人生高度 
来源:CSDN 
原文:https://blog.csdn.net/robert_chen1988/article/details/73065719 
版权声明:本文为博主原创文章,转载请附上博文链接!
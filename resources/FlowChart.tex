% https://www.jianshu.com/p/2d01d5eaaa77
\documentclass[UTF8]{ctexart}
\usepackage{tikz}
\usetikzlibrary{shapes,arrows}

\begin{document}
\pagestyle{empty} % 无页眉页脚

\tikzstyle{startstop} = [rectangle,rounded corners, minimum width=3cm,minimum height=1cm,text centered, draw=black,fill=red!30]
\tikzstyle{io} = [trapezium, trapezium left angle = 70,trapezium right angle=110,minimum width=3cm,minimum height=1cm,text centered,draw=black,fill=blue!30]
\tikzstyle{process} = [rectangle,minimum width=3cm,minimum height=1cm,text centered,text width =3cm,draw=black,fill=orange!30]
\tikzstyle{decision} = [diamond,minimum width=3cm,minimum height=1cm,text centered,draw=black,fill=green!30]
\tikzstyle{arrow} = [thick,->,>=stealth]
% ====================================================================
% 定义node和箭头的属性
%================================================================================
% # 节点形状
% rectangle:矩形,可加圆角(rounded corners)
% trapezium:平行四边形
% diamond:菱形
% # 尺寸
% minimum width
% minimum height
% # 文本
% text centered:文本居中
% # 文本宽度
% text width=3cm:文本超过3cm时会自动换行
% # 边框
% draw
% # 填充颜色
% fill
% # 线粗:
% thick:粗
% thin:细
% # 箭头
% ->:反向箭头
% <-:正向箭头
% <->:双向箭头
% # 虚线
% dashed
% # 箭头形状
% >=stealth

\begin{tikzpicture}[node distance=2cm]
\node (start) [startstop] {Start};
\node (input1) [io,below of=start] {Input};
\node (process1) [process,below of=input1] {Process 1};
\node (decision1) [decision,below of=process1,yshift=-0.5cm] {Decession 1};
\node (process2a) [process,below of=decision1,yshift=-0.5cm] {Process 2aaaaaa aaaaaaa aaaa};
\node (process2b) [process,right of =decision1,xshift=2cm] {Process 2b};
\node (out1) [io,below of=process2a] {Output};
\node (stop) [startstop,below of=out1] {Stop};
% ======================================================================
% 创建节点
% \node (decision1) [decision,below of=process1,yshift=-0.5cm] {Decession 1};
% =====================================================================
% # name
% (decision1):这个节点的name,后面需要用这个name调用这个节点。
% # 属性
% decision:需要调用的节点的属性
% # 位置
% below of=process1:定义节点的位置
% left of:
% right of:
% # 偏移,对位置进行微调
% yshift:
% xshift:
% # title
% {Decession 1}:结果显示的标题
\draw [arrow] (start) -- (input1);
\draw [arrow] (input1) -- (process1);
\draw [arrow] (process1) -- (decision1);
\draw [arrow] (decision1) -- node[anchor=east] {yes} (process2a);
\draw [arrow] (decision1) -- node[anchor=south] {no} (process2b);
\draw [arrow] (process2b) |- (process1);
\draw [arrow] (process2a) -- (out1);
\draw [arrow] (out1) -- (stop);
% ===================================================================
% 画箭头
% \draw [arrow] (decision1) -- node[anchor=east] {yes} (process2a);
% ====================================================================
% # 属性
% [arrow]:需要调用的箭头的属性
% (decision1):箭头的其实位置
% (process2a):箭头的末端位置
% # 线型
% --:直线
% |-:先竖线后横线
% -|:向横线后竖线
% # 文字:如果需要在箭头上添加文字
% {yes}:需要添加的文字
% # 文字的位置,上南下北左东右西(与地图方位不一致)
% [anchor=east]:
% [anchor=south]:
% [anchor=west]:
% [anchor=north]:
% [anchor=center]:

\end{tikzpicture}

\end{document}

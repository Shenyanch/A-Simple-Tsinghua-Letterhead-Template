\Topic{数学中的对应思想和代数表示}
  \Teach{}
  \Grade{高一}
  % \Name{郑皓天}\FirstTime{20181207}\CurrentTime{20181207}
  % \Name{林叶}\FirstTime{20180908}\CurrentTime{20181125}
  %\Name{1v2}\FirstTime{20181028}\CurrentTime{20181117}
  % \Name{林叶}\FirstTime{20180908}\CurrentTime{20181125}
  % \Name{郭文镔}\FirstTime{20181111}\CurrentTime{20181117}
  % \Name{马灿威}\FirstTime{20181111}\CurrentTime{20181111}
  % \Name{黄亭燏}\FirstTime{20181231}\CurrentTime{20181231}
  \newtheorem*{Theorem}{定理}
  \makefront
\vspace{-1.5em}
\startexercise
% \begin{exercise}{\heiti 课前检测}\\
%   \item%【三角函数、增减性】
%     函数$y=\sin x$的单调减区间为:\tk
%     \begin{answer}
%       $ \left[2k\piup+\dfrac{\piup}{2},2k\piup+\dfrac{3\piup}{2}\right] \left(k\inZ\right)$
%     \end{answer}
%   \item
%     函数$y=2\sin\Bigl(2x-\dfrac{\piup}3 \Bigr)$的单调减区间为\tk.
%     \begin{answer}
%       $\Bigl[k\piup+\dfrac{5\piup}{12},k\piup+\dfrac{11\piup}{12} \Bigr],k\in\mathbb{Z}$
%     \end{answer}
%   \item%《习题化知识清单》P77单元检测12
%     设$\omega>0$,若函数$f(x)=2\sin \omega x(\omega>0)$在区间$\Bigl[-\dfrac{\piup}3,\dfrac{\piup}4 \Bigr]$上单调递增,则$\omega$取值范围是\tk.
%     \begin{answer}
%       $\Bigl(0,\dfrac32\Bigr]$
%     \end{answer}
%   \item
%     已知函数$f(x)=\sin\Bp{\omega x+\dfrac{\piup}4}(\omega>0)$在区间$\Bigl[-\dfrac{3\piup}4,\piup \Bigr]$上的增函数,则$\omega$取值范围是\tk.
%     \begin{answer}
%       $\Bigl(0,\dfrac14\Bigr]\bigcup\Bigl[\dfrac53,\dfrac94\Bigr]$
%     \end{answer}
% \end{exercise}
\section{集合与对应思想}
  “对应”是一个极基本的数学概念。
  代数方法(笛卡尔模式):引入字母表示有关的量,找出它们之间的关系,建立等式(方程)、不等式,或发现其他的性质(如整除、互质),从而解决问题。
  \subsection{函数与方程思想}
    使运动与变化巧妙联系
    方法概述
    函数的思想,就是用运动和变化的观点,分析和研究数学中的数量关系,建立函数关系或构造函数,运用函数的图像和性质去分析问题、转化问题,从而使问题获得解决的数学思想.
    方程的思想,就是分析数学问题中变量间的等量关系,建立方程或方程组,或者构造方程,通过解方程或方程组,或者运用方程的性质去分析、转化问题,使问题获得解决的数学思想.
    方法应用
    (1)函数与不等式的相互转化.对函数y=f(x),当y>0时,就化为不等式f(x)>0,借助于函数的图象和性质可解决有关问题,而研究函数的性质也离不开不等式.
    (2)数列的通项与前n项和是自变量为正整数的函数,用函数的观点去处理数列问题十分重要.
    (3)解析几何中的许多问题,例如直线与二次曲线的位置关系问题,需要通过解二元方程组才能解决.这都涉及二次方程与二次函数的有关理论.
    (4)立体几何中有关线段的长、面积、体积的计算,经常需要运用列方程或建立函数表达式的方法加以解决.
    解题思维
    1.构造函数从而利用函数的性质解题
    对于一些问题,利用构造函数的方法较容易解决,但是如何构造函数是一个难点,不妨从以下几个方面入手:(1)把一个代数式看成一个函数;(2)把方程化为函数;(3)把参数看作变量,从而构造一个函数来帮助解题.
    2.利用方程思想去解决含复杂变量的等式问题
    对于一个含变量的等式,想到把这个等式看作一个含未知数的方程,通过对这个方程的观察和研究,往往能使问题变得容易解决.
\section{分类与整合思想}
\section{数形结合思想}
\section{转化与化归思想}
\section{类比与比较思想}
% \section{第一章节}
%   \begin{description}
%     \item [label]
%   \end{description}
%   \begin{exercise}
%     \item%LaTeX-master/sanjiaohanshu/sanjiaohanshu-gaokao.tex 4
%       函数$f(x)=\cos(\omega x+\varphi)$的部分图象如图所示,则$f(x)$的单调递减区间为\xz
%       \begin{minipage}[b]{0.8\linewidth}
%         \vspace{2.5em}
%         \xx{$\Bigl(k\piup-\dfrac{1}{4},k\piup+\dfrac{3}{4}\Bigr),k\in\mathbb{Z}$}
%           {$ \Bigl(2k\piup-\dfrac{1}{4},2k\piup+\dfrac{3}{4}\Bigr),k\in\mathbb{Z}$}
%           {$ \Bigl(k-\dfrac{1}{4},k+\dfrac{3}{4}\Bigr),k\in\mathbb{Z}$}
%           {$\Bigl(2k-\dfrac{1}{4},2k+\dfrac{3}{4}\Bigr),k\in\mathbb{Z} $}
%       \end{minipage}\hfill
%       \begin{minipage}[h]{0.2\linewidth}
%         \vspace{-3cm}
%         \begin{tikzpicture}
%           \node[below left](O) at(0,0) {\small$\bm{O}$};
%           \draw(0,1)node[right]{\tiny$1$}--(0.1,1);
%           \clip(-1.2,-1.2) rectangle (2,1.5);
%           \draw[->,>=stealth](-1.2,0)--(2,0) node[below left] (x){$x$};
%           \draw[->,>=stealth](0,-1.2)--(0,1.5) node[below right] (y){$y$};
%           \draw[domain=-1.2:2,samples=1000] plot(\x,{cos((pi*(\x)+1/4*pi) r)});
%           \node[below] (A)at (0.25,0){$\frac{1}{4}$};
%           \node[below] (B)at (1.25,0){$\frac{5}{4}$};
%         \end{tikzpicture}
%       \end{minipage}
%       \begin{answer}
%         D
%       \end{answer}
%   \end{exercise}

% \newpage
% \section{课后作业}
%   \begin{exercise}
%
%   \end{exercise}
\stopexercise

\newpage
\section{参考答案}
\begin{multicols}{2}
  \printanswer
\end{multicols}

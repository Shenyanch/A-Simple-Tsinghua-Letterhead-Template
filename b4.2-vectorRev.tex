% 节录自b5-FinExRev.tex
\Topic{平面向量复习}
  \Teach{}
  \Grade{高一}
  % \Name{郑皓天}\FirstTime{20181207}\CurrentTime{20181207}
  % \Name{林叶}\FirstTime{20180908}\CurrentTime{20181125}
  %\Name{1v2}\FirstTime{20181028}\CurrentTime{20181117}
  % \Name{林叶}\FirstTime{20180908}\CurrentTime{20181125}
  % \Name{郭文镔}\FirstTime{20181111}\CurrentTime{20181117}
  % \Name{马灿威}\FirstTime{20181111}\CurrentTime{20181111}
  \Name{黄亭燏}\FirstTime{20181231}\CurrentTime{20190112}
  \newtheorem*{Theorem}{定理}
  \makefront
\vspace{-1.5em}
\startexercise
\section{要点归纳}
  \subsection{五种常见向量}
    \begin{enumerate}[label=\arabic*)]
      \item 单位向量:模为1的向量.
      \item 零向量:模为0的向量.
      \item 平行(共线向量):方向相同或相反或其一为零向量的两个向量.
      \item 相等向量:模相等,方向相同的向量.
      \item 相反向量:模相等,方向相反的向量.
    \end{enumerate}
  \subsection{平面向量运算律}
    \begin{enumerate}[label=\arabic*)]
      \item 交换律:
        $\bm a+\bm b=\bm b+\bm a$,\quad
        $\bm a\cdot\bm b=\bm b\cdot\bm a$
      \item 结合律:
        $(\bm{a}+\bm{b})+\bm{c}=\bm{a}+(\bm{b}+\bm{c})$,\quad
        $(\lambda \bm a)\cdot\bm{b}=\lambda(\bm a\cdot\bm b)=\bm{a}\cdot(\lambda\bm{b})$
      \item 分配律:
        $(\lambda+\mu)\bm{a}=\lambda\bm{a}+\mu\bm{a}$,\quad
        $\lambda(\bm{a}+\bm{b})=\lambda\bm{a}+\lambda\bm{b}$,\quad
        $(\bm a+\bm b)\cdot \bm c=\bm a\cdot\bm c+\bm b\cdot \bm c$
      \item 重要公式:(记号$\bm a^2=\bm a\cdot\bm a$)
        $(\bm a+\bm b)(\bm a-\bm b)=\bm a^2-\bm b^2$,\quad
        $(\bm a\pm\bm b)^2=\bm a^2\pm2\bm a\cdot\bm b+\bm b^2$.
    \end{enumerate}
  \subsection{两个重要定理}
    \begin{enumerate}[label=\arabic*)]
      \item 向量共线定理:
        向量$\bm{a}~(\bm{a}\ne\bm{0})$与向量$\bm{b}$共线,当且仅当存在唯一的实数$ \lambda $,使得$\bm{b}=\lambda\bm{a}$.\\
        {\kaishu
         证明三点共线的方法:\circled{1}$\vv{AB}=\lambda\vv{AC}$,则$A$,$B$,$C$三点共线;\circled{2}$\vv{OA}=\lambda\vv{OB}+\mu\vv{OC}$,若$\lambda+\mu=1$,则$A$,$B$,$C$三点共线.
        }
      \item 平面向量基本定理:
        如果$ \bm{e}_1,\bm{e}_2 $是同一平面内的两个\CJKunderdot{不共线}的向量,
        则那么对于这一平面内的任意向量$ \bm{a} $,有且只有一对实数$ \lambda_1,~\lambda_2 $,使$\bm{a}=\lambda_1\bm{e}_1+\lambda_2\bm{e}_2$.
        其中,不共线的向量$\bm{e}_1, \bm{e}_2$叫做表示这一平面内所有向量的一组\CJKunderdot{基底}.\\
        {\kaishu 平面向量基本定理应用技巧:
          \begin{enumerate}[label=\circled{\arabic*}]
            \item 构造某一向量在同一基底下的两种不同表达形式,
              根据向量分解的唯一性求解.即:\\
              {\kaishu 以$\bm e_1$,$\bm e_2$为基底,且$\bm a=x_1\bm e_1+y_1\bm e_2=x_2\bm e_1+y_2\bm e_2$,则$\begin{cases}x_1=x_2\\y_1=y_2\end{cases}$}
            \item 构造两个共线向量在同一基底下的表达形式,
              根据向量共线定理求解.即:\\
              {\kaishu 以$\bm e_1$,$\bm e_2$为基底,且$\bm a=x_1\bm e_1+y_1\bm e_2$,$\bm b=x_2\bm e_1+y_2\bm e_2$,且$\bm a\varparallel\bm b$,则$x_1y_2-x_2y_1=0$}
            \item 将题目中的已知条件转化成
              $\lambda_1\bm e_1+\lambda_2\bm e_2=\bm 0$的形式($\bm e_1$,$\bm e_2$不共线),根据$\lambda_1=\lambda_2=0$求解.
          \end{enumerate}}
    \end{enumerate}
  \subsection{平面向量平行、垂直的等价条件}
    设$\bm a=(x_1,y_1)$,$\bm b=(x_2,y_2)$,则:
    \begin{enumerate}[label=\arabic*)]
      \item $\bm a\varparallel\bm b$$\Leftrightarrow$$x_1y_2-x_2y_1=0$.
      \item $\bm a\perp\bm b$
            $\Leftrightarrow$$\bm a\cdot\bm b=0$
            $\Leftrightarrow$$x_1x_2+y_1y_2=0$.
    \end{enumerate}
  \subsection{平面向量数量积相关量求解}
    \begin{enumerate}[label=\arabic*)]
      \item 向量夹角:设$\bm a=(x_1,y_1)$,$\bm b=(x_2,y_2)$,则
        $\cos\vangle{\bm a}{\bm b}=\dfrac{\bm{a}\bm{\cdot}\bm{b}}{\abs{\bm{a}}\abs{\bm{b}}}=\dfrac{x_1x_2+y_1y_2}{\sqrt{x_1^2+y_1^2}\sqrt{x_2^2+y_2^2}} \quad \left(\vangle{\bm a}{\bm b}\in\left[0,\piup\right]$
      \item 向量模长:若$\bm a=(x,y)$,则$\abs{\bm a}=\sqrt{\bm a\cdot\bm a}=\sqrt{x^2+y^2}$
      \item 向量投影:向量$\bm a$在$\bm b$方向上的投影为  $\abs{\bm{a}}\cos\theta=\dfrac{\bm a\cdot\bm b}{\abs{\bm b}}$
    \end{enumerate}
\begin{exercise}{\textbf{习题}}
  \item%【向量的线性运算】
    (2017 \textbullet {\kaishu 广东深圳二模})如图所示,正方形$ABCD$中,$M$是$BC$的中点,若$\vv{AC}=\lambda\vv{AM}+\mu\vv{BD}$,则$\lambda+\mu$等于\xz
    \xx{$\dfrac{4}3$}
     {$\dfrac{5}3$}
     {$\dfrac{15}8$}
     {$2$}
    \begin{center}
      \begin{tikzpicture}
        \coordinate[label=left:$A$](A)at(0,0);
        \coordinate[label=right:$B$](B)at(3.5,0);
        \coordinate[label=left:$D$](D)at(0,3.5);
        \coordinate[label=right:$C$](C)at(3.5,3.5);
        \coordinate[label=right:$M$](M)at($(B)!0.5!(C)$);
        \draw (A)--(B)--(C)--(D)--cycle;
        \draw[->,>=latex] (A)--(C);
        \draw[->,>=latex] (A)--(M);
        \draw[->,>=latex] (B)--(D);
      \end{tikzpicture}
    \end{center}
    \begin{answer}
      B
    \end{answer}
  \item%福州一中学2016-2017学年高一下学期期末考试数学…….doc-3【向量共线】
    % (2017 \textbullet {\kaishu 福州一中} 3)
    已知向量$\bm a$,$\bm b$不共线,且$\bm c=\lambda\bm a+\bm b$,$\bm d=\bm a+(2\lambda-1)\bm b$,若$\bm c$与$\bm d$方向相反,则实数$\lambda$的值为\xz
    \xx{$1$}
     {$-\dfrac12$}
     {$1$或$-\dfrac12$}
     {$-1$或$-\dfrac12$}
    \begin{answer}
      B
    \end{answer}
  \item%福州三中2017高一下数学期末卷…….doc-5【向量投影,基底表示】
    % (2017 \textbullet {\kaishu 福州三中} 5)
    设$\bm e_1$,$\bm e_2$为单位向量,且$\bm e_1$,$\bm e_2$的夹角为$\dfrac{\piup}3$,若$\bm a=\bm e_1-3\bm e_2$,$\bm b=\bm e_1+\bm e_2$,则向量$\bm a$在$\bm b$方向上的射影为\xz
    \xx{$-\sqrt3$}
     {$\sqrt3$}
     {$-\dfrac{\sqrt{10}}5$}
     {$\dfrac{\sqrt{10}}5$}
    \begin{answer}
      A
    \end{answer}
  \item%【向量坐标法在平面几何的应用,三角函数定义】
    %如图,
    半径为$\sqrt{3}$的扇形$AOB$的圆心角为120\degree,点$C$在$\arc{AB}$上,且$\angle{COB}=30\degree$,若$\vv{OC}=\lambda\vv{OA}+\mu\vv{OB}$,则$\lambda+\mu$等于\xz
    \xx{$\sqrt{3}$}
     {$\dfrac{\sqrt{3}}3$}
     {$\dfrac{4\sqrt{3}}3$}
     {$2\sqrt{3}$}
    \begin{answer}
      A
    \end{answer}
  \item%【平面向量几何应用:垂直问题】
    直角坐标系$xOy$中,$\vv{AB}=(2,1)$,$\vv{AC}=(3,k)$,若$\triangle{ABC}$是直角三角形,则$k$的可能值个数是\xz
    \xx{1}{2}{3}{4}
    \begin{answer}
      B
    \end{answer}
  \item%福建师大附中2016-2017高一下期末考试数学试题…….doc-6【数量积,三角形形状】
    % (2017 \textbullet {\kaishu 师大附中} 6)
    若点$O$是$\triangle{ABC}$平面内一点,且满足$(\vv{OB}-\vv{OC})\cdot(\vv{OB}+\vv{OC}-2\vv{OA})=0$,则$\triangle{ABC}$形状为\xz
    \xx{钝角三角形}{等腰三角形}{直角三角形}{锐角三角形}
    \begin{answer}
      B
    \end{answer}
  \item%福州三中2017高一下数学期末卷…….doc-16【向量投影,基底表示】
    % (2017 \textbullet {\kaishu 福州三中} 16)
    已知$\bm a$,$\bm b$是平面内两个相互垂直的单位向量,若向量$\bm c$满足$(\bm a-\bm c)\cdot(\bm b-\bm c)=0$,则$|\bm c|$的最大值是\tk.
    \begin{answer}
      $\sqrt2$
    \end{answer}
  \item%%福建师大附中2016-2017高一下期末考试数学试题…….doc-15【向量夹角,线性运算模长】
    % (2017 \textbullet {\kaishu 师大附中} 15)
    已知单位向量$\bm a$,$\bm b$的夹角为$\dfracp{}3$,那么$|\bm a-2\bm b|=$\tk.
    \begin{answer}
      $\sqrt3$
    \end{answer}
  \item%【平面向量的模与夹角】
    已知$\triangle{ABC}$是正三角形,若$\vv{AC}-\lambda\vv{AB}$与向量$\vv{AC}$的夹角大于90\degree,则实数$\lambda$的取值范围是\tk.
    \begin{answer}
      $(2,+\infty)$
    \end{answer}
  \item%福建师大附中2016-2017高一下期末考试数学试题…….doc-17【数量积,几何】
    % (2017 \textbullet {\kaishu 师大附中} 17)
    在$\triangle{ABC}$中,$|\vv{AD}|=|\vv{BD}|=|\vv{CD}|$,$|\vv{AB}|=3$,则$\vv{AB}\cdot\vv{AD}=$\tk.
    \begin{answer}
      $\dfrac92$
    \end{answer}
  \item%【向量坐标法在平面几何的应用,直线方程】
    在$\mathrm{Rt}\triangle{ABC}$中,$CA=CB=2$,$M$,$N$是斜边$AB$上的两个动点,且$MN=\sqrt{2}$,则$\vv{CM}\cdot\vv{CN}$的取值范围是\tk.
    \begin{answer}
      $\Bigl[\dfrac{3}2,2\Bigr]$
    \end{answer}
  \item%福州一中学2016-2017学年高一下学期期末考试数学…….doc-14【数量积,外心】
    % (2017 \textbullet {\kaishu 福州一中} 14)
    $\triangle{ABC}$中,$CA=4$,$CB=6$,点$O$为$\triangle{ABC}$的外心,则$\vv{CO}\cdot\vv{AB}=$\tk.
    \begin{answer}
      5
    \end{answer}
  \item%福建师大附中2016-2017高一下期末考试数学试题…….doc-19【向量共线、夹角、模长】
    % (2017 \textbullet {\kaishu 师大附中} 19)
    已知$\bm a$,$\bm b$为两个不共线向量,$\abs{\bm a}=2$,$\abs{\bm b}=1$,$\bm c=2\bm a-\bm b$,$\bm d=\bm a+k\bm b$.\\
    (I)若$\bm c\varparallel\bm d$,求实数$k$;\\
    (II)若$k=-7$,且$\bm c\perp\bm d$,求$\bm a$与$\bm b$的夹角.
    \begin{answer}
      (I)$k=-\dfrac12$
      (II)$\vangle{\bm a}{\bm b}=\dfrac{\piup}3$
    \end{answer}
  \vspace{3cm}
  \item%福州一中学2016-2017学年高一下学期期末考试数学…….doc-15【数量积,垂直】
    % (2017 \textbullet {\kaishu 福州一中} 15)
    已知$\bm a=(\cos\alpha,k\sin\alpha)$,$\bm b=(\cos\beta,\sin\beta)$($k>0$,$0<\alpha<\beta<\dfrac{\piup}2$),且$\bm a+\bm b$与$\bm a-\bm b$相互垂直.\\
    (1)求$k$的值;\\
    (2)若$\bm a\cdot\bm b=\dfrac45$且$\cos\beta=\dfrac35$,求$\sin\alpha$的值.
    \begin{answer}
      (1)$k=1$;
      (2)$\sin\alpha=\dfrac7{25}$
    \end{answer}
    \vspace{4cm}
  \vspace{3.5cm}
  \item%【平面向量基本定理】
    在$\triangle{OAB}$的边$OA$,$OB$上分别取点$M$,$N$,使得$\vv{OA}=3\vv{OM}$,$\vv{OB}=4\vv{ON}$,设线段$AN$与$BM$交于点$P$,
    记$\vv{OA}=\bm a$,$\vv{OB}=\bm b$,用$\bm a$,$\bm b$表示向量$\vv{OP}$.
    \begin{answer}
      $\vv{OP}=\dfrac3{11}\bm a+\dfrac3{11}\bm b$
    \end{answer}
  \vspace{9cm}
  \item%【平面向量基本定理】
    在$\triangle{OAB}$中,$\vv{OA}=4\vv{OC}$,$\vv{OB}=2\vv{OD}$,设线段$AD$与$BC$交于点$M$,
    记$\vv{OA}=\bm a$,$\vv{OB}=\bm b$.\\
    (1)用$\bm a$,$\bm b$表示向量$\vv{OP}$.\\
    (2)已知在线段$AC$上取一点$E$,在线段$BD$上取一点$F$,使$EF$过点$M$,设$\vv{OE}=p\vv{OA}$,$\vv{OF}=q\vv{OA}$,求证$\dfrac1{7p}+\dfrac3{7q}=1$
    \begin{answer}
      (1)$\vv{OP}=\dfrac17\bm a+\dfrac37\bm b$
      (2)略
    \end{answer}
  \vspace{9cm}
\end{exercise}

\newpage
\section{课后作业}
\begin{exercise}
  \item%【向量的线性运算】
    若点$D$在$\triangle{ABC}$的边$BC$上,且$\vv{CD}=4\vv{DB}=r\vv{AB}+s\vv{AC}$,则$3r+s$的值为\xz
    \xx{$\dfrac{16}5$}
     {$\dfrac{12}5$}
     {$\dfrac{8}5$}
     {$\dfrac{4}5$}
    \begin{answer}
      C
    \end{answer}
  \item%福州屏东中学2016-2017学年高一下学期期末考试数学试题.doc-4【向量共线】
    % (2017 \textbullet {\kaishu 屏东中学} 4)
    若$A(-1,1)$,$B(1,3)$,$C(x,5)$,且$\vv{AB}=\lambda\vv{BC}$,则实数$\lambda$等于\xz
    \xx{1}{2}{3}{4}
    \begin{answer}
      1
    \end{answer}
  \item%福建师大附中2016-2017高一下期末考试数学试题…….doc-3【向量投影,坐标表示】
    % (2017 \textbullet {\kaishu 师大附中} 3)
    若$\bm a=(2,1)$,$\bm b=(3,4)$,则向量$\bm b$在向量$\bm a$方向上的投影为\xz
    \xx{$2\sqrt5$}
     {$2$}
     {$\sqrt5$}
     {$10$}
    \begin{answer}
      A
    \end{answer}
  \item%【向量表示】
    设$D$为$\triangle{ABC}$所在平面内一点,$\vv{BD}=3\vv{CD}$,则\xz
    \xx{$\vv{AD}=-\dfrac13\vv{AB}+\dfrac43\vv{AC}$}
     {$\vv{AD}=\dfrac43\vv{AB}-\dfrac13\vv{AC}$}
     {$\vv{AD}=\dfrac23\vv{AB}-\dfrac12\vv{AC}$}
     {$\vv{AD}=-\dfrac12\vv{AB}+\dfrac32\vv{AC}$}
    \begin{answer}
      D
    \end{answer}
  \item%【向量夹角、模长】
    已知$|\bm a|=1$,$\bm a\cdot\bm b=\dfrac12$,$|\bm a-\bm b|^2=1$,则$\bm a$与$\bm b$的夹角等于\xz
    \xx{30\degree}{45\degree}{60\degree}{120\degree}
    \begin{answer}
      C
    \end{answer}
  \item%【向量共线】
    已知向量$\bm a=(2,3)$,$\bm b=(-1,2)$,若$m\bm a+4\bm b$与$\bm a-2\bm b$共线,则$m$的值为\xz
    \xx{$\dfrac12$}{$2$}{$-\dfrac12$}{$-2$}
    \begin{answer}
      D
    \end{answer}
  \item%【向量几何应用】
    在平面四边形$ABCD$中,若$AC=3$,$BD=2$,则$(\vv{AB}+\vv{DC})\cdot(\vv{AC}+\vv{BD})=$\tk.
    \begin{answer}
      5
    \end{answer}
  \item%【向量垂直】
    平面向量$\bm a=(\sqrt{3},-1)$,$\bm=\Bp{\dfrac12,\dfrac{\sqrt3}2}$,若存在不同时为0的实数$k$和$t$,使$\bm x=\bm a+(t^2-3)\bm b$,$\bm y=-k\bm a+t\bm b$,且$\bm x\perp\bm y$,试求函数关系式$k=f(t)$.
    \begin{answer}
      $k=f(t)=\dfrac14(t^3-3t)$
    \end{answer}
  \vspace{2.5cm}
  \item%福州三中2017高一下数学期末卷…….doc-18【向量垂直,模长,共线】
    % (2017 \textbullet {\kaishu 福州三中} 18)
    平面内的向量$\bm a=(3,2)$,$\bm b=(-1,2)$,$\bm c=(4,1)$.\\
    (I)若$(\bm a+k\bm c)\perp(2\bm b-\bm a)$,求实数$k$的值;\\
    (II)若向量$\bm d$满足$\bm d\varparallel\bm c$,且$\abs{\bm d}=\sqrt{34}$,求向量$\bm d$的坐标.
    \begin{answer}
      (I)$k=-\dfrac{11}{18}$
      (II)$\bm d=(4\sqrt2,\sqrt2)$或$\bm d=(-4\sqrt2,-\sqrt2)$
    \end{answer}
  \vspace{5cm}
  \item
    已知点$P$是$\triangle{ABC}$内一点,且满足条件$\vv{AP}+\vv{AP}+\vv{AP}=\bm 0$,设点$Q$为$CP$的延长线与$AB$的交点,令$\vv{CP}=\bm p$,试用向量$\bm p$表示$\vv{CQ}$.
    \begin{answer}
      $\vv{CQ}=2\bm p$
    \end{answer}
  \vspace{6cm}
  \item%《2018天利38套:高考真题单元专题训练(文)》专题18平面向量的概念与运算 P63p31【2010•江苏】
    (2010 \textbullet {\kaishu 江苏})
    在平面直角坐标系$xOy$中,已知点$A(-1,-2)$,$B(2,3)$,$C(-2,-1)$.\\
    (I)求以线段$AB$,$AC$为邻边的平行四边形的两条对角线的长;\\
    (II)设实数$t$满足$(\vv{AB}-t\vv{OC})\cdot\vv{OC}=0$,求$t$的值.
    \begin{answer}
      (I)两条对角线长分别为$4\sqrt2$,$2\sqrt{10}$;
      (II)$t=-\dfrac{11}5$
    \end{answer}
  \item%《2018天利38套:高考真题单元专题训练(理)ISBN978-7-223-03438-8》专题18平面向量的应用 P72p17【2014•陕西】
    (2014 \textbullet {\kaishu 陕西})
    在直角坐标系$xOy$中,已知点$A(1,1)$,$B(2,3)$,$C(3,2)$,点$P(x,y)$在$\triangle{ABC}$三边围成的区域(含边界)上.\\
    (I)若$\vv{PA}+\vv{PB}+\vv{PC}=\bm 0$,求$\abs{\vv{OP}}$;\\
    (II)设$\vv{OP}=m\vv{AB}+n\vv{AC}$($m,n\inR$),用$x$,$y$表示$m-n$,并求$m-n$的最大值.
    \begin{answer}
      (I)$\abs{\vv{OP}}=2\sqrt2$;
      (II)$(x,y)=(m+2n,2m+n)$,$m-n$最大值为1.
    \end{answer}
\end{exercise}
\stopexercise

\newpage
\section{参考答案}
\begin{multicols}{2}
  \printanswer
\end{multicols}

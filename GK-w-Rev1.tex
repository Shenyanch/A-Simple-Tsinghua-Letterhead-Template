\Topic{等差数列与等比数列}
  \Teach{}
  \Grade{高三}
  % \Name{郑皓天}\FirstTime{20181207}\CurrentTime{20181207}
  % \Name{林叶}\FirstTime{20180908}\CurrentTime{20181125}
  %\Name{1v2}\FirstTime{20181028}\CurrentTime{20181117}
  % \Name{林叶}\FirstTime{20180908}\CurrentTime{20181125}
  % \Name{郭文镔}\FirstTime{20181111}\CurrentTime{20181117}
  % \Name{马灿威}\FirstTime{20181111}\CurrentTime{20181111}
  % \Name{黄亭燏}\FirstTime{20181231}\CurrentTime{20181231}
  % \Name{王睿妍}\FirstTime{20190129}\CurrentTime{}
  \Name{郑旭晶}\FirstTime{20190423}\CurrentTime{20190505}
  \newtheorem*{Theorem}{定理}
  \makefront
\vspace{-1.5em}
\startexercise
\section{数列基本概念}
  按照一定的顺序排列的一列数叫做\CJKunderdot{数列},数列中的每一个数叫做这个数列的\CJKunderdot{项}.排在第一位的数称作数列的\CJKunderdot{首项},排在第二位的称为数列的第$ 2 $项$ \cdots\cdots $排在第$ n $位的称为这个数列的第$ n $项.数列的一般形式为\[a_1,~a_2,~a_3,~\cdots a_n,~\cdots, \]简记为$ \{a_n\} $.
  \subsection{数列与函数}
    在函数的意义下,数列是定义域为正整数集$ \mathbf{N^*} $(或它的有限子集$ \left\{1,2,3,\cdots,n\right\} $)的特殊函数,数列的通项公式就是相应的函数解析式,即$ a_n=f(n)~(n\in\mathbf{N^*}) $.
  \subsection{通项公式}
    如果数列$\{a_n\}$的第$ n $项与序号$ n $之间的关系可以用一个式子来表示,那么这个式子叫做这个数列的\CJKunderdot{通项公式}.
  \subsection{递推公式}
    如果已知数列$\{a_n\}$的第一项(或前几项),且从第二项(或某一项)开始任何一项$ a_n $与它的前一项$ a_{n-1} $(或前几项)间的关系可以用一个式子来表示,那么这个式子叫做数列$\{a_n\}$的递推公式.
  \subsection{数列的分类}
    类比函数的性质及其分类,对数列进行恰当的分类可以更深刻的理解和认识数列.
    \begin{enumerate}[1)]
      % \item 根据项的个数:
      %   \begin{enumerate}[i)]
      %     \item 有穷数列:项数有限的数列;
      %     \item 无穷数列:项数无限的数列;
      %   \end{enumerate}
      \item 根据项的变化趋势:
        \begin{enumerate}[i)]
          \item 递增数列:从第二项起每一项都大于它的前一项的数列;
          \item 递减数列:从第二项起每一项都小于它的前一项的数列;
          \item 常数列:各项相等的数列;
          \item 摆动数列:从第二项起,前后两项变化规律不定的数列.
        \end{enumerate}
      % 递增数列和递减数列统称单调数列;
      % \item 根据项的(绝对值)大小是否有限制:
      %   \begin{enumerate}[i)]
      %     \item 有界数列:$ \forall n\in\mathbf{N^*},\abs{a_n}\le M~(M\text{为常数}) $;
      %     \item 无界数列:$ \forall M\in\mathbf{R^+} ,\exists n\in\mathbf{N^*},\text{使得}\abs{a_n}>M.$
        \end{enumerate}
    \end{enumerate}
\section{等差数列}
  \subsection{基本性质}
    \subsubsection{定义}
      一般地,如果一个数列从第二项开始,每一项与前一项的差等于同一个常数,那么这个数列就叫做等差数列,这个常数叫做这个数列的公差,常用字母$ d $表示.\\
      {\kaishu 注:目前大部分等差数列考题都可以通过转化为$ a_1 $和$ d $求出.}
    \subsubsection{通项公式}
      如果等差数列$\{a_n\}$的通项公式是$ a_n=a_1+(n-1)d ,n\in\mathbf{N^*}$,其中$ a_1 $为首项,$ d $为公差.\\
      % \begin{proof}
      %   由给定条件可得:\begin{equation*}
      %   \begin{aligned}
      %    a_2-a_1& =d\\
      %   a_3-a_2&=d\\
      %   \vdots&\\
      %   a_n-a_{n-1}&=d.
      %   \end{aligned}
      %   \end{equation*}
      %   等号两边累加可得:~$ a_n-a_1=(n-1)d .$即:~$$a_n=a_1+(n-1)d$$
      % \end{proof}
    \subsubsection{等差中项}
      \begin{enumerate}[1)]
        \item 如果$ A=\dfrac{a+b}{2} ,$则称$ A $为$ a $和$ b $的等差中项;
        \item 等差数列中,等间隔的三项$a_{n-p},~a_n,~a_{n+p} (n,p\in\mathbf{N^*}~\text{且}~n<p) $满足:$ 2a_n=a_{n-p}+a_{n+p} $;
        \item 在等差数列$ \left\{a_n\right\} $中,若有$ k+l=m+n \left(k,l,m,n\in\mathbf{N^*}\right)$,则有$ a_k+a_l=a_m+a_n $.
      \end{enumerate}
    \subsubsection{前$ n $项和公式}
      设等差数列$\{a_n\}$的公差为$ d $,则其前$ n $项和$ S_n=\dfrac{n\left(a_1+a_n\right)}{2} $或$ S_n=na_1+\dfrac{n(n-1)}{2}d $.
      % \begin{proof}
      %   在等差数列中,根据性质$ a_k+a_l=a_m+a_n~(k+l=m+n) $可得$$ a_1+a_n=a_2+a_{n-1}=\cdots=a_k+a_{n-k+1} ~\left(k\le\dfrac{n}{2}\right)$$
      %   \begin{equation*}
      %   \begin{aligned}
      %   S_n&=a_1+a_2+a_3+\cdots+a_n\\
      %    &=\left(a_1+a_n\right)+\left(a_2+a_{n-1}\right)+\cdots+\left(a_k+a_{n-k+1}\right)\\
      %   &=\dfrac{n(a_1+a_n)}{2}\\
      %   &=\dfrac{n(a_1+a_1+(n-1)d}{2}=na_1+\dfrac{n(n-1)}{2}d.
      %   \end{aligned}
      %   \end{equation*}
      % \end{proof}
  \subsection{性质扩充}
    \subsubsection{等差数列的常用性质}
      \begin{enumerate}[(1)]
      \item 通项公式的推广:$ a_n=a_m+\left(n-m\right)d \left(n,m\in\mathbf{N^*}\right)$;
      \item 若$\{a_n\}$是等差数列,公差为$ d $,则$\{a_{2n}\}$ 也是等差数列,公差为$ 2d $;
      \item 若$\{a_n\},~\{b_n\}$是等差数列,则$ \left\{pa_n+qb_n\right\}~(p,q\text{是常数}) $也是等差数列;
      \item 若$\{a_n\}$是等差数列,公差为$ d $,则$ a_k,~a_{k+m},~ a_{k+2m},~a_{k+3m},\cdots\left(k,m\in\mathbf{N^*}\right)$组成公差为$ md $的等差数列.
      \end{enumerate}
    \subsubsection{与和有关的性质}
      \begin{enumerate}[(1)]
        \item 若$\{a_n\}$是等差数列,则$ \dfrac{S_n}{n} $也是等差数列,其首项与$ \{a_n\} $的首项相同,公差是$\{a_n\}$的公差的$ \dfrac{1}{2} $;
        \item 若$ S_m,S_{2m},S_{3m} $分别是$\{a_n\}$的前$ m $项,前$ 2m $项,前$ 3m $项的和,则$ S_m,~S_{2m}-S_m,~S_{3m}-S_{2m} $成等差数列,公差为$m^2d$;
        \item 关于非零等差数列奇数项和与偶数项和的性质
        \begin{enumerate}[i)]
          \item 若项数为$ 2n $,则$ S_{\text{偶}}-S_{\text{奇}}=nd ,~\dfrac{S_{\text{偶}}}{S_{\text{奇}}}=\dfrac{a_{n+1}}{a_n}$.
          \item 若项数为$ 2n-1 $,则$  S_{\text{偶}}=(n-1)a_n,~S_{\text{奇}}-S_{\text{偶}}=a_n,~\dfrac{S_{\text{奇}}}{S_{\text{偶}}}=\dfrac{n}{n-1}$.
          \item 若两个等差数列$\{a_n\}$、$\{b_n\}$的前$ n $项和分别为$ S_n\text{、}~T_n,~ $则$ \dfrac{a_n}{b_n}=\dfrac{S_{2n-1}}{T_{2n-1}} $.
        \end{enumerate}
        \item $\{\dfrac{S_n}{n}\}$成等差数列.
      \end{enumerate}
    \subsubsection{等差数列前$ n $项和的最值问题}
      \begin{enumerate}[1)]
        \item 二次函数法:当公差$d\ne0$时,将$ S_n $看作关于$ n $的二次函数,运用配方法,借助函数的单调性及数形结合,使问题得解;
        \item 通项公式法:求使$ a_n\ge0 \left(\text{或}a_n\le0\right)$成立的最大$ n $值即可得到$ S_n $的最大(或最小)值;
        \item 不等式法:借助$ S_n $最大时,有$\Bigg\{\begin{aligned}
      S_n\ge S_{n-1},\\
      S_n\ge S_{n+1}.
      \end{aligned}~(n\ge2,n\in\mathbf{N^*})$,解此不等式组确定$ n $的范围,进而确定$ n $的值和对应$ S_n $的值.
      \end{enumerate}
  \begin{exercise}{\heiti 习题}\\
    \item 已知四个数成等差数列,它们的和为26,中间两项的积为40,求这四个数.
  \end{exercise}


\newpage
\section{课后作业}
  % \begin{exercise}{\heiti 练习}
  % \end{exercise}
  \begin{exercise}
  \end{exercise}
\stopexercise

\newpage
\section{参考答案}
\begin{multicols}{2}
  \printanswer
\end{multicols}

\Topic{三角函数图像与性质}
  \Teach{}
  \Grade{高三}
  % \Name{郑皓天}\FirstTime{20181207}\CurrentTime{20181207}
  % \Name{林叶}\FirstTime{20180908}\CurrentTime{20181125}
  %\Name{1v2}\FirstTime{20181028}\CurrentTime{20181117}
  % \Name{林叶}\FirstTime{20180908}\CurrentTime{20181125}
  % \Name{郭文镔}\FirstTime{20181111}\CurrentTime{20181117}
  % \Name{马灿威}\FirstTime{20181111}\CurrentTime{20181111}
  % \Name{黄亭燏}\FirstTime{20181231}\CurrentTime{20181231}
  % \Name{王睿妍}\FirstTime{20190129}\CurrentTime{}
  \Name{郑旭晶}\FirstTime{20190423}\CurrentTime{20190507}
  \newtheorem*{Theorem}{定理}
  \makefront
\vspace{-1.5em}
\startexercise
\section{知识要点}
  \subsection{三角函数的图像和性质}
    \subsubsection{正弦函数}
      \begin{center}
        \begin{tikzpicture}[scale=0.7]
          \coordinate[label=below right:$O$] (O) at(0,0);
          \coordinate[label=below :\small$-\piup$] (t1) at(-pi,0);
          \coordinate[label=below :\small$\piup$] (t2) at(pi,0);
          \draw[->,>=latex](-3.5*pi,0)--(3.5*pi,0)node[below](x) {$x$};
          \draw[->,>=latex](0,-1.5)--(0,1.5)node[right](y) {\small $y=\sin(x)$};
          \draw [domain=-3*pi:3*pi,samples=1000] plot(\x,{sin(\x r)});
          \draw[densely dashed](pi/2,0)node[below](pi){$\dfrac{\piup}{2}$}--++(0,1);
          \draw[densely dashed](-pi/2,0)node[above](-pi){$-\dfrac{\piup}{2}$}--++(0,-1);
          \draw[densely dashed](0,1)node[left](max){$1$}--++(pi/2,0);
          \draw[densely dashed](0,-1)node[right](min){$-1$}--++(-pi/2,0);
        \end{tikzpicture}
      \end{center}
      \vspace{-0.9cm}
      \begin{enumerate}[label=\circled{\arabic*}]
        \item 定义域:$x\inR$;\quad 值域:$ \left[-1,1\right] $ ;\quad 奇偶性:奇函数;
        \item 对称轴:$x=k\piup+\dfrac{\piup}{2}\left(k\inZ\right)$;\quad 对称中心:$\left(k\piup,0\right)\left(k\inZ\right)$;\quad 最小正周期:$T=2\piup$;
        \item 单调递增区间:$ \left[2k\piup-\dfrac{\piup}{2},2k\piup+\dfrac{\piup}{2}\right]\left(k\inZ\right) $;\quad
              单调递减区间:$ \left[2k\piup+\dfrac{\piup}{2},2k\piup+\dfrac{3\piup}{2}\right] \left(k\inZ\right)$.
      \end{enumerate}
    \subsubsection{余弦函数}
      \begin{center}
        \begin{tikzpicture}[scale=0.7]
          \coordinate[label=below right:\small$O$] (O) at(0,0);
          \coordinate[label=below :\small $\frac{\piup}{2}$] (t1) at(pi/2,0);
          \coordinate[label=below :\small $-\frac{\piup}{2}$] (t2) at(-pi/2,0);
          \coordinate[label=below left :\small $1$] (t3) at(0,1);
          \draw[->,>=latex](-3.5*pi,0)--(3.5*pi,0)node[below](x) {$x$};
          \draw[->,>=latex](0,-1.5)--(0,1.5)node[right](y) {\small $y=\cos(x)$};
          \draw [domain=-3*pi:3*pi,samples=1000] plot(\x,{cos(\x r)});
          \draw[densely dashed](pi,0)node[below left](pi){\small $\piup$}--++(0,-1);
          \draw[densely dashed](-pi,0)node[below left](-pi){\small $-\piup$}--++(0,-1);
          \draw[densely dashed](-pi,-1)--(0,-1)node[below left](min){$-1$}--++(pi,0);
        \end{tikzpicture}
      \end{center}
      \vspace{-0.7cm}
      \begin{enumerate}[label=\circled{\arabic*}]
        \item 定义域:$x\inR$;\quad 值域:$ \left[-1,1\right] $;\quad 奇偶性:偶函数;
        \item 对称轴:$ x=k\piup \left(k\inZ\right) $;\quad 对称中心:$\left(k\piup+\dfrac{\piup}{2},0\right)\left(k\inZ\right)$;\quad 最小正周期:$ T=2\piup  $;
        \item 单调递增区间:$ \left[2k\piup-\piup,2k\piup\right] \left(k\inZ\right)$;\quad
              单调递减区间:$ \left[2k\piup,2k\piup+\piup\right]\left(k\inZ\right) $.
      \end{enumerate}
    \subsubsection{正切函数}
      \vspace{-0.5cm}
      \begin{center}
        \begin{tikzpicture}[scale=0.7]
          \coordinate[label=below right:$O$] (O) at(0,0);
          %\coordinate[label=below :$\dfrac{\pi}{2}$] (t1) at(pi/2,0);
          %\coordinate[label=below :$2\pi$] (t2) at(2*pi,0);
          \draw[->,>=latex](-pi,0)--(pi,0)node[below](x) {$x$};
          \draw[->,>=latex](0,-1.5)--(0,2)node[right](y) {\small $y=\tan(x)$};
          \draw [domain=-pi/3:1/3*pi,samples=1000] plot(\x,{tan(\x r)});
          \draw[densely dashed](2*pi/5,1.5)--++(0,-1.5)node[below right](pi){$\frac{\pi}{2}$}--++(0,-1.5);
          \draw[densely dashed](-2*pi/5,1.5)--++(0,-1.5)node[below left](pi){$-\frac{\pi}{2}$}--++(0,-1.5);
        \end{tikzpicture}
      \end{center}
      \vspace{-0.7cm}
      \begin{enumerate}[label=\circled{\arabic*}]
        \item 定义域:$\Bigl\{x\Bigm|x\ne k\piup+\dfrac{\piup}2,k\inZ\Bigr\}$;\quad 值域:$ \mathbf{R} $;\quad 奇偶性:奇函数;
        \item 对称中心:$\left(\dfrac{k\piup}2,0\right)\left(k\inZ\right)$;\quad 无对称轴;\quad 最小正周期:$ T=\piup  $;
        \item 单调递增区间:$ \left(k\piup-\dfrac{\piup}{2},k\piup+\dfrac{\piup}{2}\right) \left(k\inZ\right)$;
        \end{enumerate}
    \begin{exercise}
      \item %《2019金考卷双测20套(文)ISBN978-7-5371-9890-5》题型5三角函数、三角恒等变换P15p1【2018•全国III卷】【三角函数,恒等变换,周期】\\
        \source{2018文}{全国III卷}
        函数$f(x)=\dfrac{\tan\alpha}{1+\tan^2\alpha}$的最小正周期为\xz
        \xx{$\dfrac{\piup}4$}
         {$\dfrac{\piup}2$}
         {$\piup$}
         {$2\piup$}
        \begin{answer}
          C
        \end{answer}
      \vspace{1.5em}
      \item %《2019金考卷双测20套(文)ISBN978-7-5371-9890-5》题型5三角函数、三角恒等变换P15p2【2018•全国I卷】【三角函数,恒等变换,周期】\\
        \source{2018文}{全国I卷}
        已知函数$f(x)=2\cos^2x-\sin^2x+2$,则\xz
        \xx{$f(x)$的最小正周期为$\piup$,最大值为3}
         {$f(x)$的最小正周期为$\piup$,最大值为4}
         {$f(x)$的最小正周期为$2\piup$,最大值为3}
         {$f(x)$的最小正周期为$2\piup$,最大值为4}
        \begin{answer}
          B
        \end{answer}
      \vspace{1.5em}
      \item%福建师大附中2016-2017高一下期末考试数学试题…….doc-14【三角函数取值范围】
        % (2017 \textbullet {\kaishu 师大附中} 14)
        函数$y=\sqrt{\cos x-\dfrac12}$的定义域为\tk.
        \begin{answer}
          $\Bigl[-\dfrac{\piup}3+2k\piup,\dfrac{\piup}3+2k\piup\Bigr]$,$k\inZ$
        \end{answer}
      \vspace{1.5em}
      \item%福州第三中中学2015-2016学年高一数学第二学期期末检测.doc-14【三角函数性质 综合判断】
        % (2016 \textbullet {\kaishu 福州三中} 14)
        关于函数$f(x)=2\sin\Bp{2x+\dfrac{\piup}3}$($x\inR$),有下列说法:\\
        \circled{1}由$f(x_1)=f(x_2)=0$可得$x_1-x_2$必是$\piup$的整数倍;
        \circled{2}$y=f(x)$的表达式可改写为$f(x)=2\cos\Bp{2x-\dfrac{\piup}6}$;
        \circled{3}$y=f(x)$的图像关于点$\Bp{-\dfrac{\piup}6,0}$对称;
        \circled{4}$y=f(x)$的图像关于直线$x=\dfrac{7\piup}{12}$对称.\\
        其中说法正确的序号是\tk.
        \begin{answer}
          \circled{2}\circled{3}\circled{4}
        \end{answer}
      \vspace{1.5em}
      \item%福建师大附中2016-2017高一下期末考试数学试题…….doc-10【三角函数取值范围,方程解】
        % (2017 \textbullet {\kaishu 师大附中} 10)
        若方程$\cos\Bp{2x+\dfracp{}4}=m$在区间$\Bigl[0,\dfracp{}2\Bigr]$上有两个实根,则实数$m$取值范围是\xz
        \xx{$\Bigl[-1,-\dfrac{\sqrt2}2\Bigr]$}
         {$\Bigl(-1,-\dfrac{\sqrt2}2\Bigr]$}
         {$\Bigl[\dfrac{\sqrt2}2,1\Bigr]$}
         {$\Bigl[\dfrac{\sqrt2}2,1\Bigr)$}
        \begin{answer}
          B
        \end{answer}
      \vspace{1.5em}
      \item%福建师大附中2015-2016学年高一数学第二学期期末检测.doc-22【三角函数性质】
        % (2016 \textbullet {\kaishu 师大附中} 22)
        已知函数$f(x)=3\sin\Bp{\dfrac{x}2+\dfrac{\piup}6}+3$,$x\inR$.\\
        (I)求函数$f(x)$的单调增区间;\\
        (II)若$x\in\Bigl[\dfrac{\piup}3,\dfrac{4\piup}3\Bigr]$,求$f(x)$的最大值和最小值,
        并指出$f(x)$取得最值时相应$x$的值.
        \begin{answer}
          (I)$\Bigl[-\dfrac{4\piup}3+4k\piup,\dfrac{2\piup}3+4k\piup\Bigr]$,$k\inZ$;
          (II)当$x=\dfrac{4\piup}3$时,取最小值$f(x)_{\min}=\dfrac92$;当$x=\dfrac{2\piup}3$时,取最大值$f(x)_{\max}=6$.
        \end{answer}
      \vspace{5cm}
    \end{exercise}
\subsection{三角函数图像的(线性)变换}
  \begin{description}
    \item 函数$y=f(x)$图像经平移或伸缩变换后的图像解析式:{\kaishu 坐标变量的变化与图像相反}
      \[\begin{aligned}
        y=f(x)&\xrightarrow[\text{平移}\abs{a}\text{个单位}]{\text{向左}(a>0)\text{或向右}(a<0)}y=f(x+a)\hspace{3em}
        y=f(x)\xrightarrow[\text{纵坐标不变}]{\text{横坐标变为原来的}k\text{倍}}y=f\Bp{\dfrac{x}{k}}\\
        y=f(x)&\xrightarrow[\text{平移}\abs{a}\text{个单位}]{\text{向下}(a>0)\text{或向上}(a<0)}y+a=f(x)\hspace{3em}
        y=f(x)\xrightarrow[\text{横坐标不变}]{\text{纵坐标变为原来的}A\text{倍}}\dfrac{y}{A}=f(x)
      \end{aligned}\]
    \item 由函数$y=\sin(x)$的图象经过变换得到$y=A\sin\left(\omega x+\varphi\right)$的图象方法\\
      \begin{minipage}[h]{0.45\linewidth}
        \hspace{-2em}\circled{1} 先平移后伸缩
          \[\begin{aligned}
            y=\sin x&\xrightarrow[\text{平移}\abs{\varphi}\text{个单位}]{\text{向左}(\varphi>0)\text{或向右}(\varphi<0)}y=\sin\left(x+\varphi\right)\\
            &\xrightarrow[\text{纵坐标不变}]{\text{横坐标变为原来的}\tfrac{1}{\omega}}y=\sin\left(\omega x+\varphi\right)\\
            &\xrightarrow[\text{横坐标不变}]{\text{纵坐标变为原来的}A\text{倍}}y=A\sin\left(\omega x+\varphi\right)
          \end{aligned}\]
      \end{minipage}\hfill
      \begin{minipage}[h]{0.55\linewidth}
        \circled{2} 先伸缩后平移
          \[\begin{aligned}
            y=\sin x&\xrightarrow[\text{纵坐标不变}]{\text{横坐标变为原来的}\tfrac{1}{\omega}}y=\sin\omega x\\
            &\xrightarrow[\text{平移}\abs{\tfrac{\varphi}{\omega}}\text{个单位}]{\text{向左}(\varphi>0)\text{或向右}(\varphi<0)}y=\sin\biggl[\omega\Bp{x+\dfrac{\varphi}{\omega}}\biggr]\\
            &\xrightarrow[\text{横坐标不变}]{\text{纵坐标变为原来的}A\text{倍}}y=A\sin\left(\omega x+\varphi\right)
          \end{aligned}\]
      \end{minipage}
    \item 由图象求函数$y=A\sin\left(\omega x+\varphi\right)$的解析式一般步骤:
      \begin{enumerate}[label=\arabic*\degree]
        \item 由函数的最值确定$ A $的取值;
        \item 由函数的周期确定$ \omega $的值, 周期:$ T=\dfrac{2\pi}{\abs{\omega}} $;
        \item 由函数图象最高点(最低点)的坐标得到关于$ \varphi $的方程,再由$ \varphi $的范围确定$ \varphi $的值.
      \end{enumerate}
  \end{description}
  \begin{exercise}
    \item%《2018天利38套:高考真题单元专题训练(理)ISBN978-7-223-03393-0》专题14三角函数的图像与性质P53p4【2015•全国新课标】【正弦曲线图像】
         %LaTeX-master/sanjiaohanshu/sanjiaohanshu-gaokao.tex 4
      {\kaishu (2015 \textbullet 全国新课标)}
      函数$f(x)=\cos(\omega x+\varphi)$的部分图象如图所示,则$f(x)$的单调递减区间为\xz
      \begin{minipage}[b]{0.8\linewidth}
        \vspace{2.5em}
        \xx{$\Bigl(k\piup-\dfrac{1}{4},k\piup+\dfrac{3}{4}\Bigr),k\in\mathbb{Z}$}
          {$ \Bigl(2k\piup-\dfrac{1}{4},2k\piup+\dfrac{3}{4}\Bigr),k\in\mathbb{Z}$}
          {$ \Bigl(k-\dfrac{1}{4},k+\dfrac{3}{4}\Bigr),k\in\mathbb{Z}$}
          {$\Bigl(2k-\dfrac{1}{4},2k+\dfrac{3}{4}\Bigr),k\in\mathbb{Z} $}
      \end{minipage}\hfill
      \begin{minipage}[h]{0.2\linewidth}
        \vspace{-3cm}
        \begin{tikzpicture}[scale=0.9]
          \node[below left](O) at(0,0) {\small$\bm{O}$};
          \draw(0,1)node[right]{\tiny$1$}--(0.1,1);
          \clip(-1.2,-1.2) rectangle (2,1.5);
          \draw[->,>=stealth](-1.2,0)--(2,0) node[below left] (x){$x$};
          \draw[->,>=stealth](0,-1.2)--(0,1.5) node[below right] (y){$y$};
          \draw[domain=-1.2:2,samples=1000] plot(\x,{cos((pi*(\x)+1/4*pi) r)});
          \node[below] (A)at (0.25,0){$\frac{1}{4}$};
          \node[below] (B)at (1.25,0){$\frac{5}{4}$};
        \end{tikzpicture}
      \end{minipage}
      \begin{answer}
        D
      \end{answer}
    \vspace{1.5em}
    \item %《2019金考卷双测20套(文)ISBN978-7-5371-9890-5》题型5三角函数、三角恒等变换P15p6【2018•南宁摸底联考】【正弦曲线图像】\\
        {\kaishu (2018 \textbullet 南宁摸底联考(文))}
        如图,函数$f(x)=A\sin(2x+\varphi)$($A>0$,$|\varphi|<\dfrac{\piup}2$)的图像经过点$(0,\sqrt3)$,则函数$f(x)$的解析式为\xz
        \begin{minipage}[t]{0.7\linewidth}
          \xx{$f(x)=2\sin\Bigl(2x-\dfrac{\piup}3\Bigr)$}
           {$f(x)=2\sin\Bigl(2x+\dfrac{\piup}3\Bigr)$}
           {$f(x)=2\sin\Bigl(2x+\dfrac{\piup}6\Bigr)$}
           {$f(x)=2\sin\Bigl(2x-\dfrac{\piup}6\Bigr)$}
        \end{minipage}\hfill
        \begin{minipage}[h]{0.3\linewidth}
           \vspace{0.5cm}
           \begin{center}
             \begin{tikzpicture}[scale=0.8]
               \coordinate[label=below left:$O$] (O) at(0,0);
               % \coordinate[label=above :\small$\tfrac{\piup}3$] (t1) at(pi/3,0);
               \draw[->,>=latex](-0.2*pi,0)--(1.2*pi,0)node[below](x) {$x$};
               \draw[->,>=latex](0,-1.5)--(0,1.5)node[right](y) {\small $y$};
               \draw [domain=-0.1*pi:0.9*pi,samples=100] plot(\x,{sin((4*\x+pi/3) r)});
               % \draw[densely dashed](7*pi/12,0)node[above](pi){$\tfrac{7\piup}{12}$}--++(0,-1);
               \draw[densely dashed](0,-1)node[left](min){$-2$}--++(7*pi/12,0);
             \end{tikzpicture}
           \end{center}
        \end{minipage}
        \begin{answer}
          B
        \end{answer}
    \vspace{1.5em}
    \item%福州三中中学2015-2016学年高一数学第二学期期末检测.docx-9【正弦曲线图像】
      % (2016 \textbullet {\kaishu 福州三中} 9)
      将函数$y=\sin\Bigl(x-\dfrac{\piup}3\Bigr)$的图像上所有点的横坐标伸长到原来的2倍(纵坐标不变),再将所得的图像向左平移$\dfrac{\piup}3$个单位,得到的函数图像对应的解析式是\xz
      \xx{$y=\sin\dfrac x2$}
        {$y=\sin\Bigl(\dfrac x2-\dfrac{\piup}2\Bigr)$}
        {$y=\sin\Bigl(\dfrac{x}2-\dfrac{\piup}6\Bigr)$}
        {$y=\sin\Bigl(2x-\dfrac{\piup}6\Bigr)$}
      \begin{answer}
        C
      \end{answer}
    \vspace{1.5em}
    \item%福州一中2015-2016学年高一数学第二学期期末检测.docx-5【正弦曲线图像】
      % (2016 \textbullet {\kaishu 福州一中} 5)
      函数$y=\sin\Bp{2x+\dfracp{}3}$的图像向右平移$\dfrac{\piup}6$个单位,所得的图像对应的函数\xz
      \xx{为非奇非偶函数}
       {图像的对称中心为$(2k\piup,0)$($k\inZ$)}
       {为奇函数}
       {在$\Bigl[-\dfrac{\piup}3,\dfrac{\piup}6\Bigr]$上单调递增}
      \begin{answer}
        C
      \end{answer}
    \vspace{1.5em}
    \item%《2018天利38套:高考真题单元专题训练(理)ISBN978-7-223-03393-0》专题14三角函数的图像与性质P53p8【2017•天津】【正弦曲线解析式】
          % {\kaishu (2017 \textbullet 天津)}
          设函数$f(x)=2\sin(\omega x+\varphi)$,$x\inR$,其中$\omega>0$,$\abs{\varphi}<\piup$,若$f\Bp{\dfrac{5\piup}8}=2$,$f\Bp{\dfrac{11\piup}8}=0$,且$f(x)$的最小正周期大于$\piup$,则\xz
          \xx{$\omega=\dfrac23$,$\varphi=\dfrac{\piup}{12}$}
           {$\omega=\dfrac23$,$\varphi=-\dfrac{11\piup}{12}$}
           {$\omega=\dfrac13$,$\varphi=-\dfrac{11\piup}{24}$}
           {$\omega=\dfrac13$,$\varphi=\dfrac{7\piup}{24}$}
          \begin{answer}
            A
          \end{answer}
    \vspace{1.5em}
    \item%《2018天利38套:高考真题单元专题训练(理)ISBN978-7-223-03393-0》专题14三角函数的图像与性质P54p18【2017•山东】【正弦曲线解析式,三角恒等变换】
          % {\kaishu (2017 \textbullet 山东)}
          设函数$f(x)=\sin\Bp{\omega x-\dfrac{\piup}6}+\sin\Bp{\omega x-\dfrac{\piup}2}$,其中$0<\omega<3$.已知$f\Bp{\dfrac{\piup}6}=0$.\\
          (I)求$\omega$;\\
          (II)将函数$y=f(x)$的图像上各点的横坐标伸长为原来的2倍(纵坐标不变),再将得到的图像向左平移$\dfrac{\piup}4$个单位,得到函数$y=g(x)$的图像,求$g(x)$在$\Bigl[-\dfrac{\piup}4,\dfrac{3\piup}4\Bigr]$上的最小值.
          \begin{answer}
            (I)$f(x)=\sqrt3\sin\Bp{\omega x-\dfrac{\piup}3}$,$\omega=2$.
            (II)$g(x)=\sqrt3\sin\Bp{x-\dfrac{\piup}{12}}$,当$x=-\dfrac{\piup}4$时,$g(x)$取得最小值$-\dfrac32$
          \end{answer}
    \vspace{5cm}
  \end{exercise}


\newpage
\section{课后作业}
  % \begin{exercise}{\heiti 练习}
  % \end{exercise}
  \begin{exercise}
    \item%高中数学习题解法辞典.pdf 例2-1-19
      已知$\sin\Bigl(\dfrac{\piup}2+2x\Bigr)=-\dfrac12$,则$x=$\tk.
      \begin{answer}
        k\piup\pm\dfrac{\piup}3(k\in\mathbb{Z})
      \end{answer}
    \vspace{1.5em}
    \item%高中数学习题解法辞典.pdf 例2-2-4
      函数$y=\sqrt{25-x^2}+\lg\sin\Bigl(x+\dfrac{\piup}3\Bigr)$的定义域为\tk.
      \begin{answer}
        $\Bigl[-5,-\dfrac{4\piup}3\Bigr)\bigcup\Bigl(-\dfrac{\piup}3,\dfrac{2\piup}3\Bigr)$
      \end{answer}
    \vspace{1.5em}
    \item%福州三中中学2015-2016学年高一数学第二学期期末检测.docx-9
      % (福州三中中学2015-2016学年高一数学第二学期期末检测9)
      将函数$y=\sin\Bigl(x-\dfrac{\piup}3\Bigr)$的图像上所有点的横坐标伸长到原来的2倍(纵坐标不变),再将所得的图像向左平移$\dfrac{\piup}3$个单位,得到的函数图像对应的解析式是\xz
      \xx{$y=\sin\dfrac x2$}
        {$y=\sin\Bigl(\dfrac x2-\dfrac{\piup}2\Bigr)$}
        {$y=\sin\Bigl(\dfrac{x}2-\dfrac{\piup}6\Bigr)$}
        {$y=\sin\Bigl(2x-\dfrac{\piup}6\Bigr)$}
      \begin{answer}
        C
      \end{answer}
    \vspace{1.5em}
    \item%LaTeX-master/sanjiaohanshu/gaokaosection.tex 13
       已知函数$f(x)=\Bigg\{\begin{aligned}
      \sin(x+a),x\le 0\\\cos (x+b),x>0
      \end{aligned}$是偶函数,则下列结论可能成立的是\xz
       \xx{$ a=\dfrac{\piup}{4},b=-\dfrac{\piup}{4}$}
        {$ a=\dfrac{2\piup}{3},b=\dfrac{\piup}{6}$}
        {$a=\dfrac{\piup}{3},b=\dfrac{\piup}{6} $}
        {$ a=\dfrac{5\piup}{6},b=\dfrac{2\piup}{3}$}
      \begin{answer}
        C
      \end{answer}
    \vspace{1.5em}
    \item%《习题化知识清单》P72知识3-3
      若函数$y=2\cos(2x+\varphi)$是偶函数,且在$\Bigl(0,\dfrac{\piup}4\Bigr)$上是增函数,则实数$\varphi$可能是\xz
      \xx{$-\dfrac{\piup}2$}
        {0}
        {{$\dfrac{\piup}2$}}
        {$\piup$}
      \begin{answer}
        D
      \end{answer}
    \vspace{1.5em}
    \item%福州屏东中学2016-2017学年高一下学期期末考试数学试题.doc-10【正弦曲线图像】
      % (2017 \textbullet {\kaishu 屏东中学} 10)
      函数$f(x)=\sin(\omega x+\phi)$(其中$\abs{\phi}<\dfrac{\piup}2$)的图像如图所示,为了得到$y=\sin\omega x$的图像,只需把$y=f(x)$的图像上所有点(\hspace{2.5em})个长度单位.\\
      \begin{minipage}[b]{0.8\linewidth}
        \vspace{2.5em}
        \xx{向右平移$\dfrac{\piup}6$}
         {向右平移$\dfrac{\piup}{12}$}
         {向左平移$\dfrac{\piup}6$}
         {向左平移$\dfrac{\piup}{12}$}
      \end{minipage}\hfill
      \begin{minipage}[h]{0.2\linewidth}
        \vspace{-2.7cm}
        \begin{center}
          \begin{tikzpicture}[scale=0.8]
            \coordinate[label=below left:$O$] (O) at(0,0);
            \coordinate[label=above :\small$\tfrac{\piup}3$] (t1) at(pi/3,0);
            \draw[->,>=latex](-0.2*pi,0)--(1.2*pi,0)node[below](x) {$x$};
            \draw[->,>=latex](0,-1.5)--(0,1.5)node[right](y) {\small $y$};
            \draw [domain=-0.1*pi:0.9*pi,samples=100] plot(\x,{sin((2*\x+pi/3) r)});
            \draw[densely dashed](7*pi/12,0)node[above](pi){$\tfrac{7\piup}{12}$}--++(0,-1);
            \draw[densely dashed](0,-1)node[left](min){$-1$}--++(7*pi/12,0);
          \end{tikzpicture}
        \end{center}
      \end{minipage}
       \begin{answer}
         A
       \end{answer}
      \vspace{-0.9cm}
    \vspace{1.5em}
    \item%《2018天利38套:高考真题单元专题训练(理)ISBN978-7-223-03393-0》专题14三角函数的图像与性质P53p4【2017•全国新课标】【正弦曲线性质】
        {\kaishu (2017 \textbullet 全国新课标)}
        设函数$f(x)=\cos\Bp{x+\dfrac{\piup}3}$,则下列结论错误的是\xz
        \xx{$f(x)$的一个周期为$-2\piup$}
         {$y=f(x)$的图像关于直线$x=\dfrac{8\piup}3$对称}
         {$f(x+\piup)$的一个零点为$x=\dfrac{\piup}6$}
         {$f(x)$在$\Bp{\dfrac{\piup}2,\piup}$单调递减}
        \begin{answer}
          D
        \end{answer}
  \end{exercise}
\stopexercise

\newpage
\section{参考答案}
\begin{multicols}{2}
  \printanswer
\end{multicols}
